%\chapterimage{Geometrijska.jpg} % Chapter heading image

\chapter{Osnove laserjev}
V tem poglavju bomo na kratko opisali delovanje laserjev, ki so 
danes eno najpomembnejših orodij v optiki. Njihova prelomna
iznajdba v šestdesetih letih dvajsetega stoletja je omogočila
opazovanje povsem novih pojavov in uporabo optičnih metod
v številnih panogah industrije, medicine in telekomunikacij.

Do zdaj smo svetlobo obravnavali klasično, za
opis delovanja laserjev pa klasična optika ne zadošča in 
svetlobo moramo obravnavati kvantno. V kvantni optiki 
svetlobe ne opišemo kot valovanje, temveč kot 
prenos paketov (kvantov) energije, imenovanih fotoni. 
V tem poglavju bomo predstavili osnovne procese interakcije 
fotonov s snovjo in pojasnili, v katerih primerih se svetloba v snovi
absorbira in v katerih ojači. Opisali bomo preprost model laserja 
in na koncu na kratko opisali lastnosti laserske svetlobe.

\section{Sevanje črnega telesa}
V klasični sliki smo za opis snovi uporabili Lorentzev model, 
v katerem smo snov sestavili iz kroglic pozitivnega in 
negativnega naboja (glej razdelka~\ref{chap:lomni} in 
\ref{chap:AnizoLorentz}).
V tem modelu so kroglice med seboj povezane z vzmetmi, frekvenco
in amplitudo nihanja kroglic pa lahko zvezno spreminjamo.

V kvantnem opisu snovi atomi oziroma molekule zavzemajo
le stanja z diskretnimi vrednostmi energije. Obenem kvantno
obravnavamo tudi svetlobo, ki predstavlja potovanje diskretnih
kvantov energije, imenovanih fotoni. 
Energija fotona je sorazmerna z njegovo frekvenco $\nu$:
\boxeq{eq:foton}{
E_f = h\nu = \hslash \omega,
}
pri čemer je sorazmernostni faktor Planckova konstanta
$h = 6,62 \cdot 10^{-34}~\si{Js}$, imenovana 
po nemškem fiziku in nobelovcu Maxu Plancku (1858--1947).
Zaradi krajšega zapisa navadno uporabljamo reducirano Planckovo konstanto
$\hslash = h/2\pi$, 

Zamislimo si toplotni zalogovnik pri temperaturi $T$, v katerem 
je votlina s črnimi stenami. Znotraj votline se nahajajo atomi z diskretnimi
energijskimi nivoji $E_0, E_1, E_2$ ... Hkrati je v votlini zaradi 
končne temperature prisotno tudi termično sevanje v obliki 
elektromagnetnega valovanja (slika~\ref{fig:11_votlina}\,a). 
Oba sistema naj bosta v termičnem ravnovesju.

Atomi v votlini zasedajo različna diskretna  energijska stanja,
ki ustrezajo posameznim energijskim nivojem 
(slika~\ref{fig:11_votlina}\,b). Število atomov na izbranem
energijskem nivoju $i$ imenujemo zasedenost nivoja in jo označimo z $N_i$.
Razmerje med zasedenostmi nivojev podaja
Maxwell-Boltzmannova statistika:
\beq
\frac{N_2}{N_1} = \frac{e^{-E_2/k_BT}}{e^{-E_1/k_BT}} = e^{-(E_2-E_1)/k_BT},
\label{eq:11_01}
\eeq
pri čemer so $N_1$ zasedenost nižjega nivoja, $N_2$ zasedenost višjega 
nivoja in $k_B$ Boltzmannova konstanta, imenovana po avstrijskem
fiziku Ludwigu Boltzmannu (1844--1906). Njena vrednost znaša 
$k_B = 1,38 \cdot 10^{-23}~\si{J/K}$.
\begin{figure}[h!]
\centering
\def\svgwidth{80truemm} 
\input{slike/11_votlina.pdf_tex}
\caption{Zamislimo si votlino s črnimi stenami, v kateri so 
atomi in elektromagnetno valovanje v termičnem ravnovesju~(a). 
Atomi zasedajo le diskretne
energijske nivoje $E_i$, katerih zasedenost poda zasedbeno 
število $N_i$~(b). 
}
\label{fig:11_votlina}
\vglue-5truemm
\end{figure}

Elektromagnetno valovanje v votlini bomo opisali s spektrom sevanja
črnega telesa. Povprečno število fotonov $\langle n \rangle$ 
z energijo $\hslash \omega$ oziroma porazdelitev 
termičnega sevanja po frekvencah podaja 
Bose-Einsteinova statistika:
\beq
\langle n \rangle = \frac{1}{e^{\hslash \omega/k_B T}-1}.
\label{eq:11_02}
\eeq
Povprečna energija fotonov pri dani frekvenci je produkt energije fotona
in povprečnega števila fotonov pri tej frekvenci:
\beq
\langle W \rangle = \hslash \omega \langle n \rangle.
\label{eq:11_03}
\eeq
Za izračun spektra moramo upoštevati še ustrezno gostoto stanj, 
to je število različnih lastnih nihanj v votlini v bližini dane 
frekvence na enoto prostornine. Vzemimo kocko s stranico $L$ in 
preštejmo število lastnih nihajnih načinov (stoječih valovanj) v njej.
Valovne vektorje stoječih valovanj na splošno zapišemo v obliki:
\beq
\mathbf{k} = \left(\frac{\pi l}{L}, \frac{\pi m}{L}, \frac{\pi n}{L}\right)\!\!,
\label{eq:11_04}
\eeq
pri čemer so $l$, $m$ in $n$ cela števila. Valovni vektorji tako tvorijo 
mrežo v prvem oktantu prostora valovnih vektorjev.
Za velike vrednosti $k$ lahko število lastnih nihanj pri valovnih 
številih med $k$ in $k+dk$ izračunamo kot prostornino krogelne 
lupine pri $k$, deljeno s prostornino $\pi/L$, ki pripada posamezni 
mrežni točki. Dodamo še možnost dveh različnih polarizacij in dobimo:
\beq
dN = \frac{2}{8}4\pi k^2 dk \left(\frac{L}{\pi}\right)^3\!\!.
\label{eq:11_05}
\eeq
Odvisnost od $k$ prevedemo na frekvenčno odvisnost, delimo s prostornino
in za gostoto stanj, to je število lastnih nihanj na frekvenčni interval
na enoto prostornine, dobimo:
\beq
\varrho (\omega) = \frac{\omega^2}{\pi^2c^3}.
\label{eq:11_06}
\eeq
Čeprav smo gostoto stanj izpeljali za votlino v obliki kocke, 
je rezultat za dovolj velike vrednosti $k$ in $\omega$ splošno veljaven. 

Zdaj lahko izračunamo spekter izsevane svetlobe, tako da povprečno 
energijo izbranega lastnega nihanja (enačba~\ref{eq:11_03}) 
pomnožimo z gostoto stanj (enačba~\ref{eq:11_06}). 
Spektralno gostoto energije $u(\omega)$, ki je enaka energiji 
elektromagnetnega valovanja na enoto prostornine na enoto frekvence, 
opisuje Planckov zakon za sevanje črnega telesa v termičnem ravnovesju:
\boxeq{eq:Planck}{
u(\omega) =  
\frac{\hslash \omega^3}{\pi^2c^3}\frac{1}{e^{\hslash \omega/k_bT}-1}.
}
Spekter termičnega sevanja je močno odvisen od temperature telesa. Z naraščajočo
temperaturo njegova vrednost narašča, vrh spektra pa se pomika k višjim
frekvencam (slika~\ref{fig:11_Planck}).
\begin{figure}[h!]
\centering
\def\svgwidth{140truemm} 
\input{slike/11_Planck.pdf_tex}
\caption{Povprečna energija posameznega lastnega nihanja v odvisnosti
od njegove frekvence~(a), gostota stanj  oziroma lasnih nihanjih načinov
v votlini~(b) in njun produkt, to je Planckov spekter sevanja
črnega telesa pri dveh različnih temperaturah~(c). Velja $T_2>T_1$.
}
\label{fig:11_Planck}
\end{figure}

\section{Kvantni opis interakcije svetlobe s snovjo}
Nadaljujmo z obravnavo votline, v kateri so atomi 
in elektromagnetno valovanje v termičnem ravnovesju. 
Interakcijo med atomi in elektromagnetnim valovanjem 
opišemo s tremi različnimi procesi: absorpcijo, spontanim
sevanjem in stimuliranim sevanjem. Zaradi enostavnosti še
privzamemo, da imajo atomi samo dva energijska nivoja, nižjega pri energiji
$E_1$ in višjega pri $E_2$. Razliko med njima izrazimo s frekvenco kot:
\beq
E_2 -E_1 = \hslash \omega_{21}.
\label{eq:11_07}
\eeq
\vglue3truemm
\begin{figure}[h!]
\centering
\def\svgwidth{140truemm} 
\input{slike/11_procesi.pdf_tex}
\caption{Trije osnovni procesi interakcije svetlobe s snovjo:
absorpcija, pri kateri se foton absorbira, atom pa preide iz nižjega 
v višje stanje~(a); spontana emisija, pri kateri atom spontano preide 
iz višjega v nižje stanje in pri tem izseva foton~(b), in stimulirana 
emisija, pri kateri vpadni foton sproži prehod
atoma iz višjega v nižje stanje, izsevani foton pa je identično enak vpadnemu~(c).
}
\label{fig:11_procesi}
\end{figure}

\subsection*{Absorpcija} 
Absorpcija je proces, pri katerem atom absorbira vpadni foton in pri tem 
preide iz nižjega v višje energijsko stanje (slika~\ref{fig:11_procesi}\,a). 
Verjetnost za absorpcijo je močno odvisna od energije vpadnega fotona: če je 
njegova energija $\hslash \omega$ enaka razliki energij med nivojema atoma $E_2-E_1$, 
se foton lahko absorbira, sicer se ne more. Pomemben parameter pri izračunu 
verjetnosti za absorpcijo je tako spektralna gostota energije 
pri frekvenci prehoda med stanjema $u(\omega_{21})$. 
Število atomskih prehodov na časovno enoto je sorazmerno tudi zasedenosti
nižjega nivoja, tako da ga na splošno zapišemo kot:
\beq
\frac{dN_\mathrm{abs}}{dt} = B_{12}u(\omega_{21}) N_1,
\label{eq:11_08}
\eeq
pri čemer sorazmernostno konstanto označimo z $B_{12}$. 

\subsection*{Spontano sevanje}
Spontano sevanje (spontana emisija) je proces, pri katerem atom, 
ki se nahaja v višjem energijskem stanju, spontano preide v nižje 
stanje in ob tem izseva foton (slika~\ref{fig:11_procesi}\,b). Energija
izsevanega fotona je enaka razliki energij med atomskima nivojema.

Število spontanih prehodov na časovno enoto je sorazmerno številu
atomov v višjem stanju in ga zapišemo kot:
\beq
\frac{dN_\mathrm{sp}}{dt} = A_{21}N_2,
\label{eq:11_09}
\eeq
pri čemer smo vpeljali sorazmernostno konstanto $A_{21}$. Tipična vrednost 
za dovoljene prehode je $A_{21}\sim 10^{6}$--$10^8$, za prepovedane pa okoli 
štiri velikostne rede manj. Inverzna vrednost konstante $A_{21}$ 
ustreza življenjskemu času višjega nivoja oziroma naravnemu razpadnemu času višjega
stanja:
\beq
\tau = \frac{1}{A_{21}}.
\label{eq:11_10}
\eeq
Življenjski čas zgornjih nivojev je tako tipično $\tau\sim10$--$100~\si{ns}$. 
V primerjavi s stimuliranim sevanjem je spontano sevanje v delujočem laserju 
zanemarljivo in ga bomo zato pri računih pogosto zanemarili. 

\begin{remark}
Verjetnost za prehod atoma iz višjega v nižje energijsko stanje s 
spontanim sevanjem opisuje konstanta $A_{21}$. Izračunamo jo lahko z
uporabo Fermijevega zlatega pravila, ki pove verjetnost za prehod 
iz stanja $|2\rangle$ v stanje $|1\rangle$ z dipolnim sevanjem na časovno enoto: 
\beq
A_{21} = \frac{\omega_{21}^3 e_0^2\langle 2|\mathbf{r}|1\rangle^2}
{3h \varepsilon_0 c_0^3}.
\label{eq:11_11}
\eeq 
\end{remark}

\subsection*{Stimulirano sevanje}
Tretji način interakcije svetlobe s snovjo je za delovanje laserja
najpomembnejši. Stimulirano sevanje je namreč proces, pri katerem 
vpadni foton sproži prehod atoma iz višjega
v nižji energijski nivo, ob tem pa nastane foton, ki je identičen vpadnemu.
Tako iz enega vpadnega fotona nastaneta dva izhodna fotona, ki potujeta
v isti smeri, imata enako frekvenco, fazo in polarizacijo. Ta proces lahko
poteče le v primeru, da je energija vpadnega fotona enaka razliki energij 
med atomskima nivojema.

Podobno kot smo zapisali število prehodov na časovno enoto za absorpcijo 
(enačba~\ref{eq:11_08}), zapišemo tudi število prehodov na časovno enoto 
za stimulirano emisijo kot:
\beq
\frac{dN_\mathrm{st}}{dt} = B_{21}u(\omega_{21}) N_2.
\label{eq:11_12}
\eeq
Ker mora biti na začetku atom v višjem stanju, je število prehodov sorazmerno
$N_2$, sorazmernostni koeficient pa označimo z $B_{21}$. Tipične vrednosti
koeficientov so $B_{21}\sim 10^{16}$--$10^{20}~\si{m^3/Js^2}$.

\subsection*{Einsteinovi koeficienti}
Koeficiente $A$, $B_{12}$ in $B_{21}$, ki smo jih vpeljali kot sorazmernostne
koeficiente pri opisu interakcije svetlobe s snovjo, imenujemo Einsteinovi 
koeficienti. Izkaže se, da njihove vrednosti niso neodvisne, ampak veljata med 
njimi sledeči zvezi:
\beq
B_{12}=B_{21}= B
\label{eq:11_13}
\eeq
in 
\beq
\frac{A}{B} = \frac{\hslash \omega_{21}^3}{\pi^2c_0^3}.
\label{eq:11_14}
\eeq
\begin{example}{\bf Zveza med Einsteinovimi koeficienti.}
Zvezi med Einsteinovimi koeficienti (enačbi~\ref{eq:11_13} in \ref{eq:11_14}) 
lahko izračunamo na preprostem primeru. Naj imajo atomi samo dva nivoja, nižjega in 
višjega, energija vpadnih fotonov pa naj bo enaka razliki energij med nivojema
$\hslash \omega  = E_2 - E_1= \hslash \omega_{21}$. 
Celotno število atomov, ki so ali v nižjem ali v višjem stanju, je konstantno
in velja $N_1 + N_2 = N$. 

Spreminjanje zasedenosti nižjega stanja zapišemo kot vsoto treh prispevkov:
\beq
\frac{dN_1}{dt} = A_{21}N_2 + B_{21}u(\omega_{21})N_2 - B_{12}u(\omega_{21})N_1.
\label{eq:11_15}
\eeq
Prvi člen v vsoti opisuje povečanje zasedenosti nižjega nivoja zaradi 
spontanega sevanja (enačba~\ref{eq:11_09}), drugi člen povečanje 
zaradi stimuliranega sevanja (enačba~\ref{eq:11_12}) in tretji člen
zmanjševanje zasedenosti zaradi absorpcije (enačba~\ref{eq:11_08}).
Za spreminjanje zasedenosti višjega stanja je izraz zelo podoben, razlikuje
se le v predznaku, saj se vsota zasedenosti obeh stanj ohranja.

V stacionarnem stanju v termičnem ravnovesju se zasedenosti nivojev 
ne spreminjata in velja:
\beq
A_{21}N_2 + B_{21}u(\omega_{21})N_2 - B_{12}u(\omega_{21})N_1 = 0.
\label{eq:11_16}
\eeq
Izrazimo spektralno gostoto energije:
\beq
u(\omega_{21}) = \frac{A_{21}N_2}{B_{12}N_1-B_{21}N_2} = 
\frac{A_{21}}{B_{12}N_1/N_2 - B_{21}}.
\label{eq:11_17}
\eeq
Upoštevamo Boltzmannovo porazdelitev za zasedenost nivojev (
enačba~\ref{eq:11_01}) in dobimo:
\beq
u(\omega_{21}) = \frac{A_{21}}{B_{12}e^{\hslash \omega_{21}/k_B T} - B_{21}}.
\label{eq:11_18}
\eeq
Ker je v termičnem ravnovesju tudi elektromagnetno valovanje, ga 
lahko opišemo s Planckovim zakonom. Enačbo~(\ref{eq:11_18}) 
izenačimo z enačbo~(\ref{eq:Planck}):
\beq
u(\omega_{21}) = \frac{A_{21}/B_{21}}{e^{\hslash \omega_{21}/k_B T}B_{12}/B_{21} - 1} = 
\frac{\hslash \omega^3}{\pi^2c^3}\frac{1}{e^{\hslash \omega/k_bT}-1}
\eeq
in primerjamo koeficiente. Da se ujema funkcijska odvisnost v 
imenovalcu, mora veljati:
\beq
B_{12} = B_{21}.
\label{eq:11_19}
\eeq
Ker sta koeficienta enaka, ju lahko poenostavljeno označimo z $B$. 
Preostali faktorji se združijo v obliko:
\beq
\frac{A_{21}}{B} = \frac{\hslash \omega_{21}^3}{\pi^2c_0^3}
\label{eq:11_20}
\eeq
s čimer smo pokazali še enačbo~(\ref{eq:11_14}).

Parametre $A$ in $B$ imenujemo po Albertu Einsteinu, ki je že leta 
1916 proučeval interakcije svetlobe s snovjo in predlagal opisani model. 
Na podlagi modela je napovedal možnost ojačenja svetlobe s stimulirano 
emisijo in tako napovedal delovanje laserja. Do prvega delujočega laserja 
je potem minilo še  več kot 40 let. 
\end{example}

\section{Interakcija svetlobe s snovjo v optičnem resonatorju}
Optični resonatorji so naprave, znotraj katerih lahko vzbudimo 
stoječe elektromagnetno valovanje. Resonatorje navadno poznamo iz glasbe,
za svetlobo pa si lahko resonatorje mislimo kot zaprto škatlo
z zrcali na notranjih straneh, od katerih se svetloba odbija.
Frekvence vzbujenih stoječih valovanj (lastnih nihanj) so točno 
določene z geometrijo resonatorja in robnimi pogoji na njegovih stenah. 
Frekvence lastnih nihanj imenujemo tudi resonančne frekvence. 
Zaradi izgub v resonatorju črte v resonančnem spektru niso 
neskončno ozke, temveč imajo neko končno širino.

Po drugi strani tudi energija med posameznima atomskima nivojema 
ni povsem točno določena. Zaradi končnega razpadnega časa 
(enačba~\ref{eq:11_10}) ima spekter prehoda (absorpcije ali 
stimulirane emisije) neko končno širino. Tako imenovano
atomsko spektralno črto opišemo s funkcijo $g(\omega)$,
ki je Lorentzeve ali  Gaussove oblike okoli osrednje 
frekvence prehoda $\omega_{21}$. Zanjo velja:
\beq
\int_{-\infty}^\infty g(\omega) d\omega = 1.
\label{eq:11_21}
\eeq

V primeru sevanja črnega telesa smo lahko privzeli, 
da je njegov spekter znotraj atomske spektralne širine 
približno konstanten (slika~\ref{fig:11_g}\,a) in enak $u(\omega_{21})$.
To smo tudi upoštevali v zapisu enačb (\ref{eq:11_08} in \ref{eq:11_12}). 

Pri interakcijah svetlobe in snovi v resonatorju pa je treba
upoštevati prekrivanje resonančnega spektra resonatorja in 
spektra atomskega sistema. Oba imata končno širino, vendar 
je širina resonanc navadno dosti ožja od naravne širine 
emisijske oziroma absorpcijske črte (slika~\ref{fig:11_g}\,b).
\begin{figure}[h!]
\centering
\def\svgwidth{120truemm} 
\input{slike/11_g.pdf_tex}
\caption{Spekter sevanja črnega telesa $u(\omega)$ je bistveno širši 
od atomske spektralne črte $g(\omega)$ (a), spekter svetlobe iz 
resonatorja $u(\omega)$ pa je navadno bistveno ožji od atomske 
spektralne črte $g(\omega)$ (b). V laserjih velja drugi primer.
}
\label{fig:11_g}
\end{figure}

Na splošno moramo pri zapisu izrazov za absorpcijo in stimulirano emisijo 
upoštevamo širini obeh črt. Število prehodov z absorpcijo je tako:
\beq
\frac{dN_\mathrm{abs}}{dt} = BN_1 \int_{-\infty}^{\infty} g(\omega) u(\omega) d\omega
\label{eq:11_22}
\eeq
in za stimulirano emisijo:
\beq
\frac{dN_\mathrm{st}}{dt} = BN_2 \int_{-\infty}^{\infty} g(\omega) u(\omega) d\omega.
\label{eq:11_23}
\eeq
Prvi skrajni primer, ko je širina spektra črnega telesa bistveno širša
od atomske spektralne črte, smo že opisali, zdaj pa poglejmo še drugo skrajnost, 
ko je spektralna širina resonančnih črt, podana s funkcijo $u(\omega)$, bistveno 
ožja od atomske spektralne črte, podane z $g(\omega)$. V tem primeru 
lahko privzamemo, da je vrednost $g(\omega)$ znotraj integrala konstantna in 
upoštevamo le njeno vrednost pri resonančni frekvenci $\omega_R$.

Integral tudi v tem primeru preprosto izračunamo:
\beq
\int_{-\infty}^{\infty} g(\omega) u(\omega) d\omega = g(\omega_R) \int_{-\infty}^{\infty}
u(\omega) d\omega = g(\omega_R) w,
\label{eq:11_24}
\eeq
pri čemer je $\omega_R$ izbrana resonančna frekvenca resonatorja, $w$ pa gostota
energije elektromagnetnega valovanja. Število prehodov pri absorpciji
in stimuliranem sevanju na časovno enoto v tem približku zapišemo kot:
\beq
\frac{dN_\mathrm{abs}}{dt} = B N_1 g(\omega_R) w
\label{eq:11_25}
\eeq
in
\beq
\frac{dN_\mathrm{st}}{dt} = B N_2 g(\omega_R) w.
\label{eq:11_26}
\eeq
Ponovimo: $g(\omega_R)$ je vrednost atomske spektralne črte $g$ pri 
frekvenci $\omega_R$, ki ustreza monokromatskemu resonančnemu 
elektromagnetnemu valovanju znotraj resonatorja, $w$ pa je gostota
energije valovanja pri resonančni frekvenci $\omega_R$. 

\begin{remark}
Kljub temu da ima funkcija $g(\omega)$ navadno Lorentzovo ali Gaussovo 
obliko, jo pri računanju zaradi preprostosti velikokrat nadomestimo 
s škatlasto funkcijo. Njena vrednost je izven nekega ustreznega 
intervala $\Delta \omega$ enaka 0, znotraj tega intervala pa ima 
funkcija konstantno vrednost $g = 1/\Delta \omega$.
\end{remark}

\section{Ojačevanje svetlobe}
Fotoni, ki vpadajo na snov, se lahko v njej absorbirajo ali v njej sprožijo
stimulirano emisijo, pri čemer absorpcija fotonov vpadni svetlobni tok 
zmanjšuje, stimulirana emisija pa ga povečuje. Pri pravih pogojih lahko
dosežemo, da je moč izhodne svetlobe večja od moči vpadne svetlobe in tak
sistem imenujemo optični ojačevalnik.

Naj svetloba vpada vzdolž osi $z$ pod pravim kotom na tanko plast snovi. 
Debelina plasti naj bo $dz$, njena prostornina pa $dV = S dz$. V plasti naj 
bo $N_1$ atomov v nižjem energijskem stanju in $N_2$ v višjem. 
Moč vpadne svetlobe naj bo $P$ (slika~\ref{fig:11_absorpcija}). Ko 
svetloba prehaja skozi snov, se njena moč zmanjšuje zaradi absorpcije 
in povečuje zaradi stimuliranega sevanja. Spontano sevanje 
zanemarimo, saj je spontano izsevana svetloba enakomerno 
porazdeljena v vseh smereh in le zanemarljiv delež je usmerjen 
vzdolž osi $z$. Obravnavajmo primer, ko je spekter vpadne svetlobe
zelo ozek, na primer iz optičnega resonatorja.
\begin{figure}[h!]
\centering
\def\svgwidth{70truemm} 
\input{slike/11_absorpcija.pdf_tex}
\caption{Svetlobna moč se zmanjšuje zaradi absorpcije in povečuje
zaradi stimulirane emisije.}
\vglue-4truemm
\label{fig:11_absorpcija}
\end{figure}

Sprememba svetlobne moči ob prehodu skozi plast snovi je:
\beq
dP = \hslash \omega \left(\frac{dN_\mathrm{st}}{dt}- \frac{dN_\mathrm{abs}}{dt}\right)\!\!,
\label{eq:11_27}
\eeq
pri čemer smo prispevek spontanega sevanja že zanemarili.
Z uporabo enačb~(\ref{eq:11_25} in \ref{eq:11_26}) dobimo:
\beq
dP = 
\hslash \omega \left(N_2Bg(\omega) w - N_1Bg(\omega)w\right) = 
\hslash \omega\,B g(\omega) w\, (N_2 - N_1).
\label{eq:11_28}
\eeq
Upoštevamo zvezo $w = j/c$ (enačba~\ref{eq:j}) in enačbo delimo s $S$:
\beq
dj = \hslash \omega\,B g(\omega) \frac{j}{c}\frac{N_2-N_1}{V}dz.
\label{eq:11_29}
\eeq
Zapisano enačbo lahko v poenostavljeni obliki zapišemo kot:
\boxeq{eq:ojacenje}{
dj = \sigma\,\frac{N_2-N_1}{V}jdz = \gamma j dz.
}
Pri tem smo vpeljali sipalni presek za absorpcijo in stimulirano emisijo:
\beq
\sigma (\omega)= \frac{\hslash \omega\, Bg(\omega)}{c}.
\label{eq:11_30}
\eeq
Sipalni presek je odvisen od snovi in od frekvence vpadnega valovanja. 
Tipične vrednosti sipalnega preseka so $\sigma \sim 10^{-20}~\si{m^2}$.

Drugi parameter, ki smo ga vpeljali v enačbi~(\ref{eq:ojacenje}), 
je koeficient ojačenja $\gamma$:
\beq
\gamma = \sigma \,\frac{N_2-N_1}{V}
\label{eq:11_31}
\eeq
in je odvisen od zasedenosti posameznih stanj. 
Kadar je snov v termičnem ravnovesju in velja $N_2 < N_1$, je $\gamma <0$
in svetlobna moč se po prehodu skozi snov zmanjša. 

Da dosežemo
ojačenje svetlobnega toka, mora biti koeficient ojačenja $\gamma >0$. 
Iz enačbe~(\ref{eq:11_31}) sledi, da je ta pogoj izpolnjen, kadar je $N_2>N_1$.
Ojačenje svetlobe v snovi lahko dosežemo le v primeru, ko je zasedenost
višjega nivoja večja od zasedenosti nižjega nivoja. Tako neravnovesno 
stanje imenujemo stanje obrnjene zasedenosti. V termičnem ravnovesju 
obrnjene zasedenosti ne moremo doseči, zato moramo za ojačenje svetlobe
sistem atomov vzdrževati z dovajanjem energije od zunaj.

\section{Trinivojski sistemi}
Da lahko v nekem sistemu svetlobo ojačujemo, moramo med 
dvema nivojema doseči obrnjeno zasedenost. Izkaže se, da v dvonivojskem 
sistemu s še tako močnim črpanjem obrnjene zasedenosti ne moremo doseči, 
ampak moramo v interakcijski proces med snovjo in svetlobo vključiti 
vsaj tri atomske nivoje. Takrat govorimo o trinivojskem sistemu 
(slika~\ref{fig:11_3nivojski}).
\begin{figure}[h!]
\centering
\def\svgwidth{100truemm} 
\input{slike/11_3nivojski.pdf_tex}
\caption{Obrnjeno zasedenost dosežemo v vsaj trinivojskem sistemu (a). Energijske
nivoje označimo z $E_0$, $E_1$ in $E_2$, z istimi indeksi tudi ustrezne 
zasedenosti $N_0$, $N_1$ in $N_2$, 
prehode med nivoji (absorpcijo, spontano in stimulirano emisijo) pa s pripadajočimi
Einsteinovimi koeficienti $A$ in $B$. Parameter $r$ označuje črpanje. Primer
trinivojskega sistema, v katerem je laserski prehod med prvim in drugim vzbujenim
nivojem (b).
}
\vglue-5truemm
\label{fig:11_3nivojski}
\end{figure}

Naj bo celotno število atomov v sistemu $N$, zasedenosti posameznih nivojev pa 
označimo z indeksom nivoja. Potem velja:
\beq
N = N_0 + N_1 + N_2.
\label{eq:11_32}
\eeq
Pri obravnavi bomo privzeli, da je zasedenost osnovnega (najnižjega) nivoja daleč 
največja $N_0 \gg N_1, N_2$ in da zato velja $N_0 \approx N$. 

Na sistem trinivojskih atomov naj vpada svetloba s frekvenco:
\beq
\omega_{21} = \frac{E_2-E_1}{\hslash},
\label{eq:11_33}
\eeq
ki ustreza energijski razliki med prvim in drugim vzbujenim stanjem.  Če želimo
vpadno svetlobo ojačevati, moramo doseči obrnjeno zasedenost med nivojema 1 in 2.
To naredimo z zunanjim črpalnim mehanizmom, s katerim neprestano vzbujamo atome 
iz osnovnega v drugi vzbujeni nivo. Črpanje neodvisno od črpalnega mehanizma
zapišemo s koeficientom $r$:
\beq
\frac{dN_2}{dt} = r\, N_0.
\label{eq:11_34}
\eeq
Zasedbene enačbe za vse tri nivoje zapišemo z upoštevanjem vseh prehodov med njimi. 
Poglejmo podrobneje spreminjanje zasedenosti drugega vzbujenega nivoja 
(slika~\ref{fig:11_3nivojski}\,a). Zasedenost
se povečuje zaradi črpanja (enačba~\ref{eq:11_34}) in zmanjšuje zaradi spontanih
prehodov v prvo vzbujeno stanje ($A_{21}$) in v osnovno stanje ($A_{20})$. V prisotnosti
svetlobe se zasedenost povečuje še zaradi absorpcije ($B_{12}$) in zmanjšuje
zaradi stimuliranega sevanja ($B_{21}$). Vse prispevke združimo v eno enačbo in 
podobno sklepamo še za ostala dva nivoja. Dobimo sistem enačb:
\begin{align}
\frac{dN_2}{dt} &= rN_0 - A_{20}N_2 - A_{21}N_2 + B_{21}w(\omega_{21})g(\omega_{21}) 
(N_1-N_2),\label{eq:11_35}\\
\frac{dN_1}{dt} &= - A_{10}N_1 + A_{21}N_2 - B_{21}w(\omega_{21})g(\omega_{21}) 
(N_1-N_2),\label{eq:11_36}\\
\frac{dN_0}{dt} &= - rN_0 + A_{20}N_2 + A_{10}N_1.\label{eq:11_37}
\end{align}
Sistem je zapleten, vendar lahko že takoj naredimo nekaj poenostavitev. Namesto
$N_0$ vstavimo $N$, ki je konstanten, poleg tega lahko brez škode spontane prehode
iz drugega vzbujenega nivoja v osnovni nivo ($A_{20}$) zanemarimo. Spontanih prehodov
iz prvega v osnovni nivo ($A_{10}$) seveda ne smemo zanemariti, saj je to ključni
mehanizem, ki prazni nižji vzbujeni nivo.

Zanima nas stacionarno stanje, ko so časovni odvodi enaki nič. 
Iz enačbe~(\ref{eq:11_37}) dobimo:
\beq
N_1 = \frac{rN_0}{A_{10}} \approx \frac{rN}{A_{10}}.
\label{eq:11_38}
\eeq
Enačba~(\ref{eq:11_36}) se v stacionarnem stanju prepiše v:
\beq
\left(A_{21}+ B_{21}w(\omega_{21})g(\omega_{21})\right)N_2 = 
\left(A_{10}+ B_{21}w(\omega_{21})g(\omega_{21})\right)N_1.
\label{eq:11_39}
\eeq
Tretja enačba (enačba~\ref{eq:11_35}) ne da nove informacije, 
saj je vsota zasedenosti konstantna.

Iz enačb~(\ref{eq:11_38} in \ref{eq:11_39}) izrazimo razliko 
zasedenosti vzbujenih stanj in dobimo:
\beq
N_2-N_1 = \frac{A_{10}-A_{21}}{A_{21} + B_{21}w(\omega_{21})g(\omega_{21})}
\cdot \frac{rN}{A_{10}}.
\label{eq:11_40}
\eeq
Za dosego obrnjene zasedenosti v trinivojskem sistemu mora torej 
veljati $A_{10}>A_{21}$: verjetnost za prehod iz prvega vzbujenega nivoja v 
osnovni nivo mora biti večja od verjetnosti za prehod iz 
drugega vzbujenega v prvi vzbujeni nivo. Povedano drugače: 
za dosego obrnjene zasedenosti se mora srednji nivo 
prazniti hitreje od najvišjega nivoja. 

V praksi navadno uporabljamo sisteme, v katerih je $A_{10}\gg A_{21}$.
V tem primeru lahko izraz za obrnjeno zasedenost (enačba~\ref{eq:11_40}) 
še dodatno poenostavimo in dobimo:
\beq
N_2-N_1 = \frac{rN}{A_{21}}\cdot\frac{1}{1+B_{21}w(\omega_{21})g(\omega_{21})/A_{21}}.
\label{eq:11_41}
\eeq
Upoštevamo zvezo $w = j/c$ in zapišemo:
\beq
N_2-N_1 = \frac{rN}{A_{21}}\cdot\frac{1}{1+\frac{B_{21}g(\omega_{21})}{A_{21}c}j} = 
\frac{rN}{A_{21}}\cdot \frac{1}{1+j/j_s}.
\label{eq:11_42}
\eeq
Vpeljali smo saturacijsko gostoto toka:
\beq
j_s = \frac{A_{21}c}{B_{21}g(\omega_{21})},
\label{eq:11_42a}
\eeq
ki je odvisna od snovi in od frekvence vpadne svetlobe. Vidimo, da pri isti gostoti
vpadnega svetlobnega toka $j$ dosežemo večjo obrnjeno zasedenost pri 
večji saturacijski gostoti toka $j_s$ in seveda pri močnejšem črpanju.

Vstavimo izračunano obrnjeno zasedenost (enačba~\ref{eq:11_42}) v enačbo za 
optično ojačevanje (enačba~\ref{eq:ojacenje}):
\beq
dj = \sigma (\omega_{21})\frac{rN}{VA_{21}}\cdot \frac{1}{1 + j/j_s}j\,dz.
\label{eq:11_43}
\eeq
Vpeljemo koeficient ojačenja pri nizkih močeh oziroma nizkih gostotah vpadnega 
energijskega toka:
\beq
G = \sigma(\omega_{21})\frac{rN}{VA_{21}}.
\label{eq:11_44}
\eeq
Optično ojačevanje potem strnjeno zapišemo v obliki:
\boxeq{eq:3ojacevanje}{
dj = \frac{G j\,dz}{1+j/j_s}.
}

Enačbo~(\ref{eq:3ojacevanje}) integriramo:
\beq
\int_{j_0}^j \frac{dj}{j}\left(1 + j/j_s\right) = \int_0^z G\, dz,
\label{eq:11_45}
\eeq
pri čemer $j_0$ označuje vpadno gostoto svetlobnega toka, $j$ pa po 
prepotovani razdalji $z$. Dobimo:
\beq
\ln\frac{j}{j_0} + \frac{j-j_0}{j_s} = Gz.
\label{eq:11_46}
\eeq
Zapisana enačba podaja spreminjanja gostote
energijskega toka v odvisnosti od prepotovane razdalje v snovi. Odvisnosti
$j(z)$ ne moremo enostavno zapisati, lahko pa jo narišemo.
Na sliki~\ref{fig:11_ojacenje} je prikazano naraščanje
relativne gostote svetlobnega toka $j(z)/j_0$ v odvisnosti od $z$ pri 
izbrani vrednosti saturacijskega toka $j_s/j_0 = 20$.
\begin{figure}[h!]
\centering
\def\svgwidth{70truemm} 
\input{slike/11_ojacenje.pdf_tex}
\caption{Ojačenje svetlobe v trinivojskem sistemu z obrnjeno zasedenostjo.
Pri majhnih vpadnih močeh je naraščanje eksponentno, pri velikih močeh 
pa linearno, saj se ojačenje nasiti.
}
\label{fig:11_ojacenje}
\end{figure}

Naraščanje gostote svetlobnega toka (enačba~\ref{eq:11_46}) 
lahko preprosto zapišemo v limitnih primerih. 
Kadar so vpadne moči majhne in je $j \ll j_s$, drugi člen
na levi strani enačbe zanemarimo in ostane eksponentno 
naraščajoča odvisnost od dolžine prepotovane poti po snovi:
\beq
j = j_0e^{Gz}.
\label{eq:11_47}
\eeq
Eksponentno naraščanje vidimo na sliki~\ref{fig:11_ojacenje}
pri majhnih vrednostih $j$. Pri velikih vpadnih močeh,
za katere velja $j \gg j_s$, naraščanje svetlobne moči preide
v linearno obnašanje. To pokažemo tudi z limitnim računom, tako
da zanemarimo prvi člen v enačbi~(\ref{eq:11_46}). Dobimo:
\beq
j = j_0 + Gj_s\, z.
\label{eq:11_48}
\eeq
Opisani pojav imenujemo nasičenje ojačenja. Svetloba se pri 
majhnih vrednostih ojačuje, saj je prehodov s stimulirano 
emisijo več kot absorpcije. Pri velikih intenzitetah pa začne
vzbujenih atomov, ki bi lahko izsevali svetlobo, primanjkovati.
Svetlobna moč narašča zato le še linearno s prepotovano razdaljo, 
saj je v novih prostorninskih enotah še nekaj vzbujenih atomov, 
ki lahko s stimulirano emisijo oddajo svetlobo. Parameter, ki pove, 
ali se bo svetloba eksponentno ojačevala ali bo njena
intenziteta naraščala le še linearno, je ravno saturacijska
gostota svetlobnega toka $j_s$ (enačba~\ref{eq:11_42a}). 

Poglejmo to razlago še matematično. V enačbi~(\ref{eq:11_48})
je ključni parameter produkt $Gj_s$. Vstavimo enačbi
za koeficient ojačenja pri nizkih močeh ter saturacijsko 
gostoto~(\ref{eq:11_42a} in \ref{eq:11_44}) ter
upoštevamo enačbo~(\ref{eq:11_30}) za izračun preseka
za absorpcijo in stimulirano emisijo. Sledi:
\beq
Gj_s = \frac{\hslash \omega_{21}B_{21}g}{c}\cdot \frac{r N}{V A_{21}} 
\cdot \frac{A_{21}c}{B_{21}g}=
\hslash \omega_{21}\frac{rN}{V}.
\label{eq:11_49}
\eeq
Povečanje gostote svetlobnega toka v plasti debeline $dz$ potem zapišemo kot:
\beq
dj = \hslash \omega_{21}\frac{rN}{V}dz,
\label{eq:11_50}
\eeq
kar pomeni, da v izbrani plasti snovi vsi atomi, 
ki jih s črpalnim mehanizmom vzbudimo v drugi vzbujeni
nivo, na osnovi stimulirane emisije prispevajo dodatni 
svetlobni tok k vpadnemu elektromagnetnemu valovanju.
Takrat torej vso razpoložljivo energijo sistema počrpamo in ojačenje se nasiti. 

\section{Laser}
Laser je naprava, v kateri izkoristimo stimulirano emisijo 
za nastanek koherentne in monokromatske izhodne svetlobe. Dodatna odlika
laserke svetlobe je njena ozka usmerjenost (majhna divergenca svetlobnega snopa)
in posledično zelo velike gostote svetlobnega toka v snopu. Poglejmo preprost
model laserja.

Laser\footnote{Laser je pravzaprav kratica za {\it Light Amplification by Stimulated
Emission of Radiation}, to je ojačevanje svetlobe s stimuliranim sevanjem.} 
v grobem sestavljajo trije osnovni sestavni deli: ojačevalno sredstvo, 
v katerem se svetloba ojačuje, črpalni sistem, ki vzdržuje obrnjeno zasedenost
v ojačevalnem sredstvu, in resonator, ki zagotavlja prevlado stimulirane emisije.

Ojačevalno sredstvo je lahko plin, trdna snov ali tudi tekočina. Zelo razširjeni
plinski laserji so helij-neonski (He-Ne) laser, v katerem je ojačevalno sredstvo
mešanica plinov helija in neona; najbolj razširjenji laserji pa so diodni laserji,
v katerih se svetloba ojačuje na pn stiku polprevodnika. 

Črpalni mehanizem je odvisen od izbire ojačevalnega sredstva, najpogostejši način je 
z električnim tokom. V helij-neonskem laserju, na primer, v razelektritveni 
cevi elektroni trkajo z atomi helija. Vzbujeni atomi helija s trki prenesejo 
energijo na atome neona in tako vzdržujemo obrnjeno zasedenost. 

Resonatorji sta navadno dve ukrivljeni zrcali, od katerih ima eno odbojnost
kar se da veliko $R_1 \approx 100\%$, drugo pa malo manjšo $R_2 < 100\%$, 
da svetloba sploh lahko izhaja iz laserja. Med zrcali resonatorja postavimo
ojačevalno sredstvo in svetloba, ki se od zrcal odbija, se pri vsakem prehodu
ojačevalnega sredstva ojači.

Izhajanje svetlobe skozi izhodno zrcalo predstavlja poglavitne izgube resonatorja. 
Ko se izgube na prelet resonatorja izenačijo z ojačevanjem svetlobe na prelet, deluje
laser v stacionarnem stanju. Poglejmo ta proces podrobneje.

Naj bo dolžina laserskega resonatorja $L$, dolžina ojačevalnega stedstva v resonatorju pa $L'$. V enem 
obhodu resonatorja, to je od enega zrcala do drugega in nazaj, svetloba dvakrat preide skozi ojačevalno
sredstvo. Povečanje gostote svetlobnega toka na en obhod je tako:
\beq
\Delta j = \frac{G j 2L'}{1+j/j_s}.
\eeq
Izračunamo še povečanje energije elektromagnetnega valovanja, pri čemer upoštevamo zvezo:
$W = wV = jV/c$ in dobimo:
\beq
\Delta W = \frac{GW 2L'}{1+W/W_s},
\eeq
pri čemer je $W_s = j_s V/c$ saturacijska energija. 

Po drugi strani se energija iz resonatorja tudi izgublja. Deloma zaradi namenskega izhajanja
svetlobe iz laserja skozi izhodno zrcalo, deloma zaradi izgub v resonatorju zaradi absorpcije, sipanja:
\beq
\Delta W = -\Lambda W,
\eeq
pri čemer parameter $\Lambda$ opisuje izgube na obhod resonatorja.

Stacionarno stanje je doseženo, ko se ojačenje na obhod resonatorja po velikosti ravno izenači z izgubami:
\beq
\Delta W = \Lambda W = \frac{GW 2L'}{1+W/W_s}.
\eeq
Enačba ima dve rešitvi:
\beq
W=0 \qquad \mathrm{in} \qquad W = W_s \left(\frac{G}{G_\mathrm{pr}} - 1\right),
\eeq
pri čemer je $G_\mathrm{pr} = \lambda/2L'$ ojačenje na pragu. Vrednost parametra $G$ 
je odvisna od črpanja. Dokler je črpanje šibko in vrednost $G$ majhna, laser ne sveti in velja prva rešitev. 
Ko je črpanje dovolj veliko, je $G>G_\mathrm{pr}$ in energija v laserju narašča linearno s parametrom $G$.
Takrat laser na osnovi stimulirane emisije oddaja usmerjeni koherentni izhodni snop svetlobe, kar
je karakteristika laserjev.

Izhodni svetlobni tok iz laserja je:
\beq
P = (1-\mathcal{R}_2) W_s \left( \frac{G}{G_\mathrm{pr}}-1\right) \frac{c}{2L},
\eeq
pri čemer smo upoštevali čas obhoda resonatorja, v katerem je izsevana izračunana energija.
