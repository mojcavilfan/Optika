%\chapterimage{Geometrijska.jpg} % Chapter heading image

\chapter{Osnove laserjev}

\section{Kvantni opis interakcije svetlobe s snovjo}
Za razliko od Lorentzovega modela, v katerem smo privzeli, da je snov sestavljena
iz kroglic pozitivnega in negativnega naboja, ki sta povezani z vzmetjo
in lahko oscilirata s poljubno frekvenco oziroma amplitudo, pri kvantnem
opisu upoštevamo, da lahko zavzamejo atomi oziroma molekule stanja z diskretnimi vrednostmi
energije. Obenem upoštevamo tudi, da elektromagnetno valovanje snovi
lahko odda oziroma od nje dobi le diskretne ``pakete'' energije, ki jih imenujemo
fotoni. Energija fotona je 
\boxeq{eq:foton}{
E_f = \hslash \omega.
}
Zamislimo si toplotni rezervoar pri temperaturi $T$, v katerem je votlina s črnimi
stenami. V votlini so atomi z energijskimi nivoji $E_0, E_1, E_2...$. Hkrati je 
v votlini prisotno tudi termično sevanje oziroma elektromagnetno valovanje. Oba sistema
sta v termičnem ravnovesju.

Razmerje med številom atomov, ki se nahajajo v različnih  diskretnih energijskih
stanjih (zasedenost nivojev), podaja Maxwell-Boltzmannova statistika. 
\beq
\frac{N_2}{N_1} = e^{-(E_2-E_1)/k_BT},
\label{eq:11_01}
\eeq
pri čemer so $k_B$ Boltzmannova konstanta v vrednosti $k_B = 1,38 \cdot 10^{-23}~\si{J/K}$, 
$N_1$ zasedenost nižjega nivoja, $N_2$ pa zasedenost višjega nivoja.
Razmerje med številom fotonov pri različnih frekvencah oziroma porazdelitev 
termičnega sevanja po frekvencah podaja Bose-Einsteinova statistika, ki ob 
ustreznem upoštevanju gostote stanj vodi do spektra, ki ga opisuje Planckov zakon
za termično sevanje:
\beq
u(\omega)d\omega = \hslash \omega \frac{\omega^2}{\pi^2c_0^3}\frac{1}{e^{\hslash \omega/k_bT}-1} d\omega = 
\frac{\hslash \omega^3}{\pi^2c_0^3}\frac{1}{e^{\hslash \omega/k_bT}-1}d\omega.
\label{eq:Planck}
\eeq
Spekter $u(\omega)$ opisuje elektromagnetno energijo na enoto prostornine na na 
en hertz spektralne širine. 

Zaradi interakcije med elektromagnetnim poljem in atomi v votlini prihaja do treh različnih
vrst procesov: absorpcije, spontanega sevanja in stimuliranega sevanja oziroma stimulirane
emisije. Pri emisijskih pojavih prehajajo atomi iz višjih na nižje energijske nivoje, pri 
absorpciji pa iz nižje na višje energijske nivoje. Razlika med nivojema atoma naj bo $E_{21} = 
\hslash \omega_{21}$.

Spontano sevanje: Atom v vzbujenem stanju spontano preide v nižje stanje, pri čemer izseva
foton. Število spontatnih prehodov na časovno enoto je:
\beq
\frac{dN_\mathrm{sp}}{dt} = A_{21}N_2,
\eeq
pri čemer je:
\beq
A_{21} = \frac{P_\mathrm{dip}}{\hslash \omega_{21}} = 
\frac{1}{\hslash \omega_{21}}\frac{\omega_{21}^4 e_0^2\left(\langle 2|\mathbf{r}|1\rangle\rangle)^2}{3\pi \varepsilon_0 c_0^3},
\eeq
ki označuje verjetnost za prehod iz stanja $|2\rangle$ v stanje $|1\rangle$ na časovno enoto
z dipolnim sevanjem (Fermijevo zlato pravilo). Tipična vrednost je $A_{21}  = A \sim 10^{-9}$ za 
dovoljene prehode, kar ustreza življenjskemu času zgornjih nivojev $\tau = 1/A \sim 1~\si{ns}$. 
Spontano sevanje bomo pogosto zanemarili. 

Absorpcija: Ko foton z ustrezno energijo vpade na atom, se lahko absorbira, pri čemer atom 
preide v vzbujeno stanje. Verjetnost za prehod je močno odvisna od energije fotona: če je 
njegova energija približno enaka energiji med nivojema atoma, se foton lahko absorbira, sicer
se ne more. Zato je število prehodov na časovno enoto odvisno tudi od spektralne gostote 
energije $u(\omega_{21})$ pri frekvenci prehoda med stanjema:
\beq
\frac{dN_\mathrm{abs}}{dt} = B_{12}u(\omega_{21}) N_1.
\eeq
Sorazmernostno konstanto smo označili z $B_{12}$, število prehodov pa je seveda odvisno
tudi od celotnega števila atomov v nižjem nivoju $N_1$.

Stimulirano sevanje: Ko foton z ustrezno energijo vpade na vzbujen atom, lahko atom preide
iz višjega v nižje stanje, pri tem pa izseva foton, ki je enak vpadnemu fotonu. Iz enega 
vpadnega fotona in atoma v vzbujenem stanju tako dobimo dva identična fotona, ki potujeta v isti smeri
z enako energijo. Število prehodov na časovno enoto zapišemo kot:
\beq
\frac{dN_\mathrm{st}}{dt} = B_{21}u(\omega_{21}) N_2,
\eeq
pri čemer je $B_{21}$ sorazmernostni koeficient, $N_2$ pa število atomov v vzbujenem stanju. 

Koeficiente $A$, $B_{12}$ in $B_{21}$ imenujemo Einsteinovi koeficienti. Njihove vrednosti 
niso neodvisne, ampak velja:
\beq
B_{12}=B_{21}= B \qquad \mathrm{in} \qquad \frac{A}{B} = \frac{\hslash \omega_{21}^3}{\pi^2c_0^3}.
\eeq

\begin{example}{\bf Zveza med Einsteinovimi koeficienti.}
Razmerje med Einsteinovimi koeficienti lahko pojasnimo na preprostem primeru. Naj interakcija
poteka le med nivojema 2 in 1 ter elektromagnetnim valovanjem pri frekvenci $\omega_{21}$. Celotno
število atomov v enem in drugem stanju je konstantno, zato $N_1 + N_2 = N$ in spreminjanje zasedenosti
posameznih nivojev zapišemo kot:
\beq
\frac{dN_1}{dt} = A_{21} + B_{21}u(\omega_{21})N_2 - B_{12}u(\omega_{21})N_1 = \frac{dN_2}{dt}.
\eeq
Prvi člen v vsoti opisuje povečanje zasedenosti spodnjega nivoja zaradi spontanega sevanja, 
drugi člen zaradi stimuliranega sevanja, tretji pa zmanjševanje zasedenosti zaradi absorpcije.  
V stacionarnem stanju v termičnem ravnovesju se zasedenosti nivojev ne spreminjata, zato velja:
\beq
A_{21} + B_{21}u(\omega_{21})N_2 - B_{12}u(\omega_{21})N_1 = 0
\eeq
in 
\beq
u(\omega_{21}) = \frac{A_{21}N_2}{B_{12}N_1-B{21}N_2} = \frac{A_{21}}{B_{12}N_1/N_2 - B_{21}}.
\eeq
Upoštevamo Boltzmannovo porazdelitev za zasedenost nivojev (enačba~\ref{eq:11_01}) in dobimo:
\beq
u(\omega_{21}) = \frac{A_{21}}{B_{12}e^{\hslash \omega_{21}/k_B T} - B_{21}}.
\eeq
Ker je tudi elektromagnetno valovanje v termičnem ravnovesju, vemo, da velja Planckov zakon 
(enačba~\ref{eq:Planck}). Primerjava izračunanega izraza in Plankovega zakona vodi do 
zvez med Einsteinovimi koeficienti:
\beq
B_{12} = B_{21}
\eeq
in
\beq
\frac{A}{B} = \frac{\hslash \omega_{21}^3}{\pi^2c_0^3}.
\eeq
Einstein proučeval interakcije svetlobe s snovjo in že leta 1916. Napovedal možnost ojačenja
svetlobe in delovanja laserja (preveri), postavil teoretične osnove za razlago in delovanje laserja.
Potem je trajalo še več desetletij, da so prvi laser naredili. 
\end{example}

\subsection*{Interakcija svetlobe s snovjo v optičnem resonatorju}
Optični resonatorji so naprave, znotraj katerih lahko vzbudimo stoječe elektromagnetno 
valovanje pri določenih frekvencah (resonančnih frekvencah). Za optično valovanje
si jih lahko zamislimo kot zaprto škatlo z zrcali v notranjosti, da se svetloba odbija.
Posamezne črte v resonančnem spektru niso neskončno ozke, je pa širina resonanc navadno dosti ožja 
od naravne širine emisijske  oziroma absorpcijske širine atomskega sistema. 
 Na procese absorpcije in stimulirane emisije v tem primeru vpliva
prekrivanje med resonančnim spektrom resonatorja in spektrom atomskega sistema. 

Spektralno odvisnost emisije oziroma absorpcije podaja naravna širina spektralne 
črte, ki jo opišemo s funkcijo $g(\omega)$, za katero velja:
\beq
\int_{-\infty}^\infty g(\omega) d\omega = 1.
\eeq
Na splošno zapišemo za absorpcijo:
\beq
\frac{dN_{abs}}{dt} = BN_1 \int_{-\infty}^{\infty} g(\omega) u(\omega) d\omega
\eeq
in za stimulirano emisijo:
\beq
\frac{dN_{st}}{dt} = BN_2 \int_{-\infty}^{\infty} g(\omega) u(\omega) d\omega.
\eeq
Širok sevalni spekter spontanega sevanja atomskega sistema in ozki resonančni vrhovi, 
ki so lahko vzbujeni v izbranem resonatorju. Prekrivanje med svetlobo, ki lahko obstaja
v resonatorju, in spektrom spontanega sevanja (razliko energij med stanjema), 
da verjetnost za prehod na časovno enoto.

V ekstremnem primeru, ko je spektralna širina resonančnih črt, 
podana s funkcijo $u(\omega)$, bistveno ožja od širine absorpcijskega oziroma
emisijskega spektra, podane z $g(\omega)$, lahko privzamemo, da je vrednost $g(\omega)$
znotraj integrala konstantna in upoštevamo le njeno vrednost pri resonančni frekvenci.
Dobimo:
\beq
\int_{-\infty}^{\infty} g(\omega) u(\omega) d\omega = g(\omega_R) \int_{-\infty}^{\infty}
u(\omega) d\omega = g(\omega_R) w,
\eeq
pri čemer je $\omega_R$ osrednja resonančna frekvenca resonatorja, $w$ pa gostota
energije elektromagnetnega valovanja. Poenostavljeno potem število prehodov pri absorpciji
in stimuliranem sevanju zapišemo kot:
\beq
\frac{dN_{abs}}{dt} = B N_1 g(\omega) w
\eeq
in
\beq
\frac{dN_{st}}{dt} = B N_2 g(\omega) w.
\eeq



\section{Ojačevanje svetlobe}

\section{Trinivojski sistemi}
\section{Laser}
