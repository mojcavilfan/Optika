%\chapterimage{Geometrijska.jpg} % Chapter heading image

\chapter{Osnove laserjev}
V tem poglavju bomo na kratko opisali delovanje laserjev, ki so 
danes eno najpomembnejših orodij v optiki. Njihova prelomna
iznajdba v šestdesetih letih dvajsetega stoletja je omogočila
opazovanje povsem novih pojavov in uporabo optičnih metod
v številnih panogah industrije, medicine in telekomunikacij.

Do zdaj smo svetlobo obravnavali klasično, za
opis delovanja laserjev pa klasična optika ne zadošča in 
svetlobo moramo obravnavati kvantno. V kvantni optiki 
svetlobe ne opišemo kot valovanje, temveč kot 
prenos paketov (kvantov) energije, imenovanih fotoni. 
V tem poglavju bomo predstavili osnovne procese interakcije 
fotonov s snovjo in pojasnili, v katerih primerih se svetloba v snovi
absorbira in v katerih ojači. Opisali bomo preprost model laserja 
in na koncu na kratko opisali lastnosti laserske svetlobe.

\section{Sevanje črnega telesa}
V klasični sliki smo za opis snovi uporabili Lorentzev model, 
v katerem smo snov sestavili iz kroglic pozitivnega in 
negativnega naboja (glej razdelka~\ref{chap:lomni} in 
\ref{chap:AnizoLorentz}).
V tem modelu so kroglice med seboj povezane z vzmetmi, frekvenco
in amplitudo nihanja kroglic pa lahko zvezno spreminjamo.

V kvantnem opisu snovi atomi oziroma molekule zavzemajo
le stanja z diskretnimi vrednostmi energije. Obenem kvantno
obravnavamo tudi svetlobo, ki predstavlja potovanje diskretnih
kvantov energije, imenovanih fotoni. 
Energija fotona je sorazmerna z njegovo frekvenco $\nu$:
\boxeq{eq:foton}{
E_f = h\nu = \hslash \omega,
}
pri čemer je sorazmernostni faktor Planckova konstanta
$h = 6,62 \cdot 10^{-34}~\si{Js}$, imenovana 
po nemškem fiziku in nobelovcu Maxu Plancku (1858--1947).
Zaradi krajšega zapisa navadno uporabljamo reducirano Planckovo konstanto
$\hslash = h/2\pi$, 

Zamislimo si toplotni zalogovnik pri temperaturi $T$, v katerem 
je votlina s črnimi stenami. Znotraj votline se nahajajo atomi z diskretnimi
energijskimi nivoji $E_0, E_1, E_2$ ... Hkrati je v votlini zaradi 
končne temperature prisotno tudi termično sevanje v obliki 
elektromagnetnega valovanja (slika~\ref{fig:11_votlina}\,a). 
Oba sistema naj bosta v termičnem ravnovesju.

Atomi v votlini zasedajo različna diskretna  energijska stanja,
ki ustrezajo posameznim energijskim nivojem 
(slika~\ref{fig:11_votlina}\,b). Število atomov na izbranem
energijskem nivoju $i$ imenujemo zasedenost nivoja in jo označimo z $N_i$.
Razmerje med zasedenostmi nivojev podaja
Maxwell-Boltzmannova statistika:
\beq
\frac{N_2}{N_1} = \frac{e^{-E_2/k_BT}}{e^{-E_1/k_BT}} = e^{-(E_2-E_1)/k_BT},
\label{eq:11_01}
\eeq
pri čemer so $N_1$ zasedenost nižjega nivoja, $N_2$ zasedenost višjega 
nivoja in $k_B$ Boltzmannova konstanta, imenovana po avstrijskem
fiziku Ludwigu Boltzmannu (1844--1906). Njena vrednost znaša 
$k_B = 1,38 \cdot 10^{-23}~\si{J/K}$.
\begin{figure}[h!]
\centering
\def\svgwidth{80truemm} 
\input{slike/11_votlina.pdf_tex}
\caption{Zamislimo si votlino s črnimi stenami, v kateri so 
atomi in elektromagnetno valovanje v termičnem ravnovesju~(a). 
Atomi zasedajo le diskretne
energijske nivoje $E_i$, katerih zasedenost poda zasedbeno 
število $N_i$~(b). 
}
\label{fig:11_votlina}
\vglue-5truemm
\end{figure}

Elektromagnetno valovanje v votlini bomo opisali s spektrom sevanja
črnega telesa. Povprečno število fotonov $\langle n \rangle$ 
z energijo $\hslash \omega$ oziroma porazdelitev 
termičnega sevanja po frekvencah podaja 
Bose-Einsteinova statistika:
\beq
\langle n \rangle = \frac{1}{e^{\hslash \omega/k_B T}-1}.
\label{eq:11_02}
\eeq
Povprečna energija fotonov pri dani frekvenci je produkt energije fotona
in povprečnega števila fotonov pri tej frekvenci:
\beq
\langle W \rangle = \hslash \omega \langle n \rangle.
\label{eq:11_03}
\eeq
Za izračun spektra moramo upoštevati še ustrezno gostoto stanj, 
to je število različnih lastnih nihanj v votlini v bližini dane 
frekvence na enoto prostornine. Vzemimo kocko s stranico $L$ in 
preštejmo število lastnih nihajnih načinov (stoječih valovanj) v njej.
Valovne vektorje stoječih valovanj na splošno zapišemo v obliki:
\beq
\mathbf{k} = \left(\frac{\pi l}{L}, \frac{\pi m}{L}, \frac{\pi n}{L}\right)\!\!,
\label{eq:11_04}
\eeq
pri čemer so $l$, $m$ in $n$ cela števila. Valovni vektorji tako tvorijo 
mrežo v prvem oktantu prostora valovnih vektorjev.
Za velike vrednosti $k$ lahko število lastnih nihanj pri valovnih 
številih med $k$ in $k+dk$ izračunamo kot prostornino krogelne 
lupine pri $k$, deljeno s prostornino $\pi/L$, ki pripada posamezni 
mrežni točki. Dodamo še možnost dveh različnih polarizacij in dobimo:
\beq
dN = \frac{2}{8}4\pi k^2 dk \left(\frac{L}{\pi}\right)^3\!\!.
\label{eq:11_05}
\eeq
Odvisnost od $k$ prevedemo na frekvenčno odvisnost, delimo s prostornino
in za gostoto stanj, to je število lastnih nihanj na frekvenčni interval
na enoto prostornine, dobimo:
\beq
\varrho (\omega) = \frac{\omega^2}{\pi^2c^3}.
\label{eq:11_06}
\eeq
Čeprav smo gostoto stanj izpeljali za votlino v obliki kocke, 
je rezultat za dovolj velike vrednosti $k$ in $\omega$ splošno veljaven. 

Zdaj lahko izračunamo spekter izsevane svetlobe, tako da povprečno 
energijo izbranega lastnega nihanja (enačba~\ref{eq:11_03}) 
pomnožimo z gostoto stanj (enačba~\ref{eq:11_06}). 
Spektralno gostoto energije $u(\omega)$, ki je enaka energiji 
elektromagnetnega valovanja na enoto prostornine na enoto frekvence, 
opisuje Planckov zakon za sevanje črnega telesa v termičnem ravnovesju:
\boxeq{eq:Planck}{
u(\omega) =  
\frac{\hslash \omega^3}{\pi^2c^3}\frac{1}{e^{\hslash \omega/k_bT}-1}.
}
Spekter termičnega sevanja je močno odvisen od temperature telesa. Z naraščajočo
temperaturo njegova vrednost narašča, vrh spektra pa se pomika k višjim
frekvencam (slika~\ref{fig:11_Planck}).
\begin{figure}[h!]
\centering
\def\svgwidth{140truemm} 
\input{slike/11_Planck.pdf_tex}
\caption{Povprečna energija posameznega lastnega nihanja v odvisnosti
od njegove frekvence~(a), gostota stanj  oziroma lasnih nihanjih načinov
v votlini~(b) in Planckov spekter sevanja
črnega telesa pri dveh različnih temperaturah~(c). Velja $T_2>T_1$.
}
\label{fig:11_Planck}
\end{figure}

\section{Kvantni opis interakcije svetlobe s snovjo}
Nadaljujmo z obravnavo votline, v kateri so atomi 
in elektromagnetno valovanje v termičnem ravnovesju. 
Interakcijo med atomi in elektromagnetnim valovanjem 
opišemo s tremi različnimi procesi: absorpcijo, spontanim
sevanjem in stimuliranim sevanjem. Poglejmo jih podrobneje.
\vglue3truemm
\begin{figure}[h!]
\centering
\def\svgwidth{140truemm} 
\input{slike/11_procesi.pdf_tex}
\caption{Trije osnovni procesi interakcije svetlobe s snovjo:
absorpcija, pri kateri se foton absorbira, atom pa preide iz nižjega v višje stanje~(a); spontana emisija, pri kateri atom spontano preide iz višjega v nižje stanje in pri 
tem izseva foton~(b), in stimulirana emisija, pri kateri vpadni foton sproži prehod
atoma iz višjega v nižje stanje, izsevani foton pa je identično enak vpadnemu~(c).
}
\label{fig:11_procesi}
\end{figure}

\subsection*{Absorpcija} 
Absorpcija je proces, pri katerem atom absorbira vpadni foton in pri tem 
preide iz nižjega v višje energijsko stanje (slika~\ref{fig:11_procesi}\,a). 
Verjetnost za absorpcijo je močno odvisna od energije vpadnega fotona: če je 
njegova energija enaka energiji med nivojema atoma, se foton lahko absorbira, sicer
se ne more. Pomemben parameter pri zapisu verjetnosti za prehod je tako spektralna
gostota energije $u(\omega)$.


Število prehodov na časovno enoto odvisno tudi od spektralne gostote 
energije $u(\omega_{21})$ pri frekvenci prehoda med stanjema:
\beq
\frac{dN_\mathrm{abs}}{dt} = B_{12}u(\omega_{21}) N_1.
\eeq
Sorazmernostno konstanto smo označili z $B_{12}$, število prehodov pa je seveda odvisno
tudi od celotnega števila atomov v nižjem nivoju $N_1$.



Pri sevalnih (emisijskih) pojavih atomi prehajajo iz višjih na nižje 
energijske nivoje, pri absorpciji pa iz nižje na višje energijske nivoje. 
Razlika med nivojema atoma naj bo $E_{21} = \hslash \omega_{21}$.

Spontano sevanje: Atom v vzbujenem stanju spontano preide v nižje stanje, pri čemer izseva
foton. Število spontatnih prehodov na časovno enoto je:
\beq
\frac{dN_\mathrm{sp}}{dt} = A_{21}N_2,
\label{eq:11_07}
\eeq
pri čemer je:
\beq
A_{21} = \frac{P_\mathrm{dip}}{\hslash \omega_{21}} = 
\frac{1}{\hslash \omega_{21}}\frac{\omega_{21}^4 e_0^2\left(\langle 2|\mathbf{r}|1\rangle\right)^2}{3\pi \varepsilon_0 c_0^3},
\eeq
ki označuje verjetnost za prehod iz stanja $|2\rangle$ v stanje $|1\rangle$ na časovno enoto
z dipolnim sevanjem (Fermijevo zlato pravilo). Tipična vrednost je $A_{21}  = A \sim 10^{-9}$ za 
dovoljene prehode, kar ustreza življenjskemu času zgornjih nivojev $\tau = 1/A \sim 1~\si{ns}$. 
Spontano sevanje bomo pogosto zanemarili. 



Stimulirano sevanje: Ko foton z ustrezno energijo vpade na vzbujen atom, lahko atom preide
iz višjega v nižje stanje, pri tem pa izseva foton, ki je enak vpadnemu fotonu. Iz enega 
vpadnega fotona in atoma v vzbujenem stanju tako dobimo dva identična fotona, ki potujeta v isti smeri
z enako energijo. Število prehodov na časovno enoto zapišemo kot:
\beq
\frac{dN_\mathrm{st}}{dt} = B_{21}u(\omega_{21}) N_2,
\eeq
pri čemer je $B_{21}$ sorazmernostni koeficient, $N_2$ pa število atomov v vzbujenem stanju. 

Koeficiente $A$, $B_{12}$ in $B_{21}$ imenujemo Einsteinovi koeficienti. Njihove vrednosti 
niso neodvisne, ampak velja:
\beq
B_{12}=B_{21}= B \qquad \mathrm{in} \qquad \frac{A}{B} = \frac{\hslash \omega_{21}^3}{\pi^2c_0^3}.
\eeq

\begin{example}{\bf Zveza med Einsteinovimi koeficienti.}
Razmerje med Einsteinovimi koeficienti lahko pojasnimo na preprostem primeru. Naj interakcija
poteka le med nivojema 2 in 1 ter elektromagnetnim valovanjem pri frekvenci $\omega_{21}$. Celotno
število atomov v enem in drugem stanju je konstantno, zato $N_1 + N_2 = N$ in spreminjanje zasedenosti
posameznih nivojev zapišemo kot:
\beq
\frac{dN_1}{dt} = A_{21} + B_{21}u(\omega_{21})N_2 - B_{12}u(\omega_{21})N_1 = \frac{dN_2}{dt}.
\eeq
Prvi člen v vsoti opisuje povečanje zasedenosti spodnjega nivoja zaradi spontanega sevanja, 
drugi člen zaradi stimuliranega sevanja, tretji pa zmanjševanje zasedenosti zaradi absorpcije.  
V stacionarnem stanju v termičnem ravnovesju se zasedenosti nivojev ne spreminjata, zato velja:
\beq
A_{21} + B_{21}u(\omega_{21})N_2 - B_{12}u(\omega_{21})N_1 = 0
\eeq
in 
\beq
u(\omega_{21}) = \frac{A_{21}N_2}{B_{12}N_1-B{21}N_2} = \frac{A_{21}}{B_{12}N_1/N_2 - B_{21}}.
\eeq
Upoštevamo Boltzmannovo porazdelitev za zasedenost nivojev (enačba~\ref{eq:11_01}) in dobimo:
\beq
u(\omega_{21}) = \frac{A_{21}}{B_{12}e^{\hslash \omega_{21}/k_B T} - B_{21}}.
\eeq
Ker je tudi elektromagnetno valovanje v termičnem ravnovesju, vemo, da velja Planckov zakon 
(enačba~\ref{eq:Planck}). Primerjava izračunanega izraza in Plankovega zakona vodi do 
zvez med Einsteinovimi koeficienti:
\beq
B_{12} = B_{21}
\eeq
in
\beq
\frac{A}{B} = \frac{\hslash \omega_{21}^3}{\pi^2c_0^3}.
\eeq
Einstein proučeval interakcije svetlobe s snovjo in že leta 1916. Napovedal možnost ojačenja
svetlobe in delovanja laserja (preveri), postavil teoretične osnove za razlago in delovanje laserja.
Potem je trajalo še več desetletij, da so prvi laser naredili. 
\end{example}

\subsection*{Interakcija svetlobe s snovjo v optičnem resonatorju}
Optični resonatorji so naprave, znotraj katerih lahko vzbudimo stoječe elektromagnetno 
valovanje pri določenih frekvencah (resonančnih frekvencah). Za optično valovanje
si jih lahko zamislimo kot zaprto škatlo z zrcali v notranjosti, da se svetloba odbija.
Posamezne črte v resonančnem spektru niso neskončno ozke, je pa širina resonanc navadno dosti ožja 
od naravne širine emisijske  oziroma absorpcijske širine atomskega sistema. 
 Na procese absorpcije in stimulirane emisije v tem primeru vpliva
prekrivanje med resonančnim spektrom resonatorja in spektrom atomskega sistema. 

Spektralno odvisnost emisije oziroma absorpcije podaja naravna širina spektralne 
črte, ki jo opišemo s funkcijo $g(\omega)$, za katero velja:
\beq
\int_{-\infty}^\infty g(\omega) d\omega = 1.
\eeq
Na splošno zapišemo za absorpcijo:
\beq
\frac{dN_{abs}}{dt} = BN_1 \int_{-\infty}^{\infty} g(\omega) u(\omega) d\omega
\eeq
in za stimulirano emisijo:
\beq
\frac{dN_{st}}{dt} = BN_2 \int_{-\infty}^{\infty} g(\omega) u(\omega) d\omega.
\eeq
Širok sevalni spekter spontanega sevanja atomskega sistema in ozki resonančni vrhovi, 
ki so lahko vzbujeni v izbranem resonatorju. Prekrivanje med svetlobo, ki lahko obstaja
v resonatorju, in spektrom spontanega sevanja (razliko energij med stanjema), 
da verjetnost za prehod na časovno enoto.

V ekstremnem primeru, ko je spektralna širina resonančnih črt, 
podana s funkcijo $u(\omega)$, bistveno ožja od širine absorpcijskega oziroma
emisijskega spektra, podane z $g(\omega)$, lahko privzamemo, da je vrednost $g(\omega)$
znotraj integrala konstantna in upoštevamo le njeno vrednost pri resonančni frekvenci.
Dobimo:
\beq
\int_{-\infty}^{\infty} g(\omega) u(\omega) d\omega = g(\omega_R) \int_{-\infty}^{\infty}
u(\omega) d\omega = g(\omega_R) w,
\eeq
pri čemer je $\omega_R$ osrednja resonančna frekvenca resonatorja, $w$ pa gostota
energije elektromagnetnega valovanja. Poenostavljeno potem število prehodov pri absorpciji
in stimuliranem sevanju zapišemo kot:
\beq
\frac{dN_{abs}}{dt} = B N_1 g(\omega) w
\eeq
in
\beq
\frac{dN_{st}}{dt} = B N_2 g(\omega) w.
\eeq
Pri tem je $g(\omega)$ vrednost funkcije $g$ pri frekvenci $\omega$, ki ustreza monokromatskemu
resonančnemu elektromagnetnemu valovanju znotraj resonatorja, $w$ pa je gostota
energije valovanja pri resonančni frekvenci $\omega$. 

Funkcija $g(\omega)$ ima pogosto Gaussovo ali Lorenzovo obliko. Pri računanju pa jo
zaradi enostavnosti velikokrat nadomestimo s škatlasto funkcijo, ki je izven intervala $\Delta \omega$
enaka 0, znotraj tega intervala pa ima konstantno vrednost $g(\omega) = 1/\Delta \omega$.

\section{Ojačevanje svetlobe}
Zanima nas proces, ko na neko snov usmerimo snop svetlobe s svetlobnim tokom $P_0$, na izhodu
iz snovi pa dobimo svetlobni snop s tokom $P_1 > P_0$.  Napravi, ki poveča moč vpadne svetlobe, 
imenujemo optični ojačevalnik. 

Opazujemo procese v tanki plasti snovi debeline $dz$, katere prostornina je $dV = S dz$. V
tej tej plasti naj bo $\tilde{N}_1$ atomov v nižjem energijskem nivoju in $\tilde{N}_2$ 
v višjem energijskem nivoju.
Pri potovanju svetlobe skozi tako snov se svetlobni tok zmanjšuje zaradi absorpcije in povečuje
zaradi stimulirane in spontane emisije. Zanima nas samo tok v smeri vpadnega valovanja (vdolž osi
$z$), zato lahko spontano sevanje, ki je enakomerno porazdeljeno po vseh smereh prostora, zanemarimo.
Snov je postavljena v resonator (ali res?)

Sprememba energijskega toka je:
\beq
dP = S dj \hslash \omega \left(\frac{dN_{st}}{dt}- \frac{dN_{abs}}{dt} = 
\hslash \omega \left( \tilde{N}_2Bg(\omega) w - \tilde{N}_1Bg(\omega)w\right) = 
\hslash \omega B g(\omega) w (\tilde{N}_2 - \tilde{N}_1\right).
\eeq
Upoštevamo $j = wc$ in hkrati predpostavimo, da sta gostoti atomov v stanjih 1 in 2 v celotni
snovi konstantni:
\beq
\tilde{N}_1 = \frac{N_1}{V}Sdz \qquad \mathrm{in}\qquad \tilde{N}_2 = \frac{N_2}{V}Sdz.
\eeq
Dobimo:
\beq
dP = S dj = \hslash \omega B g(\omega) \frac{j}{c}\frac{N_2-N_1}{V}S dz
\eeq
in
\boxeq{eq:ojacenje}{
dj = \sigma \frac{N_2-N_1}{V}jdz = \gamma j dz,
}
pri čemer smo vpeljali sipalni presek za absorpcijo in stimulirano emisijo:
\beq
\sigma (\omega)= \frac{\hslash \omega g(\omega) B}{c}.
\eeq
Tipične vrednosti sipalnega preseka so $\sigma \sim 10^{-20}~\si{m^2}$.

Če želimo doseči ojačenje svetlobnega toka, mora biti koeficient ojačenja $\gamma$ 
pozitiven. Iz enačbe .. sledi, da je ta pogoj izpoljen le, kadar je $N_2-N_1>0$
in je torej zasedenost višjega stanja večja od zasedenosti nižjega stanja. 
Tako neravnovesno stanje imenujemo stanje obrnjene zasedenosti. V termičnem 
ravnovesju obrnjene zasedenosti ne moremo doseči, zato moramo za ojačenje svetlobe
sistem atomov vzdrževati z dovajanjem energije od zunaj.

\section{Trinivojski sistemi}
Da lahko v nekem sistemu dosežemo obrnjeno zasedenost in s tem optično ojačevanje,
moramo v interakcijski proces med snovjo in svetlobo vključiti vsaj tri atomske nivoje.
Takrat govorimo o trinivojskem sistemu. V dvonivojskem obrnjene zasedenosti s še tako
močnim dovajanjem energije (črpanjem) ne moremo doseči.

Naj bo celotno število atomov v sistemu $N$, zasedenosti posameznih nivojev pa 
označimo z indeksom. Potem velja:
\beq
N = N_0 + N_1 + N_2,
\eeq
pri čemer privzamemo, da je zasedenost osnovnega nivoja daleč največja $N_0 \gg N_1, N_2$
in da je posledično $N_0 \approx N$. Prisotna je tudi svetloba s frekvenco:
\beq
\omega_{21} = \frac{E_2-E_1}{\hslash},
\eeq
pri čemer $E_1$ in $E_2$ označujeta energiji prvega in drugega vzbujenega nivoja. Naš
cilj je ojačati svetlobo s frekvenco $\omega_{21}$, kar dosežemo le ob obrnjeni 
zasedenosti med nivojema 1 in 2.
V ta namen z zunanjim črpalnim mehanizmom neprestano vzbujamo atome iz osnovnega nivoja
v drugi vzbujeni nivo, črpanje pa neodvisno od mehanizma zapišemo v koeficientom $r$
oziroma:
\beq
\frac{dN_2}{dt} = r N_0.
\eeq
Zasedbene enačbe za vse tri nivoje potem zapišemo z upoštevanjem prehodov med njimi:
\begin{align}
\frac{dN_2}{dt} &= rN_0 - A_{20}N_2 - A_{21}N_2 + B_{21}w(\omega_{21})g(\omega_{21}) (N_1-N_2),\\
\frac{dN_1}{dt} &= - A_{10}N_1 + A_{21}N_2 - B_{21}w(\omega_{21})g(\omega_{21}) (N_1-N_2),\\
\frac{dN_0}{dt} &= - rN_0 + A_{20}N_2 + A_{10}N_1.
\end{align}
Brez škode lahko spontane prehode iz drugega vzbujenega nivoja v osnovni nivo zanemarimo, poleg 
tega namesto $N_0$ vstavimo $N$. 

Zanima nas stacionarno stanje, ko so časovni odvodi enaki nič. Iz tretje enačbe dobimo
preprosto zvezo:
\beq
N_1 = \frac{rN_0}{A_{10}},
\eeq
iz druge pa:
\beq
\left(B_{21}w(\omega_{21})g(\omega_{21}) + A_{21}\right)N_2 = \left(B_{21}w(\omega_{21})g(\omega_{21}) + 
A_{10}\right)N_1.
\eeq
Sledi:
\beq
N_2 = \frac{B_{21}w(\omega_{21})g(\omega_{21}) + A_{10}}{B_{21}w(\omega_{21})g(\omega_{21}) + A_{21}}N_1
\eeq
in od tod z upoštevanjem enačbe...:
\beq
N_2-N_1 = \frac{A_{10}-A_{21}}{B_{21}w(\omega_{21})g(\omega_{21}) + A_{21}}\frac{rN}{A_{10}}.
\eeq
Za dosego obrnjene zasedenosti v trinivojskem sistemu mora torej veljati $A_{10}>A_{21}$. Verjetnost za
prehod iz nivoja 1 v osnovni nivo mora biti torej večja od verjetnost za prehod iz drugega vzbujenega
v prvi vzbujen nivo. Povedano drugače: najvišji nivo se mora prazniti počasneje od srednjega nivoja, da med
njima dosežemo obrnjeno zasedenost.

V praksi navadno uporabljamo sisteme, v katerih je $A_{10}\gg A_{21}$. V tem primeru lahko izraz
za obrnjeno zasedenost še dodatno poenostavimo in dobimo:
\beq
N_2-N_1 = \frac{rN}{A_{21}}~\frac{1}{1+B_{21}w(\omega_{21})g(\omega_{21})/A_{21}}.
\eeq
Upoštevamo še zvezo $w(\omega_{21}) = j/c$ in zapišemo:
\beq
N_2-N_1 = \frac{rN}{A_{21}}~\frac{1}{1+\frac{B_{21}g(\omega_{21})}{A_{21}c}j} = 
\frac{rN}{A_{21}}~\frac{1}{1+j/j_s}.
\eeq
Vpeljali smo saturacijsko gostoto toka $j_s = A_{21}c/B_{21}g(\omega_{21})$. 

Vstavimo izračunano obrnjeno zasedenost v enačbo za optično ojačevanje (enčba ..):
\beq
dj = \sigma (\omega_{21})\frac{rN}{VA_{21}}\frac{1}{1 + j/j_s}j dz = \frac{G j dz}{1+j/j_s},
\eeq
pri čemer 
\beq
G = \sigma(\omega_{21})\frac{rN}{VA_{21}}
\eeq
označuje koeficient ojačenja pri nizkih močeh.

Enačbo .. lahko integriramo:
\beq
\int_{j_0}^j \frac{dj}{j}\left(1 + j/j_s\right) = \int_0^z G dz
\eeq
in dobimo zvezo:
\beq
\ln\frac{j}{j_0} + \frac{j-j_0}{j_s} = Gz.
\eeq
Zapisana enačba podaja odvisnost $z(j)$, ki jo v limitnih primerih preprosto obrnemo 
v bolj uporabno odvisnost $j(z)$. Kadar so vpadne moči majhne in je $j \ll j_s$, drugi člen
na levi strani zanemarimo in ostane eksponentno naraščajoča odvisnost od prepotovane poti po snovi:
\beq
dj \approx Gjdz \qquad \Longrightarrow \qquad j = j_0e^{Gz}.
\eeq
Za velike vpadne moči, za katere velja $j \gg j_s$ svetlobna moč 
narašča le linearno:
\beq
dj \approx \frac{G j dz}{j/j_s} qquad \Longrightarrow \qquad j = j_0 + Gj_s z.
\eeq
Opisani pojav imenujemo nasičenje ojačenja. Vstavimo koeficient ojačenja
pri nizkih močeh (enačba), saturacijsko gostoto toka (enačba) in presek za absorpcijo (enačba). 
Dobimo:
\beq
Gj_s = \frac{\sigma r N A_{21}c}{V A_{21}B_{21}g(\omega_{21})} = \frac{\hslash \omega_{21}rN 
B_{21}g(\omega_{21})}{V B_{21}g(\omega_{21})} = \hslash \omega_{21}\frac{rN}{V}.
\eeq
Od tod zapišemo povečanje gostote svetlobnega toka:
\beq
dj = \hslash \omega_{21}\frac{rN}{V}dz,
\eeq
kar pomeni, da v izbrani plasti snovi vsi atomi, ki jih s črpalnim mehanizmom vzbudimo v drugi vzbujeni
nivo, na osnovi stimulirane emisije prispevajo dodatni svetlobni tok k vpadnemu elektromagnetnemu valovanju.
Takrat torej vso razpoložljivo energijo sistema počrpamo in ojačenje je nasičeno. 

\section{Laser}
Laser je naprava, v kateri izkoristimo stimulirano emisijo za nastanek koherentne in monokromatske
svetlobe. Sestavljajo jo trije osnovni sestavni deli: ojačevalno sredstvo za ojačenje svetlobe, 
črpalni sistem za vzdrževanje obrnjene zasedenosti v ojačevalnem sredstvu, in resonator, ki zagotavlja
prevlado stimulirane emisije.

Beseda laser je pravzaprav kratica(?) za Light Amplification by Stimulated Emission of Radiation, 
to je ojačevanje svetlobe s stimuliranim sevanjem. 

Ojačevalno sredstvo je lahko plin, trdna snov ali tudi tekočina. Črpalni mehanizem je z električnim
tokom, svetilka (ali drug laser), resonator pa sta dve ukrivljeni zrcali, med katera postavimo
ojačevalno sredstvo. Svetloba se odbija med zrcaloma, vsakič se pri prehodu snovi ojači. Eno zrcalo
je karseda odbojno z odbojnostjo $100~\%$, drugi pa ima malo manjšo odbojnost, da svetloba izhaja
iz laserja. 

Naj bo dolžina laserskega resonatorja $L$, dolžina ojačevalnega stedstva v resonatorju pa $L'$. V enem 
obhodu resonatorja, to je od enega zrcala do drugega in nazaj, svetloba dvakrat preide skozi ojačevalno
sredstvo. Povečanje gostote svetlobnega toka na en obhod je tako:
\beq
\Delta j = \frac{G j 2L'}{1+j/j_s}.
\eeq
Izračunamo še povečanje energije elektromagnetnega valovanja, pri čemer upoštevamo zvezo:
$W = wV = jV/c$ in dobimo:
\beq
\Delta W = \frac{GW 2L'}{1+W/W_s},
\eeq
pri čemer je $W_s = j_s V/c$ saturacijska energija. 

Po drugi strani se energija iz resonatorja tudi izgublja. Deloma zaradi namenskega izhajanja
svetlobe iz laserja skozi izhodno zrcalo, deloma zaradi izgub v resonatorju zaradi absorpcije, sipanja:
\beq
\Delta W = -\Lambda W,
\eeq
pri čemer parameter $\Lambda$ opisuje izgube na obhod resonatorja.

Stacionarno stanje je doseženo, ko se ojačenje na obhod resonatorja po velikosti ravno izenači z izgubami:
\beq
\Delta W = \Lambda W = \frac{GW 2L'}{1+W/W_s}.
\eeq
Enačba ima dve rešitvi:
\beq
W=0 \qquad \mathrm{in} \qquad W = W_s \left(\frac{G}{G_\mathrm{pr}} - 1\right),
\eeq
pri čemer je $G_\mathrm{pr} = \lambda/2L'$ ojačenje na pragu. Vrednost parametra $G$ 
je odvisna od črpanja. Dokler je črpanje šibko in vrednost $G$ majhna, laser ne sveti in velja prva rešitev. 
Ko je črpanje dovolj veliko, je $G>G_\mathrm{pr}$ in energija v laserju narašča linearno s parametrom $G$.
Takrat laser na osnovi stimulirane emisije oddaja usmerjeni koherentni izhodni snop svetlobe, kar
je karakteristika laserjev.

Izhodni svetlobni tok iz laserja je:
\beq
P = (1-\mathcal{R}_2) W_s \left( \frac{G}{G_\mathrm{pr}}-1\right) \frac{c}{2L},
\eeq
pri čemer smo upoštevali čas obhoda resonatorja, v katerem je izsevana izračunana energija.
