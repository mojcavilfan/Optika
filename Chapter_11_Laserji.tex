%\chapterimage{Geometrijska.jpg} % Chapter heading image

\chapter{Osnove laserjev}

\section{Kvantni opis interakcije svetlobe s snovjo}
Za razliko od Lorentzovega modela, v katerem smo privzeli, da je snov sestavljena
iz kroglic pozitivnega in negativnega naboja, ki sta povezani z vzmetjo
in lahko oscilirata s poljubno frekvenco oziroma amplitudo. Pri kvantnem
opisu upoštevamo, da imajo atomi oziroma molekule stanja z diskretnimi vrednostmi
energije. Obenem upoštevamo tudi, da elektromagnetno valovanje snovi
lahko odda oziroma od nje dobi le diskretne ``pakete'' energije, ki jih imenujemo
fotoni. Energija fotona je 
\boxeq{eq:foton}{
E_f = \hbar \omega.
}
Zamislimo si toplotni rezervoar pri temperaturi $T$, v katerem je votlina s črnimi
stenami. V votlini so atomi z energijskimi nivoji $E_0, E_1, E_2...$. Hkrati je 
v votlini prisotno tudi termično sevanje oziroma elektromagnetno valovanje. Oba sistema
sta v termičnem ravnovesju.

Razmerje med številom atomov, ki se nahajajo v različnih  diskretnih energijskih
stanjih (zasedenost nivojev), podaja Maxwell-Boltzmannova statistika. 
\beq
\frac{N_2}{N_1} = e^{-(E_2-E_1)/k_BT},
\eeq
pri čemer je $k_B$ Boltzmannova konstanta v vrednosti $k_B = 1,38 \cdot 10^{-23}~\si{J/K}$.
Razmerje med številom fotonov pri različnih frekvencah oziroma porazdelitev 
termičnega sevanja po frekvencah podaja Bose-Einsteinova statistika, ki ob 
ustreznem upoštevanju gostote stanj vodi do spektra, ki ga opisuje Planckov zakon
za termično sevanje:
\beq
u(\omega)d\omega = \hbar \omega \frac{\omega^2}{\pi^2\c_0^3}\frac{1}{e^{\hbar \omega/k_bT}-1} d\omega = 
\frac{\hbar \omega^3}{\pi^2c_0^3}\frac{1}{e^{\hbar \omega/k_bT}-1}d\omega.
\eeq
Spekter $u(\omega)$ opisuje elektromagnetno energijo na enoto prostornine na na 
en hertz spektralne širine. 

Zaradi interakcije med elektromagnetnim poljem in atomi v votlini prihaja do treh različnih
vrst procesov: absorpcije, spontanega sevanja in stimuliranega sevanja oziroma stimulirane
emisije. Pri emisijskih pojavih prehajajo atomi iz višjih na nižje energijske nivoje, pri 
absorpciji pa iz nižje na višje energijske nivoje. 

Spontano sevanje: Atom v vzbujenem stanju spontano preide v nižje stanje, pri čemer izseva
foton. Število spontatnih prehodov na časovno enoto je:
\beq
\frac{dN_\mathrm{sp}}{dt} = A_{21}N_2,
\eeq
pri čemer je:
\beq
A_{21} = \frac{P_{dip}}{\hbar \omega_{21}}.
\eeq














\section{Osnovni procesi interakcije}
\section{Sevanje črnega telesa}
\section{Ojačevanje svetlobe}
\section{Trinivojski sistemi}
\section{Laser}
