%\chapterimage{Geometrijska.jpg} % Chapter heading image

\chapter{Sipanje}
Pri sipanju svetlobe nas navadno zanima učinek ovir, ki jih postavimo na pot svetlobi. 
Če te ovire niso ploščate, njhovega vpliva ne moremo obravnavati z navadno uklonsko
teorijo, pri kateri oviro opišemo s preprosto aperturno funkcijo $f(x,y)$. V primeru
3D-ovir, kot je denimo dielektrična kroglica, se faza valovanja spreminja tudi vzdolž
osi $z$ in integral v objektni ravnini $xy$ ne zadošča več. 

Sipanje svetlobe vpliva na slabljenje oziroma ekstinkcijo svetlobnega toka v smeri
osi $z$. Podobno učinkuje tudi absorpcija svetlobe, zato je celotno slabljenje
svetlobe sestavljeno iz prispevkov sipanja in absorpcije. Nas bodo zanimale le
dielektrične snovi, ki svetlobe ne absorbirajo, na primer majhne steklene kroglice.

Povezava med sipanjem in modelom ploščatih ovir, kot jih obravnavamo pri uklonski teoriji, 
je količina, ki jo imenujemo sipalni presek:
\beq
\sigma_S = Q_S A.
\label{eq:07_01}
\eeq
Pri tem sta $Q_S$ sipalni izkoristek in $A$ dejanski fizični presek oziroma 2D projekcija delca.

Zaradi sipanja gostota svetlobnega toka v smeri $z$ eksponentno pojema:
\boxeq{eq:07_02}{
j = j(z=0) e^{-\mu z},
}
pri čemer je $\mu$ koeficient slabljenja (ekstinkcijski ali atenuacijski koeficient). Izračunamo ga
kot $\mu = \sigma_S \varrho_S$, pri čemer je $\sigma_S$ sipalni presek, $\varrho_S$ pa gostota sipalcev,
to je delcev, na katerih se svetloba siplje. 

Zapisano zvezo lahko preprosto izpeljemo. Svetloba naj vpada na tanko plast snovi s prostornino $V = S dz$,
v kateri je $N$ sipalcev. Vsak od teh sipalcev odvzame svetlobni tok $dj S = dP_1 = \sigma_S j$. Vsi delci
v tanki plasti snovi tako odvzamejo svetlobni tok $dP = N \sigma_S j$. Potem velja:
\beq
dP = -\sigma_S N j = - \sigma_S j \varrho_S S dz,
\label{eq:07_03}
\eeq
od koder dobimo:
\beq
dj = \frac{dP}{S} = - \sigma_S j \varrho_S dz
\label{eq:07_04}
\eeq
in 
\beq
\frac{dj}{j} = - \sigma_S \varrho_S dz = -\mu dz.
\label{eq:07_05}
\eeq
Izraz integriramo in dobimo enačbo~(\ref{eq:07_02}). Zapisana enačba, ki opisuje eksponentno
pojemanje intenzitete vpadne svetlobe zaradi sipanja je povsem analogna Beer-Lambertovem zakonu
v primeru absorpcije v snovi. 

*** Dodaj primer. Recimo mleko in črnilo, v mleku sipanje, v črnilu absorpcija?***

\section{Rayleighovo sipanje}
Zaradi enostavnosti se omejimo na sipanje na sferičnih delcih. Prvi primer naj bo Rayleighovo 
sipanje, ki dobro opisuje sipanje na delcih, ki so bistveno manjši od valovne dolžine svetlobe
$R \ll \lambda$ ozioroma $kR \ll 1$.

Predpostavimo, da je delec tako majhen, da je optično polje v vpadnem valovanju v celotnem volumnu
znotraj delca konstantno. To vpadno optično polje v delcu inducira dielektrično polarizacijo, ki  
s časom sinusno oscilira. To povzroči dipolno sevanje delca. 

V celotnem delcu se inducira polarizacija $\mathbf{P}$, ki je vzporedna z vpadnim poljem 
$\mathbf{E}$. Iz zveze:
\beq
\mathbf{D} = \varepsilon \varepsilon_0 \mathbf{E} = \varepsilon_0 \mathbf{E} + \mathbf{P}
\label{eq:07_06}
\eeq
izrazimo polarizacijo kot:
\beq
\mathbf{P} = \varepsilon_0 (\varepsilon -1)\mathbf{E},
\label{eq:07_07}
\eeq
pri čemer je 
\beq
\mathbf{E} = E(\mathbf{r}) \mathbf{e}e^{-i \omega t}.
\label{eq:07_07}
\eeq
Ustrezni električni dipolni moment delca oziroma sipalca s polmerom $R$ je potem:
\beq
\mathbf{p} = \mathbf{P}V = \mathbf{P}\left(\frac{4\pi R^3}{3}\right),
\label{eq:07_08}
\eeq
Zapisani izraz velja za delec v vakuumu. Če je delev z dielektričnostjo $\varepsilon_2$ obdan 
s snovjo z dielektričnostjo $\varepsilon_1$, pa je za sipanje pomembna le razlika v polarizaciji 
med delcem in okolico (presežna polarizacija):
\beq
\Delta \mathbf{P} = \varepsilon_0 \Delta \varepsilon \mathbf{E},
\label{eq:07_09}
\eeq
pri čemer je $\Delta \varepsilon = \varepsilon_2 - \varepsilon_1$. Dipolni moment sipalca je tako:
\beq
\Delta \mathbf{p}= \Delta \mathbf{P} V = \Delta \varepsilon 
\left(\frac{4\pi R^3}{3}\right)\varepsilon_0 E(\mathbf{r}) \mathbf{e}e^{-i \omega t} =
\Delta \mathbf{p}_0 e^{-i \omega t}.
\label{eq:07_10}
\eeq
Ko na delec vpada elektromagnetno valovanje, se v njem inducira električni dipol. Sinusno nihajoči
električni dipol pa, kot vemo, dipolno seva. Poglejmo, kako. (referenca Andrej EMP).

Inducirani točkasti dipolni moment seva kot električna dipolna antena. Sevalno polje antene 
zapišemo kot:
\beq
\mathbf{E}_\mathrm{sev} = \frac{\omega^2 \Delta p_0}{4 \pi \varepsilon_0 c_0^2}\sin \vartheta
\left(\frac{e^{ikr - i\omega t}}{r}\right) \mathbf{e}_\vartheta
\label{eq:07_11}
\eeq
in 
\beq
\mathbf{H}_\mathrm{sev} = \frac{\omega^2 \Delta p_0}{4 \pi c_0^2}\sin \vartheta
\left(\frac{e^{ikr - i\omega t}}{r}\right) \mathbf{e}_\varphi.
\label{eq:07_12}
\eeq
Gostoto energijskega toka sevalnega polja izračunamo:
\beq
\mathbf{j}_\mathrm{sev} = \langle \mathbf{S} \rangle = \langle \mathbf{E}_\mathrm{sev}
\times \mathbf{H}_\mathrm{sev}  \rangle 
= \frac{\omega^4 \Delta p_0^2 \sin^2 \vartheta}{32 \pi^2 \varepsilon_0 c_0^3 r^2}
\mathbf{e}_r.
\label{eq:07_13}
\eeq
Zaradi sevanja se vpadni energijski tok preusmeri iz vpadne smeri $\mathbf{e}_x$ v druge smeri. Posledično
svetloba v tej smeri slabi, v ostalih smereh pa se pojavi sipana svetloba. Njena intenziteta je močno odvisna
od valovne dolžine svetlobe, saj nastopa frekvenca v enačbi~(\ref{eq:07_13}) v četrti potenci.

\begin{example}{\bf Zakaj je nebo modro?}
Svetloba, ki s Sonca vpada na Zemljino površje, najprej prepotuje Zemljino atmosfero. V njej se na molekulah
Rayleighovo siplje. Zaradi močne frekvenčne odvisnosti intenzitete sipanja (enačba~\ref{eq:07_13}) se
modra svetloba s krajšo valovno dolžino siplje bistveno močneje kot rdeča svetloba. Razmerje v intenzitetah
sipane svetlobe modre barve z $\lambda = 400~\si{\nm}$ in rdeče barve z $\lambda = 800~\si{\nm}$ je tako 16. 
Ko na atmosfero vpada svetloba s Sonca, se v atmosferi torej modra svetloba najmočneje siplje in nebo je videti 
modro. 

Ob sončnih vzhodih in zahodih, ko sonce leži le malo nad obzorjem, je pot sončne svetlobe skozi atmosfero
razmeroma dolga. Večina modre svetlobe se siplje na vse strani, rdeče pa razmeroma malo. Večina 
svetlobe, ki pride do opazovalca, je tako rdečkaste barve, kar da značilno rdeče obarvane sončne zahode.
\end{example}

Pri majhnih sipalcih torej velja, da je:
\beq
\mathbf{j}_\mathrm{sev} \propto \omega^4 \Delta p_0^2 \propto \frac{(\Delta \varepsilon V)^2}{\lambda^4} 
\propto \frac{(\Delta \varepsilon)^2 R^6}{\lambda^4}.
\label{eq:07_17}
\eeq
Za sipalni izkoristek tako velja:
\beq
Q_S = \frac{\sigma_S}{A} \propto \frac{R^6}{\lambda^4 (\pi R^2)}\propto\left(\frac{R}{\lambda}\right)^4.
\label{eq:07_18}
\eeq
Četrta potenca je značila z Rayleighovo sipanje.

Izračunajmo še odvisnost gostote energijskega toka sevalnega polja od smeri. Vpadna svetloba naj bo 
nepolarizirana in naj vpada na delec v smeri osi $x$. Električno polje vpadne svetlobe lahko razdelimo
na vertikalno polarizacijo $V$ in horizontalno polarizacijo $H$. Ravnino, glede na katero sta določeni 
smeri $V$ in $H$ določata valovni vektor vpadne svetlobe $\mathbf{k}_i$ in valovni vektor sipane svetlobe
$\mathbf{k}_s$. 

Za tisti del svetlobe, ki je polariziran v smeri $V$, se detektor nahaja v ekvatorialni ravnini 
$\vartheta = 90\si{\degree}$ in $j_\mathrm{sev}$ je neodvisen od sipalnega kota $\beta$. Za tisti del 
svetlobe, ki je polarizirana v smeri $H$, pa velja $\vartheta = \pi/2-\beta$ v izrazu za $j_\mathrm{sev}$.
Skupno sipano polje je tako:
\beq
j_\mathrm{sip} = j_{\mathrm{sev}, V} + j_{\mathrm{sev}, H} = j_{\mathrm{sev}, 0} + 
j_{\mathrm{sev}, 0}\sin^2\left(\pi/2-\beta \right) = j_{\mathrm{sev}, 0}\left(1 + \cos^2\beta\right).
\label{eq:07_13}
\eeq
Za vpadno nepolarizirano svetlobo torej v smeri pravokotno na izvor dobimo dvakrat manj sipane svetlobe
kot v vzdolžni smeri. 

Celotna gostota svetlobnega toka sipane svetlobe je tako:
\boxeq{eq:Rayleigh}{
j = j_0 \frac{1 + \cos^2\beta}{2r^2}... R^6/\lambda^4...
}

*** Dodaj modro, rdeče nebo, kasneje beli oblaki. Polalrizacija sipane svetlobe.
Zakaj polarizator pomaga pri fotografiranju.

\subsection*{Sipalna matrika}
Tudi pri sipanju navadno uvedemo matrični zapis. Vsako vpadno polarizacijo lahko obravnavamo, tako
da jo razstavimo na dve komponenti: eno, ki je vzporedna s sipalno ravnino $E_\parallel$, in drugo, ki 
je nanjo pravokotna $E_\perp$. V našem zapisu je tako $E_\parallel \parallel H$ in
$E_\perp \parallel V$. 
\beq
\mathbf{E}_s = 
\left[\begin{array}{c}
E_{s,H}\\
E_{s,V}\\
\end{array}\right]
= \left[\begin{array}{c}
E_{s, \parallel}(\beta)\\
E_{s, \perp}(\beta)\\
\end{array}\right] = 
\left[\begin{array}{cc}
S_2 & S_3 \\
S_4 & S_1\\
\end{array}\right] 
\left[\begin{array}{c}
E_{i, \parallel}(\beta)\\
E_{i, \perp}(\beta)\\
\end{array}\right] 
\left(
\frac{e^{ik(r-x)}}{ikr}
\right)\!\!.
\label{eq:07_14}
\eeq
Pri tem je matrika amplitudna sipalna matrika, ki je na splošno funkcija zenitalnega
in azimutalnega kota (če delec ni povsem okrogel), ki poruši simetrijo. Vektor v zgornjem
zapisu predstavlja optično polje vpadne svetlobe, zadnji člen pa je transportni faktor,
ki je odvisen od razdalje med sipalcem in detektorjem $r$, ter od globine, v kateri
je prišlo so sipanja (z).

V praksi navadno merimo gostoto energijskega toka sipane svetlobe:
\beq
j_s \propto |E_{s, \parallel}|^2 + |E_{s, \perp}|^2.
\label{eq:07_15}
\eeq
Ker sta polarizaciji ortogonalni, mešani členi ničesar ne prispevajo. Potem sipalno matriko
za gostoto svetlobnega toka zapišemo kot:
\beq
\left[\begin{array}{c}
j_{s, \parallel}(\beta)\\
j_{s, \perp}(\beta)\\
\end{array}\right] = 
\left[\begin{array}{cc}
|S_2|^2& 0\\
0 & |S_1|^2\\
\end{array}\right] 
\left[\begin{array}{c}
j_{i, \parallel}(\beta)\\
j_{i, \perp}(\beta)\\
\end{array}\right]\!\!.
\label{eq:07_16}
\eeq

.. Kakšen komentar, zakaj je to dobro, primer? .. Zakaj so indeksi tako čudno?

\subsection*{Popravek lokalnega polja}
Pri obravnavi inducirane polarizacije posamičnega delca moramo upoštevati, da okoliški
delci okoli izbranega delca vplivajo na električno polje, ki ga zazna izbrani delec. Lokalno polje
na mestu izbranega delca je torej drugačno, kot bi bilo v primeru povsem izoliranega delca. 

V tem primeru moramo člen $\Delta \varepsilon$ v izračunu za inducirano polarizacijo nadomestiti s členom:
\beq
\Delta \mathbf{p} \propto \left(\frac{3(\varepsilon-1)}{\varepsilon+2} \right)V \mathbf{E}_i,
\label{eq:07_22}
\eeq
pri čemer je $\varepsilon = \varepsilon_2/\varepsilon_1$. Ta popravek lokalnega polja bomo podrobneje
obravnavali pri modelih lomnega količnika v poglavju (..). 


\section{Miejevo sipanje}
Sipanje svetlobe na delcih, ki so po velikosti primerljivi z valovno dolžino vpadne svetlobe, imenujemo
Miejevo sipanje po nemškem fiziku Gustavu Mieju (1868--1957). Kadar je polmer delcev $R\approx \lambda$
ali večji, približek sipalca kot točkastega dipola ni ustrezen. V tem primeru se moramo lotiti problema
bolj natančno, tako da z reševanjem Maxwellovih enačb izračunamo polje znotraj in zunaj delca ter
ustrezno upoštevamo robne pogoje med delcem in okolico. Če je delec okrogel, iščemo rešitve v sferičnih 
koordinatah. Pojav dovolj dobro opišemo, če rešujemo skalarno Helmholtzevo enačbo (enačba..) in rešitve
poiščemo z metodo separacije spremeljivk:
\beq
E_{lm}(r,\vartheta, \varphi) = R_l(r) \Theta_l^m(\vartheta) \Phi_m(\varphi) = 
E_0 e^{im\varphi}P_l^m (\cos \vartheta) Z_l (kr).
\label{eq:07_20}
\eeq
Pri tem $P$ označujejo pridružene Legendrove polinome, $Z$ pa sferične Besslove funkcije prvega
ali drugega reda. Celotno sipano polje zapišemo kot vsoto vseh prispevkov:
\beq
E = \sum_{l,m} A_{l,m}E_{l,m}.
\label{eq:07_21}
\eeq
Navedene funkcije (vektorski sferični harmoniki) tvorijo poln ortogonalni sistem stanj. V sferične
harmonike zato lahko razvijemo tudi vpadno valovanje. Celotno polje v prostoru potem zapišemo
kot superpozicijo vpadnega in sipanega valovanja. Koeficiente $A$ pred različnimi členi pa dobimo
iz robnih pogojev. 

Z računom ugotovimo, da - podobno kot pri Fabry-Perotovem interferometru - tudi v sferičnih delcih
lahko pride do resonance, kadar sta polmer delca in valovna dolžina v ustreznem razmerju. Takrat 
je sipano polje oziroma energijski tok sipanega valovanja še posebej šibek (ali velik). Po drugi
strani  pa sipalni izkoristek ni več tako izrazito odvisen od razmerka $R/\lambda$. GRAF. 

Tudi kotna odvinosti $j(\beta)$ kaže številne minimume in maksimume. Prevladuje sipanje naprej (skica).

***
Dodaj: Rayleigh - elastično na majhnih delcih, ohranja energijo, 
odvisno od valovne dolžine, molekule v zraku - modro nebo.

Debye ali Miejevo sipanje - elastično na delcih ali molekulah, ki so primerljive
z valovno dolžino sverlobe, izrazito enenakomerno sipanje. Ni zelo odvisno od
valovne dolžine, kapljice v zraku - bel oblak in megla. 

Brillouinovo sipanje - neelastično, na trdni snovi, vzbudimo fonone, majhni premiki.

Ramanovo sipanje - neelastično, razlika v frekvencah je razlika v energijskih 
nivojih atomov in molekul. Zelo uporabno za analizo materialov, kemijsko sestavo.
Stokes, anti-stokes. 

Thompsonovo sipanje na nabitih delcih? Xray?
