%\chapterimage{Geometrijska.jpg} % Chapter heading image

\chapter{Koherenca}
\label{chap:Koherenca}
Spoznali bomo koherenco, to je lastnost svetlobe, ki je tesno povezana
s pojavom interference. Obravnavali bomo časovno koherenco in zapisali 
Wiener-Hinčinov izrek ter prostorsko koherenco in z njo povezan 
Van Cittert-Zernikov izrek. Poglavje bomo zaključili z opisom holografije.

Koherenca oziroma koherentne lastnosti svetlobe se nanašajo na obstoj povezave
med fazo elektromagnetnega valovanja na nekem kraju ob nekem času ter njegovo 
fazo na nekem drugem kraju ob nekem drugem času. V poglavjih o 
uklonu (poglavje~\ref{chap:Uklon}) in interferenci (poglavje~\ref{chap:Interferenca})
smo pri zapisu elektromagnetnega valovanja privzeli, da poznamo njegovo fazo 
na vsakem kraju in ob vsakem času, s čimer smo predpostavili, 
da je elektromagnetno valovanje popolnoma koherentno. Jakost 
električnega polja popolno koherentnega valovanja zapišemo kot funkcijo $\mathbf{r}$ in $t$ 
v obliki (enačba~\ref{eq:ravnival}):
\beq
\mathbf{E} (\mathbf{r}, t) = \mathbf{E}_0(\mathbf{r}, t) 
e^{i\phi(\mathbf{r}, t)}.
\label{eq:08_01}
\eeq
Koherentno valovanje ob interferenci da značilen interferenčni 
vzorec, ki se s časom ne spreminja, nekoherentno valovanje pa ne. Pri  
delno koherentnih valovanjih interferenčne vzorce sicer opazimo, vendar z zmanjšanim
kontrastom. V realnih sistemih valovanje
ni nikoli povsem koherentno in prej ali slej se povezava med njegovo 
fazo izgubi. Ločimo časovno (tudi longitudinalno ali vzdolžno) in krajevno 
(tudi transverzalno ali prečno) koherenco. 

\section{Časovna koherenca}
Najprej bomo opisali časovno koherenco, ki povezuje valovanje na istem mestu 
ob dveh različnih časih. Za opis se omejimo na monokromatsko elektromagnetno 
valovanje v skalarnem približku. Na mestu $\mathbf{r}$ ob časih $t$ in $t+\tau$ valovanje
zapišemo kot:
\beq
E(\mathbf{r}, t) = E_0 e^{-i\omega t} e^{i\phi(\mathbf{r}, t)}
\qquad \mathrm{in} \qquad
E(\mathbf{r}, t+\tau) = E_0 e^{-i\omega (t+\tau)} e^{i\phi(\mathbf{r}, t+\tau)}.
\label{eq:08_03}
\eeq
Glede na povezavo med $\phi(\mathbf{r}, t)$ in $\phi(\mathbf{r}, t+\tau)$ 
ločimo tri vrste valovanj:
\begin{itemize}
 \item popolnoma koherentno valovanje, za katerega velja: $\phi(\mathbf{r}, t+\tau) = 
 f(\mathbf{r}, \tau, \phi(\mathbf{r}, t))$ za vsak $\tau$. 
 Če poznamo fazo valovanja ob nekem času, jo poznamo tudi ob vsakem drugem času.
 \item delno koherentno, za katerega velja:
\begin{equation}
\phi(\mathbf{r}, t)=\begin{cases}
f(\mathbf{r}, \tau, \phi(\mathbf{r}, t)); \quad \tau < \tau_c\\
\mathrm{ni~povezave}; \quad \tau > \tau_c.
\end{cases}
\label{eq:08_04}
\end{equation}
Do časa $\tau_c$, ki ga imenujemo koherenčni čas, fazo valovanja 
poznamo, pri daljših časih pa ni povezave med fazo valovanja. Večina
realnih optičnih sistemov je delno koherentnih.
 \item nekoherentno valovanje, za katerega velja, da
 sta $\phi(\mathbf{r}, t)$ in $\phi(\mathbf{r}, t+\tau)$
 povsem nepovezana in naključna za vse vrednosti $\tau$.
\end{itemize}
\begin{example}{\bf Plinska svetilka.}
Vzemimo za primer plinsko razelektritveno cev (neonsko svetilko) in 
opazujmo svetlobo, ki jo svetilka oddaja. Z uporabo barvnega 
filtra se omejimo na opazovanje ene same frekvence oziroma 
ene same spektralne komponente. Izsevana svetloba je sestavljena iz
sevalnih prispevkov vseh atomov, ki iz vzbujenega stanja prehajajo v 
nižja energijska stanja s sevanjem. V času, ki je kratek v primerjavi 
z značilnim časom med trki atomov, imajo vsa izsevana delna valovanja 
neko določeno fazo. Ob trku dveh atomov se ta faza naključno spremeni. 
Izkaže se, da je tipični čas med med trki atomov v razelektritveni cevi 
navadno le nekaj nihajnih period vidne svetlobe in shematski prikaz
izsevanega električnega polja je prikazan na sliki~\ref{fig:08_neon}.
\begin{figure}[h!]
\centering
\def\svgwidth{80truemm} 
\input{slike/08_neon.pdf_tex}
\caption{Ob trku atomov se spremeni faza izsevane svetlobe. Koherenčni
čas ocenimo kot povprečni čas med posameznimi trki.
}
\label{fig:08_neon}
\vglue-9truemm
\end{figure}

\end{example}

Časovno koherenco elektromagnetnega valovanja analiziramo z 
Michelsonovim interferometrom (glej razdelek~\ref{chap:Michelson}). 
Spomnimo se, da Michelsonov interferometer vpadni snop 
svetlobe razdeli na dva snopa, ki se nato odbijeta od zrcal in pred 
vpadom na detektor ponovno združita. Če v postavitev dodamo kolimator, 
ki iz izvorne svetlobe naredi širok vzporeden snop, dobimo tako imenovano
Twyman-Greenovo postavitev (slika~\ref{fig:08_Twyman}).
\begin{figure}[h]
\centering
\def\svgwidth{70truemm} 
\input{slike/08_TwymanGreen.pdf_tex}
\caption{Twyman-Greenova postavitev Michelsonovega interferometra. Razlika
je v kolimatorju, ki vpadno svetlobo (levo) razširi v širok snop. Snop s 
polprepustnim zrcalom razdelimo
na dva delna snopa, ki se odbijeta od zrcal, nato pa s premikanjem enega
od zrcal ustvarjamo časovni zamik med njima. Z opazovanjem 
interferenčnega vzorca svetlobe, zbrane na detektorju, določimo
koherenčni čas svetlobe.
}
\label{fig:08_Twyman}
\vglue-3truemm
\end{figure}

Vpadni snop svetlobe razdelimo na dve delni valovanji, ki po odbojih na 
zrcalih interferirata na mestu detekcije. Čeprav izvirata iz istega 
izvora, sta delni valovanji zaradi različno dolgih poti med seboj zakasnjeni. 
Časovna zakasnitev je enaka:
\beq
\tau = \frac{2 x}{c_0},
\label{eq:08_05}
\eeq
pri čemer je $x$ pomik zrcala in $2x$ razlika v dolžini poti med prvo in drugo 
vejo interferometra. 

Jakost
električnega polja na detektorju zapišemo kot vsoto dveh delnih valovanj:
\beq
E_d = E(t) + E(t+\tau).
\label{eq:08_06}
\eeq
Gostota svetlobnega toka  je sorazmerna kvadratu jakosti:
\beq
j_d \propto |E(t) + E(t+\tau)|^2 = \left(E(t) + E(t+\tau)\right) 
\left(E^*(t) + E^*(t+\tau)\right)\!,
\label{eq:08_07}
\eeq
od koder sledi:
\beq
j_d \propto |E(t)|^2 + |E(t+\tau)|^2 + E(t)E^*(t+\tau) + E^*(t)E(t+\tau).
\label{eq:08_08}
\eeq
Z detektorjem zajemamo signal v časovnem intervalu $T$, ki je bistveno 
daljši od nihajnega časa svetlobe. Zato na detektorju zaznamo povprečen signal:
\beq
\langle j_d \rangle \propto \frac{1}{T}\int_{-T/2}^{T/2} 
\left(|E(t)|^2 + |E(t+\tau)|^2 + 2 \Re \left( E(t)E^*(t+\tau)\right) \right) dt.
\label{eq:08_09}
\eeq
Kadar sta amplitudi delnih žarkov enaki, velja:
\beq
\langle j_d \rangle \propto 2\langle |E(t)|^2 \rangle + 2\frac{1}{T}\int_{-T/2}^{T/2} 
\Re \left( E(t)E^*(t+\tau)\right) dt.
\label{eq:08_10}
\eeq
Vpeljemo časovno avtokorelacijsko funkcijo polja:
\boxeq{eq:G1}{
G^{(1)}(\tau)= \lim_{T\to \infty}~\frac{1}{T}\int_{-T/2}^{T/2}E(t)E^*(t+\tau)dt
}
in signal na detektorju zapišemo kot:
\beq
\langle j_d \rangle \propto 2\langle |E(t)|^2 \rangle + 
2\Re \left( G^{(1)}(\tau)\right)\!\!.
\label{eq:08_11}
\eeq
Zapisana avtokorelacijska funkcija podaja korelacijo valovanja samega s seboj pri
dani časovni zakasnitvi $\tau$. Poglejmo nekaj primerov.

\begin{example}{\bf Avtokorelacija sinusnega valovanja.}
Obravnavajmo ravno sinusno elektromagnetno valovanje pri točno določeni frekvenci. 
Zapišemo ga v obliki:
\beq
E(t) = E_0 e^{-i\omega t} \qquad \mathrm{in} \qquad E(t+\tau)= E_0 e^{-i\omega(t+\tau)},
\label{eq:08_12}
\eeq
pri čemer je $E_0$ realno število. Potem je časovna avtokorelacijska 
funkcija (enačba~\ref{eq:G1}) enaka:
\beq
G^{(1)}(\tau) = \lim_{T\to \infty}~\frac{1}{T}
\int_{-T/2}^{T/2}E_0 e^{-i\omega t}E_0^*e^{i\omega t}e^{i\omega\tau} dt.
\label{eq:08_13}
\eeq
Sledi:
\beq
G^{(1)}(\tau) = E_0^2 e^{i\omega \tau} 
\lim_{T\to \infty}~\frac{1}{T}\int_{-T/2}^{T/2} dt= E_0^2 e^{i\omega \tau}.
\label{eq:08_14}
\eeq
Če se omejimo zgolj na realni del, velja:
\beq
G^{(1)}(\tau) = E_0^2 \cos(\omega \tau).
\label{eq:08_14a}
\eeq
Gostota svetlobnega toka na detektorju je potem enaka (enačba~\ref{eq:08_11}):
\beq
\langle j_d \rangle \propto  
2E_0^2 + 2 E_0^2 \cos(\omega \tau).
\label{eq:08_15}
\eeq
Vpeljemo $j_0$ kot gostoto energijskega toka posameznega delnega valovanja 
in izmerjeno  gostoto toka na detektorju preoblikujemo v:
\beq
\langle j_d \rangle  = 4j_0 \cos^2\frac{\omega \tau}{2}.
\label{eq:08_15a}
\eeq
Z uporabo enačbe~(\ref{eq:08_05}) preoblikujemo še argument funkcije:
\beq
\frac{\omega \tau}{2} = \omega \frac{x}{c_0} = k x = \frac{\Delta \phi}{2}.
\label{eq:08_17}
\eeq
Dobimo izraz:
\beq
\langle j_d \rangle  = 4j_0 \cos^2 (\Delta \phi/2).
\label{eq:08_15c}
\eeq
ki ga že poznamo iz poglavja o interferenci (enačba~\ref{eq:06_09}). 

Zapisani rezultat velja samo v primeru povsem koherentnega monokromatskega 
sinusnega valovanja, za katerega znamo zapisati fazo v vsakem trenutku. 
V praksi opisano obnašanje opazimo le v bližini ekvidistančne 
lege zrcal za majhne vrednosti $\Delta \phi$. Ko razliko v poteh posameznih delnih 
žarkov povečujemo, se koherenca valovanja zmanjšuje in kontrast interferenčnega vzorca bledi.
Pri zelo velikih vrednostih $x$ imata delna žarka naključni fazi, 
kar pomeni, da $\Delta \phi$ zavzema naključne vrednosti med $0$ in $2\pi$. 
Posledično je povprečje $\langle \cos \Delta \phi \rangle= 0$
in za velike vrednosti $\tau$ velja:
\beq
\langle j_d \rangle = 2 j_0.
\eeq
\begin{figure}[h]
\vglue-3truemm
\centering
\def\svgwidth{140truemm} 
\input{slike/08_sinkoh.pdf_tex}
\caption{Signal na detektorju Michelsonovega interferometra v odvisnosti od premika zrcala $x$
v primeru koherentnega monokromatskega sinusnega valovanja (a). Kadar je dolžina valovanja
končna in valovanje ni povsem monokromatsko, kontrast signala pri večjih $x$ oslabi. Karakteristična razdalja, na kateri kontrast signala oslabi, je koherenčna dolžina $L_c = c \tau_c$.
}
\label{fig:08_sinkoh}
\vglue-7truemm
\end{figure}

\end{example} 

Namesto koherenčnega časa $\tau_c$ lahko vpeljemo koherenčno dolžino 
$L_c = c_0 \tau_c$, ki označuje pot, ki jo svetloba prepotuje 
v koherenčnem času. Časovno koherenco zato imenujemo tudi 
longitudinalna ali vzdolžna koherenca, saj opazujemo zamik valovanja, ki se
premakne vzdolžno glede na smer širjenja svetlobe. Valovanje, ki je 
koherentno za $\tau < \tau_c$, je tako pri opisanem eksperimentu 
koherentno za spremembe dolžine poti, za katere velja $L < L_c$. Pri 
večjih razlikah v dolžini poti se koherenca zmanjša in kontrast
interferenčne slike izgubi. 

\section{Wiener-Hinčinov izrek}
Električno polje analiziramo in detektiramo v nekem končnem časovnem intervalu $T$. Temu 
ustrezno ga lahko razvijemo v Fourierevo vrsto:
\beq
E(t) = \sum_n A_n e^{-i \omega_0 nt} = \sum_n A_n e^{-i \omega_nt}
\eeq
pri čemer je $\omega_0 = 2\pi/T$, $n$ pa je celo število.
Podobno zapišemo tudi:
\beq
E^*(t+\tau) = \sum_m A_m^* e^{i\omega_m (t + \tau)},
\eeq
kjer je $m$ celo število. Izraza vstavimo v definicijo za časovno avtokorelacijsko funkcijo polja:
\beq
G^{(1)}(\tau) = \lim_{T\to\infty} \frac{1}{T} \int_{-T/2}^{T/2} \sum_{n,m} A_n A_m^* e^{-i\omega_n t}
e^{i\omega_m (t+\tau)} dt= \sum_{n,m} A_n A_m^*e^{i\omega_m\tau} \left(
\lim_{T\to\infty} \frac{1}{T} \int_{-T/2}^{T/2}e^{i(\omega_m - \omega_n)t}dt \right)\!\!.
\eeq
Izraz v oklepaju je v limiti za velike vrednosti $T$ enak funkciji $\delta(\omega_n - \omega_m)$, zato 
sledi:
\beq
G^{(1)}(\tau) = \sum_m|A_m|^2e^{i\omega_m\tau}.
\eeq 
V nasprotnem primeru, ko $T$ ne gre proti neskončnosti, moramo predpostaviti, da členi z različnimi
frekvencami ne interferirajo drug z drugimi. To je približek 1, ki smo ga naredili.

Izračunajmo njegovo Fourierevo transformiranko:
\beq
\frac{1}{2\pi T}\lim_{T\to \infty}\int_{-T/2}^{T/2} G^{(1)} (\tau) e^{-i\omega \tau} d\tau = 
\sum_m |A_m|^2\lim_{T\to \infty}\frac{1}{2\pi T}\int_{-T/2}^{T/2} e^{i\omega_m\tau}e^{-i\omega \tau} d\tau.
\eeq
Integral gre za velike vrednosti $T$ proti delta funkciji $\delta (\omega_m - \omega)$, kar pomeni, 
da bi vrednosti G načeloma morali poznati za vsak $\tau$. Vendar to eksperimentalno ni izvedljivo.
Navadno lahko predpostavimo, da so vrednosti G izven merilnega območja  enake 0 oziroma zanemarljivo 
majhne. To je pa približek 2, ki smo ga naredili. Sledi:
\beq
\mathcal{F}(G^{(1)})  = |A_m|^2 \propto S(\omega).
\eeq
S $S(\omega)$ smo označili spekter svetlobe -- to je intenziteto svetlobe pri frekvenci $\omega$, deljeno
z intervalom $\omega_0$. To popravi da delta omega!
\boxeq{eq:WH}{
S(\omega) \propto \lim_{T \to \infty}\frac{1}{T}\int_{-T/2}^{T/2} G^{(1)}(\tau) e^{-i\omega \tau} d\tau.
}
Z izračunom smo pokazali, da je spekter svetlobe $S(\omega)$
Fouriereva transformiranka časovne avtokorelacijske funkcije polja $G^{(1)}(\tau)$. To trditev
imenujemo Wiener-Hinčinov izrek po ameriškem matematiku Norbertu Wienerju (1894--1964) ter 
ruskemu matematiku Aleksandru Jakovljeviču Hinčinu (1894--1959).

Od tod pride ime Fouriereva spektroskopija. 

Zapisani limitni izraz velja za diskretni spekter. Če je spekter zvezen, lahko limito izvrednotimo in dobimo
zvezo:
\beq
S(\omega) \propto \int_{-\infty}^{\infty} G^{(1)}(\tau) e^{-i\omega \tau} d\tau.
\eeq
V eksperimentu pri Fourierevi spektroskopiji je pomembno, da signal na detektorju merimo oziroma
povprečimo čim dlje. Pri tem mora veljati $T \gg \tau_c$. Če bi merili krajši čas, sploh ne bi opazili, da
faza preskoči, ampak bi dobili tak rezultat kot pri povsem koherentni svetlobi. 

Druga pomembna stvar je v tem, da naredimo meritve tudi pri velikih razmikih zrcal $\Delta L \gg t_cc_0$. 
To pa zato, ker je $t_c$ v resnici statistični parameter in dolžine nekaterih 
valovnih potez lahko tudi 
zelo velike. Pravo vrednost povprečja $\tau_c = \langle \Delta t\rangle$ lahko dobimo le, če v eksperimentu
preverimo vse možne vrednosti. 

V praksi seveda ni niti $T= \infty$ niti $\Delta L = \infty$ možno doseči. Težave rešimo z upoštevanjem
t.i. okenskih funkcij. 
Če je svetloba povsem koherentna, potem je spekter delta funkcija. Drugače pa v splošnem velja
\boxeq{eq:spkor}{
\delta \omega \tau_c \gtrsim 2\pi.
}
Opisano zvezo imenujemo K\"upfm\'ullerjev princip nedoločenosti. Krajši kot je koherenčni čas, širši je 
spekter in obratno, dolg koherenčni čas da ozek spekter.

Wiener Hinčinov izrek zares velja, samo kadar intenziteto na detektorju merimo oziroma povprečimo
neskončno dolgo $-\infty < t <\infty$ 
in kadar pri zaporednih meritvah zamik med delnima žarkoma v interferometru zavzame
vse vrednosti $-\infty <\tau <\infty$. Tem zahtevam se v praksi izognemo ob predpostavki, da so 
vrednosti G znatno večje od 0 le za $\tau <\tau_c$ ter da so funkcije polja $E(t)$ na velikih
časovnih intervalih zanemarljivo majhne.


WAIT - KAKŠNE MERITVE, KAJ SPLOH MERIMO, ZAKAJ MERIMO? Merimo spektre svetil (emisijske) ali pa 
absorpcijske, tako da v snop svetlobe postavimo snov, ki jo proučujemo. 


\section{Prostorska koherenca}
Pri prostorski koherenci nas zanima faza valovanja na dveh različnih krajih ob enakem času $t$. Območje, 
znotraj katerega so faze povezane, imenujemo koherenčna ploskev. Velja
\beq
\phi (\mathbf{r}_2, t) = f(\mathbf{r}_1 - \mathbf{r}_2, t, \phi(\mathbf{r}_1, t)),
\eeq
pri čemer sta $\mathbf{r}_1$ in $\mathbf{r}_2$ znotraj iste koherenčne ploskve.  V primeru, da
$\phi (\mathbf{r}_2, t)$ in $\phi (\mathbf{r}_1,t)$ nista povezana med seboj, sta $\mathbf{r}_1$
in $\mathbf{r}_2$ na različnih koherenčnih ploskvah. Ker se faza pri tem spreminja predvsem 
v smeri pravokotno na žarke, govorimo o transverzalni koherenci.

Prostorsko koherenco analiziramo s pomočjo Youngovega eksperimenta. Zajemamo valovanje
iz dveh različnih območij valovne fronte in zanima nas, ali na oddaljenem zaslonu
opazimo interferenčni (uklonski) vzorec ali ne.

Denimo, da imamo neko običajno plinsko svetilko končnih razsežnosti, ki jo uporabimo kot
izvor svetlobe za Youngov poskus. Pokazali bomo, da je kontrast interferenčnega vzorca odvisen
od velikosti svetila in od razdalje med režama. Velikost svetila označimo z $L$, razmik med
režama pa z $D$. Interferenčni vzorec, ki ga ustvari svetloba iz osrednjega dela svetila ($y'=0$)
je premaknjen glede na interferenčni vzorec, ki ga ustvari svetloba iz robnega dela svetila 
($y'=L/2$).

Youngov poskus, interferenca: Naj svetloba iz razsežnega svetila vpada na objektni zaslon, 
v katerem sta dve reži, razmaknjeni za $D$. Velikost svetila naj bo $L$, oddaljenost med 
svetilom in objektnim zaslonom pa $z_0$. Svetloba po prehodu skozi reži vpade na opazovalni
zaslon na oddaljenosti $z_0$ od objektnega zaslona. Ko svetloba iz svetila vpada na reži, dobimo
na opazovalnem zaslonu superpozicijo interferenčnih vzorcev svetlobe iz različnih delov svetila. 
Fazni zamik med valovanjema, ki se širita skozi reži 1 in 2 zapišemo kot vsoto dveh prispevkov: 
\beq
\Delta \phi = \Delta \phi (y, y') = k D \sin \beta + kD \sin \alpha,
\eeq
pri čemer prvi člen predstavlja razliko v fazi pred vpadom na reži, drugi pa po 
prehodu rež. Pri tem velja:
\beq
\sin \alpha \approx \alpha \approx \frac{\eta}{z_0} \qquad \mathbf{in} \qquad
\sin \beta \approx \beta \approx \frac{y'}{z_0'}.
\eeq
Za interferenčni vzorec na ozkih režah velja zveza (enačba ..)
\beq
j = 4 j_0 \cos^2 \left(\frac{\Delta \phi}{2}\right).
\eeq
Interferenčni vzorec, ki ga ustvari svetloba, ki prihaja s sredine svetila (pri $y'=0$) ima
vrhove pri $kD\sin \alpha = 2\pi N$, pri čemer je $N$ celo število. Interferenčni vzorec, ki 
ga ustvari svetloba z območij $y'>0$, ima vrhove zamaknjene k drugim vrednostim $\alpha$. Ko so 
vrhovi vzorca svetlobe, ki prihaja iz roba svetila $y'=L/2$, premaknjeni ravno na minimume
vzorca svetlobe, ki prihaja iz sredine, se bodo interferenčni vzorec izpovprečil. To se zgodi, 
kadar velja:
\beq
kD \sin\beta_\mathrm{max} = kD\frac{L}{2z_0'}= \pi,
\eeq
od koder sledi:
\beq
D_\mathrm{max} = \frac{2z_0'\pi}{kL} = \frac{z_0'\lambda}{L}.
\eeq
Maksimalna razdalja med režama $D_\mathrm{max}$, pri kateri interferenčni vzorec izgine oziroma
se izpovpreči, je merilo za transverzalno koherenčno razdaljo svetlobe, ki vpada na objektni zaslon.
Vrednost $D_\mathrm{max}^2$ pa je merilo za koherenčno ploskev vpadne svetlobe.

Koherenčno razdaljo neke monokromatske svetlobe oziroma elektromagnetnega valovanja torej dobimo
tako, da v Youngovem poskusu najprej postavimo reži blizu skupaj in opazujemo interferenčni vzorec
na oddaljenem zaslonu. Ko reži razmikamo, opazujemo, kako interferenčni vzorec bledi. Ko popolnoma
zbledi, smo dosegli $D_\mathrm{max}$, ki ustreza prečni koherenčni razdalji vpadnega valovanja.

\begin{example}{\bf Merjenje velikosti zvezd.}
Z navedenim poskusom lahko izmerimo velikost zvezde, če poznamo njeno oddaljenost od Zemlje. Na ta način
so leta 1920 prvič izmerili premer katerekoli zvezde, izmerili so zvezdo $\alpha$ v ozvezdju Orion
(Betelgeza). Oddaljenost (kako vejo?) zvezde je $z_0' = 600$~svetlobnih let. Interferenčna razdalja
je zbledela na razdalji med režama $D_\mathrm{max} = 2~\si{m}$. Zvezda je sicer zelo majhna, ampak zorni
kot je pa različen od nič, zato svetloba z različnih delov zvezde vpada na zrcali pod rahlo različnima
kotoma. Gledamo interferenco med njima oziroma kdaj izgine kontrast. 
Ker je $D_\mathrm{max} = 2~\si{m}$, lahko izračunamo:
\beq
D_B = 2R = L = \frac{z_0' \lambda}{D_\mathrm{max}} \approx 1,4 \times 10^{12}~\si{m}.
\eeq
PAZI - dejansko so določili leta 1920, da je premer enak 390MKm., to je okoli 300x toliko kot Sonce. 
Pri 10 feet je izginil vzorec, valovna dolžina 550 nm, kot 0,045 sekund. Hm, polmer je 760 +120/-60 od sonca.
Preveri! Irena: izmerili, da je DB = 1000 Ds, danes vemo, da je 887 +/- 203 premere sonca. 

S teleskopi lahko dobimo podobno ločljivost, vendar jih je bolj zahtevno zgraditi s tako veliko lečo.

Na sliki modre črte označujejo pot in interferenčni vzorec svetlobe, ki prihaja iz sredine zvezde, rdeče
črte pa pot in interferenčni vzorec z roba zvezde. Naklon smeri slednje določa zorni kot zvezde. 
V eksperimentu spreminjamo razdaljo med zrcaloma M1 in M2, ki nadomeščata reži v Youngovem poskusu
in opazujemo bledenje interferenčnega vzorca. 
\end{example}

\section{Van Cittert-Zernikov izrek}
Podobno kot pri Michelsonovem interferometru v Fourierevi spektroskopiji tudi v tem eksperimentu z
detektorjem merimo dolgo časa, tako da velja:
\beq
\langle j_d \rangle \propto \langle|E_1(t)|^2 \rangle + \langle|E_2(t<+\tau)|^2 \rangle
+ 2\Re \left( \frac{1}{T}\int_{-T/2}^{T/2}E_1(t)E_2^*(t+\tau) dt \right).
\eeq
Pri tem prvi člen opisuje svetlobno polje iz prve reže, drugi člen svetlobo iz druge reže, 
tretji člen pa imenujemo navzkrižna korelacijska funkcija polja. Časovni zamik $\tau$ je enak:
\beq
\tau = \frac{l_1-l_2}{c_0} \approx \frac{kD\sin\alpha}{c_0}.
\eeq
Če je svetloba iz posamezne reže časovno koherentna, velja:
\beq
E^*(t+\tau) = e^{i \omega \tau} E^*(t)
\eeq
in navzkrižno korelacijsko funkcijo zapišemo kot:
\beq
\Gamma_{12}(\tau) = \Re \left(e^{i \omega \tau} \frac{1}{T}\int_{-T/2}^{T/2}E_1(t)E_2^*(t+\tau) dt \right) =
\Re \left(e^{i \omega \tau} J_{12}\right). 
\eeq
Z $J_{12}$, ki je enak:
\beq
J_{12} = \frac{1}{T}\int_{-T/2}^{T/2}E_1(t)E_2^*(t+\tau) dt,
\eeq
je odvisen od prostorske koherence valovanja in določa kontrast oziroma vidljivost interferenčne slike. 
nizozemskem fiziku Pietru Hendriku van Cittertu (1889-1959) in nizozemskem fiziku in nobelovcu
Fritsu Zerniku (1888--1966). 
Pokazati se da, da je vrednost $J_{12}$ povezana z intenzitetnim profilom  svetlobnega polja 
$j_{0i}(x',y')$, pri čemer $j_{0i}$ označuje gostoto izsevanega svetlobnega toka.
Potem velja: 
\beq
J_{12} = \frac{\pi}{z_0'^2}\iint j_{0i}(x', y') e^{ikx'\Delta x/z_0'}e^{iky'\Delta y/z_0'} dx'dy',
\eeq
pri čemer je $\Delta x = x_2 -x_1$ in $\Delta y = y_2-y_1$. To je Van Cittert-Zernikov izrek po 

Koordinate $(x_1, y_1)$ so koordinate prve reže oziroma odprtine v 
objektnem zaslonu, koordinate $(x_2, y_2)$ pa koordinate druge 
reže oziroma odprtine v objektnem zaslonu. Vidljivost oziroma kontrast
interferenčne slike $J_{x_1,y_1,x_2,y_2}$ je 2D Fouriereva transformiranka 
intenzitetnega profila izvora.
Velja pa seveda tudi obratno: s Fourierevo transformacijo vidljivosti 
po $\Delta x$ in $\Delta y$ lahko
dobimo intenzitetni profil svetlobnega objekta. Uporabno v astronomiji.

Izpeljava Van Cittert-Zernikovega izreka: Kontrast oziroma vidljivost interferenčnih prog
določa $J_{12}$, za katerega velja:
\beq
J_{12} = \frac{1}{T}\int_{-T/2}^{T/2}E_1(t)E_2^*(t+\tau) dt = \langle E_1(t) E_2^*(t)\rangle.
\eeq
pri čemer je $E_1$ polje na reži 1 in $E_2$ polje na reži 2. 
Lega rež naj bo določena s koordinatama $(x_1,y_1)$ in  $(x_2,y_2)$ na objektnem zaslonu.

Svetloba, ki prihaja na režo 1 ali 2, izvira iz vseh točk svetlečega zaslona $(x', y')$. Potem 
zapišemo:
\beq
E_1 \propto \iint E_0(x',y') \frac{e^{ikr_1'}}{r_1'}dx'dy'
\eeq
in podobno za polje v drugi reži:
\beq
E_2 \propto \iint E_0(x'',y'') \frac{e^{ikr_2''}}{r_2''}dx''dy''.
\eeq
Sledi:
\beq
J_{12} = \langle E_0(x',y')E_0^*(x'',y'')\rangle.
\eeq
Kadar je svetloba, ki izvira iz različnih delov svetila med seboj povsem nekorelirana,
dobimo od nič različno povprečje le v primeru, da je $x'=x''$ in $y'=y''$. To seveda velja, 
če svetilo samo po sebi ni prostorsko koherentno, denimo termična svetloba zvezd.
Potem je:
\beq
J_{12} \propto \delta(x'-x'', y'-y'') |E_0|^2.
\eeq
Vstavimo polji enačbi.. v izraz za vidnost in dobimo:
\beq
J_{12} \propto \iint_S |E_0(x',y')|^2 \frac{e^{ikr_1'}}{r_1'}\frac{e^{ikr_2'}}{r_2'}dx'dy'
\approx \frac{1}{z_0^2}\iint_S j_{0i}(x',y') e^{ik(r_1'-r_2')}dx'dy'.
\eeq
Vstavimo in razvijemo za veliko oddaljenost med izvorom in objektnim zaslonom:
\beq
r_1' = z_0' \sqrt{1+(x_1-x')^2/z_0'^2+ (y_1-y')^2/z_0'^2} \approx
z_0'-\frac{x_1x'}{z_0'}-\frac{y_1y'}{z_0'}+\frac{x_1^2}{2z_0'}+\frac{y_1^2}{2z_0'}+\frac{x'^2}{2z_0'}
+\frac{y'^2}{2z_0'}.
\eeq
Ter podobno za drugo odprtino:
\beq
r_2' = z_0' \sqrt{1+(x_2-x')^2/z_0'^2+ (y_2-y')^2/z_0'^2} \approx
z_0'-\frac{x_2x'}{z_0'}-\frac{y_2y'}{z_0'}+\frac{x_2^2}{2z_0'}+\frac{y_2^2}{2z_0'}+\frac{x'^2}{2z_0'}
+\frac{y'^2}{2z_0'}.
\eeq
Razlika, ki nastopa v eksponentu, je potem enaka:
\beq
r_1'-r_2' = \frac{x'(x_2-x_1)}{z_0'}+\frac{y'(y_2-y_1)}{z_0'}
+ \frac{(x_1^2-x_2^2)}{2z_0'^2} + \frac{(y_1^2-y_2^2)}{2z_0'^2}.
\eeq
Če sta odprtini simetrični glede na izhodišče koordinatnega sistema, sta zadnja dva člena
v izrazu enaka nič. Sicer prineseta nek konstantni fazni faktor. Rezultat je v simetričnem primeru
enak:
\beq
J_{12} \propto \frac{1}{z_0'^2}\iint j_{0i}(x',y') e^{ikx'\Delta x/z_0'}e^{iky'\Delta y/z_0'}dx'dy'.
\eeq
Dobljeni izraz velja tudi na relativno majhnih razdaljah med 
svetilom in objektnim zaslonom, saj so 
se kvadratni členi zaradi simetričnosti izničili. 


\section{Holografija}

\beq
t \propto I = I_r + I_s + 2\sqrt{I_rI_s}\cos(\phi_r-\phi_s(x,y)).
\eeq
Zapisovanje pri klasični izvenosni holografiji; branje pri holografiji. 
Fotografija holograma. 
