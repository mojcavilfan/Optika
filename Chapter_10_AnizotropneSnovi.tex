%\chapterimage{Geometrijska.jpg} % Chapter heading image

\chapter{Optika v anizotropnih snoveh}
\label{chap:AnizotropneSnovi}
V tem poglavju bomo spoznali optično anizotropne snovi, katerih odziv na 
vpadno svetlobo je odvisen od njene polarizacije. Opisali bomo 
širjenje svetlobe skozi anizotropne snovi in pojasnili razliko med optično 
enoosnimi in optično dvoosnimi snovmi. Spoznali bomo dvojni lom in 
na koncu poglavja predstavili nekaj načinov uporabe tega zanimivega pojava.

\section{Anizotropne snovi}
Do zdaj smo obravnavali izotropne snovi, v katerih je lomni količnik 
snovi enak v vseh smereh in za vse polarizacije vpadne svetlobe. 
Za anizotropne snovi pa je značilno, da imajo 
v različnih smereh različne lastnosti, na primer mehanske, toplotne, 
električne, magnetne, optične ... Primeri optično 
anizotropnih snovi so kremen, kalcit, led, sljuda, tekoči kristali
in tudi raztegnjeni polimerni vzorci, na primer celofan ali prozorni lepilni trak. 
Optično anizotropne snovi najlaže prepoznamo po barvitih vzorcih med
polarizatorjema, ki spreminjajo barvo, ko polarizatorja sučemo. 
Zakaj je tako, bomo spoznali v nadaljevanju poglavja.
\begin{figure}[ht]
\centering
\includegraphics[width=7truecm]{slike/10_foto_nematik.jpg}\hfill
\includegraphics[width=7truecm]{slike/10_foto_celofan.jpg}
\caption{Kapljica tekočega kristala, ki je anizotropna tekočina, slikana pod 
mikroskopom (levo) in ovitek iz celofana (desno). Oba vzorca sta bila postavljena
med prekrižana polarizatorja.}
\label{fig:10_nematik}
\end{figure}

Najznačilnejša optična lastnost anizotropnih snovi je odvisnost lomnega
količnika od smeri širjenja svetlobe in njene polarizacije. Pojav
je posledica odziva snovi na vpadno svetlobo, pri katerem smer in 
velikost inducirane električne polarizacije določa struktura snovi.  
Za nazorno razlago se ponovno poslužimo
Lorentzevega modela, ki smo ga podrobneje spoznali v prejšnjem 
poglavju (glej razdelek~\ref{chap:lomni}). 

Zamislimo si negativno nabito kroglico, ki predstavlja elektron, vpeto v 
mrežo mirujočih sosednjih kroglic, ki predstavljajo pozitivne ione kristala. 
Vzmeti, ki kroglice medsebojno povezujejo, niso vse enake, ampak imajo različne 
konstante vzmeti, odvisno od smeri (slika~\ref{fig:10_model}\,a). Ko na elektron
vpade svetloba, ga sila električnega polja izmakne iz ravnovesne lege, tej sili 
pa nasprotujejo sile vzmeti. Premik elektrona je večji v smeri šibkejše vzmeti, 
zato dipolni moment, ki ga povzroči premik elektrona, ni vzporeden jakosti 
vpadnega električnega
polja. 
\begin{figure}[!h]
\centering
\def\svgwidth{120truemm} 
\input{slike/10_Lorentz_a.pdf_tex}
\caption{Anizotropijo snovi lahko pojasnimo z Lorenzevim modelom, v katerem so
konstante vzmeti, s katerimi je v snov vezan elektron, različne  in $k_1 \neq k_2 \neq k_3$ (a).
V anizotropnih snoveh inducirana električna polarizacija $\mathbf{P}$ ni vzporedna 
z jakosto električnega polja $\mathbf{E}$ (b).}
\label{fig:10_model}
\end{figure}

Osnovna posledica anizotropnega odziva je, da smer inducirane električne polarizacije $\mathbf{P}$
 v anizotropnih snoveh ni vzporedna smeri jakosti optičnega električnega polja $\mathbf{E}$ 
(slika~\ref{fig:10_model}\,b). Posledično tudi 
vektor gostote električnega polja $\mathbf{D}$ ni vzporeden s smerjo $\mathbf{E}$. 
Linearni odziv snovi zapišemo kot:
\beq
\mathbf{D} = \varepsilon_0 \mathbf{E} + \mathbf{P} = \varepsilon_0 \underline{\varepsilon} \mathbf{E}.
\label{eq:10_001}
\eeq
Vpeljali smo tenzor dielektričnosti $\underline{\varepsilon}$, ki je v anizotropnih snoveh na splošno 
tenzor drugega ranga:
\beq
\underline{\varepsilon} = 
\left[\begin{array}{ccc}
\varepsilon_{xx} & \varepsilon_{xy} & \varepsilon_{xz}\\
\varepsilon_{yx} & \varepsilon_{yy} & \varepsilon_{yz}\\
\varepsilon_{zx} & \varepsilon_{zy} & \varepsilon_{zz}\\
\end{array}\right]\!\!.
\label{eq:10_002}
\eeq
\begin{remark}
Pri obravnavi se bomo omejili na prozorne optično neaktivne anizotropne snovi. 
Ker dielektričnost nastopa v izrazu za energijo elektromagnetnega polja 
(enačba~\ref{eq:03_27}), je zaradi ohranitve energije tenzor $\underline{\varepsilon}$ na splošno
Hermitski in pozitivno definiten. V našem primeru to pomeni, da je 
tenzor simetričen in ga lahko vedno diagonaliziramo, pri čemer so 
vse njegove lastne vrednosti realne in pozitivne.
\end{remark}

Tenzor dielektričnosti diagonaliziramo in v lastnem koordinatnem 
sistemu ga zapišemo kot:
\beq
\underline{\varepsilon} = 
\left[\begin{array}{ccc}
\varepsilon_{xx} & 0 & 0\\
0 & \varepsilon_{yy} &0\\
0 & 0 & \varepsilon_{zz}\\
\end{array}\right]\!\!.
\label{eq:10_003}
\eeq
Po dogovoru koordinatni sistem navadno izberemo tako, da velja 
$\varepsilon_{xx}<\varepsilon_{yy}<\varepsilon_{zz}$.

Na splošno so vse tri neničelne komponente diagonalnega tenzorja različne. 
Pri snoveh z večjo stopnjo simetrije sta dve komponenti enaki ($\varepsilon_{xx} = 
\varepsilon_{yy} \neq \varepsilon_{zz}$), v izotropnih snoveh pa so vse tri 
neničelne komponente tenzorja enake in za opis snovi zadošča skalarna oblika dielektričnosti.

\begin{example}{\bf Smeri jakosti električnega polja 
in inducirane polarizacije v anizotropni snovi.}
Naj na anizotropno snov vpada elektromagnetno valovanje vzdolž osi $z$. Jakost
električnega polja na splošno leži v ravnini $xy$ in naj z osjo $x$ oklepa kot $\alpha$. 
Po komponentah jo zapišemo kot $\mathbf{E} = (E_x, E_y, 0)$. 

Iz enačbe~(\ref{eq:10_001}) sledi zveza:
\beq
\mathbf{P} = \varepsilon_0 \left(\underline{\varepsilon} - \underline{I}\right)\mathbf{E}
\label{eq:10_004}
\eeq
Vstavimo vektor $\mathbf{E}$ in v lastnem koordinatnem sistemu tenzorja 
dielektričnosti (enačba~\ref{eq:10_003}) inducirano električno polarizacijo zapišemo kot:
\begin{align}
P_x &= (\varepsilon_{xx}-1) \varepsilon_0 E_x \qquad \mathrm{in} \label{eq:10_005}\\
P_y &= (\varepsilon_{yy}-1) \varepsilon_0 E_y.
\label{eq:10_006}
\end{align}
Kot $\beta$, ki ga inducirana električna polarizacija oklepa z osjo $x$, izračunamo iz enačbe:
\beq
\tan \beta = \frac{P_y}{P_x} = \frac{(\varepsilon_{yy}-1)E_y}{(\varepsilon_{xx}-1)E_x}
 = \frac{\varepsilon_{yy}-1}{\varepsilon_{xx}-1}\tan \alpha.
\eeq
\end{example}

\section{Ravni val v anizotropni snovi}
Poglejmo, kako se po optično anizotropni snovi širi ravni val. 
Obravnavajmo neprevodno snov in njeno dielektričnost 
zapišimo z realnim tenzorjem $\underline{\varepsilon}$ (enačba~\ref{eq:10_003}). 
Zaradi enostavnosti se omejimo na nemagnetne snovi, za katere je $\mu = 1$. 
Naj na tako snov vpada ravno potujoče sinusno valovanje z 
valovnim vektorjem $\mathbf{k}$.  

Za opis valovanja izhajamo iz Maxwellovih enačb (enačbe~\ref{eq:Maxwell1}--\ref{eq:Maxwell4}), 
pri čemer privzamemo, da sta gostoti nabojev in električnih tokov enaki nič. 
Rešitev iščemo v obliki ravnih valov:
\begin{align}
 \mathbf{E} &= \mathbf{E}_0 e^{i\mathbf{k}\cdot \mathbf{r} - i \omega t}, \label{eq:10_010}\\
 \mathbf{D} &= \mathbf{D}_0 e^{i\mathbf{k}\cdot \mathbf{r} - i \omega t} \qquad \mathrm{in}\label{eq:10_011}\\
 \mathbf{B} &= \mu_0 \mathbf{H} = \mathbf{B}_0 e^{i\mathbf{k}\cdot \mathbf{r} - i \omega t}.
 \label{eq:10_012}
\end{align}
Iz Maxwellove enačbe~(\ref{eq:Maxwell3}) dobimo z upoštevanjem nastavkov:
\beq
\nabla \cdot \left(\mathbf{D}_0 e^{i\mathbf{k}\cdot \mathbf{r} - i \omega t} \right) = 
i\,\mathbf{k}\cdot \mathbf{D} = 0 \qquad \Longrightarrow \qquad \mathbf{D} \perp \mathbf{k}.
\label{eq:10_013}
\eeq
Podobno iz enačbe~(\ref{eq:Maxwell4}) sledi:
\beq
\nabla \cdot \left(\mathbf{B}_0 e^{i\mathbf{k}\cdot \mathbf{r} - i \omega t} \right) = 
i\,\mathbf{k}\cdot \mathbf{B} = 0 \qquad \Longrightarrow \qquad \mathbf{B} \perp \mathbf{k}.
\label{eq:10_014}
\eeq
Povsem na splošno sta torej vektorja $\mathbf{D}$ in $\mathbf{B}$ pravokotna na valovni vektor $\mathbf{k}$.
Izhajajoč iz enačbe~(\ref{eq:Maxwell1}) dobimo:
\beq
\nabla \times \left(\mathbf{H}_0 e^{i\mathbf{k}\cdot \mathbf{r} - i \omega t} \right) = 
i\,\mathbf{k}\times \mathbf{H} = \frac{\partial \mathbf{D}}{\partial t} = -i \omega \mathbf{D}.
\label{eq:10_015}
\eeq
Od tod za nemagnetne snovi neposredno sledi:
\beq
\mathbf{k}\times \mathbf{B} = - \mu_0 \omega \mathbf{D} \qquad \Longrightarrow \qquad \mathbf{B} \perp \mathbf{D}.
\label{eq:10_016}
\eeq
Vektorji $\mathbf{D}$, $\mathbf{B}$ in $\mathbf{k}$ so torej medsebojno paroma pravokotni.

Uporabimo nastavke (enačbe~\ref{eq:10_010}--\ref{eq:10_012})
še v Faradayevem zakonu (enačba~\ref{eq:Maxwell2}) in dobimo:
\beq
\nabla \times \left(\mathbf{E}_0 e^{i\mathbf{k}\cdot \mathbf{r} - i \omega t} \right) = 
i\,\mathbf{k}\times \mathbf{E} = - \frac{\partial \mathbf{B}}{\partial t} = i \omega \mathbf{B}.
\label{eq:10_017}
\eeq
Sledi:
\beq
\mathbf{k}\times \mathbf{E} = \omega \mu_0 \mathbf{H} \qquad \Longrightarrow 
\qquad \mathbf{E} \perp \mathbf{H}.
\label{eq:10_018}
\eeq
Iz definicije Poyntinovega vektorja (enačba~\ref{eq:Poyntingov}) 
$\mathbf{S} = \mathbf{E} \times \mathbf{H}$ neposredno sledi medsebojna 
pravokotnost vektorjev $\mathbf{E}$, $\mathbf{H}$ in $\mathbf{S}$.

Ker v anizotropnih snoveh vektorja jakosti $\mathbf{E}$ in gostote električnega 
polja $\mathbf{D}$ nista vzporedna, tudi valovni vektor $\mathbf{k}$ in Poyntingov vektor 
$\mathbf{S}$ nista vzporedna. Energija, ki potuje v smeri Poyntingovega vektorja, se
torej v anizotropnih snoveh ne širi vzdolž valovnega vektorja (slika~\ref{fig:10_koti}). 
Ta ugotovitev bo imela zelo zanimive posledice.
\begin{figure}[!h]
\centering
\def\svgwidth{40truemm} 
\input{slike/10_koti_Sk.pdf_tex}
\caption{V anizotropnih snoveh smer Poyntingovega 
vektorja $\mathbf{S}$ ni enaka smeri valovnega vektorja $\mathbf{k}$. Kot med 
valovnim in Poyntingovim vektorjem je enak kotu med jakostjo in 
gostoto električnega polja.}
\label{fig:10_koti}
\vglue-3truemm
\end{figure}

V anizotropnih snoveh elektromagnetno valovanje ostaja transverzalno
valovanje le s stališča vektorjev $\mathbf{D}$, $\mathbf{B}$ in 
$\mathbf{k}$. Električna poljska jakost $\mathbf{E}$ pridobi 
tudi longitudinalno komponento vzdolž smeri valovnega vektorja. 
Zveza $\nabla \cdot \mathbf{D} = 0$ zato še vedno velja, zveza 
$\nabla \cdot \mathbf{E} = 0$ pa ne več. Posledično valovna
enačba za jakost električnega polja, kakršno poznamo
v izotropnih snoveh (enačba~\ref{eq:valovnaE}), v anizotropnih snoveh ne velja.

Izhajamo iz enačbe~(\ref{eq:Maxwell2}), na njej naredimo rotor in upoštevamo
enačbo~(\ref{eq:Maxwell1}). Dobimo:
\beq
\nabla \times \left( \nabla \times \mathbf{E} \right) = - \frac{\partial}{\partial t} 
\left( \nabla \times \mathbf{B} \right) = -\mu_0 \frac{\partial}{\partial t} 
\left( \nabla \times \mathbf{H} \right) = -\mu_0 \varepsilon_0 
\frac{\partial^2}{\partial t^2} \left( \underline{\varepsilon}\,\mathbf{E} \right).
\label{eq:10_021}
\eeq
Vstavimo nastavek za ravni val (enačba~\ref{eq:10_010}):
\beq
i\mathbf{k}\times \left( i \mathbf{k} \times \mathbf{E}\right) = 
\frac{\omega^2}{c_0^2}\,\underline{\varepsilon}\,\mathbf{E} = 
k_0^2\,\underline{\varepsilon}\,\mathbf{E},
\label{eq:10_022}
\eeq
pri čemer je $k_0 = \omega / c_0$. Upoštevamo zvezo $\mathbf{a}\times
\left( \mathbf{b}\times \mathbf{c}\right) = \left(\mathbf{a}\cdot 
\mathbf{c}\right) \mathbf{b} - \left(\mathbf{a}\cdot \mathbf{b}\right) \mathbf{c}$
in dobimo:
\beq
-\left(\mathbf{k}\cdot \mathbf{E}\right)\mathbf{k} + k^2 \mathbf{E} = 
k_0^2\,\underline{\varepsilon}\,\mathbf{E}.
\label{eq:10_024}
\eeq
Od tod sledi vektorska enačba za jakost električnega polja:
\boxeq{eq:10_025}{
k^2 \mathbf{E}- k_0^2\,\underline{\varepsilon}\,\mathbf{E} = \left(\mathbf{k}\cdot \mathbf{E}\right)\mathbf{k}.
}
V izotropni snovi je izraz na desni strani enačbe enak 0 
in velja $k = k_0\sqrt{\varepsilon} = k_0 n$. 

Vektorska enačba~(\ref{eq:10_025}) predstavlja tri skalarne enačbe za tri 
ortogonalne smeri. Naj bodo to smeri $x$, $y$ in $z$ v lastnem koordinatnem sistemu
tenzorja $\underline{\varepsilon}$, tako da enačbo~(\ref{eq:10_025}) po komponentah 
zapišemo kot:
\begin{align}
\left(\mathbf{k}\cdot \mathbf{E}\right) k_x &= 
\left( k^2 -k_0^2 \varepsilon_{xx}\right) E_x, \label{eq:10_026}\\
\left(\mathbf{k}\cdot \mathbf{E}\right) k_y &= 
\left( k^2 -k_0^2 \varepsilon_{yy}\right) E_y \qquad \mathrm{in}\label{eq:10_027}\\
\left(\mathbf{k}\cdot \mathbf{E}\right) k_y &= \left( k^2 -k_0^2 \varepsilon_{zz}\right) E_z.
\label{eq:10_028}
\end{align}
Enačbe preuredimo in prvo množimo s $k_x$, drugo s $k_y$ in tretjo s $k_z$:
\begin{align}
\left(\mathbf{k}\cdot \mathbf{E}\right) 
\frac{k_x^2}{k^2 - k_0^2 \varepsilon_{xx}} &= E_x k_x, \label{eq:10_029}\\
\left(\mathbf{k}\cdot \mathbf{E}\right) 
\frac{k_y^2}{k^2 - k_0^2 \varepsilon_{yy}} &= E_y k_y \qquad \mathrm{in} \label{eq:10_030}\\
\left(\mathbf{k}\cdot \mathbf{E}\right) 
\frac{k_z^2}{k^2 - k_0^2 \varepsilon_{zz}} &= E_z k_z. \label{eq:10_031}
\end{align}
Enačbe seštejemo:
\beq
\left(\mathbf{k}\cdot \mathbf{E}\right)  
\sum_{j=x,y,z} \frac{k_j^2}{k_0^2(n^2 - \varepsilon_{jj})}
 = \mathbf{k}\cdot \mathbf{E}.
\label{eq:10_032}
\eeq
Pri tem smo uporabili zvezo: $k = k_0 n$. Ker ima v anizotropni snovi 
optično polje tudi longitudinalno komponento, je $\left(\mathbf{k}\cdot 
\mathbf{E}\right) \neq 0$ in obe strani enačbe~(\ref{eq:10_032}) lahko 
delimo s tem skalarnim produktom.
Vpeljemo še smerni vektor $\mathbf{s} = (s_x, s_y, s_z)$ v smeri valovnega vektorja, 
za katerega velja:
\beq
\mathbf{s} = \frac{\mathbf{k}}{k} = \frac{\mathbf{k}}{k_0n}.
\label{eq:10_033}
\eeq
Enačbo~(\ref{eq:10_032}) potem prepišemo v:
\boxeq{eq:10_034}{
\frac{s_x^2}{n^2 - \varepsilon_{xx}}+ 
\frac{s_y^2}{n^2 - \varepsilon_{yy}}+ 
\frac{s_z^2}{n^2 - \varepsilon_{zz}}
= \frac{1}{n^2}.
}
Zapisano enačbo imenujemo Fresnelova enačba za izračun lomnega 
količnika v anizotropni snovi. Zapisana enačba je videti 
zapletena, vendar jo lahko z upoštevanjem zveze 
$s_x^2+s_y^2+s_z^2 = 1$ prevedemo v kvadratno enačbo za $n^2$.
Pri izbrani smeri valovnega vektorja $\mathbf{s}$ tako
obstajata dve različni pozitivni rešitvi za $n$, ki 
predstavljata dve različni vrednosti lomnega količnika za
svetlobo z valovnim vektorjem v smeri $\mathbf{k}$.
Označimo ti dve rešitvi z $n_1$ in $n_2$. 

Vsaki od teh dveh rešitev za lomni količnik pri dani smeri valovnega 
vektorja lahko poiščemo pripadajoč vektor gostote električnega polja
$\mathbf{D}$. Ustrezna vektorja $\mathbf{D}_{1}$ in $\mathbf{D}_{2}$ 
imenujemo lastni polarizaciji. V nadaljevanju bomo pokazali, da sta 
lastni polarizaciji linearni in medsebojno pravokotni.

Povzemimo še enkrat to pomembno ugotovitev: v anizotropnih snoveh se v dani smeri
širita dve ravni valovanji, vsako s svojim lomnim količnikom 
in svojo linearno polarizacijo, ki sta med seboj pravokotni. Pri obravnavni snovi
moramo zato splošno vpadno valovanje razstaviti na dve lastni polarizaciji, vsako 
od katerih potuje s svojo hitrostjo in pridobi svoj fazni zamik. Izhodna svetloba
je zato na splošno eliptično polarizirana.

\begin{example}{\bf Račun lomnih količnikov vzdolž lastne osi tenzorja dielektričnosti.}
Izračunajmo za začetek najpreprostejši primer, ko se svetloba širi vzdolž ene od
lastnih osi tenzorja dielektričnosti. Naj bo to os $x$, tako da je $\mathbf{k} = 
(k,0,0)$. Naša naloga je določiti lomni količnik za posamezni lastni polarizaciji.

Vstavimo izbrani valovni vektor v enačbo~(\ref{eq:10_025}) in dobimo za komponento $x$:
\beq
k^2 E_x - k_0^2 \varepsilon_{xx}E_x = k^2 E_x,
\label{eq:10_035}
\eeq
od koder sledi, da mora biti $E_x = 0$. Drugi dve enačbi, ki ju zapišemo kot:
\beq
k^2 E_y - k_0^2 \varepsilon_{yy}E_y = 0 \qquad \mathrm{in} \qquad k^2 E_z - k_0^2 \varepsilon_{zz}E_z = 0,
\label{eq:10_036}
\eeq
rešita tudi neničelni vrednosti jakosti električnega polja. 
Z upoštevanjem zveze $k = k_0 n$ sledi, da je lomni količnik za 
valovanje, polarizirano v smeri $y$, enak $n_1 = \sqrt{\varepsilon_{yy}}$ in 
za valovanje, polarizirano v smeri $z$, enak $n_2 = \sqrt{\varepsilon_{zz}}$. 

Kadar se valovanje širi v smeri, ki ni vzporedna z lastno osjo tenzorja
dielektričnosti, je račun lomnih količnikov in polarizacij bolj zapleten in 
ga bomo podrobneje obravnavali v naslednjem razdelku.
\end{example}

\begin{example}{\bf Račun lastnih polarizacij v anizotropni snovi.}
Zapisali smo Fresnelovo enačbo (enačba~\ref{eq:10_034}) za lomna količnika
in povedali, da vsakemu lomnemu količniku ustreza lastna polarizacija. Pokažimo, da
sta lastni polarizaciji linearni in medsebojno pravokotni. 

Izhajamo iz enačbe za jakost električnega polja (enačba~\ref{eq:10_025}):
\beq
k^2 \mathbf{E} - k_0^2\underline{\varepsilon}\,\mathbf{E} = 
\left(\mathbf{k}\cdot\mathbf{E}\right)\mathbf{k}.
\label{eq:10_037}
\eeq
Vstavimo zvezi $\mathbf{D}= \varepsilon_0 
\underline{\varepsilon}\,\mathbf{E}$ in $k=k_0n$. Dobimo:
\beq
k^2 \mathbf{E}-\frac{k^2 \mathbf{D}}{\varepsilon_0 n^2} = 
\left(\mathbf{k}\cdot\mathbf{E}\right)\mathbf{k}.
\label{eq:10_038}
\eeq
Enačbo delimo s $k^2$ in preuredimo, tako da jakost polja zapišemo kot vsoto dveh komponent:
\beq
\mathbf{E} = \frac{\mathbf{D}}{\varepsilon_0 n^2} +
\frac{\left(\mathbf{k}\cdot\mathbf{E}\right)\mathbf{k}}{k^2} = \mathbf{E}_\perp + \mathbf{E}_\myparallel.
\label{eq:10_039}
\eeq
Prvi člen v vsoti predstavlja komponento jakosti električnega polja, ki je pravokotna na valovni
vektor (enačba~\ref{eq:10_013}), drugi člen pa komponento, ki je valovnemu vektorju vzporedna. 
Sledi:
\beq
\mathbf{D} = n^2 \varepsilon_0\mathbf{E}_\perp
\label{eq:10_040}
\eeq
oziroma za dve lastni vrednosti lomnega količnika $\mathbf{D}_{1} = n_1^2 \varepsilon_0\mathbf{E}_{1\perp}$ in 
$\mathbf{D}_{2} = n_2^2 \varepsilon_0\mathbf{E}_{2\perp}$.

Izračunajmo vrednosti izrazov:
\beq
\mathbf{E}_1 \cdot \mathbf{D}_2 = \mathbf{E}_1 \cdot \left(\varepsilon_0 
\underline{\varepsilon}\,\mathbf{E}_2\right) =
\varepsilon_0 \sum_{ij} E_{1i}\varepsilon_{ij}E_{2j}
\label{eq:10_041}
\eeq
in 
\beq
\mathbf{E}_2 \cdot \mathbf{D}_1 = \varepsilon_0 \sum_{ij} E_{2i}\varepsilon_{ij}E_{1j} = 
\varepsilon_0 \sum_{ij} E_{1j}\varepsilon_{ij}E_{2i} = 
\varepsilon_0 \sum_{ij} E_{1i}\varepsilon_{ji}E_{2j} = 
\varepsilon_0 \sum_{ij} E_{1i}\varepsilon_{ij}E_{2j},
\label{eq:10_042}
\eeq
pri čemer smo upoštevali simetrijo tenzorja dielektričnosti $\varepsilon_{ij} = 
\varepsilon_{ji}$. 

Od tod sledi:
\beq
\mathbf{E}_1 \cdot \mathbf{D}_2 - \mathbf{E}_2 \cdot \mathbf{D}_1 = 0.
\label{eq:10_043}
\eeq
Vektorja jakosti električnega polja razstavimo na komponenti, nato pa upoštevamo,
da sta vektorja $\mathbf{D}$ pravokotna na smer valovnega vektorja $\mathbf{k}$ in zato velja
$\mathbf{E}_\myparallel \cdot \mathbf{D} = 0$. Ostane le še pravokotna komponenta jakosti
polja, ki jo izrazimo iz enačbe~(\ref{eq:10_040}):
\beq
\mathbf{E}_{1\perp} \cdot \mathbf{D}_2 - \mathbf{E}_{2\perp} \cdot \mathbf{D}_1 =
\frac{\mathbf{D}_1}{\varepsilon_0 n_1^2}\cdot \mathbf{D}_2 - 
\frac{\mathbf{D}_2}{\varepsilon_0 n_2^2}\cdot \mathbf{D}_1 = 0.
\label{eq:10_044}
\eeq
Enačbo preoblikujemo in dobimo:
\beq
\frac{\mathbf{D}_1 \cdot \mathbf{D}_2}{\varepsilon_0}\left( \frac{1}{n_1^2} - 
\frac{1}{n_2^2}\right) = 0.
\label{eq:10_045}
\eeq
V anizotropni snovi sta lomna količnika $n_1$ in $n_2$ različna, zato je člen v oklepaju 
različen od nič. Za zadostitev enačbe mora biti enak nič skalarni produkt 
gostot električnega polja, kar pomeni, da sta v anizotropni snovi lastna 
vektorja $\mathbf{D}_1$ in $\mathbf{D}_2$ med seboj pravokotna. 
\end{example}

\begin{example}{\bf Upodobitev lomnega količnika z optično indikatriso.} 
\label{ex:ind}
Različna lomna količnika in pripadajoča lastna
vektorja $\mathbf{D}_1$ in $\mathbf{D}_2$ si lahko nazorno predstavljamo, če 
uporabimo grafični pripomoček, tako imenovano optično indikatriso ali 
indeksni elipsoid. Naša prva naloga je narisati indeksni elipsoid, ki ga določa
tenzor dielektičnosti. 

Izhajamo iz enačbe za povprečno gostoto energije valovanja
v izotropni snovi (enačba~\ref{eq:w}):
\beq
\langle w \rangle = \frac{1}{2} \varepsilon \varepsilon_0 E_0^2 = \frac{1}{2} 
\mathbf{D}_0 \cdot \mathbf{E}_0. 
\label{eq:10_046}
\eeq
Za anizotropno snov je izraz podoben, vendar upoštevamo
tenzorsko naravo dielektričnosti:
\beq
\langle w \rangle = \frac{1}{2} \mathbf{D}_0 \cdot \mathbf{E}_0 = 
\frac{1}{2\varepsilon_0}\mathbf{D}_0 \cdot \left(\underline{\varepsilon}^{-1} \cdot \mathbf{D}_0
\right)\!.
\label{eq:10_047}
\eeq
V lastnem koordinatnem sistemu je inverzni tenzor oblike:
\beq
\underline{\varepsilon}^{-1} = 
\left[\begin{array}{ccc}
1/\varepsilon_{xx} & 0 & 0\\
0 & 1/\varepsilon_{yy} &0\\
0 & 0 & 1/\varepsilon_{zz}\\
\end{array}\right]\!\!,
\label{eq:10_048}
\eeq
zato lahko enačbo~(\ref{eq:10_047}) prepišemo v:
\beq
2 \varepsilon_0 \langle w \rangle = \frac{D_{x}^2}{\varepsilon_{xx}} + 
\frac{D_{y}^2}{\varepsilon_{y}} + \frac{D_{z}^2}{\varepsilon_{zz}}.
\label{eq:10_049}
\eeq
Če vpeljemo nove normirane koordinate:
$\mathbf{r} = (x,y,z) = \mathbf{D}/\sqrt{2 \varepsilon_0 \langle w\rangle}$, 
dobimo enačbo elipsoida:
\boxeq{eq:elipsoid}{
\frac{x^2}{\varepsilon_{xx}} + \frac{y^2}{\varepsilon_{yy}} + 
\frac{z^2}{\varepsilon_{zz}} = 1,
}
ki opisuje ploskve konstantne vrednosti $\langle w \rangle$ 
z lastnimi osmi $x$, $y$ in $z$ (slika~\ref{fig:10_indikatrisa}\,a). 
\end{example}

\begin{example}{\bf Uporaba optične indikatrise.} 
\label{chap:ind2}
Ko enkrat poznamo optično indikatriso (indeksni elipsoid), lahko 
geometrijsko določimo smer lastnih polarizacij in velikosti lastnih 
lomnih količnikov.
\begin{figure}[h]
\centering
\def\svgwidth{120truemm} 
\input{slike/10_indikatrisa_ravnina.pdf_tex}
\caption{Grafična upodobitev anizotropije tenzorja dielektričnosti z indeksnim
elipsoidom. Glavne osi elipsoida se ujemajo z lastnimi osmi tenzorja, velikosti
polosi pa ustrezajo diagonalnim elementom tenzorja dielektričnosti (a). Prikaz grafičnega
postopka iskanja lastnih polarizacij in lomnih količnikov pri poljubni smeri valovnega
vektorja $\mathbf{s}$ (b).}
\label{fig:10_indikatrisa}
\end{figure}
Postopek načrtovanja je sledeč: v prostoru indikatrise izberemo smerni 
vektor $\mathbf{s}$, ki ustreza smeri valovnega vektorja. Nato 
narišemo ravnino, ki je pravokotna nanj in gre skozi središče elipsoida. 
Presečišče elipsoida in navedene ravnine je elipsa. Pokazali bomo, da njeni 
glavni osi predstavljata smeri lastnih polarizacij $\mathbf{D}_{01}$ in 
$\mathbf{D}_{02}$, dolžini ustreznih polosi pa ustrezata lomnima količnikoma 
$n_1$ in $n_2$ (slika~\ref{fig:10_indikatrisa}\,b). Ta postopek se za dejanski 
izračun le redko uporablja, lahko pa z njim  
nazorno in hitro napovemo obnašanje vpadnega elektromagnetnega valovanja. 

Potrebujemo še matematični dokaz, da je zapisani postopek s konstrukcijo presečišča 
ravnine in elipsoida res ekvivalenten reševanju enačbe za širjenje svetlobe 
po anizotropni snovi (enačba~\ref{eq:10_025}). Najprej zapišemo opisani postopek
matematično: iščemo ekstreme funkcije $r^2$ pri pogoju, da leži $\mathbf{r}$
na ravnini, ki jo podaja enačba $\mathbf{s}\cdot \mathbf{r}$ = 0, in hkrati
na elipsoidu, ki ga podaja enačba~(\ref{eq:elipsoid}). Uporabili bomo vezane ekstreme
in poiskali ekstreme funkcionala $F(x,y,z)$:
\beq
F(x,y,z) = (x^2+y^2+z^2) + 2 \lambda_1 (xs_x+ys_y+zs_z) + \lambda_2
\left(\frac{x^2}{\varepsilon_{xx}}+ \frac{y^2}{\varepsilon_{yy}}+
\frac{z^2}{\varepsilon_{zz}} -1 \right)\!\!,
\label{eq:10_050}
\eeq
pri čemer sta $\lambda_1$ in $\lambda_2$ Lagrangeeva multiplikatorja. Prvi člen
v zapisu je $r^2$, drugi predstavlja enačbo ravnine in tretji enačbo elipsoida.
V ekstremu mora biti izpolnjen pogoj $\nabla F = 0$, ki ga po komponentah zapišemo kot:
\begin{align}
\frac{\partial F}{\partial x} &= 0 \qquad \Longrightarrow 
\qquad x + \lambda_1 s_x + \lambda_2 x/\varepsilon_{xx}=0, \label{eq:10_051}\\
\frac{\partial F}{\partial y} &= 0 \qquad \Longrightarrow 
\qquad y + \lambda_1 s_y + \lambda_2 y/\varepsilon_{yy}=0, \label{eq:10_052}\\
\frac{\partial F}{\partial z} &= 0 \qquad \Longrightarrow 
\qquad z + \lambda_1 s_z + \lambda_2 z/\varepsilon_{zz}=0. \label{eq:10_053}
\end{align}
Prvo enačbo množimo z $x$, drugo z $y$ in tretjo z $z$ ter jih seštejemo v:
\beq
\left(x^2+y^2+z^2 \right) + \lambda_1 \left(s_xx+s_yy+s_zz \right) + \lambda_2 
\left( \frac{x^2}{\varepsilon_{xx}} + \frac{y^2}{\varepsilon_{yy}} + 
\frac{z^2}{\varepsilon_{zz}} \right) = 0.
\label{eq:10_054}
\eeq
Ker sta vektorja $\mathbf{r} = (x,y,z)$ in $\mathbf{s} = (s_x, s_y, s_z)$ pravokotna, 
je drugi člen v enačbi~(\ref{eq:10_054}) enak nič, tretji člen pa je po 
enačbi~(\ref{eq:elipsoid}) enak 1. Sledi: 
\beq
\lambda_2 = -r^2.
\label{eq:10_055}
\eeq
Drugi Lagrangeev multiplikator izračunamo tako, da prvo enačbo (enačba~\ref{eq:10_051}) pomnožimo s 
$s_x$, enačbo~(\ref{eq:10_052}) s $s_y$ in enačbo~(\ref{eq:10_053}) s $s_z$. 
Dobljene enačbe seštejemo v:
\beq
\left(s_xx+s_yy+s_zz \right) + \lambda_1 \left(s_x^2+s_y^2+s_z^2 \right) + \lambda_2 
\left( \frac{xs_x}{\varepsilon_{xx}} + \frac{ys_y}{\varepsilon_{yy}} + 
\frac{zs_z}{\varepsilon_{zz}} \right)= 0.
\label{eq:10_056}
\eeq
Upoštevamo, da je prvi člen enak 0 in drugi enak 1, saj velja $s_x^2+s_y^2+s_z^2=1$. Dobimo:
\beq
\lambda_1 = -\lambda_2 \left( \frac{xs_x}{\varepsilon_{xx}} + \frac{ys_y}{\varepsilon_{yy}} + 
\frac{zs_z}{\varepsilon_{zz}} \right) = r^2 \left( \frac{xs_x}{\varepsilon_{xx}} +
\frac{ys_y}{\varepsilon_{yy}} + 
\frac{zs_z}{\varepsilon_{zz}} \right)\!\!.
\label{eq:10_057}
\eeq
Vstavimo vrednosti multiplikatorjev v sistem enačb~(\ref{eq:10_051}--\ref{eq:10_053}):
\begin{align}
x + r^2 \left( \frac{xs_x}{\varepsilon_{xx}} + \frac{ys_y}{\varepsilon_{yy}} + 
\frac{zs_z}{\varepsilon_{zz}} \right) s_x -r^2 x/\varepsilon_{xx}=0, \label{eq:10_058}\\
y + r^2 \left( \frac{xs_x}{\varepsilon_{xx}} + \frac{ys_y}{\varepsilon_{yy}} + 
\frac{zs_z}{\varepsilon_{zz}} \right)s_y -r^2 y/\varepsilon_{yy}=0, \label{eq:10_059}\\
z + r^2 \left( \frac{xs_x}{\varepsilon_{xx}} + \frac{ys_y}{\varepsilon_{yy}} + 
\frac{zs_z}{\varepsilon_{zz}} \right) s_z -r^2 z/\varepsilon_{zz}=0.\label{eq:10_060}
\end{align}
Enačbe preoblikujemo:
\begin{align}
x (1-r^2/\varepsilon_{xx})+ r^2 \left( \frac{xs_x}{\varepsilon_{xx}} + \frac{ys_y}{\varepsilon_{yy}} + 
\frac{zs_z}{\varepsilon_{zz}} \right) s_x=0, \label{eq:10_061}\\
y (1-r^2/\varepsilon_{yy}) + r^2 \left( \frac{xs_x}{\varepsilon_{xx}} + \frac{ys_y}{\varepsilon_{yy}} + 
\frac{zs_z}{\varepsilon_{zz}} \right)s_y=0, \label{eq:10_062}\\
z (1-r^2/\varepsilon_{zz})+ r^2 \left( \frac{xs_x}{\varepsilon_{xx}} + \frac{ys_y}{\varepsilon_{yy}} + 
\frac{zs_z}{\varepsilon_{zz}} \right) s_z=0.\label{eq:10_063}
\end{align}
Označimo komponente vektorja $\mathbf{r}= (x,y,z)$ z indeksom $r_i$ in dobimo:
\beq
r_i \left(\frac{r^2}{\varepsilon_{ii}} -1\right) =  
r^2\left( \sum_{j}\frac{r_j s_j}{\varepsilon_{jj}}\right) s_i.
\label{eq:10_064}
\eeq
Na tem mestu zaključimo matematično formulacijo opisanega geometrijskega postopka. Pokazati moramo le, 
da so zapisane enačbe~(\ref{eq:10_064}) res ekvivalentne tistim, ki jih dobimo iz
Maxwellovih enačb za širjenje ravnega vala po anizotropni snovi. Izhajamo iz enačbe (\ref{eq:10_025}):
\beq
\left(\mathbf{k}\cdot \mathbf{E}\right)\mathbf{k} = k^2 \mathbf{E}- k_0^2\underline{\varepsilon}\,\mathbf{E}
\label{eq:10_065}
\eeq
in z upoštevanjem $\mathbf{E} = \underline{\varepsilon}^{-1} \mathbf{D}/\varepsilon_0$ 
ter $\mathbf{k} = \mathbf{s}\,k = \mathbf{s}\,k_0 n$ dobimo:
\beq
k^2 \left(\mathbf{s}\frac{1}{\varepsilon_0}\cdot \underline{\varepsilon}^{-1}\mathbf{D}\right) \mathbf{s} = k^2 \frac{1}{\varepsilon_0}\underline{\varepsilon}^{-1} \mathbf{D} - \frac{k^2}{\varepsilon_0 n^2}\mathbf{D}.
\label{eq:10_066}
\eeq
Sledi:
\beq
\left(\mathbf{s} \cdot \underline{\varepsilon}^{-1}\mathbf{D}\right) \mathbf{s} =
\underline{\varepsilon}^{-1} \mathbf{D} - \frac{\mathbf{D}}{n^2} = \frac{1}{n^2}\left( n^2 
\underline{\varepsilon}^{-1} -1 \right) \mathbf{D}.
\label{eq:10_067}
\eeq
Upoštevamo, da je tenzor $\underline{\varepsilon}$ diagonalen in enačbo 
zapišemo po komponentah:
\beq
D_i \left(\frac{n^2}{\varepsilon_{ii}}-1 \right) = n^2 \left(\sum_j\frac{s_j D_j}{\varepsilon_{jj}}\right) s_i.
\label{eq:10_068}
\eeq
Zapisana enačba je po obliki povsem enaka enačbi~(\ref{eq:10_064}), ki smo jo dobili iz 
geometrijske predstavitve problema. Vlogo krajevnega vektorja $\mathbf{r}$ igra vektor gostote
električnega polja $\mathbf{D}$, velikost vektorja $r = |\mathbf{r}|$ pa ustreza velikosti
lomnega količnika $n$. S tem smo pokazali, da rešitev geometrijske konstrukcije res ustreza 
smerem lastnih vektorjev gostot električnega polja in velikosti polosi lomnima količnikoma.
\end{example}

\section{Ploskev valovnega vektorja}
V prejšnjem razdelku smo zapisali, da v anizotropnih snoveh v dani smeri valovnega vektorja
lahko potujeta dve ravni valovanji z različnima polarizacijama in lomnima količnikoma. V 
primerih~(\ref{ex:ind} in \ref{chap:ind2}) je opisan grafični postopek iskanja 
lomnih količnikov in lastnih smeri polarizacij,
tu pa opišimo drug pristop, ki je priročnejši za konkretne izračune. 

Ponovno izhajamo iz Maxwellovih enačb oziroma enačbe~(\ref{eq:10_025}):
\beq
\left(\mathbf{k}\cdot \mathbf{E}\right)\mathbf{k} = 
k^2 \mathbf{E}- k_0^2\underline{\varepsilon}\,\mathbf{E}.
\label{eq:10_069}
\eeq
Zapisana vektorska enačba predstavlja sistem treh enačb za tri komponente vektorja $\mathbf{E}$, ki 
je netrivialno rešljiv, če je determinanta ustrezne matrike sistema enaka nič. Poiščimo najprej
matriko sistema, pri čemer enačbe tudi tokrat zapišemo v lastnem sistemu tenzorja 
$\underline{\varepsilon}$.

Enačbo za komponento v smeri $x$ zapišemo kot:
\beq
(k_x E_x + k_yE_y+k_zE_z) k_x = k^2E_x-k_0^2\varepsilon_{xx}E_x.
\label{eq:10_070}
\eeq
Z upoštevanjem zveze $k^2 = k_x^2+k_y^2+k_z^2$ se enačba prepiše v:
\beq
(k_y^2+k_z^2 - k_0^2\varepsilon_{xx})E_x  - k_xk_yE_y - k_xk_zE_z = 0.
\label{eq:10_071}
\eeq
Podobno naredimo še za preostali koordinati in zapišemo enačbi:
\beq
- k_xk_yE_x + (k_x^2+k_z^2 - k_0^2\varepsilon_{yy})E_y  - k_yk_zE_z = 0
\label{eq:10_072}
\eeq
ter 
\beq
- k_xk_zE_x - k_yk_zE_y + (k_x^2+k_y^2  - k_0^2\varepsilon_{zz})E_z   = 0.
\label{eq:10_073}
\eeq
Enačbe združimo v matriko $M$:
\beq
M\cdot \mathbf{E}=
\left[\begin{array}{ccc}
k_y^2+k_z^2 - k_0^2\varepsilon_{xx} &  - k_xk_y & - k_xk_z\\
- k_xk_y & k_x^2+k_z^2 - k_0^2\varepsilon_{yy} &- k_yk_z\\
- k_xk_z & - k_yk_z & k_x^2+k_y^2  - k_0^2\varepsilon_{zz}\\
\end{array}\right] \cdot
\left[\begin{array}{c}
E_x \\
E_y \\
E_z
\end{array}\right]=0.
\label{eq:10_074}
\eeq
Da je sistem enolično rešljiv, mora biti $\det M=0$. Determinante ne bomo izpisali
za splošen primer, saj je razmeroma zapletena, bomo pa obravnavali nekaj posebnih primerov.

Najprej poglejmo primer, ko je 
$\mathbf{k}= (k_x,k_y, 0)$, s čimer se omejimo na rešitve v ravnini $xy$. 
V tem primeru se matrika $M$ precej poenostavi in dobimo:
\beq
M = \left[\begin{array}{ccc}
k_y^2- k_0^2\varepsilon_{xx} &  - k_xk_y & 0\\
- k_xk_y & k_x^2- k_0^2\varepsilon_{yy} &0\\
0& 0 & k_x^2+k_y^2  - k_0^2\varepsilon_{zz}\\
\end{array}\right]\!\!.
\label{eq:10_075}
\eeq
Determinanta matrike $M$ je v tem primeru:
\beq
\left(k_x^2+k_y^2 - k_0^2\varepsilon_{zz}\right) 
\left((k_y^2- k_0^2\varepsilon_{xx})(k_x^2- k_0^2\varepsilon_{yy})-k_x^2k_y^2\right) = 0.
\label{eq:10_076}
\eeq
Zapisana enačba ima več rešitev. Eno dobimo, kadar je prvi oklepaj enak 0 in velja:
\boxeq{eq:10_077}{
\frac{k_x^2}{k_0^2\varepsilon_{zz}}+\frac{k_y^2}{k_0^2\varepsilon_{zz}} = 1.
}
Ta enačba v ravnini $xy$ opisuje krožnico s polmerom $k_0\varepsilon_{zz}$. Pripadajoči 
lastni vektor jakosti električnega polja ima smer osi $z$.
Drugo rešitev dobimo, kadar je drugi oklepaj v enačbi~(\ref{eq:10_076}) enak 0 in velja:
\beq
(k_y^2- k_0^2\varepsilon_{xx})(k_x^2- k_0^2\varepsilon_{yy})=k_x^2k_y^2.
\label{eq:10_078}
\eeq
Člene zmnožimo, enake člene odštejemo in enačbo preoblikujemo v:
\boxeq{eq:10_080}{
\frac{k_x^2}{k_0^2 \varepsilon_{yy}}+ \frac{k_y^2}{k_0^2 \varepsilon_{xx}} = 1.
}
Zapisana druga rešitev v ravnini $xy$ opisuje elipso s polosjo $k_0\sqrt{\varepsilon_{yy}}$ v smeri $x$ in 
$k_0\sqrt{\varepsilon_{xx}}$ v smeri $y$. Rešitev enačbe~(\ref{eq:10_074}) v ravnini $xy$ opisujeta
krožnica in elipsa. Pripadajoči lastni vektor leži v ravnini $xy$.

Podoben račun lahko naredimo tudi v ravninah $xz$ in $yz$, rezultata pa sta zelo podobna. V vseh treh
ravninah tako dobimo rešitve v obliki ene krožnice in ene elipse. Če se držimo ustaljenega dogovora, 
da velja $\varepsilon_{xx}<\varepsilon_{yy}<\varepsilon_{zz}$, potem je v ravnini $xy$ elipsa znotraj krožnice,
v ravnini $yz$ je elipsa zunaj krožnice, v ravnini $xz$ pa se krožnica in elipsa sekata 
(slika~\ref{fig:10_ploskev_preseki}).

\begin{figure}[h]
\centering
\def\svgwidth{140truemm} 
\input{slike/10_ploskev_preseki.pdf_tex}
\caption{Preseki ploskve valovnega vektorja s koordinatnimi ravninami $xy$ (a), $yz$ (b) in $xz$ (c).
V prvih dveh primerih se elipsa in krožnica, ki predstavljata rešitev enačbe, ne sekata, v tretjem
primeru pa se elipsa in krožnica sekata. Lega presečišč določa smeri dveh optičnih osi v snovi (turkizni puščici).
Pri tem velja $n_x = \sqrt{\varepsilon_{xx}}$, $n_y = \sqrt{\varepsilon_{yy}}$ in $n_z = \sqrt{\varepsilon_{zz}}$.}
\label{fig:10_ploskev_preseki}
\end{figure}

Točke, v katerih se v ravnini $xz$ krožnica in elipsa sekata, določajo smeri optičnih osi. 
V splošnem sta v anizotropni snovi dve taki osi in snov imenujemo optično dvoosna snov. Če se 
svetloba širi vzdolž smeri optične osi, sta rešitvi za obe polarizaciji enaki in se torej obe 
ortogonalni polarizaciji širita z enako hitrostjo. Svetloba, ki se širi vzdolž optične osi, 
ohranja polarizacijo.

\begin{example}{\bf Smer optičnih osi v optično dvoosni snovi.}
Poiščimo smeri optičnih osi v ravnini $xz$. Zaradi simetrije zadošča določiti smer 
ene optične osi, druga je simetrična glede na koordinatni osi. 
Smer vektorja $\mathbf{k}$, vzdolž katerega leži optična os, 
določimo kot smer presečišča krožnice (enačba~\ref{eq:10_077}) in elipse (enačba~\ref{eq:10_080}). 
Naj bo $\mathbf{s}$ smerni vektor vzdolž valovnega vektorja $\mathbf{k}$ (enačba~\ref{eq:10_033}).
Potem enačbo krožnice zapišemo kot:
\beq
n^2s_x^2+n^2s_z^2 = \varepsilon_{yy},
\label{eq:10_081}
\eeq
od koder sledi:
\beq
n = \sqrt{\varepsilon_{yy}}.
\label{eq:10_082}
\eeq
Iščemo presečišče krožnice z elipso:
\beq
\frac{n^2s_x^2}{\varepsilon_{zz}} + \frac{n^2s_z^2}{\varepsilon_{xx}}=1,
\label{eq:10_083}
\eeq
za katerega velja:
\beq
\frac{1}{n} = \sqrt{\frac{s_x^2}{\varepsilon_{zz}} + \frac{s_z^2}{\varepsilon_{xx}}} = 
\frac{1}{\sqrt{\varepsilon_{yy}}}.
\label{eq:10_084}
\eeq
Naj bo smerni vektor $\mathbf{s}= (\sin\vartheta, 0, \cos\vartheta)$, pri čemer je kot
$\vartheta$ kot med optično osjo in osjo $z$. Enačbo~(\ref{eq:10_084}) prepišemo
v:
\beq
\frac{\sin^2\vartheta}{\varepsilon_{zz}} + \frac{\cos^2\vartheta}{\varepsilon_{xx}}=
\frac{1}{\varepsilon_{yy}},
\label{eq:10_085}
\eeq
od koder sledi:
\beq
\cos \vartheta = \sqrt{\frac{1/\varepsilon_{yy}-1/\varepsilon_{zz}}{1/\varepsilon_{xx}-1/\varepsilon_{zz}}}.
\label{eq:10_086}
\eeq
Od tod izračunamo kot $\vartheta$ optične osi, druga os pa leži pod kotom $-\vartheta$ glede na os $z$.
\end{example}

Vrnimo se k enačbi~(\ref{eq:10_074}) oziroma determinanti zapisane matrike $M$. Zaenkrat smo njene 
rešitve poiskali le na koordinatnih ravninah in ugotovili, da ima v poljubni smeri valovnega vektorja
na teh ravninah sistem dve rešitvi, ki ustrezata dvem ortogonalnim polarizacijam. Rešitvam, ki jih
v ravninah opisujejo krožnice, priprada lastni vektor, ki je pravokoten na ravnino izbrane krožnice, 
rešitvam, ki jih opišemo z elipsami, pa ustreza lastni vektor v ravnini izbrane elipse, ki je pravokoten
na valovni vektor. V smereh, kjer
sta rešitvi za obe polarizaciji enaki, se obe lastni polarizaciji širita z enako hitrostjo in snov 
deluje kot izotropna. To se zgodi, kadar svetloba vpada vzdolž ene od dveh optičnih osi snovi.
Rešitve v koordinatnih ravninah smo tudi grafično predstavili na sliki~\ref{fig:10_ploskev_preseki}.
Na splošno je rešitev bolj zapletena, saj so v poljubni smeri valovnega vektorja vse tri komponente 
različne od nič. Rešitev lahko grafično predstavimo s 3D geometrijsko ploskvijo, ki jo imenujemo
ploskev valovnega vektorja. Ker sta v poljubni smeri možni dve rešitvi, je ploskev valovnega vektorja
dvolistna ploskev. Notranji in zunanji list se dotikata oziroma sekata le v štirih točkah v prostoru, ki
določajo smer optičnih osi (slika~\ref{fig:10_ploskev_3D}).

\begin{figure}[h]
\centering
\def\svgwidth{140truemm} 
\input{slike/10_ploskev_3D.pdf_tex}
\caption{Ploskev valovnega vektorja (a) je dvolistna ploskev, ki ima v poljubni smeri dve rešitvi in 
se dotika sama sebe le v smereh optičnih osi (turkizni puščici). Preseki ploskve valovnega vektorja
s koordinatnimi ravninami se ujemajo s preseki, prikazanimi na sliki~\ref{fig:10_ploskev_preseki}.
Zelene puščice prikazujejo smeri lastnih vektorjev $\mathbf{D}$ za izbrano vejo rešitve pri 
določeni smeri valovnega vektorja, rdeče puščice pa pripadajoč vektor jakosti električnega polja 
$\mathbf{E}$. Vzdolž lastnih osi se smeri gostote in jakosti električnega polja ujemata.}
\label{fig:10_ploskev_3D}
\end{figure}
 
Določimo še smer električnega polja, ki ustreza izračunanima rešitvama. Za gostoto električnega polja
smo že zapisali odgovor: za rešitev na krožnici je lastni vektor $\mathbf{D}$ pravokoten
na ravnino krožnice, za rešitev na elipsi pa leži v isti ravnini tudi lastni vektor, ki 
vedno pravokoten na valovni vektor $\mathbf{k}$. Če se širi svetloba vzdolž ene od lastnih osi
sistema, je vektor jakosti električnega polja $\mathbf{E}$ vzporeden vektorju gostote polja $\mathbf{D}$.

Na splošno se smeri gostote in jakosti električnega polja ne ujemata, zato moramo smer $\mathbf{E}$
še določiti. Zaradi preprostosti se ponovno omejimo na ravnino $xy$~(\ref{eq:10_075}). Rešitev 
enačbe~(\ref{eq:10_074}) poiščemo tako, da vstavimo dobljene rešitve za $k_x$ in $k_y$.

Ko vstavimo rešitev s krožnico, za katero velja enačba~(\ref{eq:10_077}), je zadnji člen identično 
enak nič. V tej smeri je lahko jakost električnega polja poljubna, ostali dve komponenti pa morata
biti enaki nič. Rešitvi s krožnico torej ustreza vektor jakosti električnega polja $\mathbf{E}$, ki
je pravokoten na ravnino $xy$ (slika~\ref{fig:10_ploskev_3D}).

Poglejmo še rešitev z elipso (enačba~\ref{eq:10_080}). Rešitev vstavimo v matriko $M$ in dobimo
enačbo:
\beq
M \cdot \mathbf{E}= \left[\begin{array}{ccc}
-k_x^2\frac{\varepsilon_{xx}}{\varepsilon_{yy}} &  - k_xk_y & 0\\
- k_xk_y & k_y^2\frac{\varepsilon_{yy}}{\varepsilon_{xx}} &0\\
0& 0 & k_x^2+k_y^2  - k_0^2\varepsilon_{zz}\\
\end{array}\right] \cdot
\left[\begin{array}{c}
E_x \\
E_y \\
0
\end{array}\right]=0.
\label{eq:10_087}
\eeq
Od tod sledi zveza med komponentama polja:
\beq
\frac{E_y}{E_x} = -\frac{\varepsilon_{xx}}{\varepsilon_{yy}}\frac{k_x}{k_y}.
\label{eq:10_088}
\eeq
S tem smo določili smer jakosti električnega polja v ravnini $xy$. Pokažimo z računom,
da izračunana smer predstavlja smer tangente na elipso. 

Vpeljemo $X = k_x/k_0$ in $Y = k_y/k_0$
in zapišemo elipso (enačba~\ref{eq:10_080}):
\beq
\varepsilon_{xx}X^2 + \varepsilon_{yy}Y^2 = \varepsilon_{xx}\varepsilon_{yy}.
\label{eq:10_089}
\eeq
Enačbo implicitno odvajamo in dobimo:
\beq
2\varepsilon_{xx}X + 2\varepsilon_{yy}Y Y' = 0 \qquad \Longrightarrow \qquad 
Y' = - \frac{\varepsilon_{xx}}{\varepsilon_{yy}} \frac{X}{Y}.
\label{eq:10_090}
\eeq
Vidimo, da je odvod, ki določa smerni koeficient tangente na dano krivuljo, enak
izračunanemu izrazu (enačba~\ref{eq:10_088}).
\begin{figure}[h]
\centering
\def\svgwidth{130truemm} 
\input{slike/10_ploskev_tangenta.pdf_tex}
\caption{Presek ploskve valovnega vektorja z ravninama $xy$ (a) in $xz$ (b). 
Rešitve, ki ustrezajo krožnici, so neodvisna od smeri, vektorji gostote (zelena) in jakosti 
(rdeča) električnega polja so vzporedni in prakovotni na ravnino krožnice. Za rešitev, ki ustreza 
elipsi, so vektorji gostote električnega polja pravokotni na valovni vektor (zelena), vektorji
jakosti električnega polja pa tangentni na elipso (rdeča), pri čemer oboji ležiji v ravnini elipse.}
\label{fig:10_ploskev_tangenta}
\end{figure}

Pri splošni smeri valovnega vektorja v 3D je optično električno polje orientirano
v dveh ortogonalnih tangentnih smereh glede na dvolistno ploskev valovnega vektorja. Vsaki
vrednosti jakosti električnega polja ustreza vektor gostote električnega polja, ki je pravokotna
na valovni vektor. Med vektorjema $E$ in $D$ je na splošno nek kot $\gamma$, ki je odvisen od smeri valovnega
vektorja.

\begin{example}{\bf Kot med gostoto in jakostjo električnega polja.}
Dvolistna ploskev valovnega vektorja ima na splošno dve rešitvi in izračun kota med gostoto
in jakostjo elektičnega polja je razmeroma zapleten. Tukaj se omejimo le na eno geometrijsko
ravnino, naj bo to $xz$ (slika~\ref{fig:10_ploskev_tangenta}\,b). Naj bo smer valovnega vektorja
podana s smerjo $\mathbf{s}=(s_x, s_y, s_z) = (\sin\vartheta, 0, \cos\vartheta)$, pri čemer je $\vartheta$
kot med osjo $z$ in vektorjem $\mathbf{s}$. Smer gostote polja preprosto določimo, saj vemo, da
je pravokotna na $\mathbf{s}$ (enačba~\ref{eq:10_013}) in da leži v ravnini elipse. Dobimo:
\beq
\mathbf{D} = D_0 (-\cos\vartheta, 0, \sin\vartheta).
\label{eq:10_091}
\eeq
Za jakost električnega polja pa vemo, da leži v isti ravnini tangentno na elipso (enačba~\ref{eq:10_088}):
\beq
\mathbf{E} \propto (-\varepsilon_{zz}k_z, 0, \varepsilon_{xx}k_x) \propto
(-\varepsilon_{zz}s_z, 0, \varepsilon_{xx}s_x) = (-\varepsilon_{zz}\cos\vartheta, 0, \varepsilon_{xx}\sin\vartheta).
\label{eq:10_092}
\eeq
Kot med vektorjema $\gamma$ določimo s skalarnim produktom:
\beq
\cos \gamma = \frac{\mathbf{E}\cdot\mathbf{D}}{|\mathbf{E}|\,|\mathbf{D}|} = 
\frac{\varepsilon_{zz}\cos^2\vartheta + \varepsilon_{xx}\sin^2\vartheta}
{\sqrt{\varepsilon_{zz}^2\cos^2\vartheta + \varepsilon_{xx}^2\sin^2\vartheta}}.
\label{eq:10_093}
\eeq
Vzemimo primer sljude z vrednostima $\varepsilon_{xx} = 2,443$ in $\varepsilon_{zz} = 2,563$.
Pri $\vartheta = 60\si{\degree}$ je kot med gostoto in jakostjo električnega polja
enak $\gamma = 1,2\si{\degree}$. Pri kotih $\vartheta = 0$ in $\vartheta = 90\si{\degree}$ je pričakovano
$\gamma=0$ in vektorja gostote in jakosti električnega polja vzporedna.
\end{example}

\begin{example}{\bf Primeri optično dvoosnih snovi.}
Med optično dvoosne snovi snovi sodijo monokristali s triklinsko, 
monoklinsko ali otrorobmsko mrežo. Nekaj primerov lastnih vrednosti
tenzorja dielektričnosti $\underline{\varepsilon}$ oziroma njihovih korenov 
pri valovni dolžini svetlobe $590~\si{nm}$ je navedenih v spodnji tabeli.
\begin{center}
\begin{tabular}{|l|c|c|c|}
\hline
 & $n_1 = \sqrt{\varepsilon_{xx}}$ & $n_2 = \sqrt{\varepsilon_{yy}}$ & 
 $n_3 = \sqrt{\varepsilon_{zz}}$\\ \hline
sljuda & 1,563 & 1,596 & 1,601\\ \hline
topaz & 1,618 & 1,620 & 1,627 \\ \hline
\end{tabular}
\end{center}
\end{example}

\begin{remark}
Do zdaj smo spoznali grafično upodobitev lomnega količnika v anizotropni snovi 
z optično indikatriso in z ploskvijo valovnega vektorja. Poleg tega obstajajo še
druge grafične upodobitve obravnave elektromagnetnega valovanja v anizotropnih snoveh.
Ena od njih je ploskev fazne hitrosti. Dobimo jo tako, da vpeljemo vektor fazne hitrosti:
\beq
\mathbf{c} = \mathbf{s}c = \frac{\mathbf{k}}{k}c = \mathbf{k}\frac{c^2}{\omega}.
\label{eq:10_094}
\eeq
Ta zapis vstavimo v matriko sistema enačb $M$ (enačba~\ref{eq:10_074}) in dobimo:
\beq
M=
\left[\begin{array}{ccc}
c_y^2 +
c_z^2- \frac{c^4}{c_0^2}\varepsilon_{xx} &  
- c_x c_y
& - c_x c_z\\
- c_x c_y & 
c_x^2 + c_z^2- 
\frac{c^4}{c_0^2}\varepsilon_{yy} &
- c_y c_z\\
- c_x c_z& 
- c_y c_z& 
c_x^2 +c_y^2 - 
\frac{c^4}{c_0^2}\varepsilon_{zz}\\
\end{array}\right]\!\!.
\label{eq:10_095}
\eeq
Tudi v tem pristopu mora biti determinanta matrike enaka nič, da ima sistem netrivialne rešitve.
Reševanje enačbe $\det M = 0$ v prostoru $\mathbf{c} = (c_x, c_y, c_z)$ da ploskve četrtega reda, ki 
jih imenujemo ploskve fazne hitrosti. Te ploskve so podobne dvolistne oblike kot ploskev valovnega
vektorja, le da so preseki z geometrijskimi ravninami namesto v obliki elips v obliki
ovalov. Enačbi krožnice in ovala v ravnini $xy$ sta tako:
\beq
c_x^2+c_y^2 = \frac{c_0^2}{\varepsilon_{zz}}\qquad \mathrm{in}\qquad 
\frac{c_x^2}{\varepsilon_{yy}} +\frac{c_y^2}{\varepsilon_{xx}} = 
\frac{(c_x^2+c_y^2)^2}{c_0^2}.
\label{eq:10_096}
\eeq
V izbrani smeri $\mathbf{c}/c$ ponovno dobimo dve rešitvi. 

Tretja možna reprezentacija je ploskev žarkovne hitrosti. Ta je povezana s hitrostjo
širjenja energije elektromagnetnega valovanja. Kot že vemo, Poyntingov vektor
v anizotropni snovi ni vzporeden valovnemu vektorju $\mathbf{k}$, ampak je med njima kot $\gamma$. 
Vpeljemo žarkovno hitrost $\mathbf{u}$ v smeri Poyntinovega vektorja:
\beq
\mathbf{S} = \langle w \rangle \mathbf{u}.
\label{eq:10_097}
\eeq
Žarkovna hitrost predstavlja hitrost, s katero se premikajo valovne fronte vzdolž smeri
Poyntingovega vektorja. S fazno hitrostjo $c$ jo zato izrazimo kot:
\beq
u = \frac{c}{\cos \gamma}.
\label{eq:10_098}
\eeq
Žarkovna hitrost je zato vedno večja od fazne hitrosti, razen v primeru, ko svetloba
potuje vzdolž lastne osi kristala in sta Poyntingov in valovni vektor vzporedna. 

V tem primeru zapišemo sistem enačb za $\mathbf{D}$, ki ga rešujemo z matriko 
$M$ in $\det M =0$. Ponovno dobimo dvolistno ploskev žarkovne hitrosti, njena 
presečišča s koordinatnimi ravninami pa so krožnice in elipse, na primer v ravnini $xy$:
\beq
u_x^2+u_y^2 = \frac{c_0^2}{\varepsilon_{zz}}\qquad \mathrm{in}\qquad 
u_x^2 \varepsilon_{yy} +u_y^2 \varepsilon_{xx} = c_0^2.
\label{eq:10_099}
\eeq
\end{remark}

\section{Optično enoosne snovi}
Do zdaj smo splošno obravnavali optično anizotropne snovi, katerih dielektričnost
smo opisali s tenzorjem $\underline{\varepsilon}$ in tremi različnimi lastnimi vrednostmi. 
V snoveh z višjo stopnjo simetrije se tenzor poenostavi in v rotacijsko
simetričnih snoveh sta dve lastni vrednosti enaki, tretja pa se razlikuje od njiju.
Po dogovoru velja $\varepsilon_{xx} = \varepsilon_{yy}\neq \varepsilon_{zz}$.




rešitev $\vartheta =0$.
V tem primeru je optičan os snovi ena sama in kaže v smeri rotacijske osi $z$. Optična
indukatrisa je v tem primeru rotacijski elipsoid. Rečemo, da je snov optično enoosna. 



V monokristalih z večjo simetrijo (s tetragonalno, trigonalno in heksagonalno mrežo) sta
zaradi simetrije snovi dve od treh vrednosti $\varepsilon_{ii}$ enaki. Dogovorimo se, da sta 
to vrednosti $\varepsilon_{xx} = \varepsilon_{yy}$. V tem primeru se obe optični osi združita
v eno samo optično os, ki je usmerjena vdolž osi $z$. Tako snov imenujemo optično enoosno snov. 



Optična indikatrisa v optično enoosnih snoveh je rotacijski elipsoid z rotacijsko osjo v smeri osi $z$. 
V prostoru valovnega vektorja je posledično ena ploskev kar krogla, druga pa rotacijski elipsoid.
Zaradi rotacijske  simetrije okoli osi $z$ lahko koordinatni sistem vedno izberemo tako, da
komponenta valovnega vektorja, pravokotna na optično os $k_\perp$, kaže vzdož osi $x$. Splošno 
torej lahko zapišemo $\mathbf{k} = (k_x, 0, k_z) = (k_\perp, 0, k_\myparallel)$. 
Valovni vektor potem v ravnini zapišemo kot:
\beq
\mathbf{k} = \mathbf{k}_\perp + \mathbf{k}_\myparallel,
\eeq
ustrezni smerni vektor $\mathbf{s}= \mathbf{k}/(k_0n)$ pa kot:
\beq
\mathbf{s} = \mathbf{s}_\perp + \mathbf{s}_\myparallel.
\eeq
Temu ustrezno označimo tudi $\varepsilon_{xx} = \varepsilon_{yy} = \varepsilon_\perp$ in 
$\varepsilon_{zz} = \varepsilon_\myparallel$. 

Pri poljubni smeri valovnega vektorja tako obstojata dve rešitvi: ena na krožnici in druga
na elipsi. Enačbo kroga opisuje zveza:
\beq
\frac{k_\perp^2/k_0^2}{\varepsilon_\perp}+ \frac{k_\myparallel^2/k_0^2}{\varepsilon_\perp} = 1.
\eeq
Od tod sledi:
\beq
k_\perp^2 + k_\myparallel^2 = k_0^2 \varepsilon_\perp
\eeq
in
\boxeq{eq:redni}{
n_r = \sqrt{\varepsilon_\perp}.
}
Temu lastnemu valovanju rečemo redni žarek. Njegov lomni količnik je neodvisen od smeri $\mathbf{s}$,
njegova polarizacija pa je pravokotna na ravnino, ki jo določata smer optične osi in smer valovnega
vektorja $\mathbf{s}$. Vektor jakosti elektičnega polja je pravokoten na vektor gostote elektičnega polja. 

Druga rešitev je rešitev na elipsi, ki jo opisuje enačba:
\beq
\frac{k_\perp^2/k_0^2}{\varepsilon_\myparallel}+ \frac{k_\myparallel^2/k_0^2}{\varepsilon_\perp} = 
\frac{n^2s_\perp^2}{\varepsilon_\myparallel}+ \frac{n^2s_\myparallel^2}{\varepsilon_\perp} = 
1.
\eeq
Če zapišemo $s_\myparallel = \cos \vartheta$ in $s_\perp = \sin \vartheta$, pri čemer je 
$\vartheta$ kot med optično osjo in smerjo valovnega vektorja, dobimo:
\boxeq{eq:izredni}{
\frac{\sin^2\vartheta}{n_\myparallel^2}+ \frac{\cos^2\vartheta}{n_\perp^2} = \frac{1}{n_i^2}.
}
Temu lastnemu valovanju pravimo izredni žarek. Njegov lomni količnik $n_i$ je odvisen od 
smeri valovnega vektorja $\mathbf{s}$. 

Zanima nas še smer vektorja $\mathbf{E}_i$. Vemo, da je ta smer tangnentna na elipso, ki jo 
opisuje enačba za izredni žarek. Po drugi strani pa vemo, da je smer vektorja $\mathbf{D}_i$ pravokotna na
smer valovnega vektorja. Izračunamo lahko kot $\gamma$ med vektorjema $\mathbf{E}_i$ in $\mathbf{D}_i$, za
katerega velja:
\beq
\cos \gamma = \frac{\cos^2\vartheta /\varepsilon_\perp + \sin^2 \vartheta/\varepsilon_\myparallel}{
\sqrt{\cos^2\vartheta /\varepsilon_\perp^2 + \sin^2 \vartheta/\varepsilon_\myparallel^2}}.
\eeq

Dokaz: Za gostoto električnega polja vemo, da je pravokotna na valovni vektor. Če je vektor
$\mathbf{s} = (\sin\vartheta, 0, \cos\vartheta)$, potem je $\mathbf{D} = D_0(-\cos\vartheta, 0, 
\sin\vartheta)$. Za električno poljsko jakost pa vemo, da ima tangento smer na elipso:
\beq
\frac{E_z}{E_x} = 0\frac{\varepsilon_\perp}{\varepsilon_\myparallel} \frac{s_\perp}{s_\myparallel}.
\eeq
Jakost električnega polja potem zapišemo kot:
\beq
\mathbf{E} = E (-\varepsilon_\myparallel s_\myparallel, 0, \varepsilon_\perp s_\perp) = 
E (-\varepsilon_\myparallel \cos \vartheta, 0, \varepsilon_\perp \sin \vartheta).
\eeq

Za kot $\gamma$ velja:
\beq
\cos \gamma = \frac{\mathbf{E}\cdot \mathbf{D}}{ED} = \frac{\varepsilon_\myparallel \cos^2\vartheta
+ \varepsilon_\perp \sin^2\vartheta}{{\sqrt{\cos^2\vartheta}+ \sin^2\vartheta}
\sqrt{\varepsilon_\myparallel^2 \cos^2\vartheta + \varepsilon_\perp^2 \sin^2\vartheta}},
\eeq
od koder sledi enačba (..). Vidimo, da sta pri kotih $\vartheta = 0$ in $\vartheta = 90\si{\degree}$
vrednosti $\gamma=0$ in tako vektorja gostote in jakosti električnega polja vzporedna. 
To je seveda razumljivo, saj sta to lastni osi sistema oziroma lastni osi tenzorja $\underline{\varepsilon}$.

Primeri enoosnih snovi:
\begin{center}
\begin{tabular}{|c|c|c|c|} \hline
 & $n_\perp$ & $n_\myparallel$ & $\Delta n$\\ \hline
kalcit & 1,658 & 1,486 & -0,172\\ \hline
kremen & 1,544 & 1,553 & 0,009\\ \hline
led & 1,309 & 1,313 & 0,004 \\ \hline
\end{tabular}
\end{center}
Razliko med lomnima količnikoma imenujemo optična anizotropija. Vpeljemo
$\Delta n = n_\myparallel - n_\perp$. Če je $\Delta n >0$, pravimo, da ima snov
pozitivno anizotropijo, v nasprotnem primeru pa negativno anizotropijo. Snovi s kubično 
simetrijo so optično izotropne. 

\section{Lom pri prehodu svetlobe iz izotropne v optično enoosno snov}
Zamislimo si prehod ravnega potujočega elektromagnetnega valovanj iz izotropne snovi 
z lomnim količnikom $n_0$ v anizotropno snov z ustreznima lomnima količnikoma $n_1$ in $n_2$.
V skladu z robnimi pogoji za $\mathbf{E}$ na meji dveh snovi  mora še vedno veljati, 
da se longitudinalna komponenta valovnega vektorja ohranja. Če je $\alpha$ vpadni kot,
mora veljati:
\beq
k_0 n_0 \sin \alpha = k_0 n_1(\beta) \sin \beta_1 \qquad \mathrm{in} \qquad 
k_0 n_0 \sin \alpha = k_0 n_2(\beta) \sin \beta_2.
\eeq
Iz tega sledi lomni zakon:
\beq
n_0 \sin \alpha = n_{1,2}(\beta)\sin\beta.
\eeq
Dobimo izraz, ki je podoben lomnemu zakonu, vendar je treba v tem primeru upoštevati,
da je vrednost lomnega količnika v anizotropni snovi odvisna od lomnega kota $\beta$ --
ter seveda tudi od polarizacije valovanja.

Ob tem se moramo spomniti, da v anizotropni snovi smer širjenja energije, ki jo 
podaja Poyntingov vektor $\mathbf{S}$, ni vzporedna s smerjo valovnega vektorja $\mathbf{k}$.
Zanimiv primer je pravokotni vpad nepolarizirane svetlobe na anizotropno snov. Če
ozek snop nepolarizirane svetlobe usmerimo na plast optično anizotropne snovi, se 
energijski tok tega snopa v snovi razdeli na dva delna snopa, ki se širita
v rahlo različnih smereh. Če je plast snovi dovolj debela, se delna snopa zaradi tega 
prostorsko ločita. Na izstopni strani dobimo dva ločena razmaknjena 
snopa, ki se širita v smeri pravokotno na mejo in sta medsebojno ortogonalno linearno
polarizirana.


Če ploščico optično dvoosne snovi položimo na vzorec, vzorec skozi ploščico vidimo
dvojno. Premaknjeni sliki vzorca sta med seboj pravokotno linearno polarizirani.

\begin{example}
Posebni primer dvojnega loma je primer, ko leži optična os v vpadni ravnini. Izračun
lomnih kotov $\beta_1$ in $\beta_2$ ter ustreznih lastnih polarizacij je v splošnem
primeru, ko optična os enoosne snovi ne leži v vpadni ravnini in niti ni pravokotna nanjo,
dokaj zapleten. Analitična obravnava je relativno preprosta le, če optična os leži v
vpadni ravnini in je ali vzporedna ali pa pravokotna na mejo. Račun naredimo za primer,
ko je optična os vzporedna z mejno ploskvijo.

V tem primeru TE komponenta vpadne svetlobe predstavlja tisti del elektromagnetnega
valovanja, ki se lomi kot redni žarek z lomnim količnikom $n_r$. Zato velja:
\beq
n_0 \sin\alpha = n_r \sin \beta_1 = \sqrt{\varepsilon_\perp}\sin\beta_1.
\eeq
Po drugi strani TM komponenta vpadne svetlobe predstavlja tisti del, ki se lomi
kot izredni žarek z lomnim količnikom $n_i$. Zanj velja:
\beq
n_0 \sin \alpha = n_i(\beta_2)\sin\beta_2.
\eeq
Smer žarka je na optično os nagnjena pod kotom $\vartheta_2 = \pi/2-\beta_2$. 
Zapišemo:
\beq
\frac{1}{n_i^2} = \frac{\sin^2(\pi/2-\beta_2)}{\varepsilon_\myparallel} + 
\frac{\cos^2(\pi/2-\beta_2)}{\varepsilon_\perp} = 
\frac{\cos^2\beta_2}{\varepsilon_\myparallel} + 
\frac{\sin^2\beta_2}{\varepsilon_\perp}.
\eeq
Zapišemo lomni zakon:
\beq
n_0 \sin \alpha = n_i \sin\beta_2,
\eeq
pri čemer je lomni količnik:
\beq
n_i = \frac{1}{\sqrt{\cos^2\beta_2/\varepsilon_\myparallel + 
\sin^2\beta_2/\varepsilon_\perp}}.
\eeq
Iz enačbe .. izračunamo lomni kot. Pri tem moramo biti pozorni, da je le vektor $\mathbf{D}$
transverzalen na valovni vektor, vektor $\mathbf{E}$ pa ima tudi longitudinalno 
komponento in oklepa z vektorjem $\mathbf{D}$ kot $\gamma$.

Vstavimo izraz za lomni količnik v lomni zakon in ga kvadriramo:
\beq
n_0^2 \sin^2\alpha \left(\frac{\cos^2\beta_2}{\varepsilon_\myparallel} + 
\frac{\sin^2\beta_2}{\varepsilon_\perp}\right) = \sin^2\beta_2.
\eeq
Z uporabo zveze med kotnimi funkcijami $\cos^2\varphi = 1 - \sin^2 \varphi$ dobimo:
\beq
\sin^2\beta_2 = 
\frac{n_0^2 \sin^2\alpha}{\varepsilon_\myparallel} + 
\left(\frac{n_0^2 \sin^2\alpha}{\varepsilon_\perp} - \frac{n_0^2 \sin^2\alpha}{\varepsilon_\myparallel}
\right)\sin^2\beta_2.
\eeq
Sledi:
\beq
(\varepsilon_\myparallel - \varepsilon_\perp) n_0^2 \sin^2\alpha \sin^2\beta_2 + \varepsilon_\perp
n_0^2 \sin^2\alpha = \varepsilon_\myparallel \varepsilon_\perp \sin^2\beta_2.
\eeq
Od tod izračunamo lomni kot za izredni žarek:
\beq
\sin\beta_2 = \frac{n_0 \sin \alpha}{\sqrt{\varepsilon_\myparallel - \left(\varepsilon_\myparallel
/\varepsilon_\perp -1\right)n_0^2 \sin^2\alpha}}.
\eeq
\end{example}


V smeri optičnih osi se vpadna polarizacija elektromagnetnega valovanja nemoteno širi naprej,
ne da bi se pri tem pretvarjala v drugo polarizacijo. To ne velja za Poyntingov vektor. Posledično
opazimo zanimiv pojav razcepa polarizacije svetlobe, ki mu pravimo konični dvojni lom. Samo za dvoosne snovi?

\section{Optične komponente na osnovi dvolomnih snovi}
Dvolomne snovi so osnova za večino optičnih komponent, s katerimi učinkujemo na polarizacijo
svetlobe, oziroma naprav, ki se odzivajo na vpadno elektromagnetno valovanje v odvisnosti
od njegove polarizacije.

\subsection*{Polarizatorji}
Polarizatorje uporabimo, da iz nepolarizirane svetlobe ustvarijo  linearno polarizirano 
svetlobo. Tak primer je Nicolova prizma, ki jo imenunjemo po škotskem geologu in fiziku
Williamu Nicolu (1770--1851). Prizma temelji na dvolomnosti, da sta lomna količnika za 
različni polarizaciji različna. Prizmo odrežemo pod takim kotom, da se ena lastna polarizacija
ob vpadu na stranico totalno odbije, druga polarizacija pa je prepuščena. Kristal iz kalcita, 
na primer, prerežemo in damo vmes plast prozornega kanadskega balzama z lomnim
količnikom $n_0=1,55$.  Lomni količnik kalcita je $n_\perp = 1,658$, lomni količnik
$n_\myparallel = 1,486$. Redni žarek se na meji med kristalom in vmesno plastjo 
totalno odbije. Izredni žarek pa je prepuščen. In ker smo pri tem blizu Brewstrovega kota,
je odbojnost za TM polarizacijo zelo majhna, tako da je praktično v celoti prepuščena. Na izhodu
tako dobimo TM linearno polarizirano valovanje. 

Drugi primer je Wollastonova prizma, poimenovana po angleškem kemiku in fiziku 
Williamu Hyde Wollastonu (1766--1828). Prizmo naredimo tako, da kristal prerežemo po 
diagonali in drugi del zasučemo glede na os prvega, tako da je optična os v drugem delu
obrnjena pravokotno glede na optično os v prvem delu. Na meji med obema deloma se potem 
valovanje razdeli na dve linearno polarizirani komponenti, ki zapustita prizmo vsaka v svoji
smeri. 

\subsection*{Retardacijske ploščice}
Retardacijske ploščice med dvema ortogonalnima komponentama polarizacije ustvarijo
fazni zamik (retardacijo). Njihov učinek opišemo z Jonesovo matriko:
\beq
M = 
\left[\begin{array}{cc}
e^{-i\Delta \Phi/2} & 0\\
0 & e^{i\Delta \Phi/2}\\
\end{array}\right]\!\!.
\eeq
Pri tem je $\Delta \Phi = k_0 L (n_i-n_r)$, kjer je $L$ dolžina kristala, $n_i$ in $n_r$
pa označujeta redni in izredni lomni količnik snovi. Če je $\Delta \Phi= \pi/2$, taki 
retardacijski ploščici pravimo ploščica $\lambda/4$. Taka ploščia iz linearne polarizacije
naredi cirkularno polarizirano svetlobo. Kadar je fazni zamik $\pi$, gre za ploščico $\lambda/2$.

\subsection*{Kompenzatorji}
Retardacijske ploščice imajo željeno retardacijo le za določeno valovno dolžino. Če 
želimo element, ki je uporaben za različne valovne dolžine, uporabimo tako imenovani
kompenzator. Z njim lahko izničimo oz kompneziramo poljuben fazni zamik, do katerega pride
v neki preiskovani snovi. Primer kompenzatorjev je Soleil-Babinetov kompenzator, v katerem
retardacijo reguliramo s pomikanjnjem klinastega elementa iz anizotropne snovi. Poleg opisanega
je še cela vrsta polarizatorjev in kompenzatorjev (Nomarski, Glan-Foucaul, Glan-Thompson, S{\'e}narmont, 
Rochon).

\section{Konoskopija}
Konoskopija je eksperimentalna metoda za analizo in vizualizacijo optično anizotropnih snovi. 
Navadno želimo izvedeti, ali je snov optično enoosna ali dvoosna ter določiti orientacijo 
optičnih osi. Določimo lahko tudi glavne lomne količnike $n_1$, $n_2$ in $n_3$. 

Med prekrižani polarizacijski foliji damo proučevano snov. Osvetlitev izvedemo s koničnim
snopom svetlobe. Na opazovalnem zaslonu se pojavijo svetle in temne proge, odvisno
od dvolomnosti snovi. Ločimo dve vrsti linij na zaslonu: to so lahko izogire, pri katerih vpadna
polarizacija ustreza eni od lastnih polarizacij $\mathbf{D}_1$ ali $\mathbf{D}_2$, ali
izokrome, pri katerih za dano valovno dolžino $\lambda$ dobimo fazni zamik $N2\pi$. 
Če delamo poskus z belo svetlobo, so izokrome mavričnih barv, izogire pa so vedno temne.
Trik: v eksperimentu lahko konično svetlobo dobimo tako, da z laserjem posvetimo
na difuzor (na primer prosojni paus papir).



\section{Inducirana anizotropija}
