%\chapterimage{Geometrijska.jpg} % Chapter heading image

\chapter{Optika v anizotropnih snoveh}
\label{chap:AnizotropneSnovi}
V tem poglavju bomo spoznali optične pojave v snoveh z optično anizotropijo, kar pomeni,
da imajo v različnih smereh različne optične lastnosti. Spoznali bomo razliko med
optično enoosnimi in optično dvoosnimi snovmi in poračunali prehod svetlobe skozi njih. 
Na koncu poglavja bomo pogledali nekaj načinov uporabe in nastanka dvolomnosti.

\section{Anizotropne snovi}
Za anizotropne snovi je značilno, da imajo v različnih smereh različne lastnosti. To so lahko
mehanske, toplotne, električne, magnetne, optične ... V optično anizotropnih snoveh je tako
odziv snovi, to je inducirana električna polarizacija, odvisna ne le od smeri priključenega
zunanjega polja, temveč tudi od njegove smeri. Za nazorno razumevanje se ponovno poslužimo
Lorentzevega modela, ki smo ga podrobneje spoznali v prejšnjem poglavju. Zamislimo si, 
da je negativno nabita kroglica, ki predstavlja elektron, vpeta v mrežo mirujočih sosednjih 
pozitivnih nabojev z vzmetmi, ki imajo v eni smeri različno konstanto vzmeti kot v drugi 
(slika~\ref{fig:10_model}).
\begin{figure}[!h]
\centering
\def\svgwidth{90truemm} 
%\input{slike/10_model.pdf_tex}
\caption{Anizotropijo snovi lahko pojasnimo tudi z Lorenzevim modelom, v katerem so
konstante vzmeti različne v raličnih smereh.}
\label{fig:10_model}
\end{figure}

Osnovna posledica anizotropnega odziva je, da smer inducirane električne polarizacije $\mathbf{P}$
ni več vzporedna smeri jakosti optičnega električnega polja $\mathbf{E}$. Posledično tudi 
vektor gostote električnega polja $\mathbf{D}$ ni več vzporeden s smerjo $\mathbf{E}$. 
Na splošno velja:
\beq
\mathbf{D} = \varepsilon_0 \mathbf{E} + \mathbf{P} = \varepsilon_0 \underline{\varepsilon} \mathbf{E}.
\label{eq:10_001}
\eeq
Za opis anizotropnih snovi dielektričnost $\varepsilon$ ni skalar, ampak je tenzor drugega
ranga.

V nadaljevanju se bomo omejili na snovi, ki ne absorbirajo svetlobe in so zato njihovi 
tenzorji dielektričnosti realni. S tem opisom zajamemo prozorne anizotropne snovi, ki so
tipično minerali, na primer kalcit ali  sljuda. Zaradi povezave z energijo elektromagnetnega polja
mora biti tenzor $\underline{\varepsilon}$ Hermitski in pozitivno definiten. To pomeni, da je 
simetričen in ga lahko vedno diagonaliziramo. Vse njegove lastne vrednosti so realne in pozitivne. 

\section{Lomni količnik}
Obravnavajmo optično anizotropno neprevodno snov in njeno dielektričnost opišemo s tenzorjem. Zaradi
enostavnosti se omejimo na snovi, za katere je $\mu = 1$. Naj na anizotropno snov vpada
ravno potujoče sinusno valovanje z valovnim vektorjem $\mathbf{k}$. Naša naloga je poiskati
lomni količnik snovi v odvisnosti od smeri valovnega vektorja ter od polarizacije vpadnega
valovanja. 

Uskladi s tretjim poglavjem in skrajšaj!

Izhajamo iz Maxwellovih enačb (enačbe~\ref{eq:Maxwell1}--\ref{eq:Maxwell4}), 
pri čemer privzamemo, da sta gostoti nabojev in električnih tokov enaki nič. 
Rešitev iščemo v obliki ravnih valov:
\begin{align}
 \mathbf{E} &= \mathbf{E}_0 e^{i\mathbf{k}\cdot \mathbf{r} - i \omega t} \label{eq:10_002} \\
 \mathbf{D} &= \mathbf{D}_0 e^{i\mathbf{k}\cdot \mathbf{r} - i \omega t} \label{eq:10_003}\\
 \mathbf{B} &= \mu_0 \mathbf{H} = \mathbf{B}_0 e^{i\mathbf{k}\cdot \mathbf{r} - i \omega t}.
 \label{eq:10_004}
\end{align}
Iz Maxwellove enačbe~(\ref{eq:Maxwell4}) z upoštevanjem nastavkov dobimo:
\beq
\nabla \cdot \left(\mathbf{B}_0 e^{i\mathbf{k}\cdot \mathbf{r} - i \omega t} \right) = 
i~\mathbf{k}\cdot \mathbf{B}_0 = 0 \qquad \Longrightarrow \qquad \mathbf{B}_0 \perp \mathbf{k}.
\label{eq:10_005}
\eeq
Podobno iz enačbe~(\ref{eq:Maxwell3}) sledi:
\beq
\nabla \cdot \left(\mathbf{D}_0 e^{i\mathbf{k}\cdot \mathbf{r} - i \omega t} \right) = 
i~\mathbf{k}\cdot \mathbf{D}_0 = 0 \qquad \Longrightarrow \qquad \mathbf{D}_0 \perp \mathbf{k}.
\label{eq:10_006}
\eeq
Povsem na splošno sta torej vektorja $\mathbf{D}$ in $\mathbf{B}$ pravokotna na valovni vektor $\mathbf{k}$.
Izhajajoč iz enačbe~(\ref{eq:Maxwell1}) dobimo:
\beq
\nabla \times \left(\mathbf{H}_0 e^{i\mathbf{k}\cdot \mathbf{r} - i \omega t} \right) = 
i~\mathbf{k}\times \mathbf{H}_0 = \frac{\partial \mathbf{D}}{\partial t} = -i \omega \mathbf{D}.
\label{eq:10_007}
\eeq
Od tod za nemagnetne snovi neposredno sledi:
\beq
\mathbf{k}\times \mathbf{B}_0 = - \mu_0 \omega \mathbf{D} \qquad \Longrightarrow \qquad \mathbf{B}_0 \perp \mathbf{D}_0.
\label{eq:10_008}
\eeq
Trije vektorji $\mathbf{D}$, $\mathbf{B}$ in $\mathbf{k}$ so torej medseboj paroma pravokotni.

Poglejmo še Faradayev zakon (enačbo~\ref{eq:Maxwell2}), iz katere sledi:
\beq
\nabla \times \left(\mathbf{E}_0 e^{i\mathbf{k}\cdot \mathbf{r} - i \omega t} \right) = 
i~\mathbf{k}\times \mathbf{E}_0 = - \frac{\partial \mathbf{B}}{\partial t} = i \omega \mathbf{B}.
\label{eq:10_009}
\eeq
Dobimo:
\beq
\mathbf{k}\times \mathbf{E}_0 = \omega \mu_0 \mathbf{H} \qquad \Longrightarrow \qquad \mathbf{E}_0 \perp \mathbf{H}_0.
\label{eq:10_010}
\eeq
Spomnimo se še definicije Poyntinovega vektorja (enačba~\ref{eq:Poyntingov}):
\beq
\mathbf{S} = \mathbf{E} \times \mathbf{H},
\label{eq:10_011}
\eeq
od koder, z upoštevanjem enačbe~(\ref{eq:10_010}), sledi medsebojna pravokotnost vektorjev $\mathbf{E}$, $\mathbf{H}$ in 
$\mathbf{S}$. 

A optično anizotropnih snoveh vektorja jakosti in gostote električnega polja nista vzporedna, zato tudi 
valovni vektor in Poyntingov vektor nista vzporedna. Energija, ki potuje v smeri Poyntingovega vektorja, se
torej v anizotropnih snoveh ne širi vzdolž valovnega vektorja. 
\begin{figure}[!h]
\centering
\def\svgwidth{90truemm} 
%\input{slike/10_koti.pdf_tex}
\caption{V anizotropnih snoveh smer širjenja energije (Poyntingovega 
vektorja) ni enaka smeri valovnega vektorja.}
\label{fig:10_koti}
\end{figure}

V anizotropnih snoveh elektromagnetno valovanje ostaja transverzalno
valovanje le s stališča vektorjev $\mathbf{D}$, $\mathbf{B}$ in 
$\mathbf{k}$. Električna poljska jakost $\mathbf{E}$ pa pridobi 
tudi longitudinalno komponento vzdolž smeri valovnega vektorja. 
Zveza $\nabla \cdot \mathbf{D} = 0$ zato še vedno velja, zveza 
$\nabla \cdot \mathbf{E} = 0$ pa ne več. Posledično tudi valovna
enačba za jakost električnega polja, kakor smo je vajeni
v izotropnih snoveh, v anizotropnih snoveh ne velja. Zapišimo 
analogno enačbo za anizotropne snovi. 

Izhajamo iz Maxwellove enačbe (\ref{eq:Maxwell2}) in na njej naredimo rotor:
\beq
\nabla \times \left( \nabla \times \mathbf{E} \right) = - \frac{\partial}{\partial t} 
\left( \nabla \times \mathbf{B} \right) = -\mu_0 \frac{\partial}{\partial t} 
\left( \nabla \times \mathbf{H} \right).
\label{eq:10_012}
\eeq
Z upoštevanjem Maxwellove enačbe () dobimo:
\beq
\nabla \times \left( \nabla \times \mathbf{E} \right) = -\mu_0 \varepsilon_0 
\frac{\partial^2}{\partial t^2} \left( \underline{\varepsilon}\,\mathbf{E} \right).
\label{eq:10_013}
\eeq
Vstavimo nastavek za ravni val in dobimo:
\beq
i\mathbf{k}\times \left( i \mathbf{k} \times \mathbf{E}\right) = 
\frac{\omega^2}{c_0^2}\underline{\varepsilon} \mathbf{E} = 
k_0^2 \underline{\varepsilon} \mathbf{E},
\label{eq:10_014}
\eeq
pri čemer smo vpeljali $k_0 = \omega / c_0$. Upoštevamo zvezo:
\beq
\mathbf{a}\times\left( \mathbf{b}\times \mathbf{c}\right) = \left(\mathbf{a}\cdot 
\mathbf{c}\right) \mathbf{b} - \left(\mathbf{a}\cdot \mathbf{b}\right) \mathbf{c}
\label{eq:10_015}
\eeq
in dobimo:
\beq
-\left(\mathbf{k}\cdot \mathbf{E}\right)\mathbf{k} + k^2 \mathbf{E} = k_0^2 \underline{\varepsilon} \mathbf{E}.
\label{eq:10_016}
\eeq
Od tod dobimo vektorsko enačbo za jakost električnega polja:
\boxeq{eq:10_017}{
k^2 - k_0\underline{\varepsilon}\mathbf{E} = \left(\mathbf{k}\cdot \mathbf{E}\right)\mathbf{k}.
}
V izotropni snovi je izraz na desni strani enačbe enak 0 in velja $k = k_0\sqrt{\varepsilon} = k_0 n$. V anizotropnih
snoveh pa je ta izraz na splošno različen od nič. Tudi v anizotropnih snoveh lomni količnik
definiramo na podoben način $\mathbf{k}= k_0 n \mathbf{s}$, pri čemer je $\mathbf{s}$ smerni vektor
valovnih front.

Ta vektorska enačba predstavlja tri skalarne enačbe za tri smeri. Za nadaljni izračun si izberemo,
da so to smeri $x$, $y$ in $z$ v koordinatnem sistemu, v katerem je $\underline{\varepsilon}$ diagonalen.
V izbranem koordinatnem sistemu tenzor dielektričnosti zapišemo kot:
\beq
\underline{\varepsilon} = 
\left[\begin{array}{ccc}
\varepsilon_{xx} & 0 & 0\\
0& \varepsilon_{yy}& 0\\
0& 0 & \varepsilon_{zz}\\
\end{array}\right]\!\!.
\label{eq:10_018}
\eeq
Enačbo~(\ref{eq:10_017}) zapišemo po komponentah in dobimo sistem treh enačb:
\begin{align}
\left(\mathbf{k}\cdot \mathbf{E}\right) k_x &= \left( k^2 -k_0^2 \varepsilon_{xx}\right) E_x \\
\left(\mathbf{k}\cdot \mathbf{E}\right) k_y &= \left( k^2 -k_0^2 \varepsilon_{yy}\right) E_y \\
\left(\mathbf{k}\cdot \mathbf{E}\right) k_y &= \left( k^2 -k_0^2 \varepsilon_{zz}\right) E_z 
\label{eq:10_019}
\end{align}






\section{Ploskev valovnega vektorja}
\section{Ploskev fazne in žarkovne hitrosti}
\section{Optično enoosne snovi}
\section{Optične komponente na osnovi dvolomnih snovi}
\section{Konoskopija}
\section{Inducirana anizotropija}
