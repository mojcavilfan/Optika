%\chapterimage{Geometrijska.jpg} % Chapter heading image

\chapter{Optika v anizotropnih snoveh}
\label{chap:AnizotropneSnovi}
V tem poglavju bomo spoznali optične pojave v snoveh z optično anizotropijo, kar pomeni,
da imajo v različnih smereh različne optične lastnosti. Spoznali bomo razliko med
optično enoosnimi in optično dvoosnimi snovmi in poračunali prehod svetlobe skozi njih. 
Na koncu poglavja bomo pogledali nekaj načinov uporabe in nastanka dvolomnosti.

\section{Anizotropne snovi}
Za anizotropne snovi je značilno, da imajo v različnih smereh različne lastnosti. To so lahko
mehanske, toplotne, električne, magnetne, optične ... V optično anizotropnih snoveh je tako
odziv snovi, to je inducirana električna polarizacija, odvisna ne le od smeri priključenega
zunanjega polja, temveč tudi od njegove smeri. Za nazorno razumevanje se ponovno poslužimo
Lorentzevega modela, ki smo ga podrobneje spoznali v prejšnjem poglavju. Zamislimo si, 
da je negativno nabita kroglica, ki predstavlja elektron, vpeta v mrežo mirujočih sosednjih 
pozitivnih nabojev z vzmetmi, ki imajo v eni smeri različno konstanto vzmeti kot v drugi 
(slika~\ref{fig:10_model}).
\begin{figure}[!h]
\centering
\def\svgwidth{90truemm} 
%\input{slike/10_model.pdf_tex}
\caption{Anizotropijo snovi lahko pojasnimo tudi z Lorenzevim modelom, v katerem so
konstante vzmeti različne v raličnih smereh.}
\label{fig:10_model}
\end{figure}

Osnovna posledica anizotropnega odziva je, da smer inducirane električne polarizacije $\mathbf{P}$
ni več vzporedna smeri jakosti optičnega električnega polja $\mathbf{E}$. Posledično tudi 
vektor gostote električnega polja $\mathbf{D}$ ni več vzporeden s smerjo $\mathbf{E}$. 
Na splošno velja:
\beq
\mathbf{D} = \varepsilon_0 \mathbf{E} + \mathbf{P} = \varepsilon_0 \underline{\varepsilon} \mathbf{E}.
\label{eq:10_001}
\eeq
Za opis anizotropnih snovi dielektričnost $\varepsilon$ ni skalar, ampak je tenzor drugega
ranga.

V nadaljevanju se bomo omejili na snovi, ki ne absorbirajo svetlobe in so zato njihovi 
tenzorji dielektričnosti realni. S tem opisom zajamemo prozorne anizotropne snovi, ki so
tipično minerali, na primer kalcit ali  sljuda. Zaradi povezave z energijo elektromagnetnega polja
mora biti tenzor $\underline{\varepsilon}$ Hermitski in pozitivno definiten. 
To pomeni, da je  simetričen in ga lahko vedno diagonaliziramo. Vse njegove lastne vrednosti so realne in pozitivne.

Tenzor dielektričnosti na splošno zapišemo kot:
\beq
\underline{\varepsilon} = 
\left[\begin{array}{ccc}
\varepsilon_{xx} & \varepsilon_{xy} & \varepsilon_{xz}\\
\varepsilon_{yx} & \varepsilon_{yy} & \varepsilon_{yz}\\
\varepsilon_{zx} & \varepsilon_{zy} & \varepsilon_{zz}\\
\end{array}\right]\!\!.
\eeq
V lastnem koordinatnem sistemu ga potem zapišemo:
\beq
\underline{\varepsilon} = 
\left[\begin{array}{ccc}
\varepsilon_{xx} & 0 & 0\\
0 & \varepsilon_{yy} &0\\
0 & 0 & \varepsilon_{zz}\\
\end{array}\right]\!\!.
\eeq
Če recimo svetloba vpada v smeri $z$, in ima torej jakost
elektičnega polja komponenti v smeri $E_x$ in $E_y$. 
V lastnem sistemu potem električno polarizacijo zapišemo kot:
\begin{align}
P_x &= (\varepsilon_{xx}-1) \varepsilon_0 E_x \\
P_y &= (\varepsilon_{yy}-1) \varepsilon_0 E_y.
\end{align}
Če je vpadni kot jakosti električnega polja $\alpha$, je kot
inducirane polarizacije enak:
\beq
\tan \beta = \frac{P_y}{P_x} = \frac{(\varepsilon_{yy}-1)E_y}{(\varepsilon_{xx}-1)E_x}
 = \frac{\varepsilon_{yy}-1}{\varepsilon_{xx}-1}\tan \alpha.
\eeq

\section{Lomni količnik}
Obravnavajmo optično anizotropno neprevodno snov in njeno dielektričnost opišemo s tenzorjem. Zaradi
enostavnosti se omejimo na snovi, za katere je $\mu = 1$. Naj na anizotropno snov vpada
ravno potujoče sinusno valovanje z valovnim vektorjem $\mathbf{k}$. Naša naloga je poiskati
lomni količnik snovi v odvisnosti od smeri valovnega vektorja ter od polarizacije vpadnega
valovanja. 

Uskladi s tretjim poglavjem in skrajšaj!

Izhajamo iz Maxwellovih enačb (enačbe~\ref{eq:Maxwell1}--\ref{eq:Maxwell4}), 
pri čemer privzamemo, da sta gostoti nabojev in električnih tokov enaki nič. 
Rešitev iščemo v obliki ravnih valov:
\begin{align}
 \mathbf{E} &= \mathbf{E}_0 e^{i\mathbf{k}\cdot \mathbf{r} - i \omega t} \label{eq:10_002} \\
 \mathbf{D} &= \mathbf{D}_0 e^{i\mathbf{k}\cdot \mathbf{r} - i \omega t} \label{eq:10_003}\\
 \mathbf{B} &= \mu_0 \mathbf{H} = \mathbf{B}_0 e^{i\mathbf{k}\cdot \mathbf{r} - i \omega t}.
 \label{eq:10_004}
\end{align}
Iz Maxwellove enačbe~(\ref{eq:Maxwell4}) z upoštevanjem nastavkov dobimo:
\beq
\nabla \cdot \left(\mathbf{B}_0 e^{i\mathbf{k}\cdot \mathbf{r} - i \omega t} \right) = 
i~\mathbf{k}\cdot \mathbf{B}_0 = 0 \qquad \Longrightarrow \qquad \mathbf{B}_0 \perp \mathbf{k}.
\label{eq:10_005}
\eeq
Podobno iz enačbe~(\ref{eq:Maxwell3}) sledi:
\beq
\nabla \cdot \left(\mathbf{D}_0 e^{i\mathbf{k}\cdot \mathbf{r} - i \omega t} \right) = 
i~\mathbf{k}\cdot \mathbf{D}_0 = 0 \qquad \Longrightarrow \qquad \mathbf{D}_0 \perp \mathbf{k}.
\label{eq:10_006}
\eeq
Povsem na splošno sta torej vektorja $\mathbf{D}$ in $\mathbf{B}$ pravokotna na valovni vektor $\mathbf{k}$.
Izhajajoč iz enačbe~(\ref{eq:Maxwell1}) dobimo:
\beq
\nabla \times \left(\mathbf{H}_0 e^{i\mathbf{k}\cdot \mathbf{r} - i \omega t} \right) = 
i~\mathbf{k}\times \mathbf{H}_0 = \frac{\partial \mathbf{D}}{\partial t} = -i \omega \mathbf{D}.
\label{eq:10_007}
\eeq
Od tod za nemagnetne snovi neposredno sledi:
\beq
\mathbf{k}\times \mathbf{B}_0 = - \mu_0 \omega \mathbf{D} \qquad \Longrightarrow \qquad \mathbf{B}_0 \perp \mathbf{D}_0.
\label{eq:10_008}
\eeq
Trije vektorji $\mathbf{D}$, $\mathbf{B}$ in $\mathbf{k}$ so torej medseboj paroma pravokotni.

Poglejmo še Faradayev zakon (enačbo~\ref{eq:Maxwell2}), iz katere sledi:
\beq
\nabla \times \left(\mathbf{E}_0 e^{i\mathbf{k}\cdot \mathbf{r} - i \omega t} \right) = 
i~\mathbf{k}\times \mathbf{E}_0 = - \frac{\partial \mathbf{B}}{\partial t} = i \omega \mathbf{B}.
\label{eq:10_009}
\eeq
Dobimo:
\beq
\mathbf{k}\times \mathbf{E}_0 = \omega \mu_0 \mathbf{H} \qquad \Longrightarrow \qquad \mathbf{E}_0 \perp \mathbf{H}_0.
\label{eq:10_010}
\eeq
Spomnimo se še definicije Poyntinovega vektorja (enačba~\ref{eq:Poyntingov}):
\beq
\mathbf{S} = \mathbf{E} \times \mathbf{H},
\label{eq:10_011}
\eeq
od koder, z upoštevanjem enačbe~(\ref{eq:10_010}), sledi medsebojna pravokotnost vektorjev $\mathbf{E}$, $\mathbf{H}$ in 
$\mathbf{S}$. 

A optično anizotropnih snoveh vektorja jakosti in gostote električnega polja nista vzporedna, zato tudi 
valovni vektor in Poyntingov vektor nista vzporedna. Energija, ki potuje v smeri Poyntingovega vektorja, se
torej v anizotropnih snoveh ne širi vzdolž valovnega vektorja. 
\begin{figure}[!h]
\centering
\def\svgwidth{90truemm} 
%\input{slike/10_koti.pdf_tex}
\caption{V anizotropnih snoveh smer širjenja energije (Poyntingovega 
vektorja) ni enaka smeri valovnega vektorja.}
\label{fig:10_koti}
\end{figure}

V anizotropnih snoveh elektromagnetno valovanje ostaja transverzalno
valovanje le s stališča vektorjev $\mathbf{D}$, $\mathbf{B}$ in 
$\mathbf{k}$. Električna poljska jakost $\mathbf{E}$ pa pridobi 
tudi longitudinalno komponento vzdolž smeri valovnega vektorja. 
Zveza $\nabla \cdot \mathbf{D} = 0$ zato še vedno velja, zveza 
$\nabla \cdot \mathbf{E} = 0$ pa ne več. Posledično tudi valovna
enačba za jakost električnega polja, kakor smo je vajeni
v izotropnih snoveh, v anizotropnih snoveh ne velja. Zapišimo 
analogno enačbo za anizotropne snovi. 

Izhajamo iz Maxwellove enačbe (\ref{eq:Maxwell2}) in na njej naredimo rotor:
\beq
\nabla \times \left( \nabla \times \mathbf{E} \right) = - \frac{\partial}{\partial t} 
\left( \nabla \times \mathbf{B} \right) = -\mu_0 \frac{\partial}{\partial t} 
\left( \nabla \times \mathbf{H} \right).
\label{eq:10_012}
\eeq
Z upoštevanjem Maxwellove enačbe () dobimo:
\beq
\nabla \times \left( \nabla \times \mathbf{E} \right) = -\mu_0 \varepsilon_0 
\frac{\partial^2}{\partial t^2} \left( \underline{\varepsilon}\,\mathbf{E} \right).
\label{eq:10_013}
\eeq
Vstavimo nastavek za ravni val in dobimo:
\beq
i\mathbf{k}\times \left( i \mathbf{k} \times \mathbf{E}\right) = 
\frac{\omega^2}{c_0^2}\underline{\varepsilon} \mathbf{E} = 
k_0^2 \underline{\varepsilon} \mathbf{E},
\label{eq:10_014}
\eeq
pri čemer smo vpeljali $k_0 = \omega / c_0$. Upoštevamo zvezo:
\beq
\mathbf{a}\times\left( \mathbf{b}\times \mathbf{c}\right) = \left(\mathbf{a}\cdot 
\mathbf{c}\right) \mathbf{b} - \left(\mathbf{a}\cdot \mathbf{b}\right) \mathbf{c}
\label{eq:10_015}
\eeq
in dobimo:
\beq
-\left(\mathbf{k}\cdot \mathbf{E}\right)\mathbf{k} + k^2 \mathbf{E} = 
k_0^2 \underline{\varepsilon} \mathbf{E}.
\label{eq:10_016}
\eeq
Od tod dobimo vektorsko enačbo za jakost električnega polja:
\boxeq{eq:10_017}{
k^2 \mathbf{E}- k_0\underline{\varepsilon}\mathbf{E} = \left(\mathbf{k}\cdot \mathbf{E}\right)\mathbf{k}.
}
V izotropni snovi je izraz na desni strani enačbe enak 0 in velja $k = k_0\sqrt{\varepsilon} = k_0 n$. V anizotropnih
snoveh pa je ta izraz na splošno različen od nič. Tudi v anizotropnih snoveh lomni količnik
definiramo na podoben način $\mathbf{k}= k_0 n \mathbf{s}$, pri čemer je $\mathbf{s}$ smerni vektor
valovnih front.

Ta vektorska enačba predstavlja tri skalarne enačbe za tri smeri. Za nadaljni izračun si izberemo,
da so to smeri $x$, $y$ in $z$ v koordinatnem sistemu, v katerem je $\underline{\varepsilon}$ diagonalen.
V izbranem koordinatnem sistemu tenzor dielektričnosti zapišemo kot:
\beq
\underline{\varepsilon} = 
\left[\begin{array}{ccc}
\varepsilon_{xx} & 0 & 0\\
0& \varepsilon_{yy}& 0\\
0& 0 & \varepsilon_{zz}\\
\end{array}\right]\!\!.
\label{eq:10_018}
\eeq
Enačbo~(\ref{eq:10_017}) zapišemo po komponentah in dobimo sistem treh enačb:
\begin{align}
\left(\mathbf{k}\cdot \mathbf{E}\right) k_x &= \left( k^2 -k_0^2 \varepsilon_{xx}\right) E_x \\
\left(\mathbf{k}\cdot \mathbf{E}\right) k_y &= \left( k^2 -k_0^2 \varepsilon_{yy}\right) E_y \\
\left(\mathbf{k}\cdot \mathbf{E}\right) k_y &= \left( k^2 -k_0^2 \varepsilon_{zz}\right) E_z.
\label{eq:10_019}
\end{align}
Gornje enačbe množimo prvo s $k_x$, drugo s $k_y$ in tretjo s $k_z$ ter dobimo:
\begin{align}
\left(\mathbf{k}\cdot \mathbf{E}\right) 
\frac{k_x^2}{k^2 - k_0^2 \varepsilon_{xx}} &= E_x k_x \\
\left(\mathbf{k}\cdot \mathbf{E}\right) 
\frac{k_y^2}{k^2 - k_0^2 \varepsilon_{yy}} &= E_y k_y \\
\left(\mathbf{k}\cdot \mathbf{E}\right) 
\frac{k_z^2}{k^2 - k_0^2 \varepsilon_{zz}} &= E_z k_z. \\
\label{eq:10_020}
\end{align}
Enačbe seštejemo:
\beq
\left(\mathbf{k}\cdot \mathbf{E}\right)  
\left(
\sum_{j=1}^3 \frac{k_j^2}{k_0^2(n^2 - \varepsilon_{jj})}
\right) = \left(\mathbf{k}\cdot \mathbf{E}\right).
\label{eq:10_021}
\eeq
Ker vemo, da ima v anizotropni snovi optično polje tudi
longitudinalno komponento, velja 
$\left(\mathbf{k}\cdot \mathbf{E}\right) \neq 0$, zato lahko enačno
na obeh straneh enačbe delimo s tem skalarnim produktom. Zapišemo 
še smerni vektor valovnih front po komponentah $\mathbf{s} = (s_x, s_y, s_z)$
in dobimo:
\boxeq{eq:10_022}{
\sum_{j=1}^3 \frac{s_j^2}{(n^2 - \varepsilon_{jj})} = \frac{1}{n^2}.
}
Zapisano enačbo imenujemo Fresnelova enačba za izračun lomnega 
količnika v anizotropni snovi.

Kako uporabimo Fresnelovo enačbo? V snovi izberemo smer širjenja
valovnih front. Ustrezni smerni vektor označimo s  $\mathbf{s}$
 in ga zapišemo v koordinatnem sistemu, v katerem je tenzor
 dielektričnosti diagonalen. Komponente vektorja $\mathbf{s}$ vstavimo
 v Fresnelovo enačbo. Dobimo enačbo četrtega reda za $n$ oziroma
 kvadratno enačbo za $n^2$. Obstajata dve različni pozitivni rešitvi
 za n, ki predstavljata dve možni vrednosti lomnega količnika za
 izbrano smer $\mathbf{s}$. Vsaki izmed njiju ustreza drugačna
 polarizacija elektromagnetnega valovanja oziroma druga smer vektorja
 $\mathbf{D}$. 
 Rešitvi Fresnelove enabe pri danem $\mathbf{s}$ imenujemo lastna lomna
 količnika in ju označimo z $n_1$ in $n_2$. Ustrezna vektorja $\mathbf{D}_{01}$
 in $\mathbf{D}_{02}$ imenujemo lastni polarizaciji. Pokazali bomo, 
 da za ti dve lastni polarizaciji velja $\mathbf{D}_{01} \perp \mathbf{D}_{02}$. 

Dokaz:
Izhajamo iz enačbe za jakost električnega polja:
\beq
k^2 \mathbf{E} - k_0\underline{\varepsilon}\mathbf{E} = \left(\mathbf{k}\cdot\mathbf{E}\right)\mathbf{k}.
\eeq
Upoštevamo zvezo $\mathbf{D}= \varepsilon_0 
\underline{\varepsilon} \mathbf{E}$ in dobimo:
\beq
k^2 \mathbf{E}-\frac{k^2 \mathbf{D}}{\varepsilon_0 n^2} = 
\left(\mathbf{k}\cdot\mathbf{E}\right)\mathbf{k}
\eeq
oziroma
\beq
\mathbf{E} - \frac{\mathbf{D}}{\varepsilon_0 n^2} = 
\frac{\left(\mathbf{k}\cdot\mathbf{E}\right)\mathbf{k}}{k^2} = \mathbf{E}_\parallel.
\eeq
Pri tem smo desno stran enačbe izenačili s komponento jakosti električnega polja
v smeri valovnega vektorja $E_\parallel$. Ker celotno jakost električnega
polja zapišemo kot vsoto dveh komponent, ene vzporedne z valovnim
vektorjem $E_\parallel$, in ene, pravokotne nanj $E_\perp$:
\beq
\mathbf{E} = \mathbf{E}_\perp + \mathbf{E}_\parallel,
\eeq
sklepamo, da velja:
\beq
\mathbf{E}_\perp = \frac{\mathbf{D}}{\varepsilon_0 n^2}.
\eeq
Od tod sledi:
\beq
\mathbf{D} = n^2 \varepsilon \mathbf{E}_\perp
\eeq
oziroma za dve različni lastni vrednosti:
\beq
\mathbf{D}_{1} = n_1^2 \varepsilon \mathbf{E}_{\perp 1} \qquad \mathrm{in} \qquad
\mathbf{D}_{2} = n_2^2 \varepsilon \mathbf{E}_{\perp 2}. 
\eeq
Potem pogledamo izraza:
\beq
\mathbf{E}_1 \cdot \mathbf{D}_2 = \varepsilon_0 \sum_{ij} E_{1i}\varepsilon_{ij}E_{2j}
\eeq
in 
\beq
\mathbf{E}_2 \cdot \mathbf{D}_1 = \varepsilon_0 \sum_{ij} E_{2i}\varepsilon_{ij}E_{1j} = 
\varepsilon_0 \sum_{ij} E_{1j}\varepsilon_{ij}E_{2i} = 
\varepsilon_0 \sum_{ij} E_{1i}\varepsilon_{ji}E_{2j} = 
\varepsilon_0 \sum_{ij} E_{1i}\varepsilon_{ij}E_{2j},
\eeq
pri čemer smo upoštevali tudi simetrijo tenzorja dielektričnosti $\varepsilon_{ij} = 
\varepsilon_{ji}$. 

Od tod sledi:
\beq
\mathbf{E}_1 \cdot \mathbf{D}_2 - \mathbf{E}_2 \cdot \mathbf{D}_1 = 0.
\eeq
Vemo, da je sta vektorja $\mathbf{D}$ pravokotna na smer valovnega vektorja $\mathbf{k}$,
zato je $\mathbf{E}_\parallel \cdot \mathbf{D} = 0$. Potem velja:
\beq
\mathbf{E}_{1\perp} \cdot \mathbf{D}_2 - \mathbf{E}_{2\perp} \cdot \mathbf{D}_1 =
\frac{\mathbf{D}_1}{\varepsilon_0 n_1^2}\cdot \mathbf{D}_2 - 
\frac{\mathbf{D}_2}{\varepsilon_0 n_2^2}\cdot \mathbf{D}_1 = 0,
\eeq
pri čemer smo upoštevali zvezo... Enačbo preoblikujemo in dobimo:
\beq
\frac{\mathbf{D}_1 \cdot \mathbf{D}_2}{\varepsilon_0}\left( \frac{1}{n_1^2} - 
\frac{1}{n_2^2}\right) = 0.
\eeq
Če sta lomna količnika različna, je člen v oklepaju različen od nič in mora 
biti za izpolnitev enačbe biti enak nič skalarni produkt gostot električnega
polja. S tem zaključimo dokaz, da sta lastna vektorja $\mathbf{D}_1$ in 
$\mathbf{D}_2$ med seboj pravokotna. 

V primeru, da sta lomna količnika $n_1$ in $n_2$ enaka, sta smeri gostote
električnega polja poljubni saj je enačba... vedno izpolnjena. Navadno 
izberemo ortogonalni smeri.

Vidimo tudi, da so vektorji $\mathbf{D}_1$ in $\mathbf{D}_2$ ter razmerje med njima
realni. To pomeni, da sta lastni polarizaciji linearni. 
Konec dokaza.

V optično anizotropni snovi se lahko vzdolž izbrane smeri $\mathbf{s}$ 
širita dve linearno polarizirani lastni valovanji, ki sta med seboj pravokotni. 
Vsako od obeh valovanj ima svojo fazno hitrost $c_1= c_0/n_1$ in $c_2 = c_0/n_2$,
ki ju določata lastna lomna količnika.

Pri določanju smeri lastnih vektorjev $\mathbf{D}_1$ in $\mathbf{D}_2$ 
ter ustreznih lomnih količnikov si lahko pomagamo z grafičnim pripomočkom, imenovanim
optična indikatrisa oziroma indeksni elipsoid. 

\subsection*{Optična indikatrisa}
Optična indikatrisa oziroma indeksni elipsoid je povezan z grafično predstavitvijo
tenzorja $\underline{\varepsilon}$. Kot vemo, lahko povprečno energijo oziroma
energijsko gostoto elektromagnetnega valovanja v izotropni snovi zapišemo kot:
\beq
\langle w \rangle = \frac{1}{2} \varepsilon \varepsilon_0 E_0^2 = \frac{1}{2} 
\left(\mathbf{D}_0 \cdot \mathbf{E}_0\right). 
\eeq
V anizotropni snovi je izraz podoben, pri čemer moramo seveda upoštevati
tenzorsko naravo dielektričnosti:
\beq
\langle w \rangle = \frac{1}{2} \left(\mathbf{D}_0 \cdot \mathbf{E}_0\right) = 
\frac{1}{2}\left(\mathbf{D}_0 \cdot \underline{\varepsilon}^{-1} \cdot \mathbf{D}_0
\right) \frac{1}{\varepsilon_0}.
\eeq
V lastnem koordinatnem sistemu izraz izpišemo:
\beq
2 \varepsilon_0 \langle w \rangle = \frac{D_{x}}{\varepsilon_{xx}} + 
\frac{D_{y}}{\varepsilon_{y}} + \frac{D_{z}}{\varepsilon_{zz}}.
\eeq
Če vpeljemo nove normirane koordinate:
$\mathbf{r} = \frac{\mathbf{D}}{\sqrt{2 \varepsilon_0 \langle w\rangle}}$, 
dobimo enačbo elipsoida:
\boxeq{eq:elipsoid}{
\frac{x^2}{\varepsilon_{xx}} + \frac{y^2}{\varepsilon_{yy}} + 
\frac{z^2}{\varepsilon_{zz}} = 1. 
}
To je enačba elipsoida, ki opisuje ploskve konstantne vrednosti $\langle w \rangle$ 
v anizotropni snovi z lastnimi osmi $x$, $y$ in $z$. 

Ko enkrat poznamo elipsoid, ki predstavlja optično indikatriso, 
lahko določimo smer lastnih polarizacij in velikosti lastnih 
lomnih količnikov. Navodilo, kako to naredimo:
 
V prostoru indikatrise izberemo smerni vektor $\mathbf{s}$. Nato 
narišemo ploskev, ki je pravokotna nanj in gre skozi središče elipsoida. 
Presečišče elipsoida in navedene ploskve je elipsa. Njeni glavni osi
predstavljata smeri lastnih polarizacij $\mathbf{D}_{01}$ in $\mathbf{D}_{02}$,
dolžini ustreznih polosi pa ustrezata lomnima količnikoma $n_1$ in $n_2$. 

S tem receptom lahko nazorno in hitro ugotovimo, kakšno obnašanje
vpadnega elektromagnetnega valovanja lahko pričakujemo.  Če navedeni
recept matematično formuliramo, pravimo, da iščemo vezane ekstreme funkcije $r^2$
pri pogoju $\mathbf{s}\cdot \mathbf{r}$ = 0 in enačbi... Prvi pogoj omeji 
rešitve na ravnino, pravokotno na smerni vektor $\mathbf{s}$, drugi na elipsoid, 
iščemo pa največjo in najmanjšo 
oddaljenost od izhodišča, ki podata glavni polosi elipse. Gre torej za
iskanje ekstremov funkcionala $F(x,y,z)$:
\beq
F(x,y,z) = (x^2+y^2+z^2) + 2 \lambda_1 (xs_x+ys_y+zs_z) + \lambda_2
\left(\frac{x^2}{\varepsilon_{xx}}+ \frac{y^2}{\varepsilon_{yy}}+
\frac{z^2}{\varepsilon_{zz}} \right),
\eeq
pri čemer sta $\lambda_1$ in $\lambda_2$ Langrangeeva multiplikatorja.

Dokaz, da navedeni recept deluje. Ekstrem dobimo, ko je $\nabla F = 0$.
To zapišemo po komponentah:
\begin{align}
\frac{\partial F}{\partial x} &= 0 \qquad \Longrightarrow \qquad x + \lambda_1 s_x + \lambda_2 x/\varepsilon_{xx}=0 \\
\frac{\partial F}{\partial y} &= 0 \qquad \Longrightarrow \qquad y + \lambda_1 s_y + \lambda_2 y/\varepsilon_{yy}=0 \\
\frac{\partial F}{\partial z} &= 0 \qquad \Longrightarrow \qquad z + \lambda_1 s_z + \lambda_2 z/\varepsilon_{zz}=0
\end{align}

Prvo enačbo množimo z $x$, drugo z $y$ in tretjo z $z$ ter jih seštejemo v:
\beq
\left(x^2+y^2+z^2 \right) + \lambda_1 \left(s_xx+s_yy+s_zz \right) + \lambda_2 
\left( \frac{x^2}{\varepsilon_{xx}} + \frac{y^2}{\varepsilon_{yy}} + 
\frac{z^2}{\varepsilon_{zz}} \right) = 0.
\eeq
Ker je vektor $\mathbf{r} = (x,y,z)$ pravokoten na vektor $\mathbf{s}$, je drugi člen enak nič, tretji 
pa je po enačbi ... enak 1. Sledi: 
\beq
\lambda_2 = -r^2.
\eeq
Potem prvo enačbo pomnožimo s $s_x$, drugo s $s_y$ in tretjo s $s_z$ in seštejemo v:
\beq
\left(s_xx+s_yy+s_zz \right) + \lambda_1 \left(s_x^2+s_y^2+s_z^2 \right) + \lambda_2 
\left( \frac{xs_x}{\varepsilon_{xx}} + \frac{ys_y}{\varepsilon_{yy}} + 
\frac{zs_z}{\varepsilon_{zz}} \right) = 0.
\eeq
Od tod sledi:
\beq
\lambda_1 = -\lambda_2 \left( \frac{xs_x}{\varepsilon_{xx}} + \frac{ys_y}{\varepsilon_{yy}} + 
\frac{zs_z}{\varepsilon_{zz}} \right) = r^2 \left( \frac{xs_x}{\varepsilon_{xx}} + \frac{ys_y}{\varepsilon_{yy}} + 
\frac{zs_z}{\varepsilon_{zz}} \right).
\eeq
Vstavimo vrednosti multiplikatorjev v sistem enačb in dobimo:
\begin{align}
x + r^2 \left( \frac{xs_x}{\varepsilon_{xx}} + \frac{ys_y}{\varepsilon_{yy}} + 
\frac{zs_z}{\varepsilon_{zz}} \right) s_x -r^2 x/\varepsilon_{xx}=0 \\
y + r^2 \left( \frac{xs_x}{\varepsilon_{xx}} + \frac{ys_y}{\varepsilon_{yy}} + 
\frac{zs_z}{\varepsilon_{zz}} \right)s_y -r^2 y/\varepsilon_{yy}=0 \\
z + r^2 \left( \frac{xs_x}{\varepsilon_{xx}} + \frac{ys_y}{\varepsilon_{yy}} + 
\frac{zs_z}{\varepsilon_{zz}} \right) s_z -r^2 z/\varepsilon_{zz}=0
\end{align}
Enačbe preoblikujemo:
\begin{align}
x (1-r^2/\varepsilon_{xx})+ r^2 \left( \frac{xs_x}{\varepsilon_{xx}} + \frac{ys_y}{\varepsilon_{yy}} + 
\frac{zs_z}{\varepsilon_{zz}} \right) s_x=0 \\
y (1-r^2/\varepsilon_{yy}) + r^2 \left( \frac{xs_x}{\varepsilon_{xx}} + \frac{ys_y}{\varepsilon_{yy}} + 
\frac{zs_z}{\varepsilon_{zz}} \right)s_y=0 \\
z (1-r^2/\varepsilon_{zz})+ r^2 \left( \frac{xs_x}{\varepsilon_{xx}} + \frac{ys_y}{\varepsilon_{yy}} + 
\frac{zs_z}{\varepsilon_{zz}} \right) s_z=0.
\end{align}
Označimo komponente vektorja $\mathbf{r}= (x,y,z)$ z indeksom $r_i$ in dobimo:
\beq
r_i \left(\frac{r^2}{\varepsilon_{ii}} -1\right) =  r^2\left( \sum_{j}\frac{r_j s_j}{\varepsilon_{jj}}\right) s_i. 
\eeq
Dobljeni sistem enačb je v vektorski obliki povsem enak sistemu enačb za $\mathbf{D}$, ki ga dobimo
iz Maxwellovih enačb. Izhajamo iz enačbe (\ref{eq:10_017}):
\beq
\left(\mathbf{k}\cdot \mathbf{E}\right)\mathbf{k} = k^2 \mathbf{E}- k_0\underline{\varepsilon}\mathbf{E}
\eeq
in z upoštevanjem
\beq
\mathbf{E} = \frac{1}{\varepsilon_0}\underline{\varepsilon}^{-1} \mathbf{D}
\eeq
dobimo:
\beq
k^2 \left(\mathbf{s}\frac{1}{\varepsilon_0}\cdot \underline{\varepsilon}^{-1}\mathbf{D}\right) \mathbf{s} = k^2 \frac{1}{\varepsilon_0}\underline{\varepsilon}^{-1} \mathbf{D} - \frac{k^2}{\varepsilon_0 n^2}\mathbf{D}.
\eeq
Sledi:
\beq
\left(\mathbf{s} \cdot \underline{\varepsilon}^{-1}\mathbf{D}\right) \mathbf{s} =
\underline{\varepsilon}^{-1} \mathbf{D} - \frac{\mathbf{D}}{n^2} = \frac{1}{n^2}\left( n^2 
\underline{\varepsilon}^{-1} -1 \right) \mathbf{D}.
\eeq
Upoštevamo, da je tenzor $\varepsilon$ diagonalen in enačbo prepišemo v obliko po komponentah:
\beq
D_i \left(\frac{n^2}{\varepsilon_{ii}}-1 \right) = n^2 \left(\sum_j\frac{s_j D_j}{\varepsilon_{jj}}\right) s_i.
\eeq
Zapisana enačba je po obliki povsem enaka kot enačba .., če le vektor $\mathbf{r}$ zamenjamo z $\mathbf{D}$
in $|\mathbf{r}|$ z $n$. Rešitev geometrijske konstrukcije torej res ustreza smerem lastnih vektorjev gostot
električnega polja in velikosti polosi lomnima količnikoma.  

\section{Ploskev valovnega vektorja}
Kljub temu da je indikatrisa zelo nazorna, pa z njo težko kaj konkretno izračunamo. Za bolj konkretni 
račun se direktno lotimo reševanja Maxwellovih enačb oziroma enačbe~(\ref{eq:10_017}:
\beq
\left(\mathbf{k}\cdot \mathbf{E}\right)\mathbf{k} = 
k^2 \mathbf{E}- k_0\underline{\varepsilon}\mathbf{E}.
\eeq
To je sistem treh enačb za 3 komponente vektorja $\mathbf{E}$, ki bo imel rešitve le, če bo 
determinanta ustrezne matrike sistema enaka nič. Enačbe tudi tokrat zapišemo v lastnem sistemu
tenzorja $\varepsilon$, tako da je ta diagonalen.

Za komponento v smeri $x$ dobimo:
\beq
(k_x E_x + k_yE_y+k_zE_z) k_x = k^2E_x-k_0^2\varepsilon_{xx}E_x.
\eeq
Izraz preoblikujemo:
\beq
k^2 E_x -k_x^2 E_x - k_0^2\varepsilon_{xx}E_x  - k_xk_yE_y - k_xk_zE_z = 0.
\eeq
Z upoštevanjem zveze $k^2 = k_x^2+k_y^2+k_z^2$ se prva enačba prepiše v:
\beq
(k_y^2+k_z^2 - k_0^2\varepsilon_{xx})E_x  - k_xk_yE_y - k_xk_zE_z = 0.
\eeq
Podobno naredimo še za preostali koordinati in zapišemo enačbi:
\beq
- k_xk_yE_x + (k_x^2+k_z^2 - k_0^2\varepsilon_{yy})E_y  - k_yk_zE_z = 0
\eeq
ter 
\beq
- k_xk_zE_x - k_yk_zE_y + (k_x^2+k_y^2  - k_0^2\varepsilon_{zz})E_z   = 0
\eeq
Enačbe združimo v matriko $M$:
\beq
M\mathbf{E}=
\left[\begin{array}{ccc}
k_y^2+k_z^2 - k_0^2\varepsilon_{xx} &  - k_xk_y & - k_xk_z\\
- k_xk_y & k_x^2+k_z^2 - k_0^2\varepsilon_{yy} &- k_yk_z\\
- k_xk_z & - k_yk_z & k_x^2+k_y^2  - k_0^2\varepsilon_{zz}\\
\end{array}\right] \cdot
\left[\begin{array}{c}
E_x \\
E_y \\
E_z
\end{array}\right]=0.
\eeq
Da je sistem enolično rešljiv, mora biti $\det M=0$. Determinante niti ne bomo izpisali
za splošen primer, saj je razmeroma zapletena. Poglejmo pa primer, denimo, ko je 
$\mathbf{k}= (k_x,k_y, 0)$





\section{Ploskev fazne in žarkovne hitrosti}
\section{Optično enoosne snovi}
\section{Optične komponente na osnovi dvolomnih snovi}
\section{Konoskopija}
\section{Inducirana anizotropija}
