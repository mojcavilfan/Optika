%\chapterimage{Geometrijska.jpg} % Chapter heading image

\chapter{Uklon}
Kadar svetloba vpada na oviro, katere velikost je primerljiva z valovno 
dolžino svetlobe, ali se širi skozi podobno veliko odprtino v zaslonu, 
opazimo enega najbolj značilnih pojavov valovne optike -- uklon.
V geometrijski optiki se svetloba širi po ravnih črtah, ki po 
najkrajši poti povezujejo začetno in končno točko. V tej sliki nastane
za oviro ostra senca, v katero se svetloba ne širi. Po uklonski teoriji
pa svetloba seže tudi v območje sence. Poleg tega, da se valovanje širi
v območje geometrijske sence za ovirami, se v njem pojavijo tudi ojačitve
in oslabitve oziroma pri svetlobi svetla in temna območja. Ta pojav
je posledica interference, ki jo bomo podrobneje obravnavali v
naslednjem poglavju. 
\section{Uklonski integral}
Pri obravnavi uklona zanemarimo vektorsko naravo polja oziroma
polarizacijo in jakost električneva polja obravnavamo kot skalarno količino
$E(\mathrm{r},t)$.

Naj ravno sinusno potujoče valovanje vpada na zaslon, v katerem je odprtina.
Tak zaslon imenujemo objektni zaslon. Zanima nas uklonska slika, ki nastane
za odprtino na tako imenovanem opazovalnem zaslonu (glej sliko~\ref{fig:05_shema}).
\begin{figure}[ht]
\centering
\def\svgwidth{120truemm} 
%\input{slike/05_shema.pdf_tex}
\caption{SLIKA}
\label{fig:05_shema}
\end{figure}

Iščemo vrednost jakosti elekktričnega polja v točki $P_0$ na opazovalnem zaslonu, 
do katere vodi vektor $\mathrm{r}_0$, v odvisnosti of vrednosti polja v točki $P$
na objektnem zaslonu, do katerega vodi vektor $\mathrm{r}$. 

Po Huygensovem načelu iz točke $P$ izhaja krogelno valovanje, katerega
amplitudo določa polje, ki vpada na objektni zaslon. Če je vpadno valovanje
ravno valovanje, velja $E(r) = E_0$ za vsak $\mathbf{r}$. Za nastanek slike
v točki na opazovalnem zaslonu moramo sešteti prispevke sekundarnih
polj po vseh točkah objektnega zaslona. Uklonsko sliko v točki $P_0$ 
tako z enačbo zapišemo kot:
\boxeq{eq:05_01}{
E_{P0} = E (\mathbf{r}_0) = \int_A\!\!\int \frac{i}{\lambda} E_0 \left( \frac{e^{ikR(\mathbf{r})}}{R(\mathbf{r})}\right) dS.
}
Pri tem integriramo po celotni površini odprtine $A$, del v oklepaju predstavlja
krajevni del krogelnih valov, $\lambda$ pa je valovna dolžina vpadne svetlobe
\begin{remark}
Prvotno je uklonski integral zapisal Fresnel, tako da je namesto predfaktorja 
$i/\lambda$ zapisal eksperimentalno določeno konstanto. Predfaktor v zapisani
obliki je izpeljal Kirchhoff. Popolnejši zapis je oblike $i \chi(\mathbf{r})/\lambda$,
pri čemer $\chi$ predstavlja smerni faktor. Kadar je značilna razsežnost odprtine v 
objektnem zaslonu $d$ bistveno manjša od oddaljenosti do zaslona, 
je $\chi \approx 1$.
\end{remark}
Namesto integracije sekundarnega polja samo na območju odprtine $A$, lahko vpeljemo
tako imenovano aperturno funkcijo $f(\mathbf{r})$. Njena vrednost je na mestu odprtine enaka $1$,
sicer je enaka $0$. Zapis z aperturno funkcijo omogoči integracijo po celotni dvodimenzionalni
ploskvi, ki opisuje objektni zaslon. Zapis je uporaben tudi za splošen primer, ko je zaslon
delno prozoren in delno prepušča svetlobo. Polje v ravnini objektnega zaslona potem zapišemo:
\beq
E(\mathbf{r}) = f(\mathbf{r})E_0.
\label{eq:05_02}
\eeq
Določimo zdaj koordinatni sistem na objektni in opazovani ravnini. Objektno ravnino
opišemo kot ravnino s koordinatama $x$ in $y$, opazovalno ravnino pa kot ravnino
s koordinatama $\xi$ in $\eta$. Izhodišči obeh koordinatnih sistemov naj ležita 
na skupni osi $z$, ki jo imenujemo optična os sistema (glej sliko~\ref{fig:05_koordinate}).
\begin{figure}[ht]
\centering
\def\svgwidth{120truemm} 
%\input{slike/05_koordinate.pdf_tex}
\caption{SLIKA}
\label{fig:05_koordinate}
\end{figure}

Točko $P$ na objektnem zaslonu potem opišemo s krajevnim vektorjem:
\beq
P: \mathbf{r} = (x,y,0),
\label{eq:05_03}
\eeq
točko $P_0$ na opazovalnem zaslonu pa s krajevnim vektorjem:
\beq
P_0: \mathbf{r}_0 = (\xi,\eta,z_0).
\label{eq:05_04}
\eeq
Uporabimo uklonski integral (enačba~\ref{eq:05_01}) in zapišemo:
\beq
E(\xi, \eta, z_0) = \frac{i}{\lambda} \int_{-\infty}^{\infty}
\int_{-\infty}^{\infty} f(x,y) E_0 \left(\frac{e^{ikR}}{R}\right) dx dy.
\label{eq:05_05}
\eeq
V izrazu za krogelne valove nastopa razdalja $R$ med točjo $P$ in $P_0$, za
katero velja:
\beq
R = \sqrt{(\xi-x)^2 + (\eta - y)^2 + z_0^2}.
\label{eq:05_06}
\eeq
Glede na to, kako natančno obravnavamo izraz za $R = R(x,y,\xi, \eta, z_0)$ v uklonskem
integralu (enačba~\ref{eq:05_05}), ločimo med različnimi uklonskimi približki. Najbolj 
znana sta Fraunhoferjev in Fresnelov približek, ki ju bomo podrobneje spoznali v nadaljevanju.

\section{Fraunhoferjev uklon}
Izhajamo iz uklonskega integrala (enačba~\ref{eq:05_05}):
\beq
E(\xi, \eta, z_0) = \frac{i}{\lambda} \int_{-\infty}^{\infty}
\int_{-\infty}^{\infty} f(x,y) E_0 \left(\frac{e^{ikR}}{R}\right) dx dy.
\eeq
V Fraunhoferjevem približku uklona, ki ga poenostavljeno imenujemo
Fraunhoferjev uklon, razvijemo razdaljo $R$ med točko v odprtini
in točko na zaslonu do prvega reda. Zapišemo:
\beq
R = \sqrt{(\xi-x)^2 + (\eta - y)^2 + z_0^2} \approx
\sqrt{\xi^2+\eta^2 +z_0^2 - 2\xi x - 2 \eta y + x^2 + y^2}.
\label{eq:05_07}
\eeq
Naj bo:
\beq
R_0 = \xi^2 + \eta^2 + z_0^2.
\label{eq:05_08}
\eeq
Uporabimo razvoj korenske funkcije do prvega reda:
\beq
\sqrt{1+x}\approx 1 + \frac{x}{2} + \dots,
\label{eq:05_09}
\eeq
višje člene v razvoju pa zanemarimo. Razvijemo in upoštevamo le linearne člene:
\beq
R = \sqrt{R_0^2 \left(1 - \frac{2\xi x}{R_0^2} - \frac{2\eta y}{R_0^2} + \frac{x^2+y^2}{R_0^2}
\right)} \approx 
R_0 - \frac{\xi x}{R_0} - \frac{\eta y}{R_0}.
\label{eq:05_10}
\eeq
Člen v uklonskem integralu, ki opisuje krogelne monokromatske valove, z upoštevanjem razvoja
(enačba~\ref{eq:05_10}) zapišemo kot:
\beq
\frac{e^{ikR}}{R} \approx \frac{e^{ikR_0} e^{-ik\xi x /R_0} e^{-ik\eta y/R_0}}{R_0}.
\label{eq:05_11}
\eeq
Ker se člen v imenovalcu spreminja znatno počasneje kot v števcu, lahko odvisnost od koordinat
$x$, $y$, $\xi$ in $\eta$ v imenovalcu zanemarimo. 

Vpeljemo še smerna oziroma uklonska kota:
\beq
\sin \vartheta_\xi \approx \vartheta_\xi = \frac{\xi}{R_0}
\qquad \mathrm{in}
\qquad
\sin \vartheta_\eta \approx \vartheta_\eta = \frac{\eta}{R_0}
\label{eq:05_13}
\eeq
ter z njimi povezani prostorski frekvenci:
\beq
\omega_\xi = k \vartheta_\xi
\qquad \mathrm{in}
\qquad
\omega_\eta = k \vartheta_\eta.
\label{eq:05_14}
\eeq
Potem zapišemo uklonski integral v Fraunhoferjevem približku kot:
\boxeq{eq:Fraunhoferjev}{
E(\omega_\xi,\omega_\eta) = \frac{i}{\lambda} \frac{E_0 e^{ikR_0}}{R_0}
\int_{-\infty}^{\infty}
\int_{-\infty}^{\infty} f(x,y) e^{-i\omega_\xi x} e^{-i \omega_\eta y} dx dy.
}
Pogosto naredimo še približek $R_0 \approx z_0$. 

Iz enačbe~(\ref{eq:Fraunhoferjev}) je razvidno, 
da je uklonska slika nekega objekta v Fraunhoferjevem uklonskem približku
enaka 2D Fourierovi transformiranki aperturne funkcije tega objekta. 


\subsection*{Fresnelovo število}
Poiščimo kriterij, ki bo določil, kdaj lahko uporabimo navedeni približek. Spomnimo se, da smo
v razvoju izraza za $R$ zanemarili kvadratni člen (enačba~\ref{eq:05_10}). Ta člen v izrazu
za sferične valove nastopa v členu $\exp(ik(x^2 + y^2)/2R_0)$ in torej prinaša nek dodatni fazni 
zamik. Da bo res zanemarljiv, mora za vsako vrednost $x$ in $y$ veljati:
\beq
\frac{k(x^2+y^2)}{2R_0} \ll 2 \pi.
\label{eq:05_15}
\eeq
Privzamemo, da je $(x^2+y^2)_\mathrm{max} = d^2$, pri čemer $d$ meri karakteristično razsežnost 
odprtine v objektnem zaslonu. Sledi
\beq
\frac{kd^2}{2R_0}  = \frac{2\pi}{\lambda}\frac{d^2}{2R_0}\ll 2 \pi.
\label{eq:05_16}
\eeq
Vpeljemo Fresnelovo število $F$:
\beq
F  = \frac{d^2}{\lambda R_0}
\label{eq:05_17}
\eeq
in Fraunhoferjev približek je veljaven, kadar
\beq
F \ll 1. 
\label{eq:05_18}
\eeq
Pogoj je ekvivalenten zahtevi, da je 
\beq
R_0 \gg \frac{d^2}{\lambda}.
\label{eq:05_19}
\eeq
Tipične vrednosti $R_0 \approx z_0$  morajo biti dosti večje od dimenzij odprtine, zato 
Fraunhoferjevem približku rečemo tudi približek daljnjega polja. 
 
Preden obravnavamo primere Fraunhoferjevega uklona, si oglejmo še interpretacijo 
približnosti Fraunhoferjevega približka. Če si zamislimo enodimenzionalno režo, 
ki se razteza v smemi osi $x$. 
\begin{figure}[ht]
\centering
\def\svgwidth{120truemm} 
%\input{slike/05_priblizek.pdf_tex}
\caption{SLIKA}
\label{fig:05_priblizek}
\end{figure}

Fazni zamik sekundarnih sferičnih valov, ki izvirajo z območja reže, se spreminja kot
\beq
\Delta \Phi = \frac{k \xi x}{R_0}.
\label{eq:05_20}
\eeq
Če opazujemo vrednost $\Delta \Phi(x)$ valov, ki potujejo v smeri $\vartheta_\xi \approx
\xi/R_0$ glede na os $z$, vidimo, da velja 
\beq
\Delta \Phi = k \varphi_\xi x.
\label{eq:05_21}
\eeq
Zapis ustreza ravnemu valovanju. Za $R_0 \gg d^2/\lambda$ se uklonjeno valovanje manifestira
kot del ravnega valovanja. 

\begin{example}{\bf Veljavnost Fraunhoferjevega približka.}
Naj bo valovna dolžina svetlobe $\lambda = 500~\si{nm}$. Če je tipična 
razsežnost odprtine $d = 1~\si{\micro\metre}$, mora biti za veljavnost 
Fraunhoferjevega približka oddaljenost od zaslona 
$R\gg 2~\si{\micro\metre}$. Pri razsežnosti
$d = 10~\si{\micro\metre}$ mora veljati $R\gg 200~\si{\micro\metre}$, 
pri $d = 100~\si{\micro\metre}$ mora biti $R\gg 2~\si{\centi\metre}$ in 
pri $d = 1~\si{\milli\metre}$ velja Fraunhoferejev približek pri 
$R\gg 2~\si{\metre}$. Pri majhnih odprtinah je torej Fraunhoferjev približek
preprosto izpolniti, že pri milimetrsih odprtinah
pa mora biti zaslon oddaljen vsaj nekaj metrov. 
\end{example}

Za veljavnost Fraunhoferjevega približka mora biti torej zaslon zelo
oddaljen od odprtine, na kateri se svetloba uklanja. Vendar za opazovanje 
Fraunhoferjevega uklona ni nujno, da zaslon postavimo zelo daleč za objektno ravnino.
Namesto tega lahko za objektni zaslon postavimo zbiralno lečo in uklonsko sliko opazujemo
v njeni goriščni ravnini. Zbiralna leča namreč snop vzporednih žarkov zbere v eni 
točki v goriščni ravnini leče. 
\begin{figure}[ht]
\centering
\def\svgwidth{120truemm} 
%\input{slike/05_2DFourier.pdf_tex}
\caption{SLIKA}
\label{fig:05_2DFourier}
\end{figure}
\begin{remark}
Opisani sistem je analogna naprava za izračun 2D-Fourierove transformacije nekega 
delno prozornega objekta ali vzorca. /pattern recognition?/
\end{remark}

\begin{example}{\bf Fraunhoferjev uklon na reži debeline $d$.}
Obravnavajmo uklon svetlobe z valovno dolžino $\lambda$, ki v obliki ravnih valov 
vpada na zaslon z režo debeline $d$. Uklonsko sliko na oddaljenem opazovalnem zaslonu
izračunamo iz enačbe~(\ref{eq:Fraunhoferjev}), pri čemer je funkcija $f(x)$ enaka 1 za
$-d/2<x<d/2$, sicer je 0. Dobimo:
\beq
E(\omega_\xi) = \frac{i}{\lambda} \frac{\tilde{E}_0 e^{ikR_0}}{R_0}
\int_{-d/2}^{d/2} e^{-i\omega_\xi x} dx \approx \frac{i}{\lambda}
\frac{\tilde{E}_0 e^{ikz}}{z}\frac{1}{-i\omega_\xi}\left(e^{-i\omega_\xi d/2}- e^{i\omega_\xi d/2}\right),
\label{eq:05_22}
\eeq
od koder z upoštevanjem izrazov za $\sin(x)$ in $k$ sledi:
\beq
E(\omega_\xi) = \frac{i}{\lambda}
\frac{\tilde{E}_0 e^{ikz}}{z}\frac{2}{\omega_\xi} \sin\left(\omega_\xi d/2\right) = 
\frac{id}{\lambda}
\frac{\tilde{E}_0 e^{ikz}}{z}
\frac{\sin\left(\omega_\xi d/2\right)}{\omega_\xi d/2}.
\label{eq:05_23}
\eeq
Gostota svetlobnega toka uklonjene svetlobe na opazovalnem zaslonu je tako:
\boxeq{eq:uklon1reza}{
j(\vartheta) = j_0 \left(\frac{\sin\left(kd\sin(\vartheta)/2\right)}{kd\sin(\vartheta)/2}\right)^2,
}
pri čemer so $k$ valovno število, $d$ debelina reže in $\vartheta$ uklonski kot. Gostoto
toka smo normirali, tako da je v smeri naravnost (pri $\vartheta = 0$) enaka $j_0$.
\begin{figure}[ht]
\centering
\def\svgwidth{120truemm} 
%\input{slike/05_1Reza.pdf_tex}
\caption{SLIKA}
\label{fig:05_1Reza}
\end{figure}

V smeri naprej je intenziteta uklonjene svetlobe največja, nato z naraščajočim
kotom pojema, dokler ne doseže vrednosti 0. Pri večjih kotih ponovno naraste, vendar 
je njena intenziteta znatno manjša kot v osrednjem vrhu. Uklonske kote, pri 
katerih se pojavijo oslabitve, izračunamo preprosto iz enačbe~(\ref{eq:uklon1reza}), tako da
števec postavimo na nič. Sledi:
\beq
kd \sin(\vartheta)/2 = N\pi \qquad \mathrm{in} \qquad \sin \vartheta = \frac{N\lambda}{d},
\label{eq:05_24}
\eeq
pri čemer je $N$ celo število. Lego in intenziteto vrhov je treba določiti numerično. 
\end{example}

\begin{example}{\bf Fraunhoferjev uklon na sistemu $N$ vzporednih rež.}
Naj bo v objektni ravnini $N$ vzporednih rež debeline $d$, perioda ponavljanja
rež pa naj bo $D$.

Podobno kot smo izračunali uklonsko sliko na eni reži, izračunamo tudi uklonsko sliko
na več vzporedih režah. Prepustnostna funkcija $f$ je tako enaka 1 v vsaki reži, med
režami in zunaj območja rež pa je enaka 0.
\beq
E(\omega_\xi) = \frac{i}{\lambda} \frac{\tilde{E}_0 e^{ikR_0}}{R_0}
\left( \int_{-d/2}^{d/2} e^{-i\omega_\xi x} dx + \int_{D-d/2}^{D+d/2} e^{-i\omega_\xi x} dx +
\int_{2D-d/2}^{2D+d/2} e^{-i\omega_\xi x} dx + ... \right).
\label{eq:05_25}
\eeq
Integrale izračunamo in dobimo:
\beq
E(\omega_\xi) = \frac{i}{-i\omega_\xi \lambda} \frac{\tilde{E}_0 e^{ikR_0}}{R_0}
\left(e^{-i\omega_\xi d/2}- e^{i\omega_\xi d/2} \right) \left(1 + e^{-i\omega_xi D} + 
e^{-i\omega_xi 2 D} + ...\right).
\label{eq:05_26}
\eeq
Sledi:
\beq
E(\omega_\xi) = \frac{id}{\lambda} \frac{\tilde{E}_0 e^{ikR_0}}{R_0}\frac{e^{-iN\omega_\xi D/2}}
{e^{-i\omega_\xi D/2}}
\frac{\sin(\omega_\xi d/2)}{\omega_\xi d/q} 
\frac{\sin(N\omega_\xi D/2)}{\sin(\omega_\xi D/2)}.
\label{eq:05_27}
\eeq
Gostota svetlobnega toka uklonjene svetlobe na opazovalnem zaslonu je tako:
\boxeq{eq:uklonNrez}{
j(\vartheta) = j_0 \left(\frac{\sin\left(kd\sin(\vartheta)/2\right)}{kd\sin(\vartheta)/2}\right)^2
\left(\frac{\sin\left(NkD\sin(\vartheta)/2\right)}{\sin\left(kD\sin(\vartheta)/2\right)}\right)^2.
}
Prvi oklepaj imenujemo strukturni faktor $S(\vartheta)$ in določa ovojnico uklonske slike. Enak je uklonski
sliki za eno samo režo debeline $d$. Drugi oklepaj imenujemo mrežni faktor $M(\vartheta)$ in je odvisen od 
razporeditve rež. 
\begin{figure}[ht]
\centering
\def\svgwidth{120truemm} 
%\input{slike/05_Nrez.pdf_tex}
\caption{SLIKA}
\label{fig:05_Nrez}
\end{figure}

Glavne vrhove v uklonski sliki dobimo, kadar je imenovalec ulomka enak nič:
\beq
kD \sin(\vartheta)/2 = n\pi \qquad \mathrm{in} \qquad \sin \vartheta = \frac{n\lambda}{D},
\label{eq:05_28}
\eeq
pri čemer je $n$ celo število. Med posamičnimi glavnimi vrhovi se pojavijo še manjši
stranski vrhovi. Poiščemo jih med zaporednimi minimumi, ki jih določajo ničle števca
ulomka:
\beq
NkD \sin(\vartheta)/2 = m\pi \qquad \mathrm{in} \qquad \sin \vartheta = \frac{m\lambda}{ND},
\label{eq:05_29}
\eeq
pri čemer je $m$ celo število. Med dvema glavnima vrhoma je tako v splošnem
$N-2$ stranskih vrhov. Pri izračunu amplitude vrhov moramo upoštevati
tudi strukturni faktor, torej ovojnico. 

Prvi primer na sliki~\ref{fig:05_Nrez} prikazuje uklonsko sliko za primer $N = 6$ in 
$D=4d$. Med dvema glavnima vrhoma so štirje stranski vrhovi. Z upoštevanjem amplitude
envelope ugotovimo, da vsak četri (preveri!) glavni vrh izgine zaradi minimuma envelope.
\end{example}

\begin{example}{\bf Fraunhoferjev uklon na okrogli odprtini s polmerom $a$.}
Kot naslednji primer si oglejmo Fraunhoferjev uklon na okrogli odprtini s polmerom $a$. 
Lego točke $P$ v objektni ravnini $xy$ opišemo v polarnih koordinatah $\varrho$ in $\varphi$
kot:
\beq
x = \varrho \cos \varphi \qquad \mathrm{in} \qquad y = \varrho \sin \varphi. 
\label{eq:05_30}
\eeq
Lego točke $P_0$ v opazovalni ravnini $\xi \eta$ zapišemo v polarnih koordinatah 
$q$ in $\phi$:
\beq
\xi = q \cos \phi \qquad \mathrm{in} \qquad \eta = q \sin \phi. 
\label{eq:05_31}
\eeq
Uklonski integral potem zapišemo kot:
\beq
E(q,\phi,z_0) = \frac{i}{\lambda} \frac{E_0e^{ikR_0}}{R_0}\int_0^{2\pi} \int_0^a
e^{-i\omega_\xi x}e^{-i\omega_\eta y}\varrho d\varrho d\varphi.
\label{eq:05_32}
\eeq
Upoštevamo enačbe~(\ref{eq:05_13} in \ref{eq:05_14}) in zapišemo:
\beq
\omega_\xi x + \omega_\eta y= \frac{qk\cos \phi}{R_0} \varrho \cos\phi + \frac{qk\sin \phi}{R_0} \varrho \sin\phi = 
 \frac{qk\sin \phi}{R_0} \cos(\varphi - \phi).
\label{eq:05_33}
\eeq
Zaradi simetrije lahko postavimo $\phi=0$ in zapišemo integral:
\beq
E(q,\phi,z_0) = \frac{i}{\lambda} \frac{E_0e^{ikR_0}}{R_0}\int_0^{2\pi} \int_0^a
e^{-iqk\varrho \cos(\varphi)/R_0)} \varrho d\varrho d\varphi.
\label{eq:05_34}
\eeq
Spomnimo se integralnega zapisa Besslovih funkcij prve vrste:
\beq
J_m(u) = \frac{(-1)^m}{2\pi} \int_0^{2\pi} e^{-iu\cos v }e^{imv}dv
\label{eq:05_35}
\eeq
in rekurzijske zveze:
\beq
\frac{d}{du}\left( u^m J_m(u)\right) = u^m J_{m-1}(u).
\label{eq:05_36}
\eeq
Primerjamo enačbi~(\ref{eq:05_34}) in (\ref{eq:05_35}) in vpeljemo
zveze $v = \varphi$, $m=0$ in $u = qk\varrho/R_0$. Pri računu integrala 
upoštevamo izraza za Besslove funkcije (enačbe~\ref{eq:05_35} in \ref{eq:05_36}) 
in dobimo:
\beq
E = \frac{i}{\lambda}\frac{E_0 e^{ikR_0}}{R_0} \pi a^2 \frac{2 J_1 (kqa/R_0}{kqa/R_0}.
\label{eq:05_37}
\eeq
Če vpeljemo še uklonski kot $\vartheta$ in zapišemo gostoto svetlobnega toka
uklonjene svetlobe:
\boxeq{eq:uklonAiry}{
j = j_0 \left(\frac{2J_1(ka\sin\vartheta}{ka\sin \vartheta}\right)^2.
}
Opisano odvisnost, ki jo imenujemo tudi Airyjev uklonski vzorec po angleškem
matematiku in astronomu Siru Georgeu Biddellu Airyu, si poglejmo podrobneje. 
Na opazovalnem zaslonu nastane rotacijsko simetrična uklonska slika, ki 
jo imenujemo Airyjev disk. 
\begin{figure}[ht]
\centering
\def\svgwidth{120truemm} 
%\input{slike/05_Airy.pdf_tex}
\caption{SLIKA}
\label{fig:05_Airy}
\end{figure}
V smeri naravnost (na sredini Airyjevega diska) 
je gostota svetlobnega toka enaka $j_0$, saj velja:
\beq
\lim_{u \to 0}\frac{J_1(u)}{u} = \frac{1}{2}.
\label{eq:05_39}
\eeq
Z naraščajočim uklonskim kotom intenziteta uklonjene svetlobe pojema, nato
spet naraste in ponovno pojema ... Ničle uklonske slike so določene
s pogojem, da je $J_1 = 0$ in se pojavijo pri vrednostih:
\beq
ka\sin\vartheta  \approx 3,83;~7,02;~10,17;~13,32 ...
\label{eq:05_40}
\eeq
Intenziteta stranskih obročev je razmeroma šibka, zato je najpomembnejši
notranji krog uklonske slike. Polmer notranjega kroga oziroma diska 
izračunamo iz zveze:
\beq
ka q_A/R_0 = \frac{2\pi a q_A}{\lambda R_0} = 3,83
\label{eq:05_41}
\eeq
in dobimo:
\beq
q_A = \frac{3,83}{\pi}\frac{\lambda R_0}{2a} = 1,22 \frac{\lambda R_0}{2a}.
\label{eq:05_42}
\eeq
\end{example}

\section{Ločljivost optičnih naprav}
Pri optičnih napravah pogosto svetloba potuje skozi lečo in 
nato vpade na svetlobni detektor. Leča pri tem deluje
kot okrogla odprtina s polmerom $a$, svetloba pa se na njej
uklanja.

Poglejmo primer zelo oddaljenega točkastega predmeta, ki ga 
preslikamo z lečo z goriščno razdaljo $f$ in premerom $D$. Slika predmeta 
nastane v goriščni ravnini, zato je $R_0=f$. Zaradi uklona potem 
točkasti predmet na detektorju vidimo kot okrogel disk s polmerom
\beq
q_A \approx 1,22 \frac{f\lambda}{D} = 1,22 \lambda \frac{f}{D} = 0,61 \frac{\lambda}{NA},
\label{eq:05_43}
\eeq
pri čemer je numerična apertura $NA$ razmerje med polmerom leče $D/2$
in goriščno razdaljo $f$:
\beq
NA = \frac{D/2}{f}.
\label{eq:05_44}
\eeq
Značilne vrednosti za numerično aperturo so $NA \lesssim 1$.

Če opazujemo dve oddaljeni točkasti telesi, bo v opazovalni ravnini vsako
od njih ustvarilo Airyjev disk. Kadar se diska začneta prekrivati, teles
na sliki ne moremo več ločiti. Ta pojav omejuje ločljivost optičnih naprav
in ga poglejmo podrobneje. 

Kriterij, ki določa, ali dva objekta na sliki še ločimo med seboj, imenujemo
Rayleighev kriterij ločljivosti po angleškemu fiziku Lordu Rayleighu. Kriterij
pravi, da objekta ločimo, če velja:
\beq
\Delta q_A \geq 1,22 \frac{\lambda}{NA} \gtrsim \lambda.
\label{eq:05_45}
\eeq
Najmanjša razlika zornih kotov:
\beq
\Delta \alpha_{\mathrm{min}} = 
\frac{\Delta q_{A\mathrm{min}}}{f} \gtrsim 1,22 \frac{\lambda}{D}.
\label{eq:05_46}
\eeq
Večjo kotno ločljivost optičnih naprav, na primer teleskopov, torej 
dosežemo pri manjših valovnih dolžinah svetlobe in večjih lečah.

\begin{example}{\bf Ločljivost teleskopa.}
S teleskopom, ki ima premer leče $D=5~\si{m}$, opazujemo
Luno, ki je od Zemlje oddaljena $L = 384 000~\si{km}$. Če je 
valovna dolžina svetlobe $500~\si{nm}$, ocenimo velikost podrobnosti
na Luninem površju, ki jih še ločimo.

Iz enačbe~(\ref{eq:05_46}) ocenimo:
\beq
\Delta l = 1,22 \frac{\lambda}{D} L \approx 50~\si{m}.
\label{eq:05_47}
\eeq
\end{example}

Na podoben način izračunamo tudi ločljivost optičnih mikroskopov. Ker
tam pri njih predmet blizu goriščne razdalje, je ločljivost
teh naprav približno enaka valovni dolžini svetlobe $\Delta l \approx \lambda$.

\section{Fresnelov uklon}

\section{*Izpeljava Kirchhoffovega uklonskega integrala}
Poglejmo še matematično izpeljavo uklonskega integrala (enačba~\ref{eq:05_01})
in najprej ponovimo Greenov teorem, ki izhaja iz Gaussovega stavka. Naj bo 
$\mathbf{W}(\mathbf{r})$ poljubno zvezno in zvezno odvedljivo vektorsko
polje. Gaussov stavek pravi, da je intergal po sklenjeni ploskvi enak
integralu divergence vektorskega polja po prostoru:
\beq
\oint_S \mathbf{W} d\mathbf{S} = \int_V \div(\mathbf{W})dV.
\label{eq:05_48}
\eeq
Zapišimo  vektorsko polje $\mathbf{W}$ s skalarnima poljema $\Psi$ in $\Phi$
v obliki:
\beq
\mathbf{W}(\mathbf{r}) = \Psi \nabla \Phi - \Phi \nabla \Psi.
\label{eq:05_49}
\eeq
Potem velja:
\beq
\div \mathbf{W}= \Psi \nabla^2 \Phi - \Phi \nabla^2 \Psi
\label{eq:05_50}
\eeq
in dobimo Greenov teorem:
\beq
\oint_S \left(\Psi \nabla \Phi - \Phi \nabla \Psi\right) d\mathbf{S}  =
\int_V \left( \Psi \nabla^2 \Phi - \Phi \nabla^2 \Psi \right) dV.
\label{eq:05_51}
\eeq

