%\chapterimage{Geometrijska.jpg} % Chapter heading image

\chapter{Uklon}
V tem poglavju bomo spoznali enega najznačilnejših pojavov valovne 
optike -- uklon. Zapisali bomo splošni uklonski integral in obravnavali
dva približka, Fraunhoferjevega in Fresnelovega. Z izračunom 
bomo pojasnili značilne uklonske slike in zapisali uklonsko
omejeno ločljivost optičnih naprav. Na koncu poglavja sledi matematično
zahtevnejša izpeljava Kirchhoffovega uklonskega integrala. 

\section{Uklonski integral}
Svetloba se uklanja, kadar vpada na oviro, katere velikost je primerljiva z valovno  
dolžino svetlobe, ali se širi skozi podobno veliko odprtino v zaslonu. 
V geometrijski optiki se svetloba širi po ravnih črtah, ki po 
najkrajši poti povezujejo začetno in končno točko. V tej sliki za oviro 
nastane ostra senca, v katero se svetloba ne širi. Po uklonski teoriji
svetloba seže deloma tudi v območje sence. Poleg tega, da se valovanje širi
v območje geometrijske sence za ovirami, se v njem pojavijo ojačitve
in oslabitve, ki predstavljajo značilne uklonske slike.

Naj ravno sinusno potujoče valovanje vpada na zaslon, v katerem je majhna odprtina.
Ta zaslon imenujemo objektni zaslon. Iščemo uklonsko sliko, ki nastane
za odprtino na tako imenovanem opazovalnem zaslonu (glej sliko~\ref{fig:05_shema}).
Pri tem  zanemarimo vektorsko naravo polja oziroma
polarizacijo in jakost električnega polja obravnavamo kot skalarno količino.

\begin{figure}[ht]
\centering
\def\svgwidth{90truemm} 
\input{slike/05_skica01.pdf_tex}
\caption{Ravni valovi vpadajo na objektni zaslon, v katerem je odprtina. 
Na opazovalnem zaslonu v točki $P_0$ opazujemo jakost električnega polja v
odvisnosti od vpadnega polja v točki $P$.}
\label{fig:05_shema}
\end{figure}

Naša naloga je poiskati jakost električnega polja v točki $P_0$ na opazovalnem zaslonu, 
do katere vodi vektor $\mathbf{r}_0$, v odvisnosti od vrednosti polja v točki $P$
na objektnem zaslonu, do katerega vodi vektor $\mathbf{r}$.

Po Huygensovem načelu iz vsake točke $P$ v odprtini izhaja sekundarno krogelno valovanje, 
katerega amplitudo določa polje, ki vpada na objektni zaslon. Če je vpadno valovanje
ravni val, velja $E(\mathbf{r}) = E_0$ za vsak $\mathbf{r}$. Za nastanek slike
v točki $P_0$  na opazovalnem zaslonu seštejemo prispevke sekundarnih
polj iz vseh točk odprtine v objektnem zaslonu. 

Uklonsko sliko v točki $P_0$ tako z enačbo zapišemo kot:
\boxeq{eq:05_01}{
E_{P0} = E (\mathbf{r}_0) = \int_A \frac{i}{\lambda} E_0~\frac{e^{ikR(\mathbf{r})}}{R(\mathbf{r})}~ dS,
}
pri čemer integriramo po celotni površini odprtine $A$. Ulomek v integralu predstavlja
krajevni del krogelnih valov, $\lambda$ pa je valovna dolžina vpadne svetlobe. Predfaktor
v obliki $i/\lambda$ bomo izpeljali v razdelku~\ref{chap:Kirchhoff}.
\begin{remark}
Uklonski integral je prvotno zapisal Fresnel, vendar je namesto predfaktorja 
$i/\lambda$ uvedel eksperimentalno določeno konstanto. Predfaktor $i/\lambda$ je 
izpeljal Kirchhoff. Popolnejši zapis predfaktorja je oblike $i \chi(\mathbf{r})/\lambda$,
pri čemer $\chi$ predstavlja smerni faktor. Kadar je značilna razsežnost odprtine v 
objektnem zaslonu $d$ bistveno manjša od oddaljenosti do opazovalnega zaslona, 
je $\chi \approx 1$ in predfaktor se prevede na zapisano obliko. 
\end{remark}

Namesto integracije sekundarnega valovanja po območju odprtine $A$, lahko vpeljemo
tako imenovano aperturno funkcijo $f(\mathbf{r})$. Njena vrednost je na mestu odprtine enaka $1$,
sicer je enaka $0$. Zapis z aperturno funkcijo omogoči integracijo po celotni dvodimenzionalni
ploskvi, ki opisuje objektni zaslon. Zapis je uporaben tudi za splošen primer, ko je zaslon
delno prozoren in le delno prepušča svetlobo. Polje v ravnini objektnega zaslona potem zapišemo kot:
\beq
E(\mathbf{r}) = f(\mathbf{r})E_0.
\label{eq:05_02}
\eeq
Določimo koordinatni sistem na objektni in opazovalni ravnini, ki ju bomo potrebovali
za izračun uklonskega integrala. Objektno ravnino
opišemo kot ravnino s koordinatama $x$ in $y$, vzporedno opazovalno ravnino pa kot ravnino
s koordinatama $\xi$ in $\eta$. Izhodišči obeh koordinatnih sistemov naj ležita 
na skupni osi $z$, ki jo imenujemo optična os sistema (glej sliko~\ref{fig:05_koordinate})
in je pravokotna na obe ravnini.
\begin{figure}[ht]
\centering
\def\svgwidth{120truemm} 
\input{slike/05_koordinate.pdf_tex}
\caption{Optična os sistema (os $z$) povezuje središči obeh ravnin, objektne ($x,y$) in 
opazovalne ($\xi, \eta$). Razdalja med točko v odprtini $P$ in točko na opazovalnem zaslonu $P_0$
je $R$.}
\label{fig:05_koordinate}
\end{figure}

Točko $P$ na objektnem zaslonu opišemo s krajevnim vektorjem:
\beq
P: \mathbf{r} = (x,y,0),
\label{eq:05_03}
\eeq
točko $P_0$ na opazovalnem zaslonu pa s krajevnim vektorjem:
\beq
P_0: \mathbf{r}_0 = (\xi,\eta,z_0).
\label{eq:05_04}
\eeq
Uporabimo uklonski integral (enačba~\ref{eq:05_01}) in zapišemo:
\boxeq{eq:05_05}{
E(\xi, \eta, z_0) = \frac{i}{\lambda} \int_{-\infty}^{\infty}
\int_{-\infty}^{\infty} f(x,y) E_0~\frac{e^{ikR}}{R}~ dx\,dy.
}
V izrazu za krogelne valove nastopa razdalja $R$ med točkama $P$ in $P_0$, za
katero velja:
\beq
R = \sqrt{(\xi-x)^2 + (\eta - y)^2 + z_0^2}.
\label{eq:05_06}
\eeq

Natančna obravnava uklonskega integrala je zelo zahtevna, zato se poslužujemo približkov.
Glede na to, kako natančno obravnavamo izraz za $R = R(x,y,\xi, \eta, z_0)$ v uklonskem
integralu (enačba~\ref{eq:05_05}), ločimo med različnimi uklonskimi približki. Najbolj 
znana sta Fraunhoferjev in Fresnelov približek, ki ju bomo podrobneje spoznali v nadaljevanju.

\section{Fraunhoferjev uklon}
V Fraunhoferjevem približku uklona, ki ga poenostavljeno imenujemo
Fraunhoferjev uklon, razvijemo razdaljo $R$ med točko v odprtini
in točko na zaslonu do prvega reda. Zapišemo:
\beq
R = \sqrt{(\xi-x)^2 + (\eta - y)^2 + z_0^2} \approx
\sqrt{\xi^2+\eta^2 +z_0^2 - 2\xi x - 2 \eta y + x^2 + y^2}.
\label{eq:05_07}
\eeq
Naj bo:
\beq
R_0 = \xi^2 + \eta^2 + z_0^2.
\label{eq:05_08}
\eeq
Uporabimo razvoj korenske funkcije do prvega reda:
\beq
\sqrt{1+x}\approx 1 + \frac{x}{2} + \dots,
\label{eq:05_09}
\eeq
višje člene v razvoju pa zanemarimo. Razvijemo in upoštevamo le linearne člene:
\beq
R = \sqrt{R_0^2 \left(1 - \frac{2\xi x}{R_0^2} - \frac{2\eta y}{R_0^2} + \frac{x^2+y^2}{R_0^2}
\right)} \approx 
R_0 - \frac{\xi x}{R_0} - \frac{\eta y}{R_0}.
\label{eq:05_10}
\eeq
Člen v uklonskem integralu, ki opisuje krogelne monokromatske valove, z upoštevanjem razvoja
(enačba~\ref{eq:05_10}) zapišemo kot:
\beq
\frac{e^{ikR}}{R} \approx \frac{e^{ikR_0} e^{-ik\xi x /R_0} e^{-ik\eta y/R_0}}{R_0}.
\label{eq:05_11}
\eeq
Ker se člen v imenovalcu spreminja znatno počasneje kot v števcu, lahko odvisnost od koordinat
$x$, $y$, $\xi$ in $\eta$ v imenovalcu zanemarimo. 

Vpeljemo še smerna oziroma uklonska kota:
\beq
\sin \vartheta_\xi \approx \vartheta_\xi = \frac{\xi}{R_0}
\qquad \mathrm{in}
\qquad
\sin \vartheta_\eta \approx \vartheta_\eta = \frac{\eta}{R_0}
\label{eq:05_13}
\eeq
ter z njimi povezani prostorski frekvenci:
\beq
\omega_\xi = k \vartheta_\xi
\qquad \mathrm{in}
\qquad
\omega_\eta = k \vartheta_\eta.
\label{eq:05_14}
\eeq
Potem zapišemo uklonski integral v Fraunhoferjevem približku kot:
\boxeq{eq:Fraunhoferjev}{
E(\omega_\xi,\omega_\eta) = \frac{i}{\lambda} \frac{E_0 e^{ikR_0}}{R_0}
\int_{-\infty}^{\infty}
\int_{-\infty}^{\infty} f(x,y) e^{-i\omega_\xi x} e^{-i \omega_\eta y} dx dy.
}
Pogosto naredimo še približek $R_0 \approx z_0$. 

Iz enačbe~(\ref{eq:Fraunhoferjev}) je razvidno, 
da je uklonska slika nekega objekta v Fraunhoferjevem uklonskem približku
enaka 2D Fourierovi transformiranki aperturne funkcije tega objekta. 


\subsection*{Fresnelovo število}
Poiščimo kriterij, ki bo določil, kdaj lahko uporabimo navedeni približek. Spomnimo se, da smo
v razvoju izraza za $R$ zanemarili kvadratni člen (enačba~\ref{eq:05_10}). Ta člen v izrazu
za sferične valove nastopa v členu $\exp(ik(x^2 + y^2)/2R_0)$ in torej prinaša nek dodatni fazni 
zamik. Da bo res zanemarljiv, mora za vsako vrednost $x$ in $y$ veljati:
\beq
\frac{k(x^2+y^2)}{2R_0} \ll 2 \pi.
\label{eq:05_15}
\eeq
Privzamemo, da je $(x^2+y^2)_\mathrm{max} = d^2$, pri čemer $d$ meri karakteristično razsežnost 
odprtine v objektnem zaslonu. Sledi
\beq
\frac{kd^2}{2R_0}  = \frac{2\pi}{\lambda}\frac{d^2}{2R_0}\ll 2 \pi.
\label{eq:05_16}
\eeq
Vpeljemo Fresnelovo število $F$:
\beq
F  = \frac{d^2}{\lambda R_0}
\label{eq:05_17}
\eeq
in Fraunhoferjev približek je veljaven, kadar
\beq
F \ll 1. 
\label{eq:05_18}
\eeq
Pogoj je ekvivalenten zahtevi, da je 
\beq
R_0 \gg \frac{d^2}{\lambda}.
\label{eq:05_19}
\eeq
Tipične vrednosti $R_0 \approx z_0$  morajo biti dosti večje od dimenzij odprtine, zato 
Fraunhoferjevem približku rečemo tudi približek daljnjega polja. 
 
Preden obravnavamo primere Fraunhoferjevega uklona, si oglejmo še interpretacijo 
približnosti Fraunhoferjevega približka. Če si zamislimo enodimenzionalno režo, 
ki se razteza v smemi osi $x$. 
\begin{figure}[ht]
\centering
\def\svgwidth{120truemm} 
%\input{slike/05_priblizek.pdf_tex}
\caption{SLIKA}
\label{fig:05_priblizek}
\end{figure}

Fazni zamik sekundarnih sferičnih valov, ki izvirajo z območja reže, se spreminja kot
\beq
\Delta \Phi = \frac{k \xi x}{R_0}.
\label{eq:05_20}
\eeq
Če opazujemo vrednost $\Delta \Phi(x)$ valov, ki potujejo v smeri $\vartheta_\xi \approx
\xi/R_0$ glede na os $z$, vidimo, da velja 
\beq
\Delta \Phi = k \varphi_\xi x.
\label{eq:05_21}
\eeq
Zapis ustreza ravnemu valovanju. Za $R_0 \gg d^2/\lambda$ se uklonjeno valovanje manifestira
kot del ravnega valovanja. 

\begin{example}{\bf Veljavnost Fraunhoferjevega približka.}
Naj bo valovna dolžina svetlobe $\lambda = 500~\si{nm}$. Če je tipična 
razsežnost odprtine $d = 1~\si{\micro\metre}$, mora biti za veljavnost 
Fraunhoferjevega približka oddaljenost od zaslona 
$R\gg 2~\si{\micro\metre}$. Pri razsežnosti
$d = 10~\si{\micro\metre}$ mora veljati $R\gg 200~\si{\micro\metre}$, 
pri $d = 100~\si{\micro\metre}$ mora biti $R\gg 2~\si{\centi\metre}$ in 
pri $d = 1~\si{\milli\metre}$ velja Fraunhoferejev približek pri 
$R\gg 2~\si{\metre}$. Pri majhnih odprtinah je torej Fraunhoferjev približek
preprosto izpolniti, že pri milimetrsih odprtinah
pa mora biti zaslon oddaljen vsaj nekaj metrov. 
\end{example}

Za veljavnost Fraunhoferjevega približka mora biti torej zaslon zelo
oddaljen od odprtine, na kateri se svetloba uklanja. Vendar za opazovanje 
Fraunhoferjevega uklona ni nujno, da zaslon postavimo zelo daleč za objektno ravnino.
Namesto tega lahko za objektni zaslon postavimo zbiralno lečo in uklonsko sliko opazujemo
v njeni goriščni ravnini. Zbiralna leča namreč snop vzporednih žarkov zbere v eni 
točki v goriščni ravnini leče. 
\begin{figure}[ht]
\centering
\def\svgwidth{120truemm} 
%\input{slike/05_2DFourier.pdf_tex}
\caption{SLIKA}
\label{fig:05_2DFourier}
\end{figure}
Pri računu smo tudi privzeli, da je optično polje, ki vpade na objektni zaslon $xy$ povsod 
konstantno in enako $E_0$. Tudi to v eksperimentu dosežemo z uporabo leče. 
Svetlobo, ki izhaja iz točkastega svetila, lahko namreč kolimiramo z lečo (in z ustreznim
spektralnim filtrom ustvarimo monokromatsko valovanje), tako da izvor postavimo 
v gorišče leče.
\begin{remark}
Opisani sistem je analogna naprava za izračun 2D-Fourierove transformacije nekega 
delno prozornega objekta ali vzorca. /pattern recognition?/
\end{remark}

\begin{example}{\bf Fraunhoferjev uklon na reži debeline $d$.}
Obravnavajmo uklon svetlobe z valovno dolžino $\lambda$, ki v obliki ravnih valov 
vpada na zaslon z režo debeline $d$. Uklonsko sliko na oddaljenem opazovalnem zaslonu
izračunamo iz enačbe~(\ref{eq:Fraunhoferjev}), pri čemer je funkcija $f(x)$ enaka 1 za
$-d/2<x<d/2$, sicer je 0. Dobimo:
\beq
E(\omega_\xi) = \frac{i}{\lambda} \frac{\tilde{E}_0 e^{ikR_0}}{R_0}
\int_{-d/2}^{d/2} e^{-i\omega_\xi x} dx \approx \frac{i}{\lambda}
\frac{\tilde{E}_0 e^{ikz}}{z}\frac{1}{-i\omega_\xi}\left(e^{-i\omega_\xi d/2}- e^{i\omega_\xi d/2}\right),
\label{eq:05_22}
\eeq
od koder z upoštevanjem izrazov za $\sin(x)$ in $k$ sledi:
\beq
E(\omega_\xi) = \frac{i}{\lambda}
\frac{\tilde{E}_0 e^{ikz}}{z}\frac{2}{\omega_\xi} \sin\left(\omega_\xi d/2\right) = 
\frac{id}{\lambda}
\frac{\tilde{E}_0 e^{ikz}}{z}
\frac{\sin\left(\omega_\xi d/2\right)}{\omega_\xi d/2}.
\label{eq:05_23}
\eeq
Gostota svetlobnega toka uklonjene svetlobe na opazovalnem zaslonu je tako:
\boxeq{eq:uklon1reza}{
j(\vartheta) = j_0 \left(\frac{\sin\left(kd\sin(\vartheta)/2\right)}{kd\sin(\vartheta)/2}\right)^2,
}
pri čemer so $k$ valovno število, $d$ debelina reže in $\vartheta$ uklonski kot. Gostoto
toka smo normirali, tako da je v smeri naravnost (pri $\vartheta = 0$) enaka $j_0$.
\begin{figure}[ht]
\centering
\def\svgwidth{120truemm} 
%\input{slike/05_1Reza.pdf_tex}
\caption{SLIKA}
\label{fig:05_1Reza}
\end{figure}

V smeri naprej je intenziteta uklonjene svetlobe največja, nato z naraščajočim
kotom pojema, dokler ne doseže vrednosti 0. Pri večjih kotih ponovno naraste, vendar 
je njena intenziteta znatno manjša kot v osrednjem vrhu. Uklonske kote, pri 
katerih se pojavijo oslabitve, izračunamo preprosto iz enačbe~(\ref{eq:uklon1reza}), tako da
števec postavimo na nič. Sledi:
\beq
kd \sin(\vartheta)/2 = N\pi \qquad \mathrm{in} \qquad \sin \vartheta = \frac{N\lambda}{d},
\label{eq:05_24}
\eeq
pri čemer je $N$ celo število. Lego in intenziteto vrhov je treba določiti numerično. 
\end{example}

\begin{example}{\bf Fraunhoferjev uklon na sistemu $N$ vzporednih rež.}
Naj bo v objektni ravnini $N$ vzporednih rež debeline $d$, perioda ponavljanja
rež pa naj bo $D$.

Podobno kot smo izračunali uklonsko sliko na eni reži, izračunamo tudi uklonsko sliko
na več vzporedih režah. Prepustnostna funkcija $f$ je tako enaka 1 v vsaki reži, med
režami in zunaj območja rež pa je enaka 0.
\beq
E(\omega_\xi) = \frac{i}{\lambda} \frac{\tilde{E}_0 e^{ikR_0}}{R_0}
\left( \int_{-d/2}^{d/2} e^{-i\omega_\xi x} dx + \int_{D-d/2}^{D+d/2} e^{-i\omega_\xi x} dx +
\int_{2D-d/2}^{2D+d/2} e^{-i\omega_\xi x} dx + ... \right).
\label{eq:05_25}
\eeq
Integrale izračunamo in dobimo:
\beq
E(\omega_\xi) = \frac{i}{-i\omega_\xi \lambda} \frac{\tilde{E}_0 e^{ikR_0}}{R_0}
\left(e^{-i\omega_\xi d/2}- e^{i\omega_\xi d/2} \right) \left(1 + e^{-i\omega_xi D} + 
e^{-i\omega_xi 2 D} + ...\right).
\label{eq:05_26}
\eeq
Sledi:
\beq
E(\omega_\xi) = \frac{id}{\lambda} \frac{\tilde{E}_0 e^{ikR_0}}{R_0}\frac{e^{-iN\omega_\xi D/2}}
{e^{-i\omega_\xi D/2}}
\frac{\sin(\omega_\xi d/2)}{\omega_\xi d/q} 
\frac{\sin(N\omega_\xi D/2)}{\sin(\omega_\xi D/2)}.
\label{eq:05_27}
\eeq
Gostota svetlobnega toka uklonjene svetlobe na opazovalnem zaslonu je tako:
\boxeq{eq:uklonNrez}{
j(\vartheta) = j_0 \left(\frac{\sin\left(kd\sin(\vartheta)/2\right)}{kd\sin(\vartheta)/2}\right)^2
\left(\frac{\sin\left(NkD\sin(\vartheta)/2\right)}{\sin\left(kD\sin(\vartheta)/2\right)}\right)^2.
}
Prvi oklepaj imenujemo strukturni faktor $S(\vartheta)$ in določa ovojnico uklonske slike. Enak je uklonski
sliki za eno samo režo debeline $d$. Drugi oklepaj imenujemo mrežni faktor $M(\vartheta)$ in je odvisen od 
razporeditve rež. 
\begin{figure}[ht]
\centering
\def\svgwidth{120truemm} 
%\input{slike/05_Nrez.pdf_tex}
\caption{SLIKA}
\label{fig:05_Nrez}
\end{figure}

Glavne vrhove v uklonski sliki dobimo, kadar je imenovalec ulomka enak nič:
\beq
kD \sin(\vartheta)/2 = n\pi \qquad \mathrm{in} \qquad \sin \vartheta = \frac{n\lambda}{D},
\label{eq:05_28}
\eeq
pri čemer je $n$ celo število. Med posamičnimi glavnimi vrhovi se pojavijo še manjši
stranski vrhovi. Poiščemo jih med zaporednimi minimumi, ki jih določajo ničle števca
ulomka:
\beq
NkD \sin(\vartheta)/2 = m\pi \qquad \mathrm{in} \qquad \sin \vartheta = \frac{m\lambda}{ND},
\label{eq:05_29}
\eeq
pri čemer je $m$ celo število. Med dvema glavnima vrhoma je tako v splošnem
$N-2$ stranskih vrhov. Pri izračunu amplitude vrhov moramo upoštevati
tudi strukturni faktor, torej ovojnico. 

Prvi primer na sliki~\ref{fig:05_Nrez} prikazuje uklonsko sliko za primer $N = 6$ in 
$D=4d$. Med dvema glavnima vrhoma so štirje stranski vrhovi. Z upoštevanjem amplitude
envelope ugotovimo, da vsak četri (preveri!) glavni vrh izgine zaradi minimuma envelope.
\end{example}

\begin{example}{\bf Fraunhoferjev uklon na okrogli odprtini s polmerom $a$.}
Kot naslednji primer si oglejmo Fraunhoferjev uklon na okrogli odprtini s polmerom $a$. 
Lego točke $P$ v objektni ravnini $xy$ opišemo v polarnih koordinatah $\varrho$ in $\varphi$
kot:
\beq
x = \varrho \cos \varphi \qquad \mathrm{in} \qquad y = \varrho \sin \varphi. 
\label{eq:05_30}
\eeq
Lego točke $P_0$ v opazovalni ravnini $\xi \eta$ zapišemo v polarnih koordinatah 
$q$ in $\phi$:
\beq
\xi = q \cos \phi \qquad \mathrm{in} \qquad \eta = q \sin \phi. 
\label{eq:05_31}
\eeq
Uklonski integral potem zapišemo kot:
\beq
E(q,\phi,z_0) = \frac{i}{\lambda} \frac{E_0e^{ikR_0}}{R_0}\int_0^{2\pi} \int_0^a
e^{-i\omega_\xi x}e^{-i\omega_\eta y}\varrho d\varrho d\varphi.
\label{eq:05_32}
\eeq
Upoštevamo enačbe~(\ref{eq:05_13} in \ref{eq:05_14}) in zapišemo:
\beq
\omega_\xi x + \omega_\eta y= \frac{qk\cos \phi}{R_0} \varrho \cos\phi + \frac{qk\sin \phi}{R_0} \varrho \sin\phi = 
 \frac{qk\sin \phi}{R_0} \cos(\varphi - \phi).
\label{eq:05_33}
\eeq
Zaradi simetrije lahko postavimo $\phi=0$ in zapišemo integral:
\beq
E(q,\phi,z_0) = \frac{i}{\lambda} \frac{E_0e^{ikR_0}}{R_0}\int_0^{2\pi} \int_0^a
e^{-iqk\varrho \cos(\varphi)/R_0)} \varrho d\varrho d\varphi.
\label{eq:05_34}
\eeq
Spomnimo se integralnega zapisa Besslovih funkcij prve vrste:
\beq
J_m(u) = \frac{(-1)^m}{2\pi} \int_0^{2\pi} e^{-iu\cos v }e^{imv}dv
\label{eq:05_35}
\eeq
in rekurzijske zveze:
\beq
\frac{d}{du}\left( u^m J_m(u)\right) = u^m J_{m-1}(u).
\label{eq:05_36}
\eeq
Primerjamo enačbi~(\ref{eq:05_34}) in (\ref{eq:05_35}) in vpeljemo
zveze $v = \varphi$, $m=0$ in $u = qk\varrho/R_0$. Pri računu integrala 
upoštevamo izraza za Besslove funkcije (enačbe~\ref{eq:05_35} in \ref{eq:05_36}) 
in dobimo:
\beq
E = \frac{i}{\lambda}\frac{E_0 e^{ikR_0}}{R_0} \pi a^2 \frac{2 J_1 (kqa/R_0}{kqa/R_0}.
\label{eq:05_37}
\eeq
Če vpeljemo še uklonski kot $\vartheta$ in zapišemo gostoto svetlobnega toka
uklonjene svetlobe:
\boxeq{eq:uklonAiry}{
j = j_0 \left(\frac{2J_1(ka\sin\vartheta}{ka\sin \vartheta}\right)^2.
}
Opisano odvisnost, ki jo imenujemo tudi Airyjev uklonski vzorec po angleškem
matematiku in astronomu Siru Georgeu Biddellu Airyu, si poglejmo podrobneje. 
Na opazovalnem zaslonu nastane rotacijsko simetrična uklonska slika, ki 
jo imenujemo Airyjev disk. 
\begin{figure}[ht]
\centering
\def\svgwidth{120truemm} 
%\input{slike/05_Airy.pdf_tex}
\caption{SLIKA}
\label{fig:05_Airy}
\end{figure}
V smeri naravnost (na sredini Airyjevega diska) 
je gostota svetlobnega toka enaka $j_0$, saj velja:
\beq
\lim_{u \to 0}\frac{J_1(u)}{u} = \frac{1}{2}.
\label{eq:05_39}
\eeq
Z naraščajočim uklonskim kotom intenziteta uklonjene svetlobe pojema, nato
spet naraste in ponovno pojema ... Ničle uklonske slike so določene
s pogojem, da je $J_1 = 0$ in se pojavijo pri vrednostih:
\beq
ka\sin\vartheta  \approx 3,83;~7,02;~10,17;~13,32 ...
\label{eq:05_40}
\eeq
Intenziteta stranskih obročev je razmeroma šibka, zato je najpomembnejši
notranji krog uklonske slike. Polmer notranjega kroga oziroma diska 
izračunamo iz zveze:
\beq
ka q_A/R_0 = \frac{2\pi a q_A}{\lambda R_0} = 3,83
\label{eq:05_41}
\eeq
in dobimo:
\beq
q_A = \frac{3,83}{\pi}\frac{\lambda R_0}{2a} = 1,22 \frac{\lambda R_0}{2a}.
\label{eq:05_42}
\eeq
\end{example}

\section{Ločljivost optičnih naprav}
Pri optičnih napravah pogosto svetloba potuje skozi lečo in 
nato vpade na svetlobni detektor. Leča pri tem deluje
kot okrogla odprtina s polmerom $a$, svetloba pa se na njej
uklanja.

Poglejmo primer zelo oddaljenega točkastega predmeta, ki ga 
preslikamo z lečo z goriščno razdaljo $f$ in premerom $D$. Slika predmeta 
nastane v goriščni ravnini, zato je $R_0=f$. Zaradi uklona potem 
točkasti predmet na detektorju vidimo kot okrogel disk s polmerom
\beq
q_A \approx 1,22 \frac{f\lambda}{D} = 1,22 \lambda \frac{f}{D} = 0,61 \frac{\lambda}{NA},
\label{eq:05_43}
\eeq
pri čemer je numerična apertura $NA$ razmerje med polmerom leče $D/2$
in goriščno razdaljo $f$:
\beq
NA = \frac{D/2}{f}.
\label{eq:05_44}
\eeq
Značilne vrednosti za numerično aperturo so $NA \lesssim 1$.

Če opazujemo dve oddaljeni točkasti telesi, bo v opazovalni ravnini vsako
od njih ustvarilo Airyjev disk. Kadar se diska začneta prekrivati, teles
na sliki ne moremo več ločiti. Ta pojav omejuje ločljivost optičnih naprav
in ga poglejmo podrobneje. 

Kriterij, ki določa, ali dva objekta na sliki še ločimo med seboj, imenujemo
Rayleighev kriterij ločljivosti po angleškemu fiziku Lordu Rayleighu. Kriterij
pravi, da objekta ločimo, če velja:
\beq
\Delta q_A \geq 1,22 \frac{\lambda}{NA} \gtrsim \lambda.
\label{eq:05_45}
\eeq
Najmanjša razlika zornih kotov:
\beq
\Delta \alpha_{\mathrm{min}} = 
\frac{\Delta q_{A\mathrm{min}}}{f} \gtrsim 1,22 \frac{\lambda}{D}.
\label{eq:05_46}
\eeq
Večjo kotno ločljivost optičnih naprav, na primer teleskopov, torej 
dosežemo pri manjših valovnih dolžinah svetlobe in večjih lečah.

\begin{example}{\bf Ločljivost teleskopa.}
S teleskopom, ki ima premer leče $D=5~\si{m}$, opazujemo
Luno, ki je od Zemlje oddaljena $L = 384 000~\si{km}$. Če je 
valovna dolžina svetlobe $500~\si{nm}$, ocenimo velikost podrobnosti
na Luninem površju, ki jih še ločimo.

Iz enačbe~(\ref{eq:05_46}) ocenimo:
\beq
\Delta l = 1,22 \frac{\lambda}{D} L \approx 50~\si{m}.
\label{eq:05_47}
\eeq
\end{example}

Na podoben način izračunamo tudi ločljivost optičnih mikroskopov. Ker
tam pri njih predmet blizu goriščne razdalje, je ločljivost
teh naprav približno enaka valovni dolžini svetlobe $\Delta l \approx \lambda$.

\section{Fresnelov uklon}
Doslej smo predpostavili, da je optično polje, ki vpade na objektni zaslon $xy$ povsod 
konstantno in enako $E_0$. Obravnavajmo zdaj primer, ko je svetloba, 
ki vpada na objektni zaslon, funkcija kraja $E_0 = E_0(x,y)$. 

Naj izvor svetlobe $S$, ki oddaja krogelno valovanje, leži v ravnini 
$x'y'$. Razdalja med ravnino izvora svetlobe in objektno ravnino
naj bo enaka $z_0'$ (slika~\ref{fig:05_Fresnel}).
\begin{figure}[ht]
\centering
\def\svgwidth{120truemm} 
%\input{slike/05_Fresnel.pdf_tex}
\caption{SLIKA}
\label{fig:05_Fresnel}
\end{figure}
Polje v točki $P$ v objektni ravnini potem zapišemo kot 
\beq
E_0 (x,y,0) = \tilde{E}_0 \frac{e^{ikR'}}{R'}.
\label{eq:05_65}
\eeq
Uklonski integral (enačba~\ref{eq:05_05}) se potem zapiše kot:
\beq
E(\xi, \eta, z_0) = \frac{i}{\lambda} \tilde{E}_0 \int_{-\infty}^{\infty}
\int_{-\infty}^{\infty} f(x,y) E_0 \left(\frac{e^{ikR'}}{R'}\right) \left(\frac{e^{ikR}}{R}\right) dx dy.
\label{eq:05_66}
\eeq
Izraza za $R$ in $R'$ razvijemo v Taylorjevo vrsto:
\beq
R^2 = (x-\xi)^2 + (y-\eta)^2 + z_0^2 \approx 
R_0 - \frac{x\xi}{R_0} - \frac{y\eta}{R_0} + \frac{x^2+y^2}{2R_0} + ...
\label{eq:05_67}
\eeq
Pri Fraunhoferjevem uklonu smo upoštevali prve tri člene, četrtega pa smo zanemarili.
Pri Fresnelovem uklonu upoštevamo tudi četrti člen v razvoju. 

Zapišemo razdaljo $R$ malo drugače:
\beq
R^2 = z_0^2 + (x-\xi)^2+(y-\eta)^2 = z_0^2 \left(1+ \frac{(x-\xi)^2+(y-\eta)^2}{z_0^2}\right).
\label{eq:05_68}
\eeq
Vpeljemo parameter $d^2 = (x-\xi)^2+(y-\eta)^2$ in izraz razvijemo
\beq
R = z_0^2 \left(1+\frac{d^2}{z_0^2}\right) \approx z_0 + \frac{d^2}{2z_0} + ...
\label{eq:05_69}
\eeq
Členi, ki smo jih zanemarili v razvoju, so zanemarljivi, če velja
odprtine:
\beq
k\left( \frac{1}{8}\frac{d^4}{z_0^3}\right) \ll 2\pi.
\label{eq:05_70}
\eeq
Pogoj preoblikujemo v:
\beq
\frac{d^4}{z_0^4} \ll \frac{8\lambda}{z_0}.
\label{eq:05_71}
\eeq
Kadar je zapisani pogoj izpolnjen, je Fresnelov približek veljaven.

\begin{example}
Naj bo razdalja med objektnim in opazovalnim zaslonom $z_0=1~\si{m}$ in 
valovna dolžina svetlobe $500~\si{nm}$. 
Pogoj za Fresnelov približek
uklona je izpolnjen, kadar je:
\beq
\frac{d^4}{z_0^4} \ll 8\frac{500~\si{nm}}{1~\si{m}} = 4\cdot 10^{-6},
\label{eq:05_72}
\eeq
kar pomeni, da mora za veljavnost približka veljati:
\beq
d\ll z_0 \sqrt[4]{4\cdot 10^{-6}} \approx 4~\si{mm}.
\label{eq:05_73}
\eeq
Izračunajmo še Fresnelovo število (enačba~\ref{eq:05_17}):
\beq
F = \frac{d^2}{\lambda R_0} \ll \frac{16 \times 10^{-6}~\si{m}^2}{500~\si{nm} \cdot 1~\si{m}} \approx 30. 
\label{eq:05_74}
\eeq
\end{example}

Podobno kot smo razvili $R$ (enačba~\ref{eq:05_69}), razvijemo tudi $R'$:
\beq
R' \approx z_0' + \frac{1}{2}\frac{(x-x')^2+(y-y')^2}{2z_0'}.
\label{eq:05_75}
\eeq
Oba izraza vstavimo v uklonski integral (enačba~\ref{eq:05_66}) in dobimo:
\boxeq{eq:05_76}{
E(\xi, \eta, z_0) = \frac{i}{\lambda} \frac{\tilde{E}_0 e^{ikz_0'+ikz_0}}{z_0'z_0} \int
\int_{-\infty}^{\infty} f(x,y) e^{\frac{ik}{2z_0'}((x-x')^2+(y-y')^2)} e^{\frac{ik}{2z_0}((x-\xi)^2+(y-\eta)^2)}  dx dy
}
Pri tem smo člene v imenovalcu ulomka pred integralom približali z $z_0'$ in $z_0$. Zapisani izraz
Fresnelovega približka je za analitično računanje precej zahtevnejši kot Fraunhoferjev uklon. Zato bomo obravnavali
le dva primera: Fresnelove conske plošče in uklon na pravokotni reži oziroma ostrem robu.

\begin{example}{\bf Fresnelova conska plošča.}
V objektni ravnini imamo okroglo odprtino s polmerom $a$. Oz $z$ naj poteka skozi središče odprtine. 
Točkasti izvor svetlobe leži na osi $z$ v razdalji $z_0'$ pred odprtino. Zanima nas intenziteta
svetlobe na osi $z$ v razdalji $z_0$ za odprtino.
\begin{figure}[ht]
\centering
\def\svgwidth{120truemm} 
%\input{slike/05_FresCona.pdf_tex}
\caption{SLIKA}
\label{fig:05_FresCona}
\end{figure}

Izhajamo iz enačbe~\ref{eq:05_76} in vstavimo $x'=y'=0$ in $\xi = \eta = 0$. Dobimo:
\beq
E(0,0, z_0) = \frac{i}{\lambda} \frac{\tilde{E}_0 e^{ikz_0'+ikz_0}}{z_0'z_0} \int_{-\infty}^{\infty}
\int_{-\infty}^{\infty} f(x,y) e^{(ik/2z_0')(x^2+y^2)} e^{(ik/2z_0)(x^2+y^2)} dx dy
\label{eq:05_77}
\eeq
Integral računamo v cilindričnih koordinatah in vpeljemo $\varrho = \sqrt{x^2 + y^2}$ ter
\beq
dx dy = 2 \pi \varrho d\varrho.
\label{eq:05_78}
\eeq
Funkcija $f=1$ za $\varrho \leq a$, sicer je enaka 0. Dobimo:
\beq
E(0,0, z_0) = \frac{2\pi i}{\lambda} \frac{\tilde{E}_0 e^{ikz_0'+ikz_0}}{z_0'z_0} \int_{0}^{a}
e^{\frac{ik\varrho^2}{2}\left(\frac{1}{z_0'}+\frac{1}{z_0}\right)}\varrho d\varrho.
\label{eq:05_79}
\eeq
Vpeljemo:
\beq
\frac{1}{L} = \frac{1}{z_0'} + \frac{1}{z_0} \qquad \mathrm{in} \qquad z_0'z_0 = L(z_0'+z_0).
\label{eq:05_80}
\eeq
Izračunamo integral:
\beq
\int_0^a e^{ik\varrho^2/2L}\varrho d\varrho = \frac{L}{ik}\left(e^{ika^2/2L}-1\right) = 
\frac{2L}{k}e^{ika^2/4L} \sin\left(\frac{ka^2}{4L}\right).
\label{eq:05_81}
\eeq
Za gostoto energijskega toka $j$ dobimo:
\beq
j = 4 j_0 \sin\left(\frac{ka^2}{4L}\right),
\label{eq:05_82}
\eeq
pri čemer smo z $j_0$ označili gostoto toka, kakršna bi bila na mestu opazovanja, 
če ne bi bilo zaslona:
\beq
j_0 = \frac{\tilde{E}_0^2}{(z_0'+z_0)^2}.
\label{eq:05_83}
\eeq
\begin{figure}[ht]
\centering
\def\svgwidth{120truemm} 
%\input{slike/05_FresCona2.pdf_tex}
\caption{SLIKA}
\label{fig:05_FresCona2}
\end{figure}
Na sliki~\ref{fig:05_FresCona2} je narisana odvisnost gostote svetlobnega toka na danem 
mestu od polmera odprtine $a$. Vidimo, da je največja vrednost, ki jo doseže $j$
štirikratnik intenzitete $j_0$, ki bi bila na mestu mestu opazovanja brez zaslona. 
Polmer $a_1$, pri katerem doseže vrednost $j$ prvi maksimum, imenujemo polmer
prve Fresnelove cone in je enak:
\beq
\frac{ka_1^2}{4L} = \frac{\pi}{2} \qquad \mathrm{in} \qquad a_1  = \sqrt{\lambda L}.
\label{eq:05_84}
\eeq
Prva Fresnelova cona svetlobo svetlobo torej zbira in deluje kot zbiralna leča. 
Sekundarni krogelni valovi, ki izhajajo iz območja znotraj prve Fresnelove cone, 
se seštevajo. Poglejmo račun za primer odprtine v velikosti prve Fresnelove
cone in $z_0' \to \infty$ oziroma $L = z_0$. 
\begin{figure}[ht]
\centering
\def\svgwidth{120truemm} 
%\input{slike/05_FresCona3.pdf_tex}
\caption{SLIKA}
\label{fig:05_FresCona3}
\end{figure}

Potem velja:
\beq
R^2 = z_0^2 + a_1^2 = z_0^2 + \lambda z_0.
\label{eq:05_85}
\eeq
Fazni zamik za žarek oziroma valovanje, ki izvira iz roba cone, je:
\beq
\phi_1 = kR = k\sqrt{z_0^2+\lambda z_0}.
\label{eq:05_86}
\eeq
Fazni zamik za valovanje iz sredine cone pa je:
\beq
\phi_0 = kz_0.
\label{eq:05_87}
\eeq
Dokler je fazna zakasnitev med žarki iz sredine in žarki z roba odprtine
manjša od $\pi$ oziroma razlika v dolžini poti manjša od $\lambda/2$, 
se žarki seštevajo in intenziteta na mestu $z_0$ narašča. Preverimo, ali 
se ta ugotovitev ujema s pogojem za prvo Fresnelovo cono:
\beq
R - z = \sqrt{z_0^2 + \lambda z_0} - z_0  \approx z_0 + \frac{\lambda}{2} - z_0 = \frac{\lambda}{2}.
\label{eq:05_88}
\eeq
Če polmer odprtine povečamo na območje $a>a_1$, se pojavijo valovi v protifazi in 
prispevki iz sredine odprtine in z roba odprtine se odštevajo. Zato z naraščajočim
polmerom intenziteta svetlobe pojema. Minimum doseže pri $a_2$, za katerega velja:
\beq
\frac{ka_2^2}{4L} = \pi
\label{eq:05_89}
\eeq
in $a_2 = \sqrt{2\lambda L}$. Kolobar z notranjim polmerom 
$a_1 = \sqrt{\lambda L}$ in zunanjim polmerom $a_2 = \sqrt{2\lambda L}$
imenujemo druga Fresnelova cona. Na splošno je N-ta Fresnelova cona
kolobar, ki ga omejujeta polmera $a_{N-1}$ in $a_N$:
\beq
a_{N-1} = \sqrt{(N-1)\lambda L} < a < a_{N} = \sqrt{N \lambda L}.
\label{eq:05_90}
\eeq
V tretji Fresnelovi coni $j$ spet narašča, nato v območju $a_3 - a_4$ spet pojema ...

{\bf Fresnelova uklonska leča}: Če ohranimo v ravnini $xy$ le tiste kolobarje oziroma
cone, ki so ali samo lihe ali samo sode, se prispevki vseh con seštevajo. Taka
naprava (conska plošča) zbira svetlobo na podoben način kot klasična leča. Gostota 
svetlobnega toka v točki $P_0$ je pri uporabi conske plošče dosti večja, kot bi bila 
v primeru, če bi izvor svetlobe točko osvetljeval neposredno. Tovrstne plošče so
uporabne za zabiranje svetlobe, kadar nimamo možnosti izdelave leč (na primer 
leče za žarke X. Glavna slabost Fresnelovih uklonskih leč je njihova kromatična
aberacija -- odvisnost goriščne razdalje od valovne dolžine svetlobe $f= f(L)$.
Goriščno razdaljo lahko izračunamo, da upoštevamo $z_0'\to \infty$:
\beq
\frac{1}{z_0'}+ \frac{1}{z_0}= \frac{1}{z_0} = \frac{1}{L} = \frac{1}{f},
\label{eq:05_91}
\eeq
od koder sledi:
\beq
f = \frac{a_1^2}{\lambda}.
\label{eq:05_92}
\eeq
Goriščno razdaljo conske leče z velikim številom plošč lahko tudi izračunamo:
\beq
a_{N+1}^2 - a_N^2 = \lambda (N+1)L - \lambda NL = \lambda L. 
\label{eq:05_93}
\eeq
Po drugi strani velja:
\beq
a_{N+1}^2 - a_N^2 = \left(a_{N+1}+a_N\right)\left(a_{N+1}- a_N\right) = 2\overline{a}\Delta a.
\label{eq:05_94}
\eeq
Od tod sledi:
\beq
f = \frac{a_1^2}{\lambda} = \frac{2 \overline{a}\Delta a}{\lambda}.
\label{eq:05_95}
\eeq
\end{example}

\begin{example}{\bf Fresnelov uklon na robu in pravokotni reži.}
Kot drugi primer uporabe Fresnelovega uklonskega približka 
izračunajmo uklonsko sliko na ostrem robu oziroma na pravokotni 
reži. Oz $z$ izberemo tako, da točkasti izvor svetlobe $S$ in
opazovalna točka $P_0$  ležita na njej, pravokotna reža s
koordinatami $x_1<x<x_2$ in $y_1<y<y_2$ pa naj leži v objektni
ravniki, ki je pravokotna na zveznico med točkama $S$ in $P_0$.
\begin{figure}[ht]
\centering
\def\svgwidth{120truemm} 
%\input{slike/05_FresPravokot.pdf_tex}
\caption{SLIKA}
\label{fig:05_FresPravokot}
\end{figure}

Uklonski integral (enačba~\ref{eq:05_76}) je potem:
\beq
E(0,0, z_0) = \frac{i}{\lambda} \frac{\tilde{E}_0 e^{ikz_0'+ikz_0}}{z_0'z_0} \int_{x_1}^{x_2}
\int_{y_1}^{y_2} e^{\frac{ik(x^2+y^2)}{2L}}  dx dy
\label{eq:05_96}
\eeq
oziroma:
\beq
E(P_0) \propto \int_{u_1}^{u_2} e^{i\frac{\pi}{2}u^2} du
\int_{v_1}^{v_2} e^{i\frac{\pi}{2}v^2} dv,
\label{eq:05_97}
\eeq
 pri čemer smo uvedli novi spremenljivki:
 \beq
 u = \sqrt{\frac{k}{\pi L}}x \qquad \mathrm{in} \qquad v = \sqrt{\frac{k}{\pi L}}y.
 \label{eq:05_98}
 \eeq
 Zapisana integrala izrazimo s Fresnelovima integraloma $S(x)$ in $C(x)$.
\begin{remark}
Fresnelova integrala sta oblike:
\beq
C(u) = \int_0^u \cos\left(\frac{1}{2}\pi x^2\right) dx \qquad \mathrm{in} \qquad 
S(v) = \int_0^v \sin\left(\frac{1}{2}\pi x^2\right) dx. 
\label{eq:05_99}
\eeq
\begin{figure}[ht]
\centering
\def\svgwidth{120truemm} 
%\input{slike/05_CS.pdf_tex}
\caption{SLIKA}
\label{fig:05_CS}
\end{figure}
To sta dve transcedentni funkciji, ki ju moramo izračunati numerično. Tipične
vrednosti so $C(-\infty) = -1/2$; $C(0) = 0$; $C(\infty) = 1/2$ in enako za $S$, 
pri čemer sta obe funkciji lihi. Skupaj ju združimo v integral:
\beq
\int_{0}^{u} e^{i\frac{\pi}{2}u^2} du = C(u)+iS(u) = \frac{1+i}{2}\erf \left(\frac{\sqrt{\pi}(1-i)u}{2}\right).
\label{eq:05_100}
\eeq
\end{remark}

Fresnelovo uklonsko sliko na pravokotni odprtini torej zapišemo kot:
\beq
E(P_0) \propto \left( C(u) + iS(u) \right) \left|_{u_1}^{u_2} \cdot 
\left( C(v) + iS(v) \right) \right|_{v_1}^{v_2}.
\label{eq:05_101}
\eeq

Na osnovi rezultata lahko obravnavamo tudi uklon na ravnem robu. 
To ustreza enodimenzionalnemu primeru $x_1 \to \infty$ oziroma $u_1 \to \infty$
in $x_2 = d$ oziroma $u_2  = d\sqrt{k/\pi L}$. Uklonska slika
je potem:
\beq
E(P_0) \propto C(u_2) + i S(u_2) + \frac{1}{2} + i \frac{1}{2},
\label{eq:05_102}
\eeq
od koder sledi:
\beq
j(P_0) \propto |E(P_0)|^2  \propto \left(C(u_2) + \frac{1}{2}\right)^2 + \left(S(u_2) + \frac{1}{2}\right)^2.
\label{eq:05_103}
\eeq
Zanima nas odvinost $j(d)$. Namesto da pri izračunu zaslon miruje in spreminjamo opazovalno
točko $P_0$ vzdolž osi $\xi$, točko $P_0$ ohranjamo in si mislimo, da se premika rob zaslona 
v smeri osi $x$, pri čemer negativne vrednosti $d$ ustrezajo geometrijski senci. 
\begin{figure}[ht]
\centering
\def\svgwidth{120truemm} 
%\input{slike/05_FresRob.pdf_tex}
\caption{SLIKA}
\label{fig:05_FresRob}
\end{figure}

\begin{remark}
Fresnelova integrala $C(u)$ in $S(u)$  lahko prikažemo grafično tako, da narišemo
krivuljo v parametrični obliki $C(t), S(t)$ v 2D-ravnini. 
\begin{figure}[ht]
\centering
\def\svgwidth{120truemm} 
%\input{slike/05_Cornu.pdf_tex}
\caption{SLIKA}
\label{fig:05_Cornu}
\end{figure}
Poglejmo nekaj lasnosti te parametrične krivulje, ki jo imenujemo tudi Cornujeva oziroma
Eulerjeva spirala. Pri vrednosti $t\to \infty$ je vrednost $(1/2, 1/2)$ in pri 
$t\to -\infty$ je vrednost $(-1/2, -1/2)$. 
Dolžina krivulje? Strmina krivulje? Ukrivljenost krivulje je sorazmerna z dolžino
trajektorije (poti). 
\end{remark}
\end{example}


\section{*Izpeljava Kirchhoffovega uklonskega integrala}
\label{chap:Kirchhoff}
Poglejmo še matematično izpeljavo uklonskega integrala (enačba~\ref{eq:05_01})
in najprej ponovimo Greenov teorem, ki izhaja iz Gaussovega stavka. Naj bo 
$\mathbf{W}(\mathbf{r})$ poljubno zvezno in zvezno odvedljivo vektorsko
polje. Gaussov stavek pravi, da je intergal po sklenjeni ploskvi enak
integralu divergence vektorskega polja po prostoru:
\beq
\oint_S \mathbf{W} d\mathbf{S} = \int_V \div(\mathbf{W})dV.
\label{eq:05_48}
\eeq
Zapišimo  vektorsko polje $\mathbf{W}$ s skalarnima poljema $\Psi$ in $\Phi$
v obliki:
\beq
\mathbf{W}(\mathbf{r}) = \Psi \nabla \Phi - \Phi \nabla \Psi,
\label{eq:05_49}
\eeq
pri čemer sta funkciji $\Psi$ in $\Phi$ zvezni in dvakrat zvezno odvedljivi.
Potem velja:
\beq
\div \mathbf{W}= \Psi \nabla^2 \Phi - \Phi \nabla^2 \Psi
\label{eq:05_50}
\eeq
in dobimo Greenov teorem:
\beq
\oint_S \left(\Psi \nabla \Phi - \Phi \nabla \Psi\right) d\mathbf{S}  =
\int_V \left( \Psi \nabla^2 \Phi - \Phi \nabla^2 \Psi \right) dV.
\label{eq:05_51}
\eeq
Za $\Psi$ in $\Phi$ izberemo funkciji, ki predstavljata skalarni rešitvi
krajevnega dela valovne enačbe. Za ti dve funkciji torej veljata zvezi:
\beq
\nabla^2 \Psi + k^2 \Psi = 0 \qquad \mathrm{in} \qquad \nabla^2 \Phi + k^2 \Phi = 0.
\label{eq:05_52}
\eeq
Z upoštevanjem teh dveh vez postane desni del Greenovega teorema 
(enačba~\ref{eq:05_51}) identično enak nič. Od tod sledi, da za funkciji $\Phi$ in $\Phi$,
ki sta rešitvi valovne enačbe, velja zveza:
\beq
\oint \left( \Psi\frac{\partial \Phi}{\partial \mathbf{n}} -
\Phi\frac{\partial \Psi}{\partial \mathbf{n}} \right) dS = 0.
\label{eq:05_53}
\eeq

Zamislimo si zdaj, da funkcija $\Psi$ opisuje sferične valove, ki
izhajajo iz izhodišča koordinatnega sistema $P_0$ in jo zapišemo kot:
\beq
\Psi = \frac{e^{ikr}}{r}.
\label{eq:05_54}
\eeq
Ker ta funkcija v izhodišču pri $r=0$ divergira, moramo izhodišče
izvzeti iz računa. Zato si okoli izhodišča zamislimo kroglo s 
polmerom $\varepsilon$ (slika~\ref{fig:05_Green}).
\begin{figure}[ht]
\centering
\def\svgwidth{120truemm} 
%\input{slike/05_Green.pdf_tex}
\caption{SLIKA}
\label{fig:05_Green}
\end{figure}
Celotna integracijska površina je tako sestavljena iz zunanje ploskve
$S$ in notranje krogle $S_\varepsilon$. Enačbo~(\ref{eq:05_53}) prepišemo v:
\beq
\oint_S \left( \frac{e^{ikr}}{r}\frac{\partial \Phi}{\partial \mathbf{n}} -
\Phi\frac{\partial}{\partial \mathbf{n}}\left( \frac{e^{ikr}}{r} \right) \right) dS -
\oint_{S\varepsilon} \left( \frac{e^{ikr}}{r}\frac{\partial \Phi}{\partial \mathbf{n}} -
\Phi\frac{\partial}{\partial \mathbf{n}}\left( \frac{e^{ikr}}{r} \right) \right)r^2 d\Omega = 0.
\label{eq:05_55}
\eeq
Ker je notranja ploskev krogla, je smer normale na ploskev $\mathbf{n}$ enaka smeri $\mathbf{r}$.
Potem zapišemo odvod:
\beq
\frac{\partial}{\partial r}\left( \frac{e^{ikr}}{r}\right)  = ik \frac{e^{ikr}}{r} - \frac{e^{ikr}}{r^2}.
\label{eq:05_56}
\eeq
Drugi člen v enačbi~(\ref{eq:05_55}) potem prepišemo v:
\beq
\oint_{S\varepsilon} \left( \frac{e^{ikr}}{r}\frac{\partial \Phi}{\partial \mathbf{n}}r^2 -
ik\frac{e^{ikr}}{r}r^2\Phi + \frac{e^{ikr}}{r^2}\Phi r^2 \right) d\Omega.
\label{eq:05_57}
\eeq
V limiti $\varepsilon \to 0$ gresta tako prvi kot 
drugi člen proti 0, tretji člen pa gre proti vrednosti $4\pi \Phi(P_0)$. Faktor $4\pi$ 
dobimo pri integraciji po polnem kotu. Rezultat povežemo in dobimo:
\beq
\Phi(P_0) = \frac{1}{4\pi} \oint_S \left( \frac{e^{ikr}}{r}\frac{\partial \Phi}{\partial \mathbf{n}} -
\Phi\frac{\partial}{\partial \mathbf{n}}\left( \frac{e^{ikr}}{r} \right) \right) dS.
\label{eq:05_58}
\eeq
Na ta način smo polje v opazovani točki $P_0$ izrazili s skalarnim poljem na sklenjeni površini
$S$, ki to točko obdaja.

Zdaj si zamislimo izvor valovanja $S$, ki oddaja krogelne valove oblike $\Phi = \exp(ikr')/r'$. Ti valovi
vpadajo na objektni zaslon z odprtino (slika~\ref{fig:05_Kirchhoff}), nato pa jih opazujemo 
v točki $P$ na drugi strani zaslona. Za izračun polja v točki $P$ moramo tako poznati polje na neki
sklenjeni ploskvi, ki to točko obdaja.
\begin{figure}[ht]
\centering
\def\svgwidth{120truemm} 
%\input{slike/05_Kirchhoff.pdf_tex}
\caption{SLIKA}
\label{fig:05_Kirchhoff}
\end{figure}

Integracijsko ploskev okoli točke $P$ izberemo tako, da delno poteka po zaslonu, povsod drugje
pa tako daleč stran, da so prispevki polja $\Phi$ zanemarljivo majhni. Polje na zaslonu in njegov odvod
naj bosta povsod, razen v odprtini, enaka nič. Privzamemo tudi, da je polje
v odprtini tako, kot da bi zaslona ne bilo. Potem za izračun polja v točki $P$ zadošča izračunati
integral le po površini odprtine. 

Izračunajmo smerni odvod:
\beq
\frac{\partial \Phi}{\partial \mathbf{n}} = (\nabla \Phi)\cdot \mathbf{n} = 
\frac{\partial \Phi}{\partial r'} \frac{\mathbf{r'}}{r'}\cdot \mathbf{n} = \frac{\partial \Phi}{\partial r'} 
\cos\left(\mathbf{r'},\mathbf{n}\right).
\label{eq:05_59}
\eeq
Vstavimo še funkcijo $\Phi$ in izračunamo:
\beq
\frac{\partial \Phi}{\partial r'} = \frac{\partial}{\partial r'}\left(\frac{e^{ikr'}}{r'}\right) = 
\frac{ikr'-1}{r'^2}e^{ikr'} \approx \frac{ikr'}{r'}e^{ikr'}.
\label{eq:05_60}
\eeq
Privzeli smo, da je razdalja $r'$ bistveno večja od valovne dolžine svetlobe, in 1 v števcu zanemarilli. 
Podobno izračunamo tudi drugi člen v enačbi~(\ref{eq:05_58}) in upoštevamo, da je $\lambda \ll r$. 
Uporabimo enačbi~(\ref{eq:05_59} in \ref{eq:05_60}) in ju vstavimo v enačbo~(\ref{eq:05_58}). Dobimo:
\beq
\Phi(P_0) = \frac{1}{4\pi} \int_S \left( \frac{e^{ikr}}{r} \frac{ik}{r'}e^{ikr'} \cos\left(\mathbf{r'},\mathbf{n}\right)-
\frac{e^{ikr'}}{r'} \frac{ik}{r}e^{ikr} \cos\left(\mathbf{r},\mathbf{n}\right)  \right) dS,
\label{eq:05_61}
\eeq
od koder sledi:
\beq
\Phi(P_0) = \frac{ik}{4\pi} \int_S \frac{e^{ikr}e^{ikr'}}{rr'}
\left( \cos\left(\mathbf{r'},\mathbf{n}\right) - \cos\left(\mathbf{r},\mathbf{n}\right) \right) dS.
\label{eq:05_62}
\eeq
Razliko kosinusov imenujemo tudi oblikovni faktor. V geometriji, ki smo jo privzeli, navadno velja
\beq
\cos\left(\mathbf{r'},\mathbf{n}\right) \approx 1 \qquad \mathrm{in} \qquad \cos\left(\mathbf{r},\mathbf{n}\right) \approx -1.
\label{eq:05_63}
\eeq
Potem se integral iz enačbe~(\ref{eq:05_62}) poenostavi v:
\beq
\Phi(P_0) = \frac{2ik}{4\pi} \int_S \frac{e^{ikr}e^{ikr'}}{rr'} dS = \frac{i}{\lambda} \int_S \frac{e^{ikr}e^{ikr'}}{rr'} dS.
\label{eq:05_64}
\eeq
S tem smo izpeljali predfaktor v prvotni enačbi~(\ref{eq:05_01}).
