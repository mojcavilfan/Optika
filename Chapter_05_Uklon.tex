%\chapterimage{Geometrijska.jpg} % Chapter heading image

\chapter{Uklon}
Kadar svetloba vpada na oviro, katere velikost je primerljiva z valovno 
dolžino svetlobe, ali se širi skozi podobno veliko odprtino v zaslonu, 
opazimo enega najbolj značilnih pojavov valovne optike -- uklon.
V geometrijski optiki se svetloba širi po ravnih črtah, ki po 
najkrajši poti povezujejo začetno in končno točko. V tej sliki nastane
za oviro ostra senca, v katero se svetloba ne širi. Po uklonski teoriji
pa svetloba seže tudi v območje sence. Poleg tega, da se valovanje širi
v območje geometrijske sence za ovirami, se v njem pojavijo tudi ojačitve
in oslabitve oziroma pri svetlobi svetla in temna območja. Ta pojav
je posledica interference, ki jo bomo podrobneje obravnavali v
naslednjem poglavju. 
\section{Uklonski integral}
Pri obravnavi uklona zanemarimo vektorsko naravo polja oziroma
polarizacijo in jakost električneva polja obravnavamo kot skalarno količino
$E(\mathrm{r},t)$.

Naj ravno sinusno potujoče valovanje vpada na zaslon, v katerem je odprtina.
Tak zaslon imenujemo objektni zaslon. Zanima nas uklonska slika, ki nastane
za odprtino na tako imenovanem opazovalnem zaslonu (glej sliko~\ref{fig:05_shema}).
\begin{figure}[ht]
\centering
\def\svgwidth{120truemm} 
%\input{slike/05_shema.pdf_tex}
\caption{SLIKA}
\label{fig:05_shema}
\end{figure}

Iščemo vrednost jakosti elekktričnega polja v točki $P_0$ na opazovalnem zaslonu, 
do katere vodi vektor $\mathrm{r}_0$, v odvisnosti of vrednosti polja v točki $P$
na objektnem zaslonu, do katerega vodi vektor $\mathrm{r}$. 

Po Huygensovem načelu iz točke $P$ izhaja krogelno valovanje, katerega
amplitudo določa polje, ki vpada na objektni zaslon. Če je vpadno valovanje
ravno valovanje, velja $E(r) = E_0$ za vsak $\mathbf{r}$. Za nastanek slike
v točki na opazovalnem zaslonu moramo sešteti prispevke sekundarnih
polj po vseh točkah objektnega zaslona. Uklonsko sliko v točki $P_0$ 
tako z enačbo zapišemo kot:
\boxeq{eq:05_01}{
E_{P0} = E (\mathbf{r}_0) = \int_A\!\!\int \frac{i}{\lambda} E_0 \left( \frac{e^{ikR(\mathbf{r})}}{R(\mathbf{r})}\right) dS.
}
Pri tem integriramo po celotni površini odprtine $A$, del v oklepaju predstavlja
krajevni del krogelnih valov, $\lambda$ pa je valovna dolžina vpadne svetlobe
\begin{remark}
Prvotno je uklonski integral zapisal Fresnel, tako da je namesto predfaktorja 
$i/\lambda$ zapisal eksperimentalno določeno konstanto. Predfaktor v zapisani
obliki je izpeljal Kirchhoff. Popolnejši zapis je oblike $i \chi(\mathbf{r})/\lambda$,
pri čemer $\chi$ predstavlja smerni faktor. Kadar je značilna razsežnost odprtine v 
objektnem zaslonu $d$ bistveno manjša od oddaljenosti do zaslona, 
je $\chi \approx 1$.
\end{remark}
Namesto integracije sekundarnega polja samo na območju odprtine $A$, lahko vpeljemo
tako imenovano aperturno funkcijo $f(\mathbf{r})$. Njena vrednost je na mestu odprtine enaka $1$,
sicer je enaka $0$. Zapis z aperturno funkcijo omogoči integracijo po celotni dvodimenzionalni
ploskvi, ki opisuje objektni zaslon. Zapis je uporaben tudi za splošen primer, ko je zaslon
delno prozoren in delno prepušča svetlobo. Polje v ravnini objektnega zaslona potem zapišemo:
\beq
E(\mathbf{r}) = f(\mathbf{r})E_0.
\label{eq:05_02}
\eeq
Določimo zdaj koordinatni sistem na objektni in opazovani ravnini. Objektno ravnino
opišemo kot ravnino s koordinatama $x$ in $y$, opazovalno ravnino pa kot ravnino
s koordinatama $\xi$ in $\eta$. Izhodišči obeh koordinatnih sistemov naj ležita 
na skupni osi $z$, ki jo imenujemo optična os sistema (glej sliko~\ref{fig:05_koordinate}).
\begin{figure}[ht]
\centering
\def\svgwidth{120truemm} 
%\input{slike/05_koordinate.pdf_tex}
\caption{SLIKA}
\label{fig:05_koordinate}
\end{figure}

Točko $P$ na objektnem zaslonu potem opišemo s krajevnim vektorjem:
\beq
P: \mathbf{r} = (x,y,0),
\label{eq:05_03}
\eeq
točko $P_0$ na opazovalnem zaslonu pa s krajevnim vektorjem:
\beq
P_0: \mathbf{r}_0 = (\xi,\eta,z_0).
\label{eq:05_04}
\eeq
Uporabimo uklonski integral (enačba~\ref{eq:05_01}) in zapišemo:
\beq
E(\xi, \eta, z_0) = \frac{i}{\lambda} \int_{-\infty}^{\infty}
\int_{-\infty}^{\infty} f(x,y) E_0 \left(\frac{e^{ikR}}{R}\right) dx dy.
\label{eq:05_05}
\eeq
V izrazu za krogelne valove nastopa razdalja $R$ med točjo $P$ in $P_0$, za
katero velja:
\beq
R = \sqrt{(\xi-x)^2 + (\eta - y)^2 + z_0^2}.
\label{eq:05_06}
\eeq
Glede na to, kako natančno obravnavamo izraz za $R = R(x,y,\xi, \eta, z_0)$ v uklonskem
integralu (enačba~\ref{eq:05_05}), ločimo med različnimi uklonskimi približki. Najbolj 
znana sta Fraunhoferjev in Fresnelov približek, ki ju bomo podrobneje spoznali v nadaljevanju.

\section{Fraunhoferjev uklon}
Izhajamo iz uklonskega integrala (enačba~\ref{eq:05_05}):
\beq
E(\xi, \eta, z_0) = \frac{i}{\lambda} \int_{-\infty}^{\infty}
\int_{-\infty}^{\infty} f(x,y) E_0 \left(\frac{e^{ikR}}{R}\right) dx dy.
\eeq
V Fraunhoferjevem približku uklona, ki ga poenostavljeno imenujemo
Fraunhoferjev uklon, razvijemo razdaljo $R$ med točko v odprtini
in točko na zaslonu do prvega reda. Zapišemo:
\beq
R = \sqrt{(\xi-x)^2 + (\eta - y)^2 + z_0^2} \approx
\sqrt{\xi^2+\eta^2 +z_0^2 - 2\xi x - 2 \eta y + x^2 + y^2}.
\label{eq:05_07}
\eeq
Naj bo:
\beq
R_0 = \xi^2 + \eta^2 + z_0^2.
\label{eq:05_08}
\eeq
Uporabimo razvoj korenske funkcije do prvega reda:
\beq
\sqrt{1+x}\approx 1 + \frac{x}{2} + \dots,
\label{eq:05_09}
\eeq
višje člene v razvoju pa zanemarimo. Razvijemo in upoštevamo le linearne člene:
\beq
R = \sqrt{R_0^2 \left(1 - \frac{2\xi x}{R_0^2} - \frac{2\eta y}{R_0^2} + \frac{x^2+y^2}{R_0^2}
\right)} \approx 
R_0 - \frac{\xi x}{R_0} - \frac{\eta y}{R_0}.
\label{eq:05_10}
\eeq
Člen v uklonskem integralu, ki opisuje krogelne monokromatske valove, z upoštevanjem razvoja
(enačba~\ref{eq:05_10}) zapišemo kot:
\beq
\frac{e^{ikR}}{R} \approx \frac{e^{ikR_0} e^{-ik\xi x /R_0} e^{-ik\eta y/R_0}}{R_0}.
\label{eq:05_11}
\eeq
Ker se člen v imenovalcu spreminja znatno počasneje kot v števcu, lahko odvisnost od koordinat
$x$, $y$, $\xi$ in $\eta$ v imenovalcu zanemarimo. 

Vpeljemo še smerna oziroma uklonska kota:
\beq
\sin \vartheta_\xi \approx \vartheta_\xi = \frac{\xi}{R_0}
\qquad \mathrm{in}
\qquad
\sin \vartheta_\eta \approx \vartheta_\eta = \frac{\eta}{R_0}
\label{eq:05_13}
\eeq
ter z njimi povezani prostorski frekvenci:
\beq
\omega_\xi = k \vartheta_\xi
\qquad \mathrm{in}
\qquad
\omega_\eta = k \vartheta_\eta.
\label{eq:05_14}
\eeq
Potem zapišemo uklonski integral v Fraunhoferjevem približku kot:
\boxeq{eq:Fraunhoferjev}{
E(\omega_\xi,\omega_\eta) = \frac{i}{\lambda} \frac{E_0 e^{ikR_0}}{R_0}
\int_{-\infty}^{\infty}
\int_{-\infty}^{\infty} f(x,y) e^{-i\omega_xi x} e^{-i \omega_\eta y} dx dy.
}
Pogosto naredimo še približek $R_0 \approx z_0$. 

Iz enačbe~(\ref{eq:Fraunhoferjev}) je razvidno, 
da je uklonska slika nekega objekta v Fraunhoferjevem uklonskem približku
enaka 2D Fourierovi transformiranki aperturne funkcije tega objekta. 


\subsection*{Fresnelovo število}
Poiščimo kriterij, ki bo določil, kdaj lahko uporabimo navedeni približek. Spomnimo se, da smo
v razvoju izraza za $R$ zanemarili kvadratni člen (enačba~\ref{eq:05_10}). Ta člen v izrazu
za sferične valove nastopa v členu $\exp(ik(x^2 + y^2)/2R_0)$ in torej prinaša nek dodatni fazni 
zamik. Da bo res zanemarljiv, mora za vsako vrednost $x$ in $y$ veljati:
\beq
\frac{k(x^2+y^2)}{2R_0} \ll 2 \pi.
\label{eq:05_15}
\eeq
Privzamemo, da je $(x^2+y^2)_\mathrm{max} = d^2$, pri čemer $d$ meri karakteristično razsežnost 
odprtine v objektnem zaslonu. Sledi
\beq
\frac{kd^2}{2R_0}  = \frac{2\pi}{\lambda}\frac{d^2}{2R_0}\ll 2 \pi.
\label{eq:05_16}
\eeq
Vpeljemo Fresnelovo število $F$:
\beq
F  = \frac{d^2}{\lambda R_0}
\label{eq:05_17}
\eeq
in Fraunhoferjev približek je veljaven, kadar
\beq
F \ll 1. 
\label{eq:05_18}
\eeq
Pogoj je ekvivalenten zahtevi, da je 
\beq
R_0 \gg \frac{d^2}{\lambda}.
\label{eq:05_19}
\eeq
Tipične vrednosti $R_0 \approx z_0$  morajo biti dosti večje od dimenzij odprtine, zato 
Fraunhoferjevem približku rečemo tudi približek daljnjega polja. 

\begin{example}{\bf Veljavnost Fraunhoferjevega približka.}
Naj bo valovna dolžina svetlobe $\lambda = 500~\si{nm}$. Če je tipična razsežnost odprtine $d = 1~\si{\micro\metre}$,
mora biti za veljavnost Fraunhoferjevega približka oddaljenost od zaslona $R\gg 2~\si{\micro\metre}$. Pri razsežnosti
$d = 10~\si{\micro\metre}$ mora veljati $R\gg 200~\si{\micro\metre}$, pri $d = 100~\si{\micro\metre}$ mora biti
$R\gg 2~\si{\centi\metre}$ in pri $d = 1~\si{\milli\metre}$ velja Fraunhoferejev približek pri $R\gg 2~\si{\metre}$. 
\end{example}
 
Preden obravnavamo primere Fraunhoferjevega uklona, si oglejmo še interpretacijo približnosti Fraunhoferjevega
približka. Če si zamislimo enodimenzionalno režo, ki se razteza v smemi osi $x$. 
\begin{figure}[ht]
\centering
\def\svgwidth{120truemm} 
%\input{slike/05_priblizek.pdf_tex}
\caption{SLIKA}
\label{fig:05_priblizek}
\end{figure}

Fazni zamik sekundarnih sferičnih valov, ki izvirajo z območja reže, se spreminja kot
\beq
\Delta \Phi = \frac{k \xi x}{R_0}.
\label{eq:05_20}
\eeq
Če opazujemo vrednost $\Delta \Phi(x)$ valov, ki potujejo v smeri $\vartheta_\xi \approx
\xi/R_0$ glede na os $z$, vidimo, da velja 
\beq
\Delta \Phi = k \varphi_\xi x.
\label{eq:05_21}
\eeq
Zapis ustreza ravnemu valovanju. Za $R_0 \gg d^2/\lambda$ se uklonjeno valovanje manifestira
kot del ravnega valovanja. 


Zapisali smo, da mora biti za veljavnost Fraunhoferjevega približka zaslon zelo
oddaljen od odprtine, na kateri se svetloba uklanja. Vendar za opazovanje 
Fraunhoferjevega uklona ni nujno, da zaslon postavimo zelo daleč za objektno ravnino.
Namesto tega lahko za objektni zaslon postavimo zbiralno lečo in uklonsko sliko opazujemo
v njeni goriščni ravnini. Zbiralna leča namreč snop vzporednih žarkov zbere v eni 
točki v goriščni ravnini leče. 
\begin{figure}[ht]
\centering
\def\svgwidth{120truemm} 
%\input{slike/05_2DFourier.pdf_tex}
\caption{SLIKA}
\label{fig:05_2DFourier}
\end{figure}
\begin{remark}
Opisani sistem je analogna naprava za izračun 2D-Fourierove transformacije nekega 
delno prozornega objekta ali vzorca. /pattern recognition?/
\end{remark}

\begin{example}{\bf Fraunhoferjev uklon na reži debeline $d$.}
 
\end{example}

\begin{example}{\bf Fraunhoferjev uklon na sistemu $N$ vzporednih rež.}
Naj bo v objektni ravnini $N$ vzporednih rež debeline $d$. Perioda ponavljanja
rež naj bo $D$. 
\end{example}

\begin{example}{\bf Fraunhoferjev uklon na okrogli odprtini s polmerom $a$.}
 
\end{example}



\section{Ločljivost optičnih naprav}

\section{Fresnelov uklon}
