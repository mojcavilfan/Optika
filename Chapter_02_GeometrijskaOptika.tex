%\chapterimage{Geometrijska.jpg} % Chapter heading image

\chapter{Geometrijska optika}
V tem poglavju bomo na kratko predstavili geometrijsko optiko,
v kateri svetlobo obravnavamo kot ravne žarke. Zapisali bomo žarkovno
enačbo, opisali prehod skozi optične elemente z matrikami ABCD in na nekaj 
primerih pokazali njihovo uporabo. Na koncu bomo opisali 
delovanje nekaterih optičnih naprav. 

\section{Optična pot in Fermatov teorem}
V uvodnem zgodovinskem pregledu smo zapisali Fermatov teorem, 
ki pravi, da svetloba med dvema točkama potuje po tisti poti, 
za katero potrebuje najmanj časa. Označimo
hitrost svetlobe v snovi s $c$, pri čemer se ta razlikuje
od hitrosti svetlobe v praznem prostoru $c_0$. Razmerje med
hitrostjo svetlobe v praznem prostoru in njeno hitrostjo
v snovi opisuje lomni količnik $n$. Velja:
\boxeq{eq:c}{
c = \frac{c_0}{n}.
}
Hitrost svetlobe v praznem prostoru je
po definiciji enaka $c_0 = 299\,\,792\,\,458~\si{m/s}$, 
lomni količnik pa je odvisen od snovi in frekvence svetlobe: za vidno svetlobo
je v vodi približno $1,3$, v steklih okoli $1,4$--$1,9$ in v diamantu $2,4$.

Naj svetloba potuje po snovi z lomnim količnikom $n$. Potem lahko zapišemo čas $t$, 
ki ga svetloba potrebuje, da prepotuje določeno pot $s$, kot:
\begin{equation}
 dt = \frac{ds}{c} = \frac{ds}{c_0/n} = \frac{nds}{c_0}.
\label{eq:02_02}
\end{equation}
Fermatov teorem pravi, da svetloba potuje po poti, za katero je:
\beq
\frac{nds}{c_0} = \mathrm{min}.
\label{eq:02_03}
\eeq
V homogeni snovi svetloba torej potuje naravnost in ne spreminja smeri.
Vpeljemo pojem optične poti $S$, to je produkta fizične poti $s$ in lomnega količnika
na tej poti $n$. V splošnem se lomni količnik $n$ lahko s krajem spreminja, zato moramo
optično pot zapisati v diferencialni obliki in jo integrirati od začetne do končne točke v
prostoru. Dobimo: 
\boxeq{eq:02_04}{
S = \int_1^2 n(\mathbf{r})ds = \mathrm{min}.
}
Zapisani izraz imenujemo princip najmanjše optične poti oziroma najmanjše optične akcije. 
Problem je podoben principu najmanjše akcije v klasični mehaniki, zato včasih govorimo  
tudi o Langrangeevi oziroma Hamiltonovi optiki.
\begin{remark}
Princip najmanjše optične poti temelji na predpostavki, ki jo je privzel 
Fermat v zapisu svojega teorema. Isti pogoj lahko formalno izpeljemo z 
obravnavo valovnih front in žarkov v valovni optiki, če naredimo limito $\lambda \to 0$.
Tako lahko izhajajoč iz Maxwellovih enačb pokažemo pravilnost Fermatovega teorema.
\end{remark}

\begin{example}
{\bf Izpeljava lomnega zakona iz Fermatovega teorema.}
Naj svetlobni žarek vpada na ravno mejo dveh snovi. Levo od meje je snov 
z lomnim količnikom $n_1$, desno pa snov z lomnim količnikom $n_2$. Naj svetloba
potuje od točke 1, ki jo izberemo na oddaljenosti $z_1$ levo 
od meje, do točke 2 na oddaljenosti $z_2$ desno od meje. Vertikalna razdalja med obema
točkama naj bo $x$ (glej sliko~\ref{fig:01_FerLom}). 
\begin{figure}[ht]
\centering
\def\svgwidth{100truemm} 
\input{slike/02_FerLom.pdf_tex}
\caption{K izračunu loma na meji dveh sredstev}
\label{fig:01_FerLom}
\end{figure}

Naša naloga je poiskati točko na meji, skozi katero potuje žarek svetlobe,
ob pogoju, da je skupna optična pot od točke 1 do točke 2 najmanjša.
Celotno optično pot zapišemo kot:
\begin{equation}
S = n_1 s_1 + n_2 s_2 = n_1 \sqrt{x_1^2+z_1^2}\, +\, n_2 \sqrt{x_2^2+z_2^2}.
\label{eq:02_05}
\end{equation}
Izrazimo parameter $x_2$ z lego iskane točke $x_1$: 
\begin{equation}
S = n_1 \sqrt{x_1^2+z_1^2}\, +\, n_2 \sqrt{(x-x_1)^2+z_2^2}.
\label{eq:02_06}
\end{equation}
Najkrajšo optično pot izračunamo tako, da poiščemo vrednost $x_1$, 
pri kateri je odvod optične poti po $x_1$ enak nič. Zapišemo:
\begin{equation}
\frac{dS}{dx_1} = \frac{2 n_1 x_1}{2 \sqrt{x_1^2+z_1^2}}+
\frac{-2n_2 (x-x_1)}{2 \sqrt{(x-x_1)^2+z_2^2}} = 0.
\label{eq:02_07}
\end{equation}
Vpadni kot $\alpha$ vpeljemo glede na normalo na mejo snovi (slika~\ref{fig:01_FerLom}):
\begin{equation}
\sin \alpha = \frac{x_1}{\sqrt{x_1^2+z_1^2}}.
\label{eq:02_08}
\end{equation}
Vpeljemo še lomni kot $\beta$, ki je kot med smerjo žarka v drugi snovi in normalo na mejo snovi: 
\begin{equation}
\sin \beta = \frac{x_2}{\sqrt{x_2^2+z_2^2}} = \frac{(x-x_1)}{\sqrt{(x-x_1)^2+z_2^2}}.
\label{eq:02_09}
\end{equation}
Vstavimo enačbi~(\ref{eq:02_08}) in (\ref{eq:02_09}) v enačbo~(\ref{eq:02_07})
in zapišemo lomni zakon v obliki, kot jo poznamo:
\boxeq{eq:lomnizakon}{
n_1 \sin \alpha = n_2 \sin \beta.
}
Podobno lahko izpeljemo tudi odbojni zakon, tako da izračunamo 
najkrajšo optično pot med točkama 1 in 3 in dobimo:
\boxeq{eq:odbojnizakon}{
\tilde{\alpha} = \alpha.
}
\end{example}

\section{Žarkovna enačba}
Poglejmo, kako se lotimo računa najkrajše optične poti v primeru, ko 
meja med dvema snovema ni ostra, ampak se lomni količnik zvezno spreminja. 
Naj bo lomni količnik v splošnem funkcija kraja: $n = n(\mathbf{r}) = n(x,y,z)$.
Numerično se tega problema lotimo tako, da snov razdelimo
na majhne elemente in na mejah med njimi uporabimo lomni ali odbojni zakon.
Analitično problem rešujemo z uporabo Euler-Lagrangeeve enačbe za minimum
funkcionala optične poti $S$:
\begin{equation}
 S = \int_1^2 n(x,y,z) ds  = \int_1^2 n(x,y,z)\,|\mathbf{v}|\,dt  = 
 \int_1^2 n(x,y,z) \sqrt{\dot{x}^2+ \dot{y}^2+\dot{z}^2} dt,
 \label{eq:02_10}
\end{equation}
pri čemer pika označuje odvod posamezne koordinate po času. Integrand
enačbe predstavlja Langrangian $L$, ki je funkcija treh koordinat in njihovih
odvodov.
\begin{equation}
L(x, y, z, \dot{x}, \dot{y}, \dot{z}) = n(x,y,z) \sqrt{\dot{x}^2+ \dot{y}^2+\dot{z}^2}.
\label{eq:02_11}
\end{equation}
Lagrangian vstavimo v Euler-Lagrangeevo enačbo\footnote{~Glej npr. P. Prelovšek, {\it Klasična
mehanika}, skripta, 2013.} in za koordinato $x$ dobimo:
\begin{equation}
 \frac{d}{dt}\left(\frac{\partial L}{\partial \dot{x}}\right) - 
 \frac{\partial L}{\partial x} = 0.
 \label{eq:02_12}
\end{equation}
Podobni enačbi zapišemo tudi za $y$ in $z$. Vstavimo Langrangian (enačba~\ref{eq:02_11}) 
v enačbo~(\ref{eq:02_12}) in dobimo:
\begin{equation}
\frac{d}{dt}\left(n \frac{\dot{x}}{\sqrt{\dot{x}^2+ \dot{y}^2+\dot{z}^2}} \right)
 = \frac{\partial n}{\partial x}\sqrt{\dot{x}^2+ \dot{y}^2+\dot{z}^2}.
  \label{eq:02_13}
\end{equation}
Pomnožimo enačbo z $dt/ds$:
\begin{equation}
\frac{d}{dt}\left(n \frac{\dot{x}}{\sqrt{\dot{x}^2+ \dot{y}^2+\dot{z}^2}} \right)
\frac{dt}{ds}
 = \frac{\partial n}{\partial x}\frac{\sqrt{\dot{x}^2+ \dot{y}^2+\dot{z}^2}\,dt}{ds}.
  \label{eq:02_14}
\end{equation}
Na desni strani enačbe uporabimo zvezo:
\begin{equation}
 \sqrt{\dot{x}^2+ \dot{y}^2+\dot{z}^2} dt = ds,
 \label{eq:02_15}
\end{equation}
na levi strani pa verižno pravilo odvajanja, po katerem lahko pokrajšamo $dt$. Dobimo:
\begin{equation}
 \frac{d}{ds} \left( n \frac{dx}{dt\sqrt{\dot{x}^2+ \dot{y}^2+\dot{z}^2}} \right)=
 \frac{\partial n}{\partial x}.
 \label{eq:02_16}
\end{equation}
Enačbo~(\ref{eq:02_15}) uporabimo še enkrat v oklepaju na levi strani. Sledi:
\begin{equation}
 \frac{d}{ds} \left( n\, \frac{dx}{ds} \right)=
 \frac{\partial n}{\partial x}.
  \label{eq:02_17}
\end{equation}
Podobni enačbi zapišemo še za koordinati $y$ in $z$. Vse tri enačbe
združimo v enotno žarkovno enačbo:
\boxeq{eq:zarkovnaenacba}{
\nabla n = \frac{d}{ds} \left( n \frac{d\mathbf{r}}{ds}\right)\!.
}
Rešitev žarkovne enačbe $\mathbf{r}(s)$ poda trajektorijo
optičnega žarka.

\begin{example}
{\bf Žarkovna enačba v homogeni snovi.} Naj svetloba potuje po snovi,
v kateri je lomni količnik konstanten. Potem je $\nabla n = 0$ in 
\begin{equation}
0 = \frac{d}{ds}\left( n \frac{d\mathbf{r}}{ds} \right) = 
n \frac{d^2 \mathbf{r}}{ds^2}.
 \label{eq:02_18}
\end{equation}
Rešitev te enačbe je ravni žarek:
\begin{equation}
 \mathbf{r} = \mathbf{a}_0+\mathbf{a}_1\,s,
  \label{eq:02_19}
\end{equation}
pri čemer vektor $\mathbf{a}_0$ opisuje lego začetne točke, 
vektor $\mathbf{a}_1$ pa ima smer od začetne do končne točke.
\end{example}

V optiki pogosto privzamemo, da smer širjenja svetlobe le malo 
odstopa od neke dane smeri. Naj bo to smer $z$, ki jo imenujemo
optična os sistema, trajektorijo žarka pa opazujemo v ravnini $xz$ 
(slika~\ref{fig:01_FerOs}). Lomni količnik naj bo funkcija $n= n(x)$.
\begin{figure}[ht]
\centering
\def\svgwidth{90truemm} 
\input{slike/02_FerOs.pdf_tex}
\caption{K zapisu žarkovne enačbe v obosnem približku}
\label{fig:01_FerOs}
\end{figure}

Predpostavko, da smer žarka le malo odstopa od smeri osi $z$, 
matematično zapišemo s pogojem:
\begin{equation}
 \frac{dx}{dz}\ll 1.
  \label{eq:02_20}
\end{equation}
Iz tega sledi, da za naklon žarka $\vartheta$, ki ga izračunamo kot:
\begin{equation}
\vartheta = \frac{dx}{dz}
 \label{eq:02_21}
\end{equation}
velja $\sin \vartheta \approx \tan \vartheta \approx \vartheta \ll 1$.
Upoštevajoč enačbo~(\ref{eq:02_20}) zapišemo:
\begin{equation}
ds = \sqrt{dx^2+dz^2} = \sqrt{\left(\left(\frac{dx}{dz}\right)^2 + 1\right)}\,\,dz \approx dz.
\label{eq:02_22}
\end{equation}
Predpostavki, da se žarek širi približno vzdolž osi $z$, pravimo
obosni ali paraksialni približek. V nadaljevanju se ga bomo še velikokrat 
posluževali. Žarkovno enačbo (enačba~\ref{eq:zarkovnaenacba}) v obosnem približku
zapišemo kot:
\begin{equation}
 \frac{dn}{dx} = \frac{d}{dz}\left(n(x)\,\frac{dx}{dz}\right) = n(x)\,\frac{d^2x}{dz^2}
  \label{eq:02_23}
\end{equation}
oziroma
\begin{equation}
\frac{d^2x}{dz^2} = \frac{1}{n(x)} \frac{dn}{dx}.
 \label{eq:02_24}
\end{equation}

\begin{example}
{\bf Žarkovna enačba v snovi s paraboličnim profilom
lomnega količnika.} Izračunajmo trajektorijo žarka svetlobe v 
obosnem približku v snovi, v kateri se lomni količnik parabolično 
spreminja z oddaljenostjo od osi $z$.  Odvisnost lomnega količnika 
od oddaljenosti $r$ od osi $z$ zapišemo kot:
\begin{equation}
 n(x) = n_0 \left(1-\frac{\alpha^2\,x^2}{2}\right)\!,
 \label{eq:02_25}
\end{equation}
pri čemer je $\alpha x$ majhen.
Vstavimo izraz za lomni količnik (enačba~\ref{eq:02_25}) v obosni približek
žarkovne enačbe (enačba~\ref{eq:02_24}) in dobimo:
\begin{equation}
 \frac{d^2x}{dz^2}= \frac{1}{n(x)}\frac{dn}{dz} = 
 \frac{-n_0 \alpha^2x}{n_0 \left(1-\alpha^2\,x^2/2\right)} 
 \approx -\frac{n_0\alpha^2x}{n_0} = -\alpha^2x.
 \label{eq:02_26}
\end{equation}
Enačbo~(\ref{eq:02_26}) preprosto rešimo in dobimo splošno 
obliko trajektorije:
\begin{equation}
 x(z) = A \cos (\alpha x) + B \sin (\alpha x).
  \label{eq:02_27}
\end{equation}
Naj bo na začetku pri $z=0$ žarek na oddaljenosti $x_0$ 
in naj se širi pod kotom $\vartheta_0$. 
Z upoštevanjem začetnih pogojev dobimo rešitev:
\begin{equation}
x (z) = x_0 \cos(\alpha x) + \frac{\vartheta_0}{\alpha} \sin (\alpha x).
 \label{eq:02_28}
\end{equation}
Rešitve obosnega približka žarkovne enačbe v snovi s paraboličnim profilom 
so torej oscilatorne funkcije s periodo $2\pi/\alpha$ (glej sliko~\ref{fig:01_FerPar}).
\begin{figure}[ht]
\centering
\def\svgwidth{100truemm} 
\input{slike/02_FerPar.pdf_tex}
\caption{Rešitev žarkovne enačbe v snovi s paraboličnim profilom je periodična. Narisanih
je šest žarkov, pri katerih je $x_0=0$, razlikujejo pa se v $\vartheta_0$.}
\label{fig:01_FerPar}
\end{figure}

\begin{figure}[ht]
\centering
\includegraphics[width=10truecm]{slike/02_Sladkor.jpg}
\caption{Potek žarka v vodni raztopini sladkorja, v kateri koncentracija sladkorja in z
njo lomni količnik naraščata z globino. Vodi smo dodali fluorescein, da vidimo potek žarka.}
\label{fig:01_Sladkor}
\end{figure}
\end{example}

\section{Transformacije žarkov z matrikami ABCD}
V prejšnjem razdelku smo zapisali žarkovno enačbo v obosnem 
približku. Ključna parametra za opis trajektorije žarka sta 
oddaljenost od optične osi $z$, ki jo označimo z $x$, in naklonski 
kot žarka glede na optično os, ki ga označimo s $\vartheta$.
Naša naloga je povezati dve točki žarka na različnih mestih v
prostoru in zapisati preslikavo med njima, če je med točkama
optični medij oziroma nek optični element. 
\begin{figure}[ht]
\centering
\def\svgwidth{90truemm} 
\input{slike/02_ABCD0.pdf_tex}
\caption{Pri danem $z$ žarek opišemo z lego $x$ in smerjo $\vartheta$. Spremembo
teh dveh parametrov opišemo s preslikavo.}
\label{fig:01_ABCD0}
\end{figure}

V splošnem vrednosti lege in naklona v točki 2 izrazimo kot 
linearno kombinacijo prvotnih vrednosti v točki 1:
\begin{align}
 x_2 = & A x_1 + B \vartheta_1 \qquad \mathrm{in}  \label{eq:02_29}\\
 \vartheta_2 = & C x_1 + D\vartheta_1.
 \label{eq:02_30}
\end{align}
Če združimo parametra $x$ in $\vartheta$ v neki točki prostora
v dvodimenzionalni vektor,
lahko preslikavo strnjeno zapišemo v matrični obliki:
\boxeq{eq:02_31}{
\left[\begin{array}{c}
x_2\\
\vartheta_2
\end{array}\right] = 
\left[\begin{array}{cc}
A& B\\
C&D
\end{array}\right]
\left[\begin{array}{c}
x_1\\
\vartheta_1
\end{array}\right]
= M \left[\begin{array}{c}
x_1\\
\vartheta_1
\end{array}\right]\!\!.
}
Matriko $M$ imenujemo transformacijska oziroma prehodna matrika 
vmesnega optičnega medija ali optičnega elementa. 

\begin{example}
\label{ex:ML}
{\bf Matrika ABCD za premik v homogeni snovi s konstantnim lomnim količnikom.} 
Prvi primer naj bo homogena snov, v kateri je lomni količnik konstanten in enak $n$. 
V točki 1 pri $z_1$ naj bosta komponenti vektorja enaki $x_1$ in 
$\vartheta_1$. Poiščimo vrednosti komponent vektorja, 
če se premaknemo za $L$ vzdolž optične osi sistema do $z_2$ 
(slika~\ref{fig:01_ABCD1}). 
\begin{figure}[ht]
\centering
\def\svgwidth{70truemm} 
\input{slike/02_ABCD1.pdf_tex}
\caption{K izračunu matrike ABCD za premik v homogeni snovi}
\label{fig:01_ABCD1}
\end{figure}

Naklon žarka se pri premiku ne spremeni, zato ostane $\vartheta_2 = \vartheta_1$. Spremeni
pa se odmik od optične osi, saj žarek potuje pod določenim kotom. 
Vrednost $x_2$ zapišemo kot:
\begin{equation}
 x_2 = x_1 + (z_2-z_1)\vartheta_1 = x_1 + L\vartheta_1.
\label{eq:02_32}
\end{equation}
Potem zapišemo sistem enačb:
\begin{align}
 x_2 =& 1\cdot x_1 + L\cdot \vartheta_1 \qquad \mathrm{in} \label{eq:02_33}\\
 \vartheta_2 =& 0\cdot x_1 + 1\cdot \vartheta_1.
 \label{eq:02_34}
\end{align}
Iz zapisa razberemo koeficiente matrike $M$ za homogeno snov dolžine $L$:
\begin{equation}
 M = \left[\begin{array}{cc}
1& L\\
0&1
\end{array}\right]\!\!.
 \label{eq:02_35}
\end{equation}
\end{example}

\begin{example}
\label{ex:Mmeja}
{\bf Matrika ABCD za prehod skozi ravno mejo med dvema snovema.} Naj svetloba vpada na 
mejo dveh snovi z različnima lomnima količnikoma $n_1$ in $n_2$. Za izračun 
prehodne matrike izberemo dve točki na žarku: eno tik pred mejo in eno tik za njo. 
Lega žarka se ob tem ne spremeni in $x_2 = x_1$. 
Spremeni pa se naklon žarka. 
\begin{figure}[ht]
\centering
\def\svgwidth{70truemm} 
\input{slike/02_ABCD2.pdf_tex}
\caption{K izračunu matrike ABCD za prehod skozi ravno mejo med dvema snovema}
\label{fig:01_ABCD2}
\end{figure}

Za majhne naklone lahko lomni zakon
(enačba~\ref{eq:lomnizakon}) razvijemo in dobimo:
\begin{equation}
n_1 \vartheta_1 = n_2 \vartheta_2.
 \label{eq:02_36}
\end{equation}
Sistem enačb je potem:
\begin{align}
 x_2 =& 1\cdot x_1 + 0\cdot \vartheta_1 \qquad \mathrm{in} \label{eq:02_37}\\
 \vartheta_2 =& 0\cdot x_1 + \frac{n_1}{n_2}\cdot \vartheta_1.
 \label{eq:02_38}
\end{align}
Matrika $M$ za prehod skozi mejo med dvema snovema je:
\begin{equation}
 M = \left[\begin{array}{cc}
1& 0\\
0&\frac{n_1}{n_2}
\end{array}\right]\!\!.
\label{eq:02_39}
\end{equation}
\end{example}

\begin{example}
\label{ex:MUmeja}
{\bf Matrika ABCD za prehod skozi ukrivljeno mejo med dvema snovema.} V prejšnjem 
primeru je bila meja med snovema z lomnima količnikoma $n_1$ in $n_2$ ravna, zdaj pa 
poglejmo še matriko za prehod skozi ukrivljeno mejo. Krivinski radij mejne ploskve
naj bo $R$, pri čemer $R$ štejemo pozitivno, če je meja konveksna glede na smer
naraščajočega $z$, in negativno, če je meja konkavna. Ponovno izberemo prvo točko
tik pred mejo, drugo pa tik za mejo. To pomeni, da se oddaljenost od optične osi
pri prehodu ohranja in $x_1 = x_2$. 

Izračun naklona žarka po prehodu je malo bolj zapleten. Za zapis lomnega 
zakona moramo namreč vpeljati kote glede na normalo na
mejo. Vpadni kot je tako $\alpha = \vartheta_1 + \varphi$, lomni kot pa 
$\beta = \vartheta_2 + \varphi$ (glej sliko~\ref{fig:01_ABCD3}).
S slike razberemo, da velja zveza:
\begin{equation}
 \sin \varphi = \frac{x}{R} \approx \varphi,
 \label{eq:02_40}
\end{equation}
pri čemer smo privzeli, da so žarki blizu optične osi. 
\begin{figure}[!h]
\centering
\def\svgwidth{70truemm} 
\input{slike/02_ABCD3.pdf_tex}
\caption{K izračunu matrike ABCD za prehod skozi ukrivljeno mejo med dvema snovema}
\label{fig:01_ABCD3}
\end{figure}

Zapišemo lomni zakon,
pri čemer privzamemo, da so vsi koti majhni:
\begin{equation}
n_1 (\vartheta_1 + \varphi) = n_2 (\vartheta_2 + \varphi).
\label{eq:02_41}
\end{equation}
Od tod izračunamo kot $\vartheta_2$:
\begin{equation}
 \vartheta_2 = \frac{n_1-n_2}{n_2}\varphi + \frac{n_1}{n_2} \vartheta_1.
 \label{eq:02_42}
 \end{equation}
Če $\varphi$ izrazimo iz enačbe~(\ref{eq:02_40}), zapišemo transformacijo koordinat:
\begin{align}
 x_2 =& 1\cdot x_1 + 0\cdot \vartheta_1 \qquad \mathrm{in} \\
 \vartheta_2 =& \frac{n_1-n_2}{n_2R}\cdot x_1 + \frac{n_1}{n_2}\cdot \vartheta_1.
 \label{eq:02_43}
\end{align}
Razberemo matriko $M$ za prehod skozi ukrivljeno mejo med dvema snovema:
\begin{equation}
 M = \left[\begin{array}{cc}
1& 0\\
\frac{n_1-n_2}{n_2R}&\frac{n_1}{n_2}
\end{array}\right]\!\!.
 \label{eq:02_44}
\end{equation}
V limitnem primeru, ko gre $R \to \infty$, se matrika prevede na matriko za
prehod skozi ravno mejo (glej primer~\ref{ex:Mmeja}).
\end{example}

Izračunajmo še determinanto izpeljanih matrik ABCD. Na splošno velja, da
je determinanta matrike ABCD enaka razmerju lomnih količnikov začetne in 
končne snovi. Če je lomni količnik snovi na koncu enak
kot na začetku (primer~\ref{ex:ML}), je $\det(M) = 1$, v nasprotnem primeru 
(primera~\ref{ex:Mmeja} in \ref{ex:MUmeja}) je $\det(M) = n_1/n_2$.

Do zdaj smo obravnavali primere posameznih prehodov. Poglejmo, kako izračunamo
prehodno matriko za primer, če svetloba potuje skozi več elementov zapored. V tem
primeru se pokaže praktičnost zapisa z matrikami, saj ob prehodu svetlobe skozi 
več elementov matrike za te elemente preprosto zmnožimo. Paziti moramo seveda
na vrsti red: matriko za element, na katerega vpade svetloba najprej, zapišemo
najbolj desno, to je najbliže vektorju, ki opisuje vpadno svetlobo. Celoten prehod
spet opiše ena sama matrika:
\begin{equation}
\left[\begin{array}{c}
x_N\\
\vartheta_N
\end{array}\right] 
= \tilde{M} \left[\begin{array}{c}
x_1\\
\vartheta_1
\end{array}\right]\!\!,
 \label{eq:02_45}
\end{equation}
pri čemer je:
\begin{equation}
\tilde{M} = M_N \cdot M_{N-1}~...~M_2 \cdot M_1.
 \label{eq:02_46}
\end{equation}
\begin{figure}[ht]
\centering
\def\svgwidth{100truemm} 
\input{slike/02_MMM.pdf_tex}
\caption{Prehod žarka skozi več optičnih elementov zapišemo kot produkt matrik posameznih prehodov.
Pri tem moramo paziti na vrstni red zapisa matrik.}
\label{fig:01_MMM}
\end{figure}

\section{Leče, zrcala}

Nujno dodati poglavja klasične geometrijske optike.

Leča? Poglejmo, kaj se zgodi s svetlobo ob prehodu skozi
tanko lečo s krivinskima radijema ploskev $R_1$ in $R_2$. Z uporabo matrik ABCD zapišemo:

Za primerjavo naredimo še prehod skozi lečo brez matrik(?, glej Strnad). 
Debela leča, lečje. 

Najpreprostejša optični element je zrcalo. Najprej z odbojem, potem z matriko?

Hm, prizma? Ali je potrebna? Ne moremo je opisati z matrikami. Za disperzijo kasneje?

\section{Optične naprave}
mikroskop, daljnogled, spektroskop, oko, korekcijska očala. Kaj astronomskega? Nujno dodati.


