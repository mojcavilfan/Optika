%\chapterimage{Geometrijska.jpg} % Chapter heading image

\chapter{Odboj in lom}
Dva izmed najosnovnejših optičnih pojavov sta odboj in lom valovanja, ki vpada na mejo
dveh sredstev. V tem poglavju bomo podrobneje spoznali, kako se svetloba odbije in lomi ter
kako se pri tem spremenita amplituda in faza valovanja. Opisali bomo totalni odboj,
vpeljali Brewstrov kot in na koncu na hitro spoznali še odboj svetlobe na meji s kovino.

\section{Robni pogoji na meji dveh dielektrikov}
Že geometrijska optika pove, kako se svetloba na meji dveh sredstev odbija 
in lomi. Vemo, da je odbojni kot enak vpadnemu (enačba~\ref{eq:odbojnizakon}), 
smer lomljenega žarka pa je odvisna od lomnih količnikov snovi (enačba~\ref{eq:lomnizakon}). 
Vendar geometrijska optika ne pove ničesar o amplitudah in fazah odbitega 
in prepuščenega vala. Za to potrebujemo valovno obravnavo svetlobe. 

Izhajamo iz Maxwellovih enačb (enačbe~\ref{eq:Maxwell1}--\ref{eq:Maxwell4}), ki 
opisujejo elektromagnetno valovanje v snovi. Za obravnavo prehoda
skozi mejo dveh snovi potrebujemo še ustrezne robne pogoje. Zapišimo jih
za mejo med snovema z lomnima količnikoma
$n_1$ in $n_2$. Na meji naj ne bo električnih tokov ali površinskih nabojev. 
V prvi snovi naj bo gostota električnega polja enaka $\mathbf{D}_1$, v drugi 
pa $\mathbf{D}_2$.

Zapišimo Gaussov zakon (enačba~\ref{eq:Maxwell3}) v integralni obliki:
\beq
\oint \mathbf{D}\cdot d\mathbf{S} = 0.
\label{eq:04_01}
\eeq
\begin{figure}[ht]
\centering
\def\svgwidth{120truemm} 
\input{slike/04_RP.pdf_tex}
\caption{K izpeljavi robnih pogojev na meji dveh snovi. Prvi robni pogoj je ohranitev
pravokotne komponente $\mathbf{D}$ (a), drugi pa ohranitev tangentne komponente 
$\mathbf{E}$ (b).}
\label{fig:04_RP}
\end{figure}

Sklenjena ploskev, po kateri integriramo, naj bo kvader z dvema stranicama
vzporednima mejni ploskvi (slika~\ref{fig:04_RP}\,a). 
Ko gre višina kvadra $h \to 0$, ostaneta le še dva prispevka k integralu:
\beq
\mathbf{D}_1 \cdot \mathbf{s}_0 + \mathbf{D}_2 \cdot (-\mathbf{s}_0) = 0,
\label{eq:04_02}
\eeq
pri čemer vektor $\mathbf{s}_0 $ opisuje normalo na ploskev kvadra in hkrati
tudi normalo na mejno ravnino. Sledi:
\boxeq{eq:RPD}{
\left( \mathbf{D}_2-\mathbf{D}_1 \right) \cdot \mathbf{s}_0 = 0 \qquad \mathrm{oziroma} \qquad
D_{1\perp} = D_{2\perp}.
}
Pri prehodu skozi mejo dveh snovi se torej ohranja normalna komponenta gostote
električnega polja. Povsem enak račun lahko naredimo za gostoto magnetnega polja, 
če izhajamo iz enačbe:
\beq
\oint \mathbf{B}\cdot d\mathbf{S} = 0.
\label{eq:04_03}
\eeq
Pri prehodu skozi mejo dveh snovi se tako ohranja normalna komponeneta
gostote magnetnega polja:
\boxeq{eq:RPB}{
\left( \mathbf{B}_2-\mathbf{B}_1 \right) \cdot \mathbf{s}_0 = 0 \qquad 
\mathrm{oziroma} \qquad B_{1\perp} = B_{2\perp}.
}

Naredimo zdaj še pravokotno zanko, ki leži v ravnini, pravokotni na mejo dveh 
sredstev (slika~\ref{fig:04_RP}\,b). Zanko 
lahko vedno izberemo tako, da leži v ravnini, ki jo tvorita vektorja normale na 
mejo $\mathbf{s}_0$ in jakosti električnega polja $\mathbf{E}$. 
Zapišemo Faradayev zakon (enačba~\ref{eq:Maxwell2}) v integralni obliki, pri 
čemer integriramo po sklenjeni zanki:
\beq
\oint \mathbf{E}\cdot d\mathbf{l} = - \frac{\partial}{\partial t}\int \mathbf{B}\cdot d\mathbf{S}.
\label{eq:04_04}
\eeq
Višino zanke naredimo kar se da majhno ($h \to 0$), zato ostaneta le še 
integrala po zgornji in spodnji stranici:
\beq
\mathbf{E}_1 \cdot d\mathbf{l} - \mathbf{E}_2 \cdot d\mathbf{l} = 0.
\label{eq:04_05}
\eeq
Ohranja se torej projekcija vektorja jakosti električnega polja na mejo. 
Ker je vektor $d\mathbf{l}$ vzporeden z mejo, vektor $d\mathbf{s}_0$ pa 
pravokoten nanjo, lahko zgornji izraz zapišemo kot:
\boxeq{eq:RPE}{
\left( \mathbf{E}_2-\mathbf{E}_1 \right) \times \mathbf{s}_0 = 0 \qquad \mathrm{oziroma} \qquad
E_{1 \parallel} = E_{2 \parallel}.
}
Podobno izpeljemo tudi robni pogoj za ohranitev tangentne komponente
jakosti magnetnega polja:
\boxeq{eq:RPH}{
\left( \mathbf{H}_2-\mathbf{H}_1 \right) \times \mathbf{s}_0 = 0 \qquad \mathrm{oziroma} \qquad
H_{1\parallel} = H_{2\parallel}.
}

\section{Odbojni in lomni zakon}
Zamislimo si ravno potujoče sinusno elektromagnetno valovanje, ki vpada na mejo dveh snovi
z lomnima količnikoma $n_1$ in $n_2$. Vemo, da se vpadno valovanje razdeli na odbito in prepuščeno (lomljeno) valovanje. Zanima nas, kakšne lastnosti morata imeti odbito in prepuščeno valovanje, da zadostita robnim pogojem. 
\begin{figure}[ht]
\centering
\def\svgwidth{120truemm} 
\input{slike/04_lom.pdf_tex}
\caption{K izpeljavi odboja in loma na meji dveh snovi (a). Pri odboju in lomu
se ohranja komponenta valovnega vektorja, ki je vzporedna z mejno ploskvijo (b).}
\label{fig:04_lom}
\end{figure}

Naj žarek svetlobe vpada na mejno ravnino pod kotom $\alpha$ glede na normalo na mejo (glej
sliko~\ref{fig:04_lom}). Njegov valovni vektor naj bo $\mathbf{k}_i$, 
krožna frekvenca $\omega_i$, faza $\delta_i$ in amplituda $\mathbf{E}_{0i}$. 
Pripadajoče valovno število
je $k_i = \omega_i n_1/c_0$. Vpadni val na splošno zapišemo kot:
\beq
\mathbf{E}_i = \mathbf{E}_{0i} e^{i\mathbf{k}_i\cdot \mathbf{r} - i \omega_i t + i \delta_i}.
\label{eq:04_06}
\eeq
Podobno zapišemo odbiti val z valovnim vektorjem $\mathbf{k}_r$, 
krožno frekvenco $\omega_r$, fazo $\delta_r$ in amplitudo $\mathbf{E}_{0r}$: 
\beq
\mathbf{E}_r = \mathbf{E}_{0r} e^{i\mathbf{k}_r\cdot \mathbf{r} - i \omega_r t + i \delta_r}
\label{eq:04_07}
\eeq
in prepuščeni val z valovnim vektorjem $\mathbf{k}_t$, 
krožno frekvenco $\omega_t$, fazo $\delta_t$ in amplitudo $\mathbf{E}_{0t}$:
\beq
\mathbf{E}_t = \mathbf{E}_{0t} e^{i\mathbf{k}_t\cdot \mathbf{r} - i \omega_i t + i \delta_t}.
\label{eq:04_08}
\eeq
Mejo med snovema postavimo v ravnino $z=0$. Iz robnega pogoja za ohranitev tangentne
komponente vektorja $\mathbf{E}$ (enačba~\ref{eq:RPE}) dobimo zvezo:
\beq
\mathbf{E}_{i\parallel} + \mathbf{E}_{r\parallel} = \mathbf{E}_{t\parallel},
\label{eq:04_09}
\eeq
oziroma izpisano:
\beq
\mathbf{E}_{0i\parallel} e^{ik_{ix}x+ik_{iy}y - i \omega_i t + i \delta_i}+
\mathbf{E}_{0r\parallel} e^{ik_{rx}x+ik_{ry}y - i \omega_r t + i \delta_r} =
\mathbf{E}_{0t\parallel} e^{ik_{tx}x+ik_{ty}y - i \omega_i t + i \delta_t}.
\label{eq:04_10}
\eeq
Pri tem smo upoštevali $z=0$. Zapisana zveza mora veljati 
ob vseh časih $t$ in za vse vrednosti $x$ in $y$. Vzemimo najprej $x=y=0$ in $t=0$. 
Od tod sledi, da so vse fazne enake in $\delta_i = \delta_r = \delta_t$.
Drugi pogoj poglejmo pri $x=y=0$ in poljubnem času $t>0$. Robni pogoj 
je izpolnjen le v primeru, da velja 
\boxeq{eq:omegakonst}{
\omega_i =
\omega_r = \omega_t
}
in se torej frekvenca pri odboju in lomu ohranja. 

Obravnavajmo zdaj primer $t=0$ pri poljubnem paru $x,y$ oziroma vektorju $\mathbf{r}$
v mejni ravnini. Da bo lahko robni pogoj izpolnjen za vsak $\mathbf{r}$,
mora veljati:
\beq
\mathbf{k}_i\cdot \mathbf{r} = \mathbf{k}_r\cdot \mathbf{r} = \mathbf{k}_t\cdot \mathbf{r} = \mathrm{konst.}
\label{eq:04_11}
\eeq
Iz tega sledi, da so projekcije vseh treh valovnih vektorjev na mejno
ravnino enake. Pomembnejša je ugotovitev, da ležijo valovni vektorji vpadnega, odbitega
in lomljenega valovanja vedno v eni ravnini. Imenujemo jo vpadna ravnina. 

Navadno izberemo, da je vpadna ravnina ravnina $xz$. Potem zapišemo 
valovne vektorje vpadne, odbite in prepuščene svetobe kot:
\begin{align}
\mathbf{k}_i  =& \left( \sin\alpha, 0, \cos \alpha\right) \frac{\omega}{c_0} n_1, \label{eq:04_12}\\
\mathbf{k}_r  =& \left( \sin\tilde{\alpha}, 0, -\cos \tilde{\alpha}\right) \frac{\omega}{c_0} n_1,\label{eq:04_13}\\
\mathbf{k}_t  =& \left( \sin\beta, 0, \cos \beta\right) \frac{\omega}{c_0} n_2.\label{eq:04_14}
\end{align}
Pri tem smo poleg vpadnega kota $\alpha$ vpeljali še odbojni kot $\alpha'$ in lomni
kot $\beta$. Iz enačbe~(\ref{eq:04_11}) neposredno sledi, da se mora ohranjati komponenta
valovnega vektorja, ki je vzporedna z mejno ravnino:
\beq
k_{ix} = k_{rx} = k_{tx}.
\label{eq:04_15}
\eeq
Najprej poglejmo prvo enakost ($k_{ix} = k_{rx}$). Vstavimo komponenti valovnih
vektorjev (enačbi~\ref{eq:04_12} in \ref{eq:04_13}) in dobimo:
\beq
\frac{\omega}{c_0} n_1 \sin \alpha  =  \frac{\omega}{c_0} n_1 \sin\tilde{\alpha},
\label{eq:04_16}
\eeq
od koder sledi odbojni zakon, ki pravi, da je odbojni kot enak vpadnemu:
\boxeq{eq:odbojni}{
\tilde\alpha = \alpha.
}

Poglejmo zdaj še drugo enakost ($k_{ix} = k_{tx}$). Vstavimo komponenti
valovnih vektorjev (enačbi~\ref{eq:04_12} in \ref{eq:04_14}) in dobimo:
\beq
\frac{\omega}{c_0} n_1 \sin \alpha  = \frac{\omega}{c_0} n_2\sin\beta.
\label{eq:04_17}
\eeq
Od tu sledi lomni zakon:
\boxeq{eq:04_18}{
n_1 \sin \alpha = n_2 \sin \beta.
}

Z odbojnim in lomnim zakonom smo izpolnili ujemanje faz v enačbi~(\ref{eq:04_10}). 
Hribi in doline vpadnega, odbitega in prepuščenega valovanja vzdolž meje 
potujejo enako hitro. Dodatno informacijo o valovanju dobimo, če
uskladimo še amplitude jakosti električnih polj. Preden izpeljemo
zveze, ki povezujejo razmerja med amplitudami, vpeljimo koordinatni sistem
in se dogovorimo o oznaki polarizacij.

Za obravnavo izberemo dve ortogonalni linearni polarizaciji, saj lahko potem
poljubno polarizacijo sestavimo kot kombinacijo teh dveh.
Mejna ravnina naj bo kot do zdaj ravnina $xy$, os $z$ pa naj bo obrnjena navzdol. 
V prvem primeru naj bo jakost elektičnega polja
vpadne, odbite in prepuščene svetlobe vzporedna z osjo $y$ in tako pravokotna
na vpadno ravnino (slika~\ref{fig:04_tetm}\,a). 
Tako valovanje imenujemo transverzalno električno (TE) valovanje. 
V literaturi pogosto najdemo oznaki $s$ (kot {\it senkrecht}, pravokotno) ali $\sigma$.

\begin{figure}[ht]
\centering
\def\svgwidth{140truemm} 
\input{slike/04_tetm.pdf_tex}
\caption{V primeru transverzalne električne (TE) polarizacije leži jakost
električnega polja pravokotno na vpadno ravnino (a),
pri transverzalni magnetni (TM) polarizaciji pa leži jakost električnega
polja v vpadni ravnini (b).}
\label{fig:04_tetm}
\end{figure}

V drugem primeru ležijo jakosti električnega polja v 
vpadni ravnini (slika~\ref{fig:04_tetm}\,b), gostote magnetnega polja 
pa so pravokotne na njih. To polarizacijo zato poimenujemo 
transverzalna magnetna (TM) polarizacija. Uporabljajo se tudi oznake
$p$ (kot {\it parallel}, vzporedno) ali $\pi$. 
Omenjena primera bomo obravnavali ločeno, začenši s TE polarizacijo. V 
obeh primerih se bomo omejili na izotropne in nemagnetne snovi ($\mu=1$).

\section{Fresnelove enačbe za transverzalno električno valovanje}
Osnovna zahteva na meji dveh snovi je izpolnjevanje robnih
pogojev, po katerih se na meji ohranja tangentna komponenta jakosti električnega
polja. Ker so vse tri jakosti vzporedne z osjo $y$, velja preprosta zveza:
\beq
E_{0i} + E_{0r}= E_{0t}
\label{eq:04_19}
\eeq
Pri tem smo upoštevali, da je polje v prvi snovi superpozicija vpadnega in odbitega
polja, v drugi snovi pa je samo prepuščeno valovanje. Drugi robni pogoj, ki zahteva
ohranitev normalne komponente gostote električnega polja (enačba~\ref{eq:RPD}) je 
vedno izpolnjen, saj je ta komponenta v obeh snoveh identično enaka nič. 
Tretji robni pogoj (enačba~\ref{eq:RPH}) zahteva
ohranitev tangentne komponente jakosti magnetnega polja. Privzamemo, da
sta snovi nemagnetni $\mu=1$ in dobimo:
\beq
-B_{0i}\cos \alpha + B_{0r}\cos \alpha = -B_{0t}\cos \beta.
\label{eq:04_20}
\eeq
Upoštevajoč zvezo med amplitudama električnega in magnetnega 
polja $E_0 = B_0 c = B_0 c_0/n$  (enačba~\ref{eq:EBc}) zapišemo robni pogoj kot:
\beq
-E_{0i}n_1\cos \alpha + E_{0r}n_1\cos \alpha = -E_{0t}n_2\cos \beta,
\label{eq:04_21}
\eeq 
oziroma:
\beq
n_1 E_{0i} \cos \alpha  -n_1 E_{0r}\cos \alpha = n_2 E_{0t}\cos \beta.
\label{eq:04_22}
\eeq
Na koncu zapišemo še četrti robni pogoj, ohranitev normalne komponente
gostote magnetnega polja (enačba~\ref{eq:RPB}):
\beq
B_{0i} \sin \alpha  + B_{0r}\sin \alpha = B_{0t}\sin \beta.
\label{eq:04_23}
\eeq
Ponovno uporabimo zvezo med amplitudami (enačba~\ref{eq:EBc}) in dobimo:
\beq
n_1  \sin \alpha \left( E_{0i} + E_{0r} \right) = n_2 E_{0t} \sin \beta.
\label{eq:04_24}
\eeq
Upoštevamo še lomni zakon (enačba~\ref{eq:04_18}) in dobimo:
\beq
E_{0i} + E_{0r} = E_{0t}.
\label{eq:04_25}
\eeq
Ta pogoj je ekvivalenten prvemu pogoju (enačba~\ref{eq:04_19}). Iz robnih
pogojev tako dobimo dve neodvisni enačbi (enačbi~\ref{eq:04_19} in \ref{eq:04_22}) 
za tri neznake: $E_{0i}$, $E_{0r}$ in $E_{0t}$. Navadno izrazimo amplitudi
odbitega in prepuščenega valovanja z amplitudo vpadnega valovanja in tako določimo 
relativno amplitudo odbite in prepuščene svetlobe. Izračunajmo ju.

Najprej pomnožimo enačbo~(\ref{eq:04_19}) z $n_2 \cos \beta$ in od nje odštejemo
enačbo~(\ref{eq:04_22}). Dobimo:
\beq
\left( n_2 \cos \beta-n_1 \cos \alpha\right) E_{0i} + 
\left( n_2 \cos \beta+n_1 \cos \alpha\right) E_{0r} = 0.
\label{eq:04_26}
\eeq
Vpeljemo amplitudno odbojnost $r = E_{0r}/E_{0i}$ 
kot razmerje odbite in vpadne amplitude valovanja. Iz enačbe~(\ref{eq:04_26}) sledi:
\boxeq{eq:TEr}{
r  = \frac{n_1 \cos \alpha - n_2 \cos \beta}
{n_1 \cos \alpha + n_2 \cos \beta},
}
pri čemer $\alpha$ in $\beta$ označujeta vpadni in lomni kot. Izraz za $r$
lahko z porabo lomnega zakona preoblikujemo:
\beq
r = \frac{\frac{n_1}{n_2} \cos \alpha - \cos \beta}{\frac{n_1}{n_2} \cos \alpha - \cos \beta} = 
\frac{\frac{\sin \beta}{\sin \alpha} \cos \alpha - \cos \beta}
{\frac{\sin \beta}{\sin \alpha} \cos \alpha - \cos \beta} = \frac{\sin \beta \cos \alpha -
\cos \beta \sin \alpha}{\sin \beta \cos \alpha + \cos \beta \sin \alpha} = 
-\frac{\sin (\alpha -\beta)}{\sin (\alpha + \beta)}.
\label{eq:04_27}
\eeq

Izračunajmo še amplitudno prepustnost $t = E_{0t}/E_{0i}$, ki jo vpeljemo
kot razmerje amplitud prepuščenega in vpadnega valovanja. Izhajamo
iz prvega robnega pogoja (enačba~\ref{eq:04_19}) in zapišemo:
\beq
t = \frac{E_{0t}}{E_{0i}} = \frac{E_{0i} + E_{0r}}{E_{0i}} = 1+ r.
\label{eq:04_28}
\eeq
Amplitudna prepustnost je tako:
\boxeq{eq:TEt}{
t = 1+r = \frac{2n_1\cos \alpha}{n_1 \cos \alpha + n_2 \cos \beta}.
}

\section{Fresnelove enačbe za transverzalno magnetno valovanje}
Izračunajmo še amplitudno odbdojnost in prepustnost za 
transverzalno magnetno polarizirano valovanje. Tudi v tem primeru začnemo
z robnimi pogoji. Na meji dveh sredstev se ohranja tangentna 
komponenta jakosti magnetnega polja (enačba~\ref{eq:RPH}):
\beq
H_{0i} + H_{0r}= H_{0t}.
\label{eq:04_29}
\eeq
Privzamemo, da so snovi nemagnetne in velja: $\mu_1 = \mu_2 = 1$. Potem enačbo
prepišemo v:
\beq
B_{0i} + B_{0r}= B_{0t}.
\label{eq:04_30}
\eeq
Iz zveze med amplitudama električnega in magnetnega polja (enačba~\ref{eq:EBc})
dobimo:
\beq
E_{0i}n_1 + E_{0r}n_1= E_{0t}n_2.
\label{eq:04_31}
\eeq
Drugi robni pogoj (enačba~\ref{eq:RPB}), ki pravi, da se ob prehodu v drugo snov ohranja
pravokotna komponenta gostote magnetnega polja, je prav tako izpolnjen, saj je $B_{\perp}=0$. 
Iz tretjega robnega pogoja, ohranitve tangentne komponente jakosti
električnega polja (enačba~\ref{eq:RPE}), sledi:
\beq
E_{0i} \cos \alpha - E_{0r}\cos \alpha = E_{0t}\cos \beta.
\label{eq:04_32}
\eeq
Predznak jakosti električnega polja smo zapisali v skladu s sliko~\ref{fig:04_tetm}. 
Za vajo lahko zapišemo še četrti robni pogoj, ki pomeni ohranitev normalne 
komponenete gostote električnega polja $D_\perp$ (enačba~\ref{eq:RPD}):
\beq
D_{0i}\sin \alpha - D_{0r}\sin \alpha = D_{0t}\sin \beta.
\label{eq:04_33}
\eeq
Upoštevamo, da je $D_0 = \varepsilon \varepsilon_0 E_0 = \varepsilon_0 n^2 E_0$ in
zapišemo:
\beq
E_{0i} n_1^2 \sin \alpha - E_{0r} n_1^2 \sin \alpha  = 
E_{0t} n_2^2 \sin \beta = E_{0t} n_1 n_2 \sin \alpha.
\label{eq:04_34}
\eeq
Pri tem smo upoštevali lomni zakon. Po krajšanju dobimo enačbo, 
ki je enaka prvemu robnemu pogoju in ne da nove informacije. 
Ostaneta dve enačbi za tri neznanke in ponovno lahko
izrazimo amplitudi odbite in prepuščene svetlobe z amplitudo vpadne. 
Enačbo (\ref{eq:04_31}) pomnožimo s $\cos \beta$, enačbo~(\ref{eq:04_32}) pa z $n_2$:
\begin{align}
E_{0i} n_1 \cos \beta + E_{0r} n_1 \cos \beta  =& E_{0t} n_2 \cos \beta \label{eq:04_35} \\
E_{0i} n_2 \cos \alpha - E_{0r} n_2 \cos \alpha  =& E_{0t} n_2 \cos \beta\label{eq:04_36}.
\end{align}
Enačbi odštejemo in dobimo:
\beq
E_{0i} \left(n_1 \cos \beta - n_2 \cos \alpha \right) + E_{0r} \left(n_1 \cos \beta + 
n_2 \cos \alpha \right) = 0.
\label{eq:04_37}
\eeq
Od tod izračunamo amplitudno odbojnost $r = E_{0r}/E_{0i}$, ki je 
razmerje med odbito in vpadno amplitudo jakosti električnega polja:
\boxeq{eq:TMr}{
r = \frac{n_2 \cos \alpha - n_1 \cos \beta}{n_2 \cos \alpha + n_1 \cos \beta}.
}
Tudi ta izraz lahko predelamo z upoštevanjem lomnega zakona:
\beq
r = \frac{\frac{n_2}{n_1} \cos \alpha - \cos \beta}{\frac{n_2}{n_1} \cos \alpha - \cos \beta} = 
\frac{\frac{\sin \alpha}{\sin \beta} \cos \alpha - \cos \beta}
{\frac{\sin \alpha}{\sin \beta} \cos \alpha - \cos \beta} = 
\frac{\sin \alpha \cos \alpha -\cos \beta \sin \beta}
{\sin \alpha \cos \alpha + \cos \beta \sin \beta}.
\label{eq:04_38}
\eeq
To lahko zapišemo z dvojnimi koti:
\beq
r = \frac{\sin(2\alpha) - \sin(2\beta)}{\sin(2\alpha) + \sin(2\beta)},
\label{eq:04_39}
\eeq
nato pa razliko in vsoto sinusov zapišemo kot produkt in dobimo:
\beq
r = \frac{2\cos(\alpha+\beta )\sin(\alpha-\beta )}
{2\sin(\alpha+\beta )\cos(\alpha-\beta )} = \frac{\tan(\alpha-\beta )}{\tan(\alpha+\beta )}.
\label{eq:04_40}
\eeq
Amplitudno prepustnost $t = E_{0t}/E_{0i}$ 
izračunamo iz enačbe~(\ref{eq:04_31}):
\boxeq{eq:TMt}{
t = \frac{n_1}{n_2}\left(1+r \right) = 
\frac{2 n_1 \cos \alpha}{n_2 \cos \alpha + n_1 \cos \beta}.
}

\begin{example}{\bf Amplitudni odbojnost in prepustnost pri pravokotnem vpadu.} 
Izračunali smo amplitudni odbojnosti $r$ in prepustnosti $t$ za obe med seboj 
pravokotni polarizaciji TE in TM. Poglejmo zdaj najpreprostejši primer, pri katerem
svetloba vpada pravokotno na mejo snovi. V tem primeru so vpadni, odbojni in 
lomni kot enaki in $\alpha = \tilde{\alpha} = \beta = 0$. Za TE polarizacijo dobimo:
\beq
r_{\mathrm{TE}} = \frac{n_1 -n_2}{n_1+n_2} \qquad \mathrm{in} \qquad
t_{\mathrm{TE}} = \frac{2n_1}{n_1+n_2},
\label{eq:04_42}
\eeq
za TM polarizacijo pa:
\beq
r_{\mathrm{TM}} = \frac{n_2 -n_1}{n_2+n_1} \qquad \mathrm{in} \qquad
t_{\mathrm{TM}} = \frac{2n_1}{n_1+n_2}.
\label{eq:04_43}
\eeq
Ker je vpad pravokoten, pričakujemo enaka rezultata za obe polarizaciji, saj 
polarizacij pri pravokotnem vpadu ne moremo ločiti med seboj. Rezultata za
prepuščeno svetlobo sta res enaka, za odbito svetlobo pa se razlikujeta za predznak.
Vzrok za to navidezno neenakost je v začetni izbiri smeri polarizacije 
pri odboju (glej sliko~\ref{fig:04_tetm}).
Odbojnost je z upoštevanjem privzete smeri tako enaka za obe polarizaciji 
-- kar je seveda pravilno.
\end{example}

\section{Energijski tok pri odboju in lomu}
V poskusih nas navadno zanima energijski tok valovanja. Gostoto energijskega
toka na splošno zapišemo kot (enačba~\ref{eq:j}):
\beq
j = \frac{1}{2} \varepsilon_0^2 E_0^2 c_0 n.
\label{eq:04_44}
\eeq
Odbojnost $\mathcal{R}$ vpeljemo kot razmerje med gostotama energijskih
tokov odbite in vpadne svetlobe:
\beq
\mathcal{R} = \frac{j_r}{j_i} = \frac{\frac{1}{2}\varepsilon_0 E_{0r}^2 c_0 n_1}
{\frac{1}{2}\varepsilon_0 E_{0i}^2 c_0 n_1} = \left(\frac{E_{0r}}{E_{0i}}\right)^2\!\!.
\label{eq:04_45}
\eeq
V razmerju jakosti električnega polja prepoznamo amplitudno odbojnost in zapišemo:
\boxeq{eq:TEMR}{
\mathcal{R} = |r|^2.
}
Ker odbita in vpadna svetloba potujeta po isti snovi, je odbojnost kar 
enaka kvadratu razmerja amplitud polj. 

Izračunajmo še prepustnost $\mathcal{T}$, ki je razmerje med gostotama svetlobnih
tokov prepuščene in vpadne svetlobe. Pri zapisu je treba biti pazljiv, saj je 
treba upoštevati projekcijo povprečnega Poyntingovega vektorja. Zato zapišemo: 
\beq
\mathcal{T} = \frac{j_t}{j_i} = \frac{\frac{1}{2}\varepsilon_0 E_{0t}^2 c_0 n_2}
{\frac{1}{2}\varepsilon_0 E_{0i}^2 c_0 n_1} \frac{\cos\beta}{\cos\alpha}= 
\left(\frac{E_{0t}}{E_{0i}}\right)^2 \frac{n_2\cos\beta}{n_1\cos\alpha}.
\label{eq:04_46}
\eeq
Prepustnost je torej enaka:
\boxeq{eq:TEMT}{
\mathcal{T} = |t|^2 \frac{n_2\cos\beta}{n_1\cos\alpha}.
}

Izračunajmo vsoto odbojnosti in prepustnosti na primeru TE polariziranega valovanja:
\begin{align}
\mathcal{R}+ \mathcal{T}=&
\left(\frac{n_1 \cos \alpha - n_2 \cos \beta}{n_1 \cos \alpha + n_2 \cos \beta}\right)^2+
\frac{4n_1n_2\cos \alpha \, \cos \beta}{\left(n_1 \cos \alpha + n_2 \cos \beta\right)^2}\\ =& 
\frac{n_1^2 \cos^2\alpha + n_2^2\cos^2\beta -2 n_1n_2 \cos \alpha \cos \beta + 
4n_1n_2\cos\alpha \cos \beta}{\left(n_1 \cos \alpha + n_2 \cos \beta\right)^2}\\ = &
\frac{n_1^2 \cos^2\alpha + n_2^2\cos^2\beta +2 n_1n_2 \cos \alpha \cos \beta}
{n_1^2 \cos^2\alpha + n_2^2\cos^2\beta +2 n_1n_2 
\cos \alpha \cos \beta} = 1.
\label{eq:04_47}
\end{align}
Če povzamemo:
\boxeq{eq:TEMRT}{
\mathcal{R}+\mathcal{T} = 1.
}
Rezultat je seveda pričakovan, saj opisuje ohranitev energije. Pogosto je najbolj
priročno izračunati odbojnost, nato pa z uporabo enačbe~(\ref{eq:TEMRT})
preprosto še prepustnost. 

Opozorimo še enkrat, da je pri izračunu prepustnosti treba upoštevati tudi 
spremembo lomnega količnika in kota širjenja svetlobe, zato:
\beq
|r|^2+ |t|^2 \neq 1.
\label{eq:04_48}
\eeq

\begin{example}{\bf Odbojnost pri pravokotnem vpadu na steklo.} 
Izračunajmo, kolikšna je odbojnost svetlobe, ki vpada pod pravim kotom iz zraka na
steklo z lomnim količnikom $n_2=1,5$. Uporabimo enačbi~(\ref{eq:TEr}) in (\ref{eq:TEMR})
in dobimo:
\beq
\mathcal{R} = \left(\frac{n_1-n_2}{n_1+n_2}\right)^2 \approx~4~\%.
\label{eq:04_49}
\eeq
Ob pravokotnem vpadu na steklo se $4~\%$ svetlobe odbije, ostala 
svetloba je prepuščena.
\end{example}

\section{Prehod v optično gostejšo snov}
Najprej si oglejmo prehajanje svetlobe iz snovi z manjšim lomnim
količnikom v snov z večjim lomni količnikom ($n_1<n_2$). To je na primer
primer odboja in loma pri vpadu svetlobe iz zraka v vodo ali steklo. 

Narišimo najprej amplitudno odbojnost in prepustnost za TE polarizacijo v odvisnosti
od vpadnega kota (slika~\ref{fig:04_redte}\,a). 
Iz enačbe~(\ref{eq:TEr}) sledi, da je amplitudna odbojnost $r_\mathrm{TE}$
za vse kote negativna. Njena velikost se zvezno spreminja od začetne vrednosti
pri pravokotnem vpadu do vrednosti $-1$, ko se vpadni kot približuje $90\si{\degree}$.
Negativni predznak pri kotu $\alpha =0$ pomeni, da se svetloba pri 
pravokotnem vpadu na optično gostejše sredstvo vedno odbije z nasprotno fazo, ne 
glede na njeno polarizacijo.
\begin{figure}[ht]
\centering
\def\svgwidth{140truemm} 
\input{slike/04_redte.pdf_tex}
\caption{nknl}
\label{fig:04_redte}
\end{figure}

Amplitudna prepustnost $t_\mathrm{TE}$, ki jo izračunamo kot $t=r+1$, je za vse
vpadne kote pozitivna in pojema od največje vrednosti pri pravokotnem vpadu
do vrednosti $0$, ko se vpadni kot približuje $90\si{\degree}$. 
Odbojnost $\mathcal{R}_\mathrm{TE}$, ki jo izračunamo kot kvadrat amplitudne odbojnosti,
je prikazana na sliki~\ref{fig:04_redte}\,b. Vrednost je po pričakovanju 
vedno pozitivna in narašča od neke začetne vrednosti pri pravokotnem
vpadu do vrednosti $1$, ko se vpadi kot približuje $90\si{\degree}$. Prepustnost
$\mathcal{R}_\mathrm{TE}$ izračunamo po enačbi $\mathcal{T} = 1 -\mathcal{R}$. 

Poglejmo še primer transverzalne magnetne polarizacije. Narišemo odvisnost
$r$ in $t$ od vpadnega kota (slika~\ref{fig:04_redtm}\,a). Zaradi začetne 
izbire smeri odbite jakosti električnega polja je predznak pri pravokotnem vpadu drugačen kot pri 
polarizaciji TE in je pozitiven. Z naraščajočim vpadnim kotom se amplitudna
odbojnost zmanjšuje in doseže vrednost $r=-1$ ko se vpadni kot
približuje $\alpha = 90\si{\degree}$. Vidimo, da pri nekem kotu zavzame  
vrednost $r = 0$. Pri tem kotu, imenujemo ga Brewstrov kot $\alpha_B$, 
je odbojnost za TM polarizacijo enaka nič. To ima pomembno in uporabno
posledico, da je celotno TM polarizirano valovanje pri Brewstrovem kotu prepuščeno. 
Ker amplitudna odbojnost pri Brewstrovem kotu zamenja predznak, se pri tem
kotu tudi spremeni faza odbite svetlobe. 
\begin{figure}[ht]
\centering
\def\svgwidth{140truemm} 
\input{slike/04_redtm.pdf_tex}
\caption{nknl}
\label{fig:04_redtm}
\end{figure}

Odbojnost in prepustnost TM polariziranega valovanja sta prikazani na 
sliki~\ref{fig:04_redtm}\,b. Odbojnost od začetne vrednosti pri $\alpha=0$
pojema z naraščajočim vpadnim kotom do vrednosti 0, ki jo doseže pri 
Brewstrovem vpadnem kotu $\alpha_B$. Nato odbojnost strmo naraste z naraščajočim vpadnim
kotom do vrednosti 1. Prepustnost je enaka $\mathcal{T} = 1- \mathcal{R}$.

\subsection*{Brewstrov kot}
Brewstrov kot smo vpeljali kot kot, pri katerem je odbojnost
TM polariziranega valovanja enaka nič. Poglejmo, pri katerem
vpadnem kotu se to zgodi. Amplitudna odbojnost $r$ je po enačbi~(\ref{eq:04_40})
enaka:
\beq
r = \frac{\tan(\alpha-\beta)}{\tan(\alpha+\beta)}.
\label{eq:04_50}
\eeq
Pri vrednosti $\alpha + \beta = \pi/2$ imenovalec v izrazu za $r$ 
naraste v neskončnost. Števec je vedno različen od nič, saj
sta po lomnem zakonu vpadni in lomljeni kot vedno različna. Posledično
je pri vrednosti $\alpha + \beta = \pi/2$ odbojnost enaka nič.
Ob tem pogoju lahko zapišemo $\cos \alpha = \sin \beta$. Iz lomnega 
zakona sledi:
\beq
n_1 \sin \alpha = n_2 \sin \beta = n_2 \cos \alpha.
\label{eq:04_51}
\eeq
Od tod seledi, da za Brewstrov kot velja:
\boxeq{eq:Brewstrovkot}{
\tan \alpha_B = \frac{n_2}{n_1}.
}
Pri prehodu zrak/steklo je $\alpha_B = 56,3\si{\degree}$ in na meji
steklo/zrak $\alpha_B = 33,6\si{\degree}$. 

Zaradi tega pojava je odbojnost za TM polarizirano valovanje pri
večjih vpadnih kotih dosti manjša kot za TE polarizirano valovanje. 
Kadar osvetljujemo mejno ploskev z nepolarizirano svetlobo, se 
pri kotu $\alpha_B$ od mejne ploskve odbije samo TE polarizirano
valovanje, TM pa je v celoti prepuščeno. Na ta preprost način
dobimo iz nepolarizirane svetlobe linearno polarizirano. 

To s pridom uporabljamo pri polarizacijskih sončnih očalih. Ker
se od bleščeče površine, na primer morske gladine, snežene pokrajine
ali gladke ceste, odbije razmero malo TM polariziranega valovanja, 
je odbita svetloba pretežno TE linearno polarizirana. Če sončna
očala delujejo kot polarizator, ki TE polarizirane svetlobe
ne prepušča, je prepuščene zelo malo odbite svetlobe.

\begin{remark}
Odstonost odboja pri Brewsterjevem kotu lahko pojasnimo tudi z mikroskopsko
sliko. Električni dipoli, ki sevajo odbito svetlobo, ne sevdajo v smeri
dipola. To je ravno v smeri pravokotno na smer vpadne svetlobe, zato 
pri pogoju $\alpha + \beta$ ni izsevane svetlobe.
\end{remark}

\section{Prehod v optično redkejšo snov}
Pri prehodu v snov z manjšim lomnim količnikom pride do novih zanimivih pojavov.
Primer tega je prehod svetlobe iz vode ali stekla v zrak, ko je $n_1>n_2$.
Če izhajamo iz lomnega zakona in zapišemo lomni kot:
\beq
\beta = \arcsin \left(\frac{n_1}{n_2}\sin \alpha\right),
\label{eq:04_52}
\eeq
vidimo, da pri določeni vrednosti argument doseže vrednost $1$ in lomni kot
$\beta = 90\si{\degree}$. Za večje
vpadne kote $\alpha$ enačba nima več rešitve. Takrat govorimo o totalnem 
ali popolnem odboju. Mejni kot totalnega odboja izračunamo kot:
\boxeq{eq:totalni}{
\alpha_m = \arcsin\left(\frac{n_2}{n_1}\right).
}
V steklu z lomnim količnikom $n_1=1,5$ je mejni kot totalnega odboja
enak $\alpha_m = 41,8\si{\degree}$, v vodi z lomnim količnikom 
$n_1 = 1,33$ pa $\alpha_m = 48,7\si{\degree}$.

Izračunajmo odbojnost in prepustnost najprej za TE polarizacijo.
Po enačbi~(\ref{eq:TEMr}) je:
\beq
r = \frac{n_1 \cos \alpha - n_2 \sqrt{1 - (\sin \alpha n_1/n_2)^2}}
{n_1 \cos \alpha + n_2 \sqrt{1 - (\sin \alpha n_1/n_2)^2}}.
\eeq
Za vrednosti vpadnega kota $\alpha > \alpha_m$ je argument korena
negativen, zato postane člen imaginaren. Zapišimo:
\beq
r = \frac{n_1 \cos \alpha - i \kappa}{n_1 \cos \alpha + i \kappa},
\eeq
pri čemer je:
\beq
\kappa = n_2 \sqrt{\left(\frac{n_1}{n_2}\sin \alpha \right)^2-1}  = 
n_2 \sqrt{\left(\frac{\sin \alpha}{\sin \alpha_m}\right)^2 -1}.
\eeq
Potem izračunamo odbojnost $\mathcal{R}$:
\beq
\mathcal{R} = |r|^2 = \frac{n_1 \cos \alpha -i \kappa}{n_1 \cos \alpha -i \kappa}
\cdot \frac{n_1 \cos \alpha +i \kappa}{n_1 \cos \alpha -i \kappa} = 1.
\eeq
Pri vpadnih kotih, ki so večji od mejnega kota totalnega odboja, se torej 
vsa vpadna svetloba odbije $\mathcal{R} = 1$, pri čemer velja
odbojni zakon. 

Narišimo zdaj amplitudni odbojnost in prepustnost za TE valovanje. Ker je 
$n_1>n_2$, je amplitudna odbojnost pri pravokotnem vpadu pozitivna. Njena
vrednost z naraščajočim vpadnim kotom narašča do mejnega kota totalnega
odboja, pri katerem naraste do $r=1$. Prepustnost, ki jo izračunamo
kot $t = 1+r$, zavzame zato pri pravokotnem vpadu vrednost nekaj nad 1,
nato pa narašča do vrednosti $t=2$, ki jo doseže pri mejnem kotu totalnega
odboja. To, da je prepustnost večja od 1, nas ne sme motiti, saj gre
razmerje med amplitudami valovanja in ne med energijskimi tokovi.

Na sliki je prikazana še odbojnost $\mathcal{R}$, ki homogeno
narašča od neke začetne vrednosti do $1$ pri mejnem kotu totalnega 
odboja. Vrednost $\mathcal{R} = 1$ ostaja za vse kote, večje
od mejnega kota. 

Za TM polarizacijo amplitudna prepustnost, ki homogeno
narašča od začetne negativne vrenosti do 1 pri mejnem
kotu totalnega odboja, pri Brewstrovem kotu zavzame vrednost 0. 
Tako sta pri prehodu TM polariziranega valovanja v optično redkejšo
snov pomembna dva kota: Brewstrov kot in mejni kot totalnega odboja.
Pri enem je odbojnost enaka nič, pri drugem pa 1. 

\section{Evanescento polje in Goos-Hanchen pojav}
V prejšnjem razdelku smo spoznali totalni odboj, do katerega
pride pri prehodu snovi iz optično gostejše v optično redkejšo 
snov. Odbojnost je pri kotih, ki so večji od mejnega kota
je enaka 1. Poglejmo pojav podrobneje.

Naj svetloba vpada na ravno mejo dveh neprevodnih, homogenih 
in izotropnih snovi, za kateri velja $n_1>n_2$.
Za prepuščeno svetlobo velja lomni zakon:
\beq
n_1 \sin \alpha = n_2 \sin \beta.
\eeq
Ko doseže lomljeni kot največjo možno vrednost $\beta = 90\si{\degree}$,
pride do totalnega odboja. Za mejni vpadni kot $\alpha$ velja:
\beq
\sin \alpha_m = \frac{n_2}{n_1},
\eeq
za lomljenega $\beta$ pa:
\beq
\cos \beta  = \sqrt{1 - \left(\frac{\sin \alpha}{\sin \alpha_m}\right)^2}
\eeq
Za vrednosti $\alpha > \alpha_m$ postane vrednost pod korenom
negativna in koren kompleksen. Zapišemo:
\beq
\cos \beta = i \sqrt{\left(\frac{\sin \alpha}{\sin \alpha_m}\right)^2-1} = i\kappa.
\eeq

Amplitudni odbojnosti za obe polarizaciji zapišemo kot:
\beq
r_{\mathrm{TE}} = \frac{n_1 \cos \alpha - i n_2 \kappa}
{n_1 \cos \alpha + i n_2 \kappa} \qquad \mathrm{in} \qquad
r_{\mathrm{TM}} = \frac{n_2 \cos \alpha - i n_1 \kappa}
{n_2 \cos \alpha + i n_1 \kappa}.
\eeq
Amplitudni odbojnosti za različni polarizaciji nista enaki.

Omejimo se na primer TE polariziranega valovanja. Valovni
vektor prepuščenega valovanja je enak:
\beq
\mathbf{k}_t = k_0 n_2 \left( \sin \beta, 0, \cos \beta \right) = 
\left( k_0 n_2 \sin \beta, 0, k_0 n_2 \cos \beta \right) = 
\left( k_0 n_1 \sin \alpha, 0, i k_0 n_2 \kappa \right).
\eeq
Prva komponenta, ki je vzporedna z mejno ravnino, se vedno ohranja,
tretja komponenta pa pri velikih vpadnih kotih postane imaginarna.
Prepuščeno valovanje v drugi snovi kot zapišemo kot:
\begin{align}
\mathbf{E}_t =& \mathbf{E}_{0t} e^{i k_0 n_2 \sin \beta x}
 e^{i k_0 n_2 \cos \beta z} e^{-i \omega t} \\
 =& \mathbf{e}_{y} E_{0t}  e^{i k_0 n_1 \sin \alpha x}
 e^{-i \omega t} e^{i k_0 n_2 (i\kappa) z} \\
 =& \mathbf{e}_{y} E_{0t}  e^{i k_0 n_1 \sin \alpha x}
 e^{-i \omega t} e^{- \varkappa z},
\end{align}
pri čemer smo vpeljali $\varkappa = k_0 n_2 \kappa$.

Amplituda jakosti električnega polja torej pojema eksponentno
z oddaljenostjo od mejne ploskve. Temu eksponentno pojemajočemu
polju v snovi z manjšim lomnim količnikom pravimo evanescnetno polje 
oziroma evanescentno valovanje. Jakost električnega polja zapišemo kot:
\beq
\mathbf{E}_t = \mathbf{e}_{y} E_{0t} e^{- \varkappa z}
\cos \left( k_0 n_1 \sin \alpha x-i \omega t\right). 
\eeq
Valovne fronte torej potujejo v smeri $x$, amplituda pa pojema
eksponentno v smeri $z$. 

Odbito polje je potem enako vpadnemu, le smer $k_z$ se spremeni. 
Celotno polje v snovi 1 potem zapišemo kot:
\begin{align}
\mathbf{E}_1 =& \mathbf{e}_{y} E_{0i} e^{ik_x x}e^{ik_z z -i \omega t} +
\mathbf{e}_{y} E_{0r} e^{ik_x x}e^{-ik_z z -i \omega t} \\
=& \mathbf{e}_{y} E_{0} \left(
e^{ik_0n_1\sin \alpha x}e^{ik_0n_1\cos \alpha z}+
e^{ik_0n_1\sin \alpha x}e^{-ik_0n_1\cos \alpha z}\right) e^{-i \omega t} \\
=& \mathbf{e}_{y} E_{0} e^{ik_0n_1\sin \alpha x-i\omega t} 2 
\cos \left(k_0 n_1\cos \alpha z\right).
\end{align}
Potem zapišemo realni del jakosti električnega polja v prvi snovi:
\boxeq{eq:ev1}{
\mathbf{E}_1 = 2 \mathbf{e}_{y} E_{0} 
\cos \left(k_0 n_1\cos \alpha z\right)\cdot 
\cos \left(k_0 n_1\sin \alpha x-\omega t\right)
}
in v drugi snovi:
\boxeq{eq:ev1}{
\mathbf{E}_2 = 2 \mathbf{e}_{y} E_{0} 
e^{-\kappa z} 
\cos \left(k_0 n_1\sin \alpha x-\omega t\right).
}
V prvi snovi se pojavi torej interferenčno modulirano polje v smeri osi $z$,
valovne fronte pa se širijo v smeri osi $x$.  V drugi snovi pa imamo
evanescentno polje, pri katerem valovne fronte potujejo v smeri osi $x$, 
amplituda polja pa pojema eksponentno v smeri $z$.

\begin{remark}
V resnici se vpadno in odbojno polje razlikujeta za več kot samo 
za predznak komponente valovnega vektorja. Pri odboju pride
namreč tudi do faznega zamika glede na vpadno polje, tako da je 
na splošno treba pri argumentu kosinusa  dodati fazni zamik.
\end{remark}

Izračunajmo še Poyntingov vektor v snoveh 1 in 2. Račun pokaže, da 
ima v obeh snoveh Poyntingov vektor samo komponento vzdolž osi $x$. 
To pomeni, da energija potuje zdolž osi $x$, vzdolž osi $z$ pa ne. 

\begin{exercise}
Izračunajmo Poyntingov vektor v snovi 2. 
Najprej zapišemo jakost električnega polja:
\beq
\mathbf{E} = \left(0, E_{0t} e^{ik_xx-i\omega t}e^{-\varkappa z}, 0\right)
 = \left(0,1,0\right) E_{0t} e^{ik_xx-i\omega t-\varkappa z},
\eeq
pri čemer je $E_{0t} = 2E_{0i} = E_0$.
Iz Maxwellove enačbe:
\beq
\nabla \times \mathbf{E} = - \frac{\partial B}{\partial t} = -\mu_0 
\frac{\partial \mathbf{H}}{\partial t} = \mu i \omega \mathbf{H}
\eeq
izpeljemo vektor jakosti magnetnega polja:
\beq
\mathbf{H} = \frac{-iE_{0t}}{\mu \omega} \left|
\begin{array}{ccc}
\mathbf{i} & \mathbf{j} & \mathbf{k} \\
\frac{\partial}{\partial x} & \frac{\partial}{\partial y} & \frac{\partial}{\partial z} \\
0&e^{ik_xx-i\omega t-\varkappa z}& 0 \\
\end{array}
\right|.
\eeq
Sledi:
\beq
\mathbf{H} =
\frac{-iE_{0t}}{\mu \omega} \left( -\varkappa  e^{ik_xx-i\omega t-\varkappa z}, 0, 
ik_x e^{ik_xx-i\omega t-\varkappa z}\right) = 
\frac{E_{0t}}{\mu \omega} e^{ik_xx-i\omega t-\varkappa z} \left(i\varkappa, 0, k_x \right).
\eeq
iz tega lahko izračunamo Poyntingov vektor:
\beq
\mathbf{S} = \mathbf{E}\times \mathbf{H}^* = \frac{E_{0t} e^{-2\varkappa z }}{\mu_0 \omega} 
\left( (0,1,0) \times (i\varkappa, 0, k_x)\right) = 
\frac{E_{0t} e^{-2\varkappa z }}{\mu_0 \omega} \left( k_x, 0, -i\varkappa \right).
\eeq
Vidimo, da je $z$ komponenta Poyntingovega vektorja imaginarna. To pomeni, 
da se svetloba v smeri $z$ ne razširja. 

Mogoče je pojav razumljivejši, če zapišemo vektorje v realni obliki. Jakost električnega
polja zapišemo kot:
\beq
\mathbf{E} = E_{0t} \left(0,1,0\right) e^{-\varkappa z} \cos(k_xx-\omega t).
\eeq
Jakost magnetnega polja izračunamo iz Maxwellove enačbe:
\beq
\nabla \times \mathbf{E} = -\mu_0 \frac{\partial \mathbf{H}}{\partial t}.
\eeq
Najprej izračunamo levo stran enačbe:
\beq
\nabla \times \mathbf{E} = \left|
\begin{array}{ccc}
\mathbf{i} & \mathbf{j} & \mathbf{k} \\
\frac{\partial}{\partial x} & \frac{\partial}{\partial y} & \frac{\partial}{\partial z} \\
0&E_{0t}e^{-\varkappa z} \cos \left(k_xx-\omega t\right)& 0 \\
\end{array}
\right| = \left[
\begin{array}{c}
-\varkappa E_{0t} e^{-\varkappa z} \cos \left(k_xx-\omega t\right) \\
0\\
-k_x E_{0t} e^{-\varkappa z} \sin \left(k_xx-\omega t\right)\\
\end{array}
\right]
\eeq
Sledi:
\beq
\frac{\partial \mathbf{H}}{\partial t} = \frac{-1}{\mu_0} \left[
\begin{array}{c}
-\varkappa E_{0t} e^{-\varkappa z} \cos \left(k_xx-\omega t\right) \\
0\\
-k_x E_{0t} e^{-\varkappa z} \sin \left(k_xx-\omega t\right)\\
\end{array}
\right]
\eeq
in 
\beq
\mathbf{H} = \frac{E_{0t}}{\mu_0} \left[
\begin{array}{c}
\varkappa e^{-\varkappa z} \int \cos \left(k_xx-\omega t\right) dt\\
0\\
k_x e^{-\varkappa z} \int\sin \left(k_xx-\omega t\right) dt\\
\end{array}
\right] = 
\frac{E_{0t}}{\mu_0} \left[
\begin{array}{c}
\frac{\varkappa}{\omega} e^{-\varkappa z} \sin \left(k_xx-\omega t\right)\\
0\\
\frac{k_x}{\omega} e^{-\varkappa z} \cos \left(k_xx-\omega t\right) dt\\
\end{array}
\right].
\eeq
Zdaj lahko izračunamo Poyntingov vektor:
\beq
\mathbf{S} = \mathbf{E} \times \mathbf{H} = 
E_{0t}e^{-\varkappa z} \left[ \begin{array}{c}
                        0\\
                        \cos(k_xx-\omega t)\\
                        0 \\
                       \end{array}\right]
\times
\frac{E_{0t}}{\mu_0\omega}e^{-\varkappa z}\left[
\begin{array}{c}
\frac{\varkappa}{\omega} e^{-\varkappa z} \sin \left(k_xx-\omega t\right)\\
0\\
\frac{k_x}{\omega} e^{-\varkappa z} \cos \left(k_xx-\omega t\right) \\
\end{array}
\right]
= \frac{E_{0t}^2}{\mu_0\omega}e^{-2\varkappa z}\left[
\begin{array}{c}
k_x\cos^2 \left(k_xx-\omega t\right)\\
0\\
\varkappa \sin \left(k_xx-\omega t\right) \cos \left(k_xx-\omega t\right)\\
\end{array}
\right].
\eeq
Pri energijskem toku na zanima časovno povprečje Poyntingovega vektorja:
\beq
\langle \mathbf{S}\rangle = \frac{E_{0t}^2}{\mu_0\omega}e^{-2\varkappa z}\left[
\begin{array}{c}
k_x/2\\
0\\
0\\
\end{array}
\right].
\eeq
Iz zapisanega vidimo, da je energijski tok v zmeri $z$ enak nič - pravokotno na mejno 
ploskev se torej energija ne prenaša. 

Kaj pa v smeri vzporedno z mejno ploskvijo? Izračunana komponenta Poyntingovega vektorja je:
\beq
\langle \mathbf{S}_x\rangle = \frac{E_{0t}^2}{2\mu_0\omega}e^{-2\varkappa z}k_0 n_1\sin \alpha,
\eeq
pri čemer smo upoštevali, da je $k_x = k_0 n_1 \sin \alpha$. Če izraz pomnožimo v števcu in imenovalcu 
z $\varepsilon_0$ in upoštevamo izraz za svetlobno hitrost, dobimo:
\beq
\langle \mathbf{S}_x\rangle = \frac{1}{2}\varepsilon_0 E_{0t}^2 e^{-2\varkappa z} c_0 n_1 \sin \alpha.
\eeq
Dobili smo izraz za gostoto svetlobnega toka, ki pa je pomnožen s faktorjem $\sin \alpha$ in seveda eksponentno
pojemajočim delom $\exp(-\varkappa z)$. 

Poyntingov vektor lahko izračunamo tudi v snovi 1. Vidimo, da ima pri $z=0$ enako vrednost. Tudi v snovi 1 je
različno od nič le povprečje komponente $x$, povprečje komponente v smeri $z$ pa je enako nič. Za razliko
od snovi 2, kjer intenziteta pojema eksponentno, se v snovi 1 intenziteta spreminja oscilatorno. 
\end{exercise}

Izračunajmo še vdorno globino $d_v$ evanescentnega polja v snovi 2. Vpeljemo jo kot inverz 
parametra $\varkappa$. Spomnimo se, da smo parameter $\varkappa$ vpeljali kot (enačba~):
\beq
\varkappa = k_0 n_2 \kappa = k_0 n_2 \sqrt{\left(\frac{\sin \alpha}{\sin \alpha_m}\right)^2-1}.
\eeq
Vdorna globina je potem:
\beq
d_v = \frac{1}{\varkappa} = \frac{\lambda_0}{2 \pi n_2 }\frac{1}{\sqrt{\left(\frac{\sin \alpha}{\sin \alpha_m}\right)^2-1}}=
\frac{\lambda_0}{2\pi}\frac{1}{\sqrt{n_1^2\sin ^2\alpha - n_2^2}}.
\eeq
Pri $\alpha = \alpha_m$ se vdorna globina povečuje proti neskončnosti. Pri $\alpha \to \pi/2$ pa vdorna
globina pade na vrednost $d_v = \lambda_0/2\pi NA$, pri čemer je $NA = \sqrt{n_1^2-n_2^2}$ numerična odprtina
ali numerična apertura. 

\subsection*{Frustrirani totalni odboj}
Če je pri totalnem odboju debela snovi 2 majhna, lahko pride to tuneliranja svetlobe iz snovi 1 v snov 3. To je 
še zlasti opazno, kadar je debelina snovi 2 po velikosti primerljiva z vdorno globino evanescentnega polja. Takrat
znaten delež vpadne svetlobe preide v snov 3. Prepustnost takega ``tunela'' za svetlobo spreminjamo
z debelino ``prepovedane plasti'' oziroma debelino reže med snovema 1 in 3. 

Primer aplikacije: čitalec prstih odtisov. Totalni odboj je močno odvisen od širine zračne reže med steklom
in prstom. Ker se ta razdalja spreminja zaradi vzorca prstnega odtisa, se to pozna na vzorcu odbite svetlobe. 

\section{Faza pri totalnem odboju}
Poglejmo, da se zgodi s fazo odbite svetlobe. Pri navadnem odboju, pri katerem sta koeficienta
amplitudne odbojnost za obe polarizaciji realna, ni težav. Pogledati moramo samo predznak $r$ in vidimo,
da se faza pri pravokotnem odboju na optično gostejšem sredstvu spremeni za $\pi$, faza pri odboju v optično
redkejšo snov pa ostane enaka. To bo imelo pomembne posledice pri računu odboja svetlobe in 
interference svetlobe na tankih plasteh (poglavje..). 

Zanimivejša je faza pri totalnem odboju. V prejšnjem razdelku smo spoznali, da atenuacijski koeficient
$\varkappa$ nastopa v izrazi kot parameter, ki določa vdorno globino polja v snovi 2. Njegova vrednost
je odvisna od vpadnega kota in tudi od polarizacije vpadnega valovanja:
\beq
\varkappa_{\mathrm{TE}}= k_0 n_2 \sqrt{\left(\frac{\sin \alpha}{\sin \alpha_m}\right)^2-1} 
\qquad \mathrm{in}\qquad
\varkappa_{\mathrm{TM}}= k_0 n_1 \sqrt{\left(\frac{\sin \alpha}{\sin \alpha_m}\right)^2-1}.
\eeq
Ker polje vdira v snov 2, pri oboju doživi fazni zamik, ki je odvisen od kota $\alpha$. Ta fazni
zamik je pole tega odvisen tudi od polarizacije svetlobe. 

Goos-Hanchen pojav. 

Izračunajmo najprej fazo za TE polarizirano valovanje: 
\beq
r = \frac{n_1 \cos \alpha - i n_2 \kappa}{n_1 \cos \alpha + i n_2 \kappa} = 
\frac{\sqrt{n_1^2\cos^2 \alpha + n_2^2 \kappa^2}e^{-i\Phi}}{\sqrt{n_1^2\cos^2 \alpha + n_2^2 \kappa^2}e^{+i\Phi}}
= e^{-2i\Phi},
\eeq
pri čemer je:
\beq
\Phi = \arctan\frac{n_2\kappa}{n_1 \cos \alpha} = \arctan\frac{n_2 \sqrt{\left(\frac{\sin \alpha}{\sin \alpha_m}\right)^2-1}}{n_1 \cos \alpha}.
\eeq
Pri $\alpha = \alpha_m$ je vrednost $\Phi=0$, pri vpadnem kotu $\alpha \to 90~\si{\degree}$ pa je
fazni zamik odbitega valovanja $2\Phi = \pi$. 

Podoben rezultat dobimo tudi za TM polarizirano valovanje, le da sta lomna količnika $n_1$ in $n_2$
zamenjana. Tako dobimo:
\beq
\Phi = \arctan\frac{n_1\kappa}{n_2 \cos \alpha} = \arctan\frac{n_1 \sqrt{\left(\frac{\sin \alpha}{\sin \alpha_m}\right)^2-1}}{n_2 \cos \alpha}.
\eeq
Ker je $n_1 > n_2$, so vrednosti faznega zamika za TM polarizacijo vedno večje od faznega zamika za TE polarizirano
valovanje. Pri tem velja:
\beq
\tan \Phi |_\mathrm{TM} = \left(\frac{n_1}{n_2}\right)^2 \tan \Phi |_\mathrm{TE}
\eeq
in posledično:
\beq
\Delta = 2\Phi_\mathrm{TM}-2\Phi_\mathrm{TE} > 0.
\eeq
Na meji steklo zrak dobimo maksimalno vrednost $\Delta > \pi/4$. To pomeni, da lahko z dvema zaporednima
totalnima odbojema dobimo fazni zamik med TE in TM polariziranim valovanjem, ki je enak $\pi/2$. Tak 
optični element, v katerem pride do dveh zaporednih totalnih odbojev torej lahko deluje kot ploščica
$\lambda/4$. S štirimi totalnimi odboji dobimo fazni zamik $\pi$ in učinek ploščice $\lambda/2$. Tovrstni
retarderji se imenujejo Fresnelovi rombi. Glavna prednost je, da je njihov učinek neodvisen od 
valovne dolžine $\lambda$, saj ta ne nastopa eksplicitno v izrazu za $\Delta$. 

\section{Optična vlakna?}
\section{Odboj na meji s kovino}
Zanimivo vprašanje je, kako se Fresnelove enačbe spremenijo, če je snov 2 prevodna. V tem primeru
vemo, da polje pojema z globino zaradi notranjih električnih tokov. Pričakujemo torej, da bo po analogiji
s totalnim odbojem, odbojnost na meji zato večja. 

Ker je zaradi meje simetrija prostora zlomljena le v smeri osi $z$, tudi v tem primeru polje
lahko pojema le z globino, to je s koordinato $z$. V prevodni snovi 2 tako zapišemo:
\beq
\mathbf{E}_2 = \mathbf{e}_y E_{0t} e^{i(k_z+i\varkappa)z} e^{ik_xx-i\omega t}.
\eeq
Drugi člen je enak kot v snovi 1, pri čemer velja $k_x = k_0 n_1 \sin \alpha$,
v prvem členu pa vpeljemo $\tilde{k}_z = k_z + i \varkappa$.

V snovi 2 velja telegrafska enačba in zato:
\beq
\tilde{\mathbf{k}}^2 = k_0^2 \mathcal{N}^2 = k_x^2 + (k_z+i\varkappa)^2 = 
k_0^2 \left(\varepsilon_2 + \frac{i\sigma_2}{\varepsilon_0 \omega}\right).
\eeq
Iz tega sledi:
\beq
\frac{k_x^2+k_z^2 - \varkappa^2}{k_0^2} + \frac{2 i \kappa k_z}{k_0^2} = \varepsilon_2 + 
\frac{i \sigma_2}{\varepsilon_0 \omega}.
\eeq
Iz te enačbe lahko izračunamo $k_z$ in $\varkappa$ v odvisnosti od $k_x = k_0 n_1 \sin \alpha$, ki
se mora ohranjati zaradi robnih pogojev. 

Rešimo enačbo, tako da $k_x$ nesemo na drugo stran in dobimo:
\beq
k_z^2 - \varkappa^2+ 2 i \kappa k_z= k_0^2 \varepsilon_2 - k_x^2 + i\frac{k_0^2\sigma_2}{\varepsilon_0 \omega}.
\eeq
Nato postopamo podobno, kot smo 
to naredili pri izračunu kompleksnega lomnega količnika. Sledi:
\beq
k_z^2 = \frac{1}{2}\left(\sqrt{(k_0^2\varepsilon_2-k_x^2)^2 + \left(\frac{\sigma k_0^2}{\varepsilon_0 \omega}\right)^2}
+ (k_0^2\varepsilon_2-k_x^2)^2 \right)
\eeq
in 
\beq
\varkappa^2 = \frac{1}{2}\left(\sqrt{(k_0^2\varepsilon_2-k_x^2)^2 + \left(\frac{\sigma k_0^2}{\varepsilon_0 \omega}\right)^2}
- (k_0^2\varepsilon_2-k_x^2)^2 \right).
\eeq
Celotno komponento valovnega vektorja v smeri $z$ v snovi 2 potem zapišemo kot:
\beq
\tilde{k}_z = k_z + i \varkappa.
\eeq

Spomnimo se, da za navadni neprevodni primer amplitudno odbojnost zapišemo kot:
\beq
r = \frac{n_1 \cos \alpha - n_2 \cos \beta}{n_1 \cos \alpha + n_2 \cos \beta} = 
\frac{k_0 n_1 \cos \alpha - k_0 n_2 \cos \beta}{k_0 n_1 \cos \alpha + k_0 n_2 \cos \beta} = 
\frac{k_{z1}-k_{z2}}{k_{z1}+k_{z2}}.
\eeq
Domnevamo, da zapišemo amplitudno odbojnost za primer kovine v enaki obliki, le 
upoštevati moramo kompleksno vrednost komponente $\tilde{k}_z = k_{2z}+ i\varkappa$.

V to se lahko prepričamo, če zapišemo robne pogoje in rešimo enačbi:
\beq
E_{0i} + E_{0r} = E_{0t}
\eeq
in 
\beq
-ik_{1z}E_{0i} + ik_{1z}E_{0r}= -(ik_{2z}-\varkappa)E_{0t}.
\eeq
Prvo enačbo pomnožimo s $\tilde{k}_z$ in enačbi odštejemo. Dobimo:
\boxeq{eq:Kovinar}{
r = \frac{E_{0r}}{E_{0i}} = \frac{k_{z1}-k_{z2}-i \varkappa}{k_{z1}+k_{z2}+i\varkappa}.
}
Realni del je povsem enak amplitudni odbojnosti v neprevodnih snoveh, drugi del pa je 
analogen kot pri totalnemu odboju na meji med dvema neprevodnima snovema. 

Za TE polarizirano valovanje potem zapišemo:
\beq
r_\mathrm{TE} = \frac{n_1 \cos \alpha - \mathcal{N}_2 \cos \beta}{n_1 \cos \alpha + \mathcal{N}_2 \cos \beta}
\eeq
in za TM polarizirano: 
\beq
r_\mathrm{TM} = \frac{\mathcal{N}_2 \cos \alpha - n_1 \cos \beta}{\mathcal{N}_2 \cos \alpha + n_1 \cos \beta}.
\eeq
Pri tem je tudi $\cos \beta$ kompleksno število, določeno formalno z zvezo $n_1 \sin \alpha = \mathcal{N}_2 \sin \beta$.

Temu ustrezno lahko rezultat razumemo kot neko vmesno rešitev, ki delno spominja
na navadni odboj, vendar ima večjo odbojnost. Kovine dobro odbijajo svetlobo
in se zato ``bleščijo''. 

Zgornji račun je preprost je za primer pravokotnega vpada, ko sta $\alpha = \beta = 0$. 
Takrat dobimo za amplitudno odbojnost:
\beq
r = \frac{n_1 - \mathcal{N}}{n_1 + \mathcal{N}} = \frac{n_1 -n_2' -in_2''}{n_1 +n_2' +in_2''}.
\eeq
Odbojnost pa se zapiše kot:
\beq
\mathcal{R} = |r|^2 = \frac{(n_1 -n_2')^2 +n_2''^2}{(n_1 +n_2')^2 +n_2''^2}.
\eeq

Če imamo zelo dobro kovino z veliko prevodnostjo in velja $\sigma/\varepsilon_0 \omega \gg \varepsilon$, potem
vemo, da velja $n''> n'$ in 
\beq
\mathcal{R} = \frac{n_1^2 +n_2''^2}{n_1^2 +n_2''^2} = 1.
\eeq
Plast dobro prevodne kovine, ki je ravna in brez prask (spolirana), tako deluje kot zrcalo!
