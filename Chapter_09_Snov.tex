%\chapterimage{Geometrijska.jpg} % Chapter heading image

\chapter{Interakcija svetlobe s snovjo}
V tem poglavju se bomo posvetili interakciji svetlobe s snovjo, ki jo 
v najpreprostejši obliki opišemo z lomnim količnikom. Spoznali bomo 
odvisnost lomnega količnika od frekvence vpadnega valovanja, zapisali
lomni količni za prevodne snovi, opisali optično aktivnost in Faradayev 
pojav.
 
\section{Lomni količnik}
V drugem poglavju smo vpeljali lomni količnik kot razmerje med hitrostjo
svetlobe v vakuumu in hitrostjo svetlobe v snovi (enačba~\ref{eq:c}):
\beq
c = \frac{c_0}{n}.
\label{eq:09_01}
\eeq
Zapisali smo, da je lomni količnik odvisen od snovi, po kateri se svetloba širi, 
in tudi od frekvence vpadne svetobe. Odvisnost lomnega količnika od
frekvence vpadnega elektromagnetnega valovanja imenujemo disperzija. 
Izračunajmo disperzijsko odvisnost na preprostem modelu. 

Obravnavajmo nemagnetne snovi, za katere je $\mu = 1$, torej 
za izračun frekvenčne odvisnosti lomnega količnika zadošča, če 
poznamo $\varepsilon(\omega)$. Snov naj bo sestavljena iz $N$
identičnih atomov oziroma molekul, ki so enakomerno porazdeljeni
po prostoru s prostornino $V$. Za opis posameznega atoma ali molekule uporabimo 
klasični model oscilatorja, ki ga imenujemo Lorentzov model 
po nizozemskem fiziku Hendriku Lorentzu (1853--1928). 

Zamislimo si, da atom oziroma molekulo sestavljata kroglica 
pozitivnega naboja, ki miruje v izhodišču pri $x=0$, in kroglica
(oblak) negativnega naboja na oddaljenosti $x$. Med njima naj 
deluje privlačna sila, ki jo v klasičnem modelu opišemo z vzmetjo
s konstanto vzmeti $k$. 
Poleg elastične sile naj na negativno nabito kroglico med 
premikanjem deluje tudi sila dušenja, ki opisuje absorpcijo svetlobe
v snovi. V klasični sliki si lahko
zamislimo, da je ravnovesna razdalja med kroglicama enaka $x_r$, 
v bolj realističnem modelu, v katerem elektron opišemo kot oblak, 
pa je ravnovesna razdalja enaka 0. Ker je razlika le v konstanti, 
to na končni rezultat ne vpliva.

Naj na tak model atoma (ali molekule) vpade elektromagnetno valovanje. Ker je 
valovna dolžina svetlobe bistveno večja od atoma, lahko privzamemo, da
je električno polje po velikost znotraj atoma homogeno. Za negativno nabito
kroglico zapišemo Newtonov zakon, pri čemer upoštevamo silo vzmeti, silo 
dušenja in silo v električnem polju, pri čemer se omejimo na polarizacijo 
polja v smeri $x$:
\beq
m \ddot{x} = -kx -\gamma m \dot{x} - e_0 E,
\label{eq:09_02}
\eeq
pri čemer so $m$ masa negativno nabite kroglice, $\gamma$ koeficient dušenja
in 
\beq
E = E_0 e^{-i\omega t}.
\label{eq:09_03}
\eeq
Pike označujejo časovne odvode.
\begin{remark}
Pri zapisu Lorentzovega modela smo privzeli, da je elektično polje, 
ki ga čuti atom, enako vpadnemu električnemu polju. 
To velja, dokler je atom sam. Če pa je obdan z ostalimi atomi, 
čuti tudi njihov vpliv in pravilno bi bilo upoštevati skupni vpliv, ki 
ga zapišemo s tako imenovanim lokalnim poljem. 
Kako izračunamo lokalno polje bomo spoznali na koncu razdelka. 
\end{remark}

Vpeljemo lastno krožno oziroma resonančno frekvenco $\omega_0 = k/m$ in 
enačbo prepišemo v:
\beq
\ddot{x} + \gamma \dot{x} + \omega_0^2 x  = - \frac{e_0}{m} E_0 e^{-i \omega t}.
\label{eq:09_04}
\eeq
Rešujemo jo z nastavkom:
\beq
x(t) = x_0(\omega) e^{-i\omega t},
\label{eq:09_05}
\eeq
pri čemer $x_0$ opisuje amplitudo nihanja. Dobimo:
\beq
-\omega^2 x_0 - i \omega \gamma x_0 + \omega_0^2 x_0  = - \frac{e_0}{m} E_0.
\label{eq:09_06}
\eeq
Od tod izrazimo amplitudo v odvisnosti od frekvence vpadnega valovanja:
\beq
x_0 = \frac{-e_0/m~E_0}{\omega_0^2 - \omega^2 - i\gamma \omega}.
\label{eq:09_07}
\eeq

Razmik med lego pozitivno in negativno nabite kroglice $x$ povzroči pojav
dipolnega momenta $\mathbf{p}$:
\beq
\mathbf{p} = -e_0 x~ \mathbf{e}_x.
\label{eq:09_08}
\eeq
Prehod iz mikroskopskega opisa z dipolnim momentom posameznega atoma 
v makroskopski opis naredimo z upoštevanjem gostote atomov v snovi 
$\varrho = N/V$. Vpeljemo inducirano električno polarizacijo $P$, ki 
je enaka celotnemu dipolnemu momentu na enoto volumna:
\beq
\mathbf{P} = \frac{\mathbf{p}N}{V} = \mathbf{p}\varrho. 
\label{eq:09_09}
\eeq
Z upoštevanjem enačb~(\ref{eq:09_07} in \ref{eq:09_08}) zapišemo 
električno polarizacijo v snovi kot:
\beq
\mathbf{P} = -e_0 x~\mathbf{e}_x\varrho = \frac{e_0^2\varrho/m}{\omega_0^2 - \omega^2 - i\gamma \omega} E_0 \mathbf{e}_x e^{-i\omega t}.
\label{eq:09_10}
\eeq
Inducirano električno polarizacijo lahko povežemo z drugimi količinami, 
s katerimi opišemo električno polje v snovi. Gostoto električnega polja
zapišemo kot:
\beq
\mathbf{D} = \varepsilon \varepsilon_0 \mathbf{E} = 
\varepsilon_0 \mathbf{E} + \mathbf{P}.
\label{eq:09_11}
\eeq
Od tod sledi linearna zveza med električno polarizacijo in jakostjo
električnega poljal v snovi:
\beq
\mathbf{P} = \varepsilon \varepsilon_0 \mathbf{E} - \varepsilon_0 \mathbf{E} = 
\varepsilon_0 (\epsilon-1) \mathbf{E}.
\label{eq:09_12}
\eeq
Enačbo~(\ref{eq:09_12}) primerjamo z enačbo~(\ref{eq:09_10}) in dobimo:
\beq
\varepsilon= 1 + \frac{e_0^2\varrho/\varepsilon_0 m}{\omega_0^2 - \omega^2 - i\gamma \omega}.
\label{eq:09_13}
\eeq
Vpeljemo še plazemsko frekvenco:
\beq
\omega_p = \sqrt{\frac{e_0^2 \varrho }{\varepsilon_0 m}}
\label{eq:09_14}
\eeq
in upoštevamo zvezo med na splošno kompleksnim lomnim količnikom:
\beq
\varepsilon = \mathcal{N}^2.
\label{eq:09_15}
\eeq
Dobimo:
\boxeq{eq:09_16}{
\varepsilon = \mathcal{N}^2 = 1+ \frac{\omega_p^2}{\omega_0^2 - \omega^2 - i\gamma \omega}.
}
\begin{remark}
V plazmi se elektroni prosto gibljejo, zato sta v tem primeru $k=0$ in $\gamma = 0$. 
Za opis plazme tako ostane le še frekvenca $\omega_p$, od tod ime plazemska frekvenca. \end{remark}









\section{Disperzija}
\section{Prevodne snovi}
\section{Optični metamateriali}
\section{Optična aktivnost}
%\section{Dikroizem}
\section{Faradayev pojav}
Uporabo dodaj in primere.
