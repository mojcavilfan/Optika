%\chapterimage{Geometrijska.jpg} % Chapter heading image

\chapter{Interakcija svetlobe s snovjo}
V tem poglavju se bomo posvetili interakciji svetlobe s snovjo, ki jo 
v najpreprostejši obliki opišemo z lomnim količnikom. Spoznali bomo 
odvisnost lomnega količnika od frekvence vpadnega valovanja, zapisali
lomni količni za prevodne snovi, opisali optično aktivnost in Faradayev 
pojav.

\section{Fazna in grupna hitrost}
V tretjem poglavju smo zapisali najpreprostejšo rešitev valovne enačbe
v neomejenem prostoru v obliki ravnih valov. Izračunali smo fazno hitrost,
to je hitrost premikanja ploskev konstantne faze oziroma valovnih front in
jo zapisali kot razmerje (enačba~\ref{eq:03_10}):
\beq
v_f = \frac{\omega}{k}.
\label{eq:09_25}
\eeq
Fazno hitrost navadno označimo s $c$ in jo zapišemo kot (enačba~\ref{eq:c}):
\beq
c = \frac{c_0}{n}.
\label{eq:09_01}
\eeq
Razmerje med hitrostjo svetlobe v vakuumu in hitrostjo svetlobe v snovi
opisuje lomni količnik $n$. Na splošno je lomni količnik odvisen 
od snovi, po kateri se svetloba širi in tudi od frekvence vpadne svetlobe.

Obravnavajmo zdaj dva ravna vala z enakima amplitudama in malo različnima 
frekvencama $\omega_1=\omega + \Delta \omega$ in $\omega_2=\omega - \Delta \omega$, 
ki potujeta v isti smeri. Vsota obeh valovanj je:
\beq
E = E_0 \cos (k_1x-\omega_1 t)+ E_0 \cos (k_2x-\omega_2 t)=
E_0 \cos\left((k-\Delta k)x-(\omega + \Delta \omega)t\right) 
+ E_0 \cos\left((k+\Delta k)x-(\omega - \Delta \omega)t\right). 
\label{eq:09_23}
\eeq
Z uporabo adicijskih izrekov za kotne funkcije izraz preoblikujemo v:
\beq
E = 2\cos \left(kx - \omega t \right)\cos \left(\Delta kx - \Delta \omega t \right) 
\label{eq:09_24}
\eeq
Skupno valovanje je torej valovanje z osrednjo krožno frekvenco $\omega$, 
ki pa je amplitudno modulirano s krožno frekvenco $\Delta \omega$ (slika~\ref{}).
Pri takem valovanju, ki si ga lahko predstavljamo kot zaporedje potujočih sunkov,
lahko določimo dve različni hitrosti. Prva je hitrost premikanja ploskev konstantne
faze oziroma premikanja vrhov hitre modulacije. Ta hitrost je fazna hitrost in
jo izračunamo z znane enačbe~(\ref{eq:09_25}). Druga hitrost pa je hitrost, s katero
se premikajo vrhovi počasne modulacije oziroma posamezni sunki. To hitrost imenujemo
grupna hitrost in jo izračunamo kot $v_g=\Delta \omega/\Delta k$ oziroma v limiti:
\boxeq{eq:09_24}{
v_g = \frac{\partial\omega}{\partial k}.
}
Izraz za grupno hitrost smo sicer izpeljali na primeru valovanja, sestavljenega iz dveh
valovanj, vendar velja na splošno za razširjanje sunkov svetlobe. Z grupno hitrostjo
se tako širi informacija, ki jo lahko prenašamo s svetlobo.

Če je lomni količnik konstanten in $\omega = k c_0/n$, sta fazna in grupna hitrost enaki.
V primeru, da je lomni količnik funkcija frekvence valovanja, se fazna in grupna hitrost
razlikujeta. Praviloma je grupna hitrost manjša od fazne, obstajajo pa tudi izjeme, ki
jih bomo spoznali v prihodnjem razdelku.




Guenther str 263 in celo poglavje!


\section{Lomni količnik}
Zapisali smo, da je lomni količnik odvisen od snovi, po kateri se svetloba širi, 
in tudi od frekvence vpadne svetobe. Odvisnost lomnega količnika od
frekvence vpadnega elektromagnetnega valovanja imenujemo disperzija. 
Izračunajmo disperzijsko odvisnost na preprostem modelu. 

Obravnavajmo nemagnetne snovi, za katere je $\mu = 1$, torej 
za izračun frekvenčne odvisnosti lomnega količnika zadošča, če 
poznamo $\varepsilon(\omega)$. Snov naj bo sestavljena iz $N$
identičnih atomov oziroma molekul, ki so enakomerno porazdeljeni
po prostoru s prostornino $V$. Za opis posameznega atoma ali molekule uporabimo 
klasični model oscilatorja, ki ga imenujemo Lorentzov model 
po nizozemskem fiziku Hendriku Lorentzu (1853--1928). 

Zamislimo si, da atom oziroma molekulo sestavljata kroglica 
pozitivnega naboja, ki miruje v izhodišču pri $x=0$, in kroglica
(oblak) negativnega naboja na oddaljenosti $x$. Med njima naj 
deluje privlačna sila, ki jo v klasičnem modelu opišemo z vzmetjo
s konstanto vzmeti $k$. 
Poleg elastične sile naj na negativno nabito kroglico med 
premikanjem deluje tudi sila dušenja, ki opisuje absorpcijo svetlobe
v snovi. V klasični sliki si lahko
zamislimo, da je ravnovesna razdalja med kroglicama enaka $x_r$, 
v bolj realističnem modelu, v katerem elektron opišemo kot oblak, 
pa je ravnovesna razdalja enaka 0. Ker je razlika le v konstanti, 
to na končni rezultat ne vpliva.

Naj na tak model atoma (ali molekule) vpade elektromagnetno valovanje. Ker je 
valovna dolžina svetlobe bistveno večja od atoma, lahko privzamemo, da
je električno polje po velikost znotraj atoma homogeno. Za negativno nabito
kroglico zapišemo Newtonov zakon, pri čemer upoštevamo silo vzmeti, silo 
dušenja in silo v električnem polju, pri čemer se omejimo na polarizacijo 
polja v smeri $x$:
\beq
m \ddot{x} = -kx -\gamma m \dot{x} - e_0 E,
\label{eq:09_02}
\eeq
pri čemer so $m$ masa negativno nabite kroglice, $\gamma$ koeficient dušenja
in 
\beq
E = E_0 e^{-i\omega t}.
\label{eq:09_03}
\eeq
Pike označujejo časovne odvode.
\begin{remark}
Pri zapisu Lorentzovega modela smo privzeli, da je elektično polje, 
ki ga čuti atom, enako vpadnemu električnemu polju. 
To velja, dokler je atom sam. Če pa je obdan z ostalimi atomi, 
čuti tudi njihov vpliv in pravilno bi bilo upoštevati skupni vpliv, ki 
ga zapišemo s tako imenovanim lokalnim poljem. 
Kako izračunamo lokalno polje bomo spoznali na koncu razdelka. 
\end{remark}

Vpeljemo lastno krožno oziroma resonančno frekvenco $\omega_0 = k/m$ in 
enačbo prepišemo v:
\beq
\ddot{x} + \gamma \dot{x} + \omega_0^2 x  = - \frac{e_0}{m} E_0 e^{-i \omega t}.
\label{eq:09_04}
\eeq
Rešujemo jo z nastavkom:
\beq
x(t) = x_0(\omega) e^{-i\omega t},
\label{eq:09_05}
\eeq
pri čemer $x_0$ opisuje amplitudo nihanja. Dobimo:
\beq
-\omega^2 x_0 - i \omega \gamma x_0 + \omega_0^2 x_0  = - \frac{e_0}{m} E_0.
\label{eq:09_06}
\eeq
Od tod izrazimo amplitudo v odvisnosti od frekvence vpadnega valovanja:
\beq
x_0 = \frac{-e_0/m~E_0}{\omega_0^2 - \omega^2 - i\gamma \omega}.
\label{eq:09_07}
\eeq

Razmik med lego pozitivno in negativno nabite kroglice $x$ povzroči pojav
dipolnega momenta $\mathbf{p}$:
\beq
\mathbf{p} = -e_0 x~ \mathbf{e}_x.
\label{eq:09_08}
\eeq
Prehod iz mikroskopskega opisa z dipolnim momentom posameznega atoma 
v makroskopski opis naredimo z upoštevanjem gostote atomov v snovi 
$\varrho = N/V$. Vpeljemo inducirano električno polarizacijo $P$, ki 
je enaka celotnemu dipolnemu momentu na enoto volumna:
\beq
\mathbf{P} = \frac{\mathbf{p}N}{V} = \mathbf{p}\varrho. 
\label{eq:09_09}
\eeq
Z upoštevanjem enačb~(\ref{eq:09_07} in \ref{eq:09_08}) zapišemo 
električno polarizacijo v snovi kot:
\beq
\mathbf{P} = -e_0 x~\mathbf{e}_x\varrho = \frac{e_0^2\varrho/m}{\omega_0^2 - \omega^2 - i\gamma \omega} E_0 \mathbf{e}_x e^{-i\omega t}.
\label{eq:09_10}
\eeq
Inducirano električno polarizacijo lahko povežemo z drugimi količinami, 
s katerimi opišemo električno polje v snovi. Gostoto električnega polja
zapišemo kot:
\beq
\mathbf{D} = \varepsilon \varepsilon_0 \mathbf{E} = 
\varepsilon_0 \mathbf{E} + \mathbf{P}.
\label{eq:09_11}
\eeq
Od tod sledi linearna zveza med električno polarizacijo in jakostjo
električnega poljal v snovi:
\beq
\mathbf{P} = \varepsilon \varepsilon_0 \mathbf{E} - \varepsilon_0 \mathbf{E} = 
\varepsilon_0 (\epsilon-1) \mathbf{E}.
\label{eq:09_12}
\eeq
Enačbo~(\ref{eq:09_12}) primerjamo z enačbo~(\ref{eq:09_10}) in dobimo:
\beq
\varepsilon= 1 + \frac{e_0^2\varrho/\varepsilon_0 m}{\omega_0^2 - \omega^2 - i\gamma \omega}.
\label{eq:09_13}
\eeq
Vpeljemo še plazemsko frekvenco:
\beq
\omega_p = \sqrt{\frac{e_0^2 \varrho }{\varepsilon_0 m}}
\label{eq:09_14}
\eeq
in upoštevamo zvezo med na splošno kompleksnim lomnim količnikom:
\beq
\varepsilon = \mathcal{N}^2.
\label{eq:09_15}
\eeq
Dobimo:
\boxeq{eq:09_16}{
\varepsilon = \mathcal{N}^2 = 1+ \frac{\omega_p^2}{\omega_0^2 - \omega^2 - i\gamma \omega}.
}
\begin{remark}
V plazmi se elektroni prosto gibljejo, zato sta v tem primeru $k=0$ in $\gamma = 0$. 
Za opis plazme tako ostane le še frekvenca $\omega_p$, od tod ime plazemska frekvenca. \end{remark}

Kompleksni lomni količnik $\mathcal{N}$ razdelimo na realni $n'$ in imaginarni del $n''$ in 
zapišemo:
\beq
\mathcal{N}^2 = (n'+n'')^2 = n'^2-n''^2+2in'n'' =  1+ \frac{\omega_p^2}{\omega_0^2 - \omega^2 - i\gamma \omega}.
\label{eq:09_17}
\eeq
Izenačimo realni in imaginarni del leve in desne strani enačbe in dobimo:
\beq
n'^2 -n''^2 = \varepsilon' = 1 + \frac{\left(\omega_0^2 - \omega^2\right)\omega_p^2}{\left(\omega_0^2 - 
\omega^2\right)^2 + \gamma^2 \omega^2}
\label{eq:09_18}
\eeq
ter 
\beq
2n'n'' = \varepsilon'' = \frac{\gamma \omega \omega_p^2}{\left(\omega_0^2 - 
\omega^2\right)^2 + \gamma^2 \omega^2}.
\label{eq:09_19}
\eeq
Zapisali smo sistem dveh enačb za realni in imaginarni del lomnega količnika. Rešujemo ga podobno, kot
smo reševali sistem enačb za lomni količnik v prevodni snovi (enačbe~\ref{eq:nkompleks2} in \ref{eq:03_72})
z rešitvami v obliki enačb~(\ref{eq:03_76} in \ref{eq:03_77}). Rešitvi sta v obliki:
\beq
n'^2 = \frac{1}{2}\left(\varepsilon'+\sqrt{\varepsilon'^2 + \varepsilon''^2}\right)
\label{eq:09_20}
\eeq
in
\beq
n''^2 = \frac{1}{2}\left(-\varepsilon'+\sqrt{\varepsilon'^2 + \varepsilon''^2}\right)\!,
\label{eq:09_21}
\eeq
pri čemer sta $\varepsilon'$ in $\varepsilon''$ podana z enačbama~(\ref{eq:09_18} in \ref{eq:09_19}).

\section{Disperzija}
V prejšnjem razdelku smo na preprostem modelu izpeljali frekvenčno odvisnost realnega in imaginarnega
dela lomnega količnika. Izrazi so razmeroma zapleteni, zato se navadno poslužimo približkov. Prvi naj 
bo približek redkega plina. 

V redkem plinu vrednost lomnega količnika le malo odstopa od lomnega količnika vakuuma 
in $\mathcal{N} \approx 1$. Ker je odstopanje lomnega količnika majhno, lahko kvadratni koren v izrazu
za lomni količnik (enačba~\ref{eq:09_17}) razvijemo in dobimo:
\beq
\mathcal{N} = \sqrt{1+ \frac{\omega_p^2}{\omega_0^2 - \omega^2 - i\gamma \omega}} \approx
1 + \frac{1}{2}\frac{\left(\omega_0^2 - \omega^2\right)\omega_p^2}{\left(\omega_0^2 - 
\omega^2\right)^2 + \gamma^2 \omega^2} + \frac{1}{2}\frac{\gamma \omega \omega_p^2}{\left(\omega_0^2 - 
\omega^2\right)^2 + \gamma^2 \omega^2}.
\label{eq:09_22}
\eeq
\begin{figure}[h!]
\centering
\def\svgwidth{100truemm} 
%\input{slike/09_nkompleks.pdf_tex}
\caption{Odvisnost realnega in imaginarnega dela in imaginarnega dela lomnega količnika od frekvence
vpadne svetlobe v preprostem Lorentzovem modelu}
\label{fig:09_nkompelks}
\end{figure}
Poglejmo najprej realni del lomnega količnika (slika~\ref{fig:09_nkompelks}\,a). Bolj kot vrednost
lomnega količnika samega je pomemben odvod lomnega količnika po krožni frekvenci, saj ta nastopa v izrazu
za grupno hitrost valovanja. 

Vidimo, da je odvod
lomnega količnika od frekvence 

disperizaja v vlaknih.


vlanka
mavrica
prizma

odvisnosti
viglviglvigl razlozi resonance, uv, elektronske... 
primeri

n< 1x xray ionosphere


\section{Prevodne snovi}
\section{Optični metamateriali}
\section{Optična aktivnost}
%\section{Dikroizem}
\section{Faradayev pojav}
Uporabo dodaj in primere.
