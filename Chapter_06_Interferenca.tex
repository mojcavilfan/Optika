%\chapterimage{Geometrijska.jpg} % Chapter heading image

\chapter{Interferenca}
Spoznali bomo, da sta uklon in interferenca zelo povezana in pravzaprav
manifestacija istega pojava -- seštevanja različnih prispevkov valovanj k 
skupnemu optičnemu polju. V poglavju o uklonu so nas zanimali predvsem uklonski 
vzorci, v poglavju o interferenci pa se bomo osredotočili na interferometrijo 
in pojave na tankih plasteh.

\section{Interferenca in interferometrija}
O interferenci govorimo, kadar je na danem mestu v prostoru jakost električnega polja
sestavljena iz več prispevkov iz različnih valovanj. Ta valovanja se na splošno
lahko razlikujejo v smeri širjenja, amplitudi, frekvenci, fazi ali polarizaciji. 

V obravnavi inteference se omejimo le na valovanja, ki imajo enako frekvenco (oziroma
se frekvence valovanj je malo razlikujejo), enako polarizacijo in konstantno fazo. 
Če bi prva zahteva ne bila izpolnjena, bi se svetlobno valovanje na danem mestu zelo
hitro spreminjalo in zaznali bi lahko le povprečno sliko. Zahteva po isti polarizaciji
mora biti izpolnjena, saj sicer valovanji ne interferirata. V računu to pomeni skalarno 
obravnavno jakosti električnega polja. Slednja zahteva o konstantni fazi 
je povezana s koherenco valovanja, ki jo bomo podrobneje obravnavali v 
poglavju~\ref{chap:Koherenca}. 

Najpreprostejši interferenčni poskusi so taki, pri katerih vpadni snop svetlobe
razdelimo na več delov. To lahko naredimo z delitvijo po valovni fronti ali delitvijo 
po amplitudi. V obeh primerih je pogoj po isti frekvenci, polarizaciji in fazi izpolnjen, 
seveda pa se valovanji razlikujeta v dodatnem faznem zamiku, ki ga pridobita po delitvi.
Dodatni fazni zamik (in z njim celotna interferenčna slika) je močno odvisen od dolžine
poti, ki jo del prvotnega snopa vpadne svetlobe prepotuje. V odvisnosti od faznega zamika
se na opazovalnem zaslonu pojavijo ojačitve (konstruktivna inferferenca) ali oslabitve 
(destruktivna interferenca). Ker je valovna dolžina svetlobe zelo majhna, že majhne 
spremembe faznega zamika povzročijo velike spremembe interferenčnega vzorca. 

Z opazovanjem interferenčnih vzorcev lahko zelo natančno določimo fazni zamik med
dvema deloma valovanja in s tem pridobimo podatke o valovanju samem, o snovi, po 
kateri se širi valovanje, ali o oddaljenosti predmeta, od katerega se je svetloba odbila. 
Zato interferometrijo, kot imenujemo metodo, ki temelji na opazovanju interference,
s pridom uporabljamo med drugim za izredno natačne meritve dolžine, 
oddaljenosti, lomnega količnika snovi oziroma njegovih sprememb ali odstopanj v frekvenci. 
Interferenčne meritve so ene najnatančnejših in zato zelo uporabnih na veliko področjih:
od prvotne opustitve obstanka etra prek izredno natačnih meritev gladkosti površine do
opazovanja gravitacijskih valov. 

V laboratorijih in industrijskih aplikacijah navadno za vir svetlobe uporabljamo laser, zato
je interferenčna slika sestavljena iz svetlih in temnih območij. V naravi interferenco
navadno opazujemo z belo dnevno svetlobo, kar da predmetom, na primer plasti olja na vodi ali
milnemu mehurčku, značilno mavrično obarvanost.

Zapišimo preprost primer interference matematično. Jakost električnega polja
obravnavajmo kot skalarno količino (se omejimo na eno polarizacijo), ki je na splošno
funkcija kraja in časa: $E= E(\mathbf{r},t)$. 

Naj imata obe vpadni valovanji enako frekvenco in pripadajoče valovno število $k$.
Izhodiščna faza valovanj naj bo enaka in jo lahko brez izgube splošnosti 
postavimo na 0, amplitudi posameznih valovanj pa označimo z $E_{10}$ in $E_{20}$. 
Valovanji potem zapišemo kot:
\beq
E_1 = E_{10} e^{i\mathbf{k}_1 \cdot \mathbf{r} - i \omega t + i \delta_1}
\qquad \mathrm{in} \qquad
E_2 = E_{20} e^{i\mathbf{k}_2 \cdot \mathbf{r} - i \omega t + i \delta_2}.
\label{eq:06_01}
\eeq
Pri tem privzamemo, da sta $E_{10}$ in $E_{20}$ realni števili. Celotno
električno polje je potem:
\beq
E = E_1 + E_2 = E_{10} e^{i\phi_1 - i \omega t} + E_{20} e^{i\phi_2 - i \omega t},
\label{eq:06_02}
\eeq
pri čemer sta $\phi_{1,2} = \mathbf{k}_{1,2} \cdot \mathbf{r} + \delta_{1,2}$. 
Zanima nas gostota svetlobnega toka $j$, ki je enaka:
\beq
j = \left|\langle \mathbf{S}\rangle \right| = \frac{1}{2}\varepsilon \varepsilon_0 |E|^2c
\label{eq:06_03}
\eeq
in 
\beq
j \propto (E_1+E_2)(E_1^*+E_2^*)  = 
\left( E_{10} e^{i\phi_1 - i \omega t} + E_{20} e^{i\phi_2 - i \omega t}\right)
\left( E_{10} e^{-i\phi_1 + i \omega t} + E_{20} e^{-i\phi_2 + i \omega t}\right).
\label{eq:06_04}
\eeq
Dobimo:
\beq
j \propto E_{10}^2 + E_{20}^2 + E_{10}E_{20} \left(e^{i\phi_1-i\phi_2}+ e^{-i\phi_1+i\phi_2}\right)
\label{eq:06_05}
\eeq
in 
\beq
j = j_1 + j_2 + 2\sqrt{j_1 j_2} \cos(\Delta \phi),
\label{eq:06_06}
\eeq
pri čemer sta $j_1$ in $j_2$ gostoti svetlobnih tokov prvega in drugega delnega valovanja,
$\Delta \phi$ pa označuje razliko faz $\phi_1-\phi_2$. Le kadar je ta razlika neodvisna
od časa, lahko v eksperimentu izmerimo tudi tretji člen v enačbi~(\ref{eq:06_06}). Sicer se 
ta člen pri meritvi izpovpreči. V odvisnosti od (časovno neodvisnega) $\Delta \phi$,  
gostota svetlobnega toka zavzema vrednosti:
\beq
\left(\sqrt{j_1}- \sqrt{j_2}\right)^2 < j < \left(\sqrt{j_1} +\sqrt{j_2}\right)^2.
\label{eq:06_07}
\eeq
\begin{figure}[ht]
\centering
\def\svgwidth{120truemm} 
%\input{slike/06_kontrast.pdf_tex}
\caption{SLIKA}
\label{fig:06_kontrast}
\end{figure}

Vpeljemo vidljivost ali kontrast interferenčnega vzorca:
\beq
v = \frac{j_\mathrm{max}- j_\mathrm{min}}{j_\mathrm{max}+ j_\mathrm{min}},
\label{eq:06_08}
\eeq
ki lahko zavzema vrednosti med 0 in 1.

V posebnem primeru, ko sta amplitudi obeh valovanj enaki $E_{01} = E_{02} = E_0$ in velja 
$j_1 = j_2 = j_0$, dobimo:
\beq
j = 2j_0 + 2j_0 \cos (\Delta \phi) = 4j_0 \cos^2 (\Delta \phi/2)
\label{eq:06_09}
\eeq
Interferenčni vzorec dveh enako močnih snopov svetlobe zavzame vrednosti med 0 
in $4j_0$, kar pomeni, da je kontrast takega vzorca enak 1. 

Kako pa je z ohranitvijo energijskega toka? Pri interferenci se energijski 
tok seveda ohranja, pride le do preporazdelitve energije. Povprečni energijski 
tok čez veliko območje prostora pa je 
\beq
\langle j \rangle = \frac{1}{2}(4j_0) = 2j_0,
\label{eq:06_10}
\eeq
kar ustreza gostoti energijskega toka začetnih dveh valovanj. 

Poglejmo interferenco dveh valovanj, ki pod kotom vpadata eno glede na drugo.
\begin{figure}[ht]
\centering
\def\svgwidth{120truemm} 
%\input{slike/06_interferenca.pdf_tex}
\caption{SLIKA}
\label{fig:06_interferenca}
\end{figure}
Potem prvo valovanje zapišemo kot:
\beq
E_1 = e^{ik_\perp x} e^{ik_z z}e^{-i\omega t}
\label{eq:06_11}
\eeq
in drugo kot:
\beq
E_1 = e^{-ik_\perp x} e^{ik_z z}e^{-i\omega t},
\label{eq:06_12}
\eeq
pri čemer sta vektorja:
\beq
\mathbf{k}_1 = (k_\perp,0, k_z) \qquad \mathrm{in} \qquad \mathbf{k}_2 = (-k_\perp,0, k_z).
\label{eq:06_13}
\eeq
Interferenca teh dveh valovanj da:
\beq
E = E_1+E_2 = E_0 e^{ik_z z -i\omega t }\left(e^{ik_\perp x}+e^{-ik_\perp x} \right) = 
E_0 e^{ik_z z -i\omega t } 2 \cos(k_\perp x).
\label{eq:06_14}
\eeq
Interferenčni vzorec, ki ga opazujemo, je:
\beq
j = 4 j_0 \cos^2(k_\perp x).
\label{eq:06_15}
\eeq

\section{Interferenca z delitvijo valovne fronte}
V večini interferometričnih ustvarimo interferenco z enim samim snopom svetlobe, 
ki ga razdelimo na dva dela. To lahko naredimo na dva načina: lahko razdelimo
valovno fronto valovanja (npr. Youngov poskus) ali pa razdelimo amplitudo
valovanja (npr. Michelsonov interferometer). 

Najpomembnejši primer interference z delitvijo valovne fronte je nedvomno
Youngov poskus (1801), ki je angleškega fizika Thomasa Younga vodil do spoznanja,
da je svetloba transverzalno valovanje. Pri tem poskusu vpadno valovanje iz 
monokromatskega izvora simetrično usmerimo na zaslon, v katerem sta dve enaki ozki
reži in opazujemo sliko na oddaljenem zaslonu\footnote{Vpadno valovTorej potrebujemoanje mora biti 
koherentno. Več o tej lastnosti valovanja 
bomo spoznali v poglavju~\ref{chap:Koherenca}.}. Uklonsko sliko 
pravzaprav že poznamo, saj jo lahko izračunamo s Fraunhoferjevim 
uklonskim približkom.
\begin{figure}[ht]
\centering
\def\svgwidth{120truemm} 
%\input{slike/06_Young.pdf_tex}
\caption{SLIKA}
\label{fig:06_Young}
\end{figure}

Naj bodo $z_0$ razdalja od objektnega zaslona do opazovalnega zaslona, $D$ razdalja 
med središčema rež, $r_1$ razdalja med sredino prve reže in izbrano točko na zaslonu
ter $r_2$ razdalja med sredino druge reže in točko na zaslonu. Potem za velike 
oddaljenosti $r_1, r_2 \gg D$ velja:
\beq
\Delta r = r_1-r_2 \approx D \sin\vartheta,
\label{eq:06_16}
\eeq
pri čemer je $\vartheta$ kot med osjo $z$ in zveznico med točko na sredini med režama in
izbrano točko na opazovalnem zaslonu. Uporabimo enačbo~(\ref{eq:06_09}) in dobimo:
\beq
j = 4 j_0 \cos^2\left(\frac{\Delta \phi}{2}\right) = 4j_0 \cos^2\left(\frac{kD\sin \vartheta}{2}\right).
\label{eq:06_17}
\eeq

V točkah, kjer se prispevka obeh valovanj seštevata, nastopi tako imenovana
konstruktivna interferenca in na zaslonu opazimo ojačitve. Pogoj za ojačitev zapišemo kot:
\beq
\frac{kD\sin \vartheta}{2} = N\pi,
\label{eq:06_18}
\eeq
pri čemer je $N$ celo število. Od tod izračunamo pogoj za kote $\vartheta_\mathrm{max}$, pri katerih
se pojavijo vrhovi:
\boxeq{eq:InterferencaMax}{
D \sin\vartheta = N \lambda.
}
Ker je leva stran enačbe enaka razliki poti od rež do točke na zaslonu, je zapis preprosto razumeti:
razlika poti od središč rež do točke na zaslonu mora biti enaka večkratniku valovne dolžine valovanja.

V vmesnih točkah, kjer se prispevka valovanj odštevata, je destruktivna
interferenca in na zaslonu opazimo oslabitve (temne proge). Podobno kot za ojačitve
zapišemo pogoj za oslabitve:
\boxeq{eq:InterferencaMin}{
D \sin\vartheta = \left(N + \frac{1}{2}\right)\lambda.
}

Vzorec, ki se pojavi v daljnjem polju oziroma na oddaljenem zaslonu,
je periodičen s periodo ponavljanja $\Delta \xi_0$. Izračunajmo jo. Velja:
\beq
\Delta \left(\frac{kD\sin \vartheta}{2}\right) = \pi.
\label{eq:06_19}
\eeq
Za majhne kote $\vartheta$ velja 
\beq
k D \Delta \vartheta = k D \frac{\Delta \xi_0}{z_0} = 2 \pi,
\label{eq:06_20}
\eeq
od koder sledi:
\beq
\Delta \xi_0 = \frac{\lambda z_0}{D}.
\label{eq:06_21}
\eeq
Bliže kot so reže, bolj razmaknjeni so uklonski vrhovi, kar je seveda v skladu
z uklonsko sliko na velikem številu rež. 

Za $D=1~\si{\micro\metre}$ je na oddaljenosti $z_0 = 1~\si{m}$ razmik med ojačitvami
$\Delta \xi_0 = 0,5~\si{m}$, medtem ko je pri $D = 1~\si{mm}$ 
vrednost $\Delta \xi_0 = 0,5~\si{mm}$. 

Pokažimo, da je dobljeni rezultat ekvivalenten ristemu, ki smo ga izpeljali pri
obravnavi Fraunhoferjevega uklona v petem poglavju. Za sistem $N$ rež širine $d$
s periodo ponavljanja $D$ je veljalo (enačba~\ref{eq:uklonNrez}):
\beq
j(\vartheta) = j_0 \left(\frac{\sin\left(kd\sin\vartheta/2\right)}{kd\sin\vartheta/2}\right)^2
\left(\frac{\sin\left(NkD\sin\vartheta/2\right)}{\sin\left(kD\sin\vartheta/2\right)}\right)^2\!\!.
\label{eq:06_22}
\eeq
Če so reže ze ozke in $d\to 0$, je strukturni faktor enak 1 in intenziteta vrhov z oddaljenostjo
os optične osi ne pojema. V primeru dveh ozkih rež, je 
\beq
j = j_0 \frac{\sin^2(NkD\sin\vartheta/2)}{\sin^2(kD\sin\vartheta/2)} = 
j_0 \frac{4 \sin^2(kD\sin\vartheta/2)\cos^2(kD \sin\vartheta/2)}{×\sin^2(kD \sin\vartheta/2)}
 = 4j_0 \cos^2\left(\frac{kD \sin\vartheta}{2},
\label{eq:06_23}
\eeq
kar je enako enačbi~(\ref{eq:06_17}).

\begin{remark}
Oglejmo si še nekaj alternativnih postavitev interferometrov z delitvijo žarka. Pri Youngovem
eksperimentu se namreč težko izognemo težavam, povezanim s končno širino reže $d$. Njegovi
sodobniki so poskušali podoben interferenčni vzorec ustvariti nad druge načine. Prvi primer
je Lloydovo zrcalo, pri katerem interferirajo žarki, ki pridejo neposredno od izvora svetlobe, 
in žarki, ki se odbijejo od zrcala. Drugi primer sta Fresnelovi zrcali, pri katerih interferirata
sva snopa svetlobe, ki se odbijata vsak od svojega zrcala. Tretja postavitev, ki jo omenimo,
je s Fresnelovo biprizmo. Svetloba, ki izhaja iz enega svetila, se na Fresnelovi biprizmi
lomi, lomljena žarka pa med seboj interferirata.
\end{remark}

\section{Interferenca z delitvijo amplitude}
Najznačilnejši primer interferometra z delitvijo amplitude je Michelsonov interferometer.
Vpadno svetlobo s polprepustnim zrcalom razdelimo na dva snopa, ki se odbijeta vsak od 
svojega zrcala in interferirata na detektorju. S premikanjem enega od zrcal spreminjamo
zakasnitev med žarkoma in na detektorju se izmenično pojavljajo ojačitve in oslabitve.
Gostota svetlobnega toka na detektorju je (enačba~):
\beq
j = 4j_0 \cos^2(\Delta \phi/2),
\label{eq:06_24}
\eeq
pri čemer je $\Delta \phi = k_0(l_1-l_2)$....


\section{Interferenca na tanki plasti}
Do zdaj smo obravnavali interferenco, pri kateri je bilo skupno valovanje sestavljeno iz le 
dveh prispevkov. Zdaj si oglejmo primer interference, ki nastane kot vsota zelo velikega 
števila delnih valovanj. 

Naj ravno valovanje vpada na tanko plast snovi z lomnim količnikom $n_2$, lomni količnik 
okolice pa naj bo enak $n_1$, pri čemer se omejimo na primer TE polariziranega valovanja.
Ob vpadu na plast se del svetlobe odbije, del pa lomi v snov po lomnem zakonu (enačba~) 
z amplitudo, ki jo določa Fresnelova enačba (). Prepuščeni val potuje po plasti do meje
na drugi strani, kjer je ga ponovno nekaj odbije, del pa je prepuščen v smeri, ki 
je enaka vpadni smeri. Odbiti del valovanja potuje 
do vpadne meje, kjer se deloma odbije, del pa ga izhaja na vpadni strani ... 

Jakost prepuščenega električnega polja je tako sestavljena iz velikega števila prispevkov, ki 
izhajajo v isti smeri, njihove amplitude pa so vedno manjše. Ker izhodni žarki 
izhajajo na različnih točkah, jih navadno z zbiralno lečo preslikamo v eno točko. Pri pravokotnem
vpadu na tanko plast teh težav ni. 

Zanima nas gostota prepuščenega svetlobnega toka $j$ v odvisnosti od 
lomnih količnikov $n_1$ in $n_2$, vpadnega kota $\alpha$ in debeline plasti $d$. 

Najprej z lomnim zakonom izračunamo kot $\beta$, pod katerim se širi svetloba v plasti:
\beq
\sin\beta = \frac{n_1}{n_2}\sin\alpha.
\label{eq:06_25}
\eeq
Amplitudna odbojnost $r$ in prepustnot $t$ na prvi meji sta enaka (enačba~):
\beq
r_{12} = \frac{n_1\cos \alpha - n_2\cos \beta}{n_1\cos \alpha + n_2\cos \beta}\qquad 
\mathrm{in}\qquad t_{12} = 1+r_{12},
\label{eq:06_26}
\eeq
na drugi meji pa:
\beq
r_{21} = \frac{n_2\cos \beta - n_1\cos \alpha}{n_2\cos \beta + n_1\cos \alpha}\qquad 
\mathrm{in}\qquad t_{21} = 1+r_{21}.
\label{eq:06_27}
\eeq
Vidimo, da velja $r_{12} = r_{21}$. Ko poznamo vse koeficiente, lahko zapišemo
posamezne prispevke jakosti električnega polja na prepuščeni strani:
\begin{align}
E_1 &= E_0\,t_{12}\,t_{21},\\
E_2 &= E_0\,t_{12}\,r_{21}\,r_{21}\,e^{i\phi}\,t_{21},\\
E_3 &= E_0\,t_{12}\,r_{21}\,r_{21}\,e^{i\phi}\,r_{21}\,r_{21}\,e^{i\phi}\,t_{21},\\
itd.
\label{eq:06_29}
\end{align}





\section{Fabry-Perotov interferometer}
\section{Večplastni nanosi}
\section{Uporaba interference}
