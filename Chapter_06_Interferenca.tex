%\chapterimage{Geometrijska.jpg} % Chapter heading image

\chapter{Interferenca}
\label{chap:Interferenca}
Spoznali bomo, da sta uklon in interferenca tesno povezana in pravzaprav
manifestacija istega pojava -- seštevanja  valovanj v 
skupno optično polje. V poglavju o uklonu so nas zanimali predvsem uklonski 
vzorci, v poglavju o interferenci pa se bomo osredotočili na interferometrijo 
in pojave na tankih plasteh.

\section{Interferenca in interferometrija}
O interferenci govorimo, kadar je na danem mestu v 
prostoru jakost električnega polja sestavljena iz 
več valovanj. Seštevanje 
dveh ali več valovanj ponekod vodi
do ojačevanja svetlobe (konstruktivna interferenca), 
drugod pa do oslabitev (destruktivna interferenca). 
Vzorci izmenjujočih se ojačitev in oslabitev sestavljajo
značilno interferenčno sliko.

Elektromagnetna valovanja se na splošno razlikujejo 
v smeri širjenja, amplitudi, frekvenci, fazi ali polarizaciji. 
Spoznali bomo, da se interferenčna slika pojavi le pri valovanjih z enako
polarizacijo, konstantno (ali zelo počasi se spreminjajočo)
začetno fazo in enako (oziroma približno enako) frekvenco. 
Če se frekvenci valovanj ne ujemata, se svetlobno valovanje na 
danem mestu zelo hitro spreminja in slika, ki jo zaznamo, se izpovpreči.
Zahteva o konstantni fazi je povezana s koherenco valovanja, 
ki jo bomo podrobneje obravnavali v poglavju~\ref{chap:Koherenca}. 

Najpreprostejši interferenčni poskusi so taki, pri katerih vpadni snop svetlobe
razdelimo na več delov. To lahko naredimo z delitvijo po valovni fronti
ali z delitvijo po amplitudi. V obeh primerih so zahteve po isti frekvenci, 
polarizaciji in začetni fazi izpolnjene, seveda pa se valovanji razlikujeta 
v dodatnem faznem zamiku, ki ga pridobita po delitvi.
Dodatni fazni zamik in z njim povezana interferenčna slika 
sta močno odvisna od dolžine poti, ki jo prepotuje del prvotnega snopa 
vpadne svetlobe. Ker je valovna dolžina svetlobe zelo majhna, že majhne 
spremembe v dolžini poti oziroma zakasnitvi žarka povzročijo velike spremembe 
interferenčnega vzorca. 

Z opazovanjem interferenčnih vzorcev lahko zelo natančno določimo fazni zamik med
dvema deloma valovanja in s tem pridobimo podatke o valovanju samem, o snovi, po 
kateri se širi valovanje, ali o oddaljenosti predmeta, od katerega se svetloba odbija. 
Zato interferometrijo, kot imenujemo metodo, ki temelji na opazovanju interference,
s pridom uporabljamo med drugim za izredno natančne meritve dolžine, 
oddaljenosti, lomnega količnika snovi oziroma njegovih sprememb ali odstopanj v frekvenci. 
Interferenčne meritve so ene najnatančnejših in so zato zelo uporabne na veliko področjih:
od prvotne opustitve obstanka etra prek izredno natančnih meritev gladkosti površine do
opazovanja gravitacijskih valov. 

V laboratorijih in industrijskih aplikacijah navadno za vir svetlobe uporabljamo laser, zato
je interferenčna slika sestavljena iz svetlih in temnih območij. V naravi interferenco
navadno opazujemo z belo dnevno svetlobo, kar da tankim plastem, na primer plasti olja na vodi ali
milnemu mehurčku, značilno mavrično obarvanost.

\section{Interferenca dveh ravnih valov}
Zapišimo preprost primer interference dveh ravnih valov z enako frekvenco. Valovna
vektorja valovanj označimo s $\mathbf{k}_1$ in $\mathbf{k}_2$, fazi valovanj z 
$\delta_1$ in $\delta_2$ ter njuni realni amplitudi z $E_{10}$ in $E_{20}$. Ker imajo 
valovanja, ki interferirajo, isto polarizacijo,
za račun interference zadošča skalarna oblika električnega polja.
Valovanji potem zapišemo kot:
\beq
E_1 = E_{10} e^{i\mathbf{k}_1 \cdot \mathbf{r} - i \omega t + i \delta_1}
\qquad \mathrm{in} \qquad
E_2 = E_{20} e^{i\mathbf{k}_2 \cdot \mathbf{r} - i \omega t + i \delta_2}.
\label{eq:06_01}
\eeq
Celotno
električno polje je vsota obeh prispevkov:
\beq
E = E_1 + E_2 = E_{10} e^{i\phi_1 - i \omega t} + E_{20} e^{i\phi_2 - i \omega t},
\label{eq:06_02}
\eeq
pri čemer sta $\phi_{1,2} = \mathbf{k}_{1,2} \cdot \mathbf{r} + \delta_{1,2}$. 
Gostota svetlobnega toka $j$ je na splošno (enačba~\ref{eq:j}):
\beq
j = \frac{1}{2}\varepsilon \varepsilon_0 |E|^2c,
\label{eq:06_03}
\eeq
zato je celotna gostota svetlobnega toka enaka:
\beq
j \propto (E_1+E_2)(E_1^*+E_2^*)  = 
\left( E_{10} e^{i\phi_1 - i \omega t} + E_{20} e^{i\phi_2 - i \omega t}\right)
\left( E_{10} e^{-i\phi_1 + i \omega t} + E_{20} e^{-i\phi_2 + i \omega t}\right)\!.
\label{eq:06_04}
\eeq
Sledi:
\beq
j \propto E_{10}^2 + E_{20}^2 + E_{10}E_{20} \left(e^{i\phi_1-i\phi_2}+ e^{-i\phi_1+i\phi_2}\right)\!.
\label{eq:06_05}
\eeq
Eksponente v oklepaju izrazimo s kotno funkcijo in upoštevajoč zvezo~(enačba~\ref{eq:06_03}) dobimo:
\boxeq{eq:06_06}{
j = j_1 + j_2 + 2\sqrt{j_1 j_2} \cos(\Delta \phi),
}
pri čemer sta $j_1$ in $j_2$ gostoti svetlobnih tokov prvega in drugega delnega valovanja,
$\Delta \phi$ pa označuje razliko faz $\phi_1-\phi_2$. Le kadar je ta razlika neodvisna
od časa oziroma se s časom zelo počasi spreminja, v eksperimentu poleg prvih dveh členov
v enačbi~(\ref{eq:06_06}) opazimo tudi tretjega. V nasprotnem primeru se tretji člen 
izpovpreči in intenziteta na opazovalnem zaslonu je enaka vsoti intenzitet posameznih 
delnih valovanj. Interference v tem primeru ne vidimo. 

Če je $\Delta \phi$ neodvisen od časa, gostota svetlobnega toka na opazovalnem zaslonu $j$
zavzema vrednosti $j_\mathrm{min}<j<j_\mathrm{max}$, za katere velja:
\beq
\left(j_1+j_2 -2\sqrt{j_1 j_2}\right) < j < \left(j_1+j_2 -2\sqrt{j_1 j_2}\right)
\label{eq:06_07}
\eeq
oziroma zapisano drugače:
\beq
\left(\sqrt{j_1}- \sqrt{j_2}\right)^2 < j < \left(\sqrt{j_1} +\sqrt{j_2}\right)^2\!\!.
\label{eq:06_07a}
\eeq
\begin{remark}
Vpeljemo lahko tudi kontrast interferenčnega vzorca, ki ga izračunamo kot:
\beq
v = \frac{j_\mathrm{max}- j_\mathrm{min}}{j_\mathrm{max}+ j_\mathrm{min}}.
\label{eq:06_08}
\eeq
Parameter $v$, ki zavzema vrednosti med 0 in 1, imenujemo tudi
vidljivost interferenčnega vzorca.
\end{remark}

Vrnimo se h krajevni odvisnosti gostote svetlobnega toka (enačba~\ref{eq:06_06}). Na
splošno se intenziteti delnih valovanj razlikujeta in skupna gostota svetlobnega
toka oscilira med neko neničelno najmanjšo in največjo vrednostjo (slika~\ref{fig:06_kontrast}).
\begin{figure}[h!]
\centering
\def\svgwidth{85truemm} 
\input{slike/06_vidljivost.pdf_tex}
\vglue1truemm
\caption{Interferenca dveh valovanj z različnima intenzitetama. 
Vrednost skupne gostote svetlobnega toka v odvisnosti od faznega 
zamika med vpadnima valovanjema oscilira med najmanjšo $j_\mathrm{min}$ 
in največjo vrednostjo $j_\mathrm{max}$. Pikčasta črta označuje skupno
gostoto toka delnih valovanj, kadar se fazna razlika hitro spreminja
in interferenčni vzorec ni viden.}
\label{fig:06_kontrast}
\end{figure}

V posebnem primeru, ko sta amplitudi obeh valovanj enaki in je $j_1 = j_2 = j_0$, 
je skupna gostota svetlobnega toka enaka (enačba~\ref{eq:06_06}):
\beq
j = 2j_0 + 2j_0 \cos (\Delta \phi) = 4j_0 \cos^2 (\Delta \phi/2).
\label{eq:06_09}
\eeq
Interferenčni vzorec dveh enako močnih snopov svetlobe zavzema vrednosti med 0 
in $4j_0$, kar pomeni, da je kontrast takega vzorca (enačba~\ref{eq:06_08}) enak 1. 
Kadar se fazna razlika med valovanjema $\Delta \phi$ hitro spreminja,
je gostota svetlobnega toka na zaslonu enaka vsoti prispevkov posameznih 
delnih valovanj in interference ne vidimo. Takrat je kontrast enak 0. 

Kako pa je z ohranitvijo energijskega toka pri interferenci? Omejimo 
se na primer, ko sta amplitudi obeh delnih valovanj enaki. Takrat
je povprečni energijski  tok čez veliko območje prostora enak (enačba~\ref{eq:06_09}):
\beq
\langle j \rangle = \langle 4j_0 \cos^2 (\Delta \phi/2) \rangle  = \frac{1}{2}(4j_0) = 2j_0.
\label{eq:06_10}
\eeq
Po pričakovanju je povprečna gostota energijskega toka interferenčnega vzorca
enaka vsoti gostot energijskega toka obeh vpadnih valovanj. Enako velja tudi 
v primeru, ko amplitudi vpadnih delnih valovanj nista enaki. Pri interferenci se 
torej energijski tok ohranja, vendar se energija prerazporedi.

\begin{example}{\bf Interferenca dveh ravnih valovanj.}
Opazujmo interferenco dveh ravnih valovanj z enakima intenzitetama, ki pod kotom 
vpadata eno glede na drugo (slika~\ref{fig:06_int}). Njuna valovna vektorja na splošno
zapišemo kot: $\mathbf{k}_1 = (k_x,0, k_z)$ in $\mathbf{k}_2 = (-k_x,0, k_z)$. 
Valovanji sta oblike:
\beq
E_1 = E_0 e^{ik_x x} e^{ik_z z}e^{-i\omega t} \qquad \mathrm{in} \qquad 
E_2 = E_0 e^{-ik_x x} e^{ik_z z}e^{-i\omega t}.
\label{eq:06_12}
\eeq
Interferenca teh dveh valovanj da:
\beq
E = E_1+E_2 = E_0 e^{ik_z z -i\omega t }\left(e^{ik_x x}+e^{-ik_x x} \right) = 
E_0 e^{ik_z z -i\omega t } 2 \cos(k_x x).
\label{eq:06_14}
\eeq
Izračunamo gostoto svetlobnega toka in za interferenčni vzorec dobimo obliko:
\beq
j = 4 j_0 \cos^2(k_x x).
\label{eq:06_15}
\eeq
\begin{figure}[!h]
\centering
\def\svgwidth{120truemm} 
\input{slike/06_interferenca_1.pdf_tex}
\caption{Interferenca dveh ravnih valovanj, ki pod kotom vpadata eno na drugo. Modra barva
označuje negativno vrednost jakosti električnega polja in rdeča pozitivno. Kjer je belo, je jakost
električnega polja enaka nič (levo). Intenzitetni interferenčni vzorec
ima značilne bele črte na mestu oslabitev, potujoče ojačitve z leve slike pa se 
na detektorju izpovprečijo v znano interferenčno sliko (desno).}
\label{fig:06_int}
\end{figure}

\end{example}

\begin{example}{\bf Interferenca dveh krožnih valovanj.}
Poglejmo še interferenco dveh krožnih valovanj v ravnini. Krožno valovanje izhaja
iz ene točke in se v koncentričnih krogih širi po ravnini navzven. V
ravnini $xy$ dve krožni valovanji zapišemo kot:
\beq
E_1 \propto \exp\left( ik\sqrt{x^2+y^2}\right)  e^{-i\omega t}
\qquad \mathrm{in} \qquad
E_2 \propto \exp\left( ik\sqrt{(x-d)^2+y^2}\right)  e^{-i\omega t}.
\label{eq:06_16a}
\eeq
Pri zapisu smo privzeli, da sta izhodišči valovanj razmaknjeni vzdolž osi $x$, 
razmik med njima pa je enak $d$. Pojemanja amplitude
z oddaljenostjo od izvora nismo zapisali. Nekaj primerov amplitudnih in pripadajočih
intenzitetnih interferenčnih vzorcev, ki 
nastanejo pri različnih razmikih $d$ in različni valovni dolžini, 
je narisanih na sliki~\ref{fig:06_intkrog}.
\begin{figure}[!h]
\centering
\includegraphics[width=140truemm]{slike/06_interferenca_krog_nov.png}
\caption{Interferenca dveh krožnih valov, pri čemer zgornja vrsta prikazuje 
amplitudo polja, spodnja pa intenziteto. Na mestih, kjer sta valovanji iz faze (destruktivna
interferenca), se pojavijo oslabitve, kjer sta valovanji v fazi (konstruktivna interferenca),
pa ojačitve. Od leve proti desni razmik med izvoroma narašča, zato se ojačitve gostijo, zadnji 
stolpec pa ponazori premik ojačitev pri spremenjeni valovni dolžini valovanja.}
\label{fig:06_intkrog}
\vglue-5truemm
\end{figure}

\end{example}

\section{Interferenca z delitvijo valovne fronte}
Pri večini interferenčnih poskusov ustvarimo interferenco z enim samim 
snopom svetlobe, ki ga razdelimo na dva dela. S tem zagotovimo isto 
frekvenco, isto začetno fazo in enako polarizacijo obeh delnih valovanj. V uvodu smo
že omenili, da lahko vpadno valovanje razdelimo na dva dela z delitvijo 
valovne fronte valovanja (npr. Youngov poskus) ali z delitvijo amplitude
valovanja (npr. Michelsonov interferometer). Najprej si poglejmo prvi primer.

Najpomembnejši primer interference z delitvijo valovne fronte je nedvomno
Youngov poskus (1801), na podlagi katerega je angleški fizik Thomas Young
sklepal, da je svetloba transverzalno valovanje. 
Pri tem poskusu vpadno valovanje iz monokromatskega izvora  
simetrično usmerimo na objektni zaslon, v katerem sta dve enaki ozki
reži, in opazujemo sliko na oddaljenem zaslonu\footnote{Vpadno valovanje mora biti 
kar se da koherentno. Young je to dosegel tako, da je vpadno svetlobo najprej 
speljal skozi eno tanko režo in jo šele nato usmeril na zaslon z dvema režama. 
Več o koherenci bomo spoznali v poglavju~\ref{chap:Koherenca}.}. Poiščimo 
lego ojačitev in oslabitev na oddaljenem opazovalnem zaslonu.
\begin{figure}[ht]
\centering
\def\svgwidth{100truemm} 
\input{slike/06_Young.pdf_tex}
\caption{Pri prvotnem Youngovem poskusu monokromatska svetloba prehaja ozko 
režo in vpada simetrično na dve ozki reži v razmiku $D$ na objektnem zaslonu. 
Na oddaljenem zaslonu opazujemo interferenčno sliko. Pri tem $z_0$ opisuje razdaljo
od objektnega do opazovalnega zaslona, $r_1$ in 
$r_2$ razdalji od rež do izbrane točke na opazovalnem zaslonu in kot 
$\vartheta$ približno smer izbrane točke glede na reži v zaslonu. 
Privzamemo, da je $z_0 \gg D$.}
\label{fig:06_Young}
\vglue-3truemm
\end{figure}

Naj bodo $z_0$ razdalja od objektnega zaslona z dvema režama 
do opazovalnega zaslona, $D$ razdalja med središčema rež, 
$r_1$ razdalja med sredino prve reže in izbrano točko na zaslonu
ter $r_2$ razdalja med sredino druge reže in isto točko na zaslonu. Potem za velike 
oddaljenosti $r_1, r_2 \gg D$ velja:
\beq
\Delta r = r_1-r_2 \approx D \sin\vartheta,
\label{eq:06_16}
\eeq
pri čemer je $\vartheta$ kot med osjo $z$ in zveznico med točko na sredini med režama in
izbrano točko na opazovalnem zaslonu. Ker ležita reži simetrično, sta intenziteti delnih
valovanj enaki, prav tako tudi njuni začetni fazi. Za izračun gostote svetlobnega toka na zaslonu 
zato lahko uporabimo enačbo~(\ref{eq:06_09}). Vstavimo fazno razliko $\Delta \phi \approx 
k\Delta r$ in z upoštevanjem enačbe~(\ref{eq:06_16}) dobimo:
\beq
j = 4 j_0 \cos^2\left(\frac{\Delta \phi}{2}\right) = 4j_0 \cos^2\left(\frac{k\Delta r}{2}\right) = 
4j_0 \cos^2\left(\frac{kD\sin \vartheta}{2}\right)\!\!.
\label{eq:06_17}
\eeq
V točkah, kjer se prispevka obeh valovanj seštejeta, nastopi 
konstruktivna interferenca in na zaslonu opazimo ojačitve. 
Izhajajoč iz enačbe~(\ref{eq:06_17}) pogoj za ojačitev zapišemo kot:
\beq
\frac{kD\sin \vartheta}{2} = N\pi,
\label{eq:06_18}
\eeq
pri čemer je $N$ celo število. 

Od tod izračunamo pogoj za kote $\vartheta_\mathrm{max}$, pri katerih
se pojavijo ojačitve:
\boxeq{eq:InterferencaMax}{
D \sin\vartheta_\mathrm{max} = N \lambda.
}
Zapis pove, da se ojačitve pojavijo, ko je razlika poti od rež 
do točke na zaslonu enaka večkratniku valovne dolžine valovanja.

V vmesnih točkah, kjer se prispevka valovanj odštejeta, nastopi 
destruktivna interferenca in na zaslonu opazimo oslabitve (neosvetljene proge). 
Pogoj za oslabitve je:
\boxeq{eq:InterferencaMin}{
D \sin\vartheta_\mathrm{min} = \left(N + \frac{1}{2}\right)\lambda.
}
Interferenčni vzorec, ki ga opazujemo na oddaljenem opazovalnem zaslonu, je za majhne 
kote periodičen. Izračunajmo periodo ponavljanja $\xi_0$ (slika~\ref{fig:06_Young}). 
Iz enačbe~(\ref{eq:06_17}) sledi zahteva:
\beq
\Delta \left(\frac{kD\sin \vartheta}{2}\right) = \pi.
\label{eq:06_19}
\eeq
Za majhne kote velja $\sin \vartheta \approx \vartheta$ in zapišemo: 
\beq
\frac{k D\,\Delta \sin\vartheta}{2} \approx \frac{k D\, \Delta \vartheta}{2}
 \approx \frac{k D\,\xi_0}{2z_0} = \pi,
\label{eq:06_20}
\eeq
od koder sledi:
\beq
\xi_0 = \frac{\lambda z_0}{D}.
\label{eq:06_21}
\eeq
Bliže kot sta reži (manjši $D$), bolj razmaknjene so ojačitve (večji $\xi_0$). 
Za $D=1~\si{\micro\metre}$ je na oddaljenosti $z_0 = 1~\si{m}$ razmik med ojačitvami za 
svetlobo z valovno dolžino $\lambda  = 500~\si{nm}$ enak $\xi_0 = 0,5~\si{m}$, 
medtem ko je pri $D = 1~\si{mm}$ na isti razdalji vrednost $\xi_0 = 0,5~\si{mm}$. 

\begin{example}{\bf Fraunhoferjeva uklonska obravnava Youngovega poskusa.}
Youngov poskus na dveh režah lahko obravnavamo tudi v Fraunhoferjevi uklonski
sliki (poglavje~\ref{chap:Fraunhofer}). Spomnimo se uklona na $N$ režah
širine $d$ s periodo ponavljanja $D$ (enačba~\ref{eq:uklonNrez}):
\beq
j(\vartheta) = j_0 \left(\frac{\sin\left(kd\sin\vartheta/2\right)}{kd\sin\vartheta/2}\right)^2
\left(\frac{\sin\left(NkD\sin\vartheta/2\right)}{\sin\left(kD\sin\vartheta/2\right)}\right)^2\!\!.
\label{eq:06_22}
\eeq
Pokažimo, da je uklonska slika za $N=2$ v limitnem primeru, ko sta
reži ozki in velja $d\to 0$, enaka izračunanemu interferenčnemu vzorcu 
(enačba~\ref{eq:06_17}). Strukturni faktor (prvi oklepaj) je za ozke reže 
enak $1$ in intenziteta vrhov z oddaljenostjo od optične osi $z$ ne pojema. Ostane:
\beq
j = j_0~\frac{\sin^2(2kD\sin\vartheta/2)}{\sin^2(kD\sin\vartheta/2)} = 
j_0~\frac{4 \sin^2(kD\sin\vartheta/2)\cos^2(kD \sin\vartheta/2)}{\sin^2(kD \sin\vartheta/2)},
\eeq
pri čemer smo uporabili izraz za zapis dvojnega kota. Ulomek krajšamo in dobimo:
\beq
j = 4j_0 \cos^2\left(\frac{kD \sin\vartheta}{2}\right)\!\!,
\label{eq:06_23}
\eeq
kar je, po pričakovanju, enako enačbi~(\ref{eq:06_17}).
\end{example}

\begin{remark}
Oglejmo si še nekaj alternativnih postavitev interferenčnih poskusov
z delitvijo žarka. Pri Youngovem poskusu se namreč težko izognemo 
težavam, povezanim s končno širino reže $d$. Njegovi
sodobniki so zato poskušali podoben interferenčni vzorec ustvariti 
nad druge načine, s čimer bi se izognili domnevam, da se interferenčni 
vzorec pojavi kot posledica pojavov na robu rež. 
Prvi primer je Lloydovo zrcalo (1834), pri katerem interferirajo žarki, 
ki vpadajo neposredno od izvora svetlobe, in žarki, ki se odbijejo od zrcala
(slika~\ref{fig:06_Lloyd}\,a). 
Drugi primer sta Fresnelovi zrcali, pri katerih interferirata
sva snopa svetlobe, ki se odbijata vsak od svojega zrcala 
(slika~\ref{fig:06_Lloyd}\,b). Omenimo še postavitev s Fresnelovo biprizmo. 
V tem primeru se svetloba, ki izhaja iz enega svetila, 
na biprizmi lomi, lomljena žarka pa med seboj 
interferirata (slika~\ref{fig:06_Lloyd}\,c).

\begin{figure}[ht]
\centering
\def\svgwidth{140truemm} 
\input{slike/06_Lloyd.pdf_tex}
\caption{Različne postavitve za opazovanje interference: Lloydovo zrcalo (a), Fresnelovi
zrcali (b) in Fresnelova biprizma (c). S črko $S$ smo označili izvore svetlobe in s  
$S'$ navidezne izvore. Črtkane črte prikazujejo poti navideznih snopov svetlobe.}
\label{fig:06_Lloyd}
\end{figure}
\vskip1truecm
\end{remark}

\begin{remark}
Youngov poskus je osnova za delovanje Rayleighovega interferometra, ki ga uporabljamo
za natančno merjenje lomnega količnika plinov. V njem svetlobo 
iz točkastega svetila (laserja) najprej razširimo v širok snop in nato usmerimo
na zaslon z dvema ozkima režama. Za režama svetloba prehaja skozi plinsko komoro, 
nato pa žarka z drugo lečo ponovno zberemo, kjer interferirata. Lege ojačitev na zaslonu
so odvisne od relativnega faznega zamika obeh delnih valovanj, ta pa je odvisen od
lomnega količnika plina, po katerem je svetloba potovala. Ker so interferenčne
meritve izredno natančne, lahko določamo vrednost $n-1$ na $10^{-8}$ natančno.

\begin{figure}[ht]
\centering
\def\svgwidth{70truemm} 
\input{slike/06_Lloyd.pdf_tex}
\caption{Različne postavitve za opazovanje interference: Lloydovo zrcalo (a), Fresnelovi
zrcali (b) in Fresnelova biprizma (c). S črko $S$ smo označili izvore svetlobe in s  
$S'$ navidezne izvore. Črtkane črte prikazujejo poti navideznih snopov svetlobe.}
\label{fig:06_RayleighInt}
\end{figure}
\end{remark}


\section{Interferenca z delitvijo amplitude}
\label{chap:Michelson}
Amplitudno delitev valovanja dosežemo s polprepustnimi zrcali. Kot že ime pove, taka
zrcala del vpadne svetlobe odbijejo in del prepustijo. 
Najznačilnejši primer interferometra z delitvijo amplitude je Michelsonov interferometer.
Z njim je Michelson pokazal, da je hitrost svetlobe v smeri gibanja Zemlje enaka hitrosti
svetlobe v smeri pravokotno na smer gibanja in tako ovrgel teorijo o obstoju etra. 

V Michelsonovem interferometru vpadno svetlobo s polprepustnim zrcalom razdelimo 
na dva snopa. Delna snopa usmerimo na  vsak od svojega zrcala in interferirata na detektorju. 
S premikanjem enega od zrcal spreminjamo
zakasnitev med žarkoma in na detektorju se izmenično pojavljajo ojačitve in oslabitve.
Gostota svetlobnega toka na detektorju je (enačba~):
\beq
j = 4j_0 \cos^2(\Delta \phi/2),
\label{eq:06_24}
\eeq
pri čemer je $\Delta \phi = k_0(l_1-l_2)$....


\section{Interferenca na tanki plasti}
Do zdaj smo obravnavali interferenco, pri kateri je bilo skupno valovanje sestavljeno iz le 
dveh prispevkov. Zdaj si oglejmo primer interference, ki nastane kot vsota zelo velikega 
števila delnih valovanj. 

Naj ravno valovanje vpada na tanko plast snovi z lomnim količnikom $n_2$, lomni količnik 
okolice pa naj bo enak $n_1$, pri čemer se omejimo na primer TE polariziranega valovanja.
Ob vpadu na plast se del svetlobe odbije, del pa lomi v snov po lomnem zakonu (enačba~) 
z amplitudo, ki jo določa Fresnelova enačba (). Prepuščeni val potuje po plasti do meje
na drugi strani, kjer je ga ponovno nekaj odbije, del pa je prepuščen v smeri, ki 
je enaka vpadni smeri. Odbiti del valovanja potuje 
do vpadne meje, kjer se deloma odbije, del pa ga izhaja na vpadni strani ... 

Jakost prepuščenega električnega polja je tako sestavljena iz velikega števila prispevkov, ki 
izhajajo v isti smeri, njihove amplitude pa so vedno manjše. Ker izhodni žarki 
izhajajo na različnih točkah, jih navadno z zbiralno lečo preslikamo v eno točko. Pri pravokotnem
vpadu na tanko plast teh težav ni. 

Zanima nas gostota prepuščenega svetlobnega toka $j$ v odvisnosti od 
lomnih količnikov $n_1$ in $n_2$, vpadnega kota $\alpha$ in debeline plasti $d$. 

Najprej z lomnim zakonom izračunamo kot $\beta$, pod katerim se širi svetloba v plasti:
\beq
\sin\beta = \frac{n_1}{n_2}\sin\alpha.
\label{eq:06_25}
\eeq
Amplitudna odbojnost $r$ in prepustnot $t$ na prvi meji sta enaka (enačba~):
\beq
r_{12} = \frac{n_1\cos \alpha - n_2\cos \beta}{n_1\cos \alpha + n_2\cos \beta}\qquad 
\mathrm{in}\qquad t_{12} = 1+r_{12},
\label{eq:06_26}
\eeq
na drugi meji pa:
\beq
r_{21} = \frac{n_2\cos \beta - n_1\cos \alpha}{n_2\cos \beta + n_1\cos \alpha}\qquad 
\mathrm{in}\qquad t_{21} = 1+r_{21}.
\label{eq:06_27}
\eeq
Vidimo, da velja $r_{12} = r_{21}$. Ko poznamo vse koeficiente, lahko zapišemo
posamezne prispevke jakosti električnega polja na prepuščeni strani:
\begin{align}
E_1 &= E_0\,t_{12}\,t_{21},\\
E_2 &= E_0\,t_{12}\,r_{21}\,r_{21}\,e^{i\phi}\,t_{21},\\
E_3 &= E_0\,t_{12}\,r_{21}\,r_{21}\,e^{i\phi}\,r_{21}\,r_{21}\,e^{i\phi}\,t_{21},\\
&...\\
E_{N+1} &= E_0\,t_{12}\,\left(r_{21}\,r_{21}\right)^N\,e^{iN\phi}\,t_{21}\\
&...
\label{eq:06_29}
\end{align}

\begin{figure}[ht]
\centering
\def\svgwidth{140truemm} 
%\input{slike/06_.pdf_tex}
\caption{SLIKA}
\label{fig:06_Nrez}
\end{figure}

Pri tem smo upoštevali tudi fazni zamik zaradi dodatne prepotovane poti v plasti:
\beq
\phi = k_0 2d \cos \beta n_2.
\label{eq:06_31}
\eeq
Dodaj izpeljavo, skico. 
Celotno polje zapišemo kot vsoto vseh teh prispevkov:
\begin{align}
E_t &= E_1+E_2+E_3+... + E_N + ... \\
& E_0\, t_{12}\,t_{21}\,\left(1 + r_{21}^2 e^{i\phi} + r_{21}^4 e^{2i\phi} + r_{21}^6 e^{3i\phi} + ... \right).
\label{eq:06_30}
\end{align}
Geometrijsko vrsto lahko seštejemo in za prepuščeno polje dobimo:
\beq
E_t = \frac{E_0 t_{21}t_{12}}{1-r_{21}^2e^{i\phi}} = E_0 t.
\label{eq:06_32}
\eeq
Zanima nas gostota prepuščenega energijskega toka glede na vpadno. Ker sta lomni količnik snovi
in smer širjenja svetlobe na izhodni strani enaka kot na vpadni, je prepustnost $T$ kar enaka $|t|^2$.
Uporabimo enačbo~(\ref{eq:06_32}) in upoštevamo zveze med amplitudnimi prepustnostmi in odbojnostmi:
$t_{21} = 1+r_{21} = 1-r_{12}$ in $t_{12} = 1+r_{12}$. Vpeljemo odbojnost $R = r_{12}^2 = R$.
Prepustnost je tako:
\beq
T
\label{eq:06_33}
\eeq



\section{Fabry-Perotov interferometer}

\section{Večplastni nanosi}
