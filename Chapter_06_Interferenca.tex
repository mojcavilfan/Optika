%\chapterimage{Geometrijska.jpg} % Chapter heading image

\chapter{Interferenca}
Uklon in interferenca sta pravzaprav manifestacija enega in istega pojava 
-- superpozicije optičnega polja. Pri poglavju o uklonu
so nas zanimale predvsem uklonski vzorci oziroma slike. V poglavju
o interferenci pa se bomo bolj posvetili aplikacijam, interferometrom.

'Interferometry' is a measurement method using the phenomenon of interference of waves (usually light, radio or sound waves). The measurements may include those of certain characteristics of the waves themselves and the materials that the waves interact with. In addition, interferometry is used to describe the techniques that use light waves for the study of changes in displacement. This displacement measuring interferometry is extensively used for calibration and mechanical stage motion control in precision machining.

By using two light beams (usually by splitting one beam into two), an interference pattern can be formed when these two beams superpose. Because the wavelength of the visible light is very short, small changes in the differences in the optical paths (distance travelled) between the two beams can be detected (as these differences will produce noticeable changes in the interference pattern). Hence, the optical interferometry has been a valuable measurement technique for more than a hundred years. Its accuracy has later been improved with the invention of lasers. 

Interferometers are widely used in science and industry for the measurement of small displacements, refractive index changes and surface irregularities. 

\section{Interferenca}
Električno polje obravnavajmo kot skalarno količino. V eksperimentih
to realiziramo tako, da uporabimo polarizirano valovanje, pri čemer
imajo so vsa valovanja enako polarizirana. Valovanje je potem v splošnem 
$E= E(\mathbf{r},t)$. 

Izračunajmo naprej interferenčni vzorec dveh ravnih valovanj z enako
polarizacijo in enako frekvenco. Valovanji zapišemo kot:
\beq
E_1 = E_{10} e^{i\mathbf{k}_1 \cdot \mathbf{r} - i \omega t + i \delta_1}
\qquad \mathrm{in} \qquad
E_2 = E_{20} e^{i\mathbf{k}_2 \cdot \mathbf{r} - i \omega t + i \delta_2}.
\label{eq:06_01}
\eeq
Pri tem privzamemo, da sta $E_{10}$ in $E_{20}$ realni števili. Celotno
električno polje je potem:
\beq
E = E_1 + E_2 = E_{10} e^{i\phi_1 - i \omega t} + E_{20} e^{i\phi_2 - i \omega t},
\label{eq:06_02}
\eeq
pri čemer sta $\phi_{1,2} = \mathbf{k}_{1,2} \cdot \mathbf{r} + \delta_{1,2}$. 
Zanima nas gostota svetlobnega toka $j$, ki je enaka:
\beq
j = \left|\langle \mathbf{S}\rangle \right| = \frac{1}{2}\varepsilon \varepsilon_0 |E|^2c
\label{eq:06_03}
\eeq
in 
\beq
j \propto (E_1+E_2)(E_1^*+E_2^*)  = 
\left( E_{10} e^{i\phi_1 - i \omega t} + E_{20} e^{i\phi_2 - i \omega t}\right)
\left( E_{10} e^{-i\phi_1 + i \omega t} + E_{20} e^{-i\phi_2 + i \omega t}\right).
\label{eq:06_04}
\eeq
Dobimo:
\beq
j \propto E_{10}^2 + E_{20}^2 + E_{10}E_{20} \left(e^{i\phi_1-i\phi_2}+ e^{-i\phi_1+i\phi_2}\right)
\label{eq:06_05}
\eeq
in 
\beq
j = j_1 + j_2 + 2\sqrt{j_1 j_2} \cos(\Delta \phi),
\label{eq:06_06}
\eeq
pri čemer sta $j_1$ in $j_2$ gostoti svetlobnih tokov prvega in drugega delnega valovanja,
$\Delta \phi$ pa označuje razliko faz $\phi_1-\phi_2$. Le kadar je ta razlika neodvisna
od časa, lahko v eksperimentu izmerimo tudi tretji člen v enačbi~(\ref{eq:06_06}). Sicer se 
ta člen pri meritvi izpovpreči. V odvisnosti od (časovno neodvisnega) $\Delta \phi$,  
gostota svetlobnega toka zavzema vrednosti:
\beq
\left(\sqrt{j_1}- \sqrt{j_2}\right)^2 < j < \left(\sqrt{j_1} +\sqrt{j_2}\right)^2.
\label{eq:06_07}
\eeq
\begin{figure}[ht]
\centering
\def\svgwidth{120truemm} 
%\input{slike/06_kontrast.pdf_tex}
\caption{SLIKA}
\label{fig:06_kontrast}
\end{figure}

Vpeljemo vidljivost ali kontrast interferenčnega vzorca:
\beq
v = \frac{j_\mathrm{max}- j_\mathrm{min}}{j_\mathrm{max}+ j_\mathrm{min}},
\label{eq:06_08}
\eeq
ki lahko zavzema vrednosti med 0 in 1.

V posebnem primeru, ko sta amplitudi obeh valovanj enaki $E_{01} = E_{02} = E_0$ in velja 
$j_1 = j_2 = j_0$, dobimo:
\beq
j = 2j_0 + 2j_0 \cos (\Delta \phi) = 4j_0 \cos^2 (\Delta \phi/2)
\label{eq:06_09}
\eeq
Interferenčni vzorec dveh enako močnih snopov svetlobe zavzame vrednosti med 0 
in $4j_0$, kar pomeni, da je kontrast takega vzorca enak 1. 

Kako pa je z ohranitvijo energijskega toka? Pri interferenci se energijski 
tok seveda ohranja, pride le do preporazdelitve energije. Povprečni energijski 
tok čez veliko območje prostora pa je 
\beq
\langle j \rangle = \frac{1}{2}(4j_0) = 2j_0,
\label{eq:06_10}
\eeq
kar ustreza gostoti energijskega toka začetnih dveh valovanj. 

Poglejmo interferenco dveh valovanj, ki pod kotom vpadata eno glede na drugo.
\begin{figure}[ht]
\centering
\def\svgwidth{120truemm} 
%\input{slike/06_interferenca.pdf_tex}
\caption{SLIKA}
\label{fig:06_interferenca}
\end{figure}
Potem prvo valovanje zapišemo kot:
\beq
E_1 = e^{ik_\perp x} e^{ik_z z}e^{-i\omega t}
\label{eq:06_11}
\eeq
in drugo kot:
\beq
E_1 = e^{-ik_\perp x} e^{ik_z z}e^{-i\omega t},
\label{eq:06_12}
\eeq
pri čemer sta vektorja:
\beq
\mathbf{k}_1 = (k_\perp,0, k_z) \qquad \mathrm{in} \qquad \mathbf{k}_2 = (-k_\perp,0, k_z).
\label{eq:06_13}
\eeq
Interferenca teh dveh valovanj da:
\beq
E = E_1+E_2 = E_0 e^{ik_z z -i\omega t }\left(e^{ik_\perp x}+e^{-ik_\perp x} \right) = 
E_0 e^{ik_z z -i\omega t } 2 \cos(k_\perp x).
\label{eq:06_14}
\eeq
Interferenčni vzorec, ki ga opazujemo, je:
\beq
j = 4 j_0 \cos^2(k_\perp x).
\label{eq:06_15}
\eeq

\section{Interferenca z delitvijo valovne fronte}
V večini interferometričnih ustvarimo interferenco z enim samim snopom svetlobe, 
ki ga razdelimo na dva dela. To lahko naredimo na dva načina: lahko razdelimo
valovno fronto valovanja (npr. Youngov poskus) ali pa razdelimo amplitudo
valovanja (npr. Michelsonov interferometer). 

Najpomembnejši primer interference z delitvijo valovne fronte je nedvomno
Youngov poskus (1801), ki je angleškega fizika Thomasa Younga vodil do spoznanja,
da je svetloba transverzalno valovanje. Pri tem poskusu vpadno valovanje iz 
monokromatskega izvora simetrično usmerimo na zaslon, v katerem sta dve enaki ozki
reži in opazujemo sliko na oddaljenem zaslonu\footnote{Vpadno valovTorej potrebujemoanje mora biti 
koherentno. Več o tej lastnosti valovanja 
bomo spoznali v poglavju~\label{chap:Koherenca}.}. Uklonsko sliko 
pravzaprav že poznamo, saj jo lahko izračunamo s Fraunhoferjevim 
uklonskim približkom.
\begin{figure}[ht]
\centering
\def\svgwidth{120truemm} 
%\input{slike/06_Young.pdf_tex}
\caption{SLIKA}
\label{fig:06_Young}
\end{figure}

Naj bodo $z_0$ razdalja od objektnega zaslona do opazovalnega zaslona, $D$ razdalja 
med središčema rež, $r_1$ razdalja med sredino prve reže in izbrano točko na zaslonu
ter $r_2$ razdalja med sredino druge reže in točko na zaslonu. Potem za velike 
oddaljenosti $r_1, r_2 \gg D$ velja:
\beq
\Delta r = r_1-r_2 \approx D \sin\vartheta,
\label{eq:06_16}
\eeq
pri čemer je $\vartheta$ kot med osjo $z$ in zveznico med točko na sredini med režama in
izbrano točko na opazovalnem zaslonu. Uporabimo enačbo~(\ref{eq:06_09}) in dobimo:
\beq
j = 4 j_0 \cos^2\left(\frac{\Delta \Phi}{2}\right) = 4j_0 \cos^2\left(\frac{kD\sin \vartheta}{2}\right).
\label{eq:06_17}
\eeq

V točkah, kjer se prispevka obeh valovanj seštevata, nastopi tako imenovana
konstruktivna interferenca in na zaslonu opazimo ojačitve. Pogoj za ojačitev zapišemo kot:
\beq
\frac{kD\sin \vartheta}{2} = N\pi,
\label{eq:06_18}
\eeq
pri čemer je $N$ celo število. Od tod izračunamo pogoj za kote $\vartheta_\mathrm{max}$, pri katerih
se pojavijo vrhovi:
\boxeq{eq:InterferencaMax}{
D \sin\vartheta = N \lambda.
}
Ker je leva stran enačbe enaka razliki poti od rež do točke na zaslonu, je zapis preprosto razumeti:
razlika poti od središč rež do točke na zaslonu mora biti enaka večkratniku valovne dolžine valovanja.

V vmesnih točkah, kjer se prispevka valovanj odštevata, je destruktivna
interferenca in na zaslonu opazimo oslabitve (temne proge). Podobno kot za ojačitve
zapišemo pogoj za oslabitve:
\boxeq{eq:InterferencaMin}{
D \sin\vartheta = \left(N + \frac{1}{2}\right)\lambda.
}

Vzorec, ki se pojavi v daljnjem polju oziroma na oddaljenem zaslonu,
je periodičen s periodo ponavljanja $\Delta \xi_0$. Izračunajmo jo. Velja:
\beq
\Delta \left(\frac{kD\sin \vartheta}{2}\right) = \pi.
\label{eq:06_19}
\eeq
Za majhne kote $\vartheta$ velja 
















\section{Delitev amplitude}
\section{Tanka plast}
\section{Fabry-Perotov interferometer}
\section{Večplastni nanosi}
\section{Uporaba interference}
