%\chapterimage{Geometrijska.jpg} % Chapter heading image

\chapter{Interferenca}
Uklon in interferenca sta pravzaprav manifestacija enega in istega pojava 
-- superpozicije optičnega polja. Pri poglavju o uklonu
so nas zanimale predvsem uklonski vzorci oziroma slike. V poglavju
o interferenci pa se bomo bolj posvetili aplikacijam, interferometrom.

\section{Interferenca}
Električno polje obravnavajmo kot skalarno količino. V eksperimentih
to realiziramo tako, da uporabimo polarizirano valovanje, pri čemer
imajo so vsa valovanja enako polarizirana. Valovanje je potem v splošnem 
$E= E(\mathbf{r},t)$. 

Izračunajmo naprej interferenčni vzorec dveh ravnih valovanj z enako
polarizacijo in enako frekvenco. Valovanji zapišemo kot:
\beq
E_1 = E_{10} e^{i\mathbf{k}_1 \cdot \mathbf{r} - i \omega t + i \delta_1}
\qquad \mathrm{in} \qquad
E_2 = E_{20} e^{i\mathbf{k}_2 \cdot \mathbf{r} - i \omega t + i \delta_2}.
\label{eq:06_01}
\eeq
Pri tem privzamemo, da sta $E_{10}$ in $E_{20}$ realni števili. Celotno
električno polje je potem:
\beq
E = E_1 + E_2 = E_{10} e^{i\phi_1 - i \omega t} + E_{20} e^{i\phi_2 - i \omega t},
\label{eq:06_02}
\eeq
pri čemer sta $\phi_{1,2} = \mathbf{k}_{1,2} \cdot \mathbf{r} + \delta_{1,2}$. 
Zanima nas gostota svetlobnega toka $j$, ki je enaka:
\beq
j = \left|\langle \mathbf{S}\rangle \right| = \frac{1}{2}\varepsilon \varepsilon_0 |E|^2c
\label{eq:06_03}
\eeq
in 
\beq
j \propto (E_1+E_2)(E_1^*+E_2^*)  = 
\left( E_{10} e^{i\phi_1 - i \omega t} + E_{20} e^{i\phi_2 - i \omega t}\right)
\left( E_{10} e^{-i\phi_1 + i \omega t} + E_{20} e^{-i\phi_2 + i \omega t}\right).
\label{eq:06_04}
\eeq
Dobimo:
\beq
j \propto E_{10}^2 + E_{20}^2 + E_{10}E_{20} \left(e^{i\phi_1-i\phi_2}+ e^{-i\phi_1+i\phi_2}\right)
\label{eq:06_05}
\eeq
in 
\beq
j = j_1 + j_2 + 2\sqrt{j_1 j_2} \cos(\Delta \phi),
\label{eq:06_06}
\eeq
pri čemer sta $j_1$ in $j_2$ gostoti svetlobnih tokov prvega in drugega delnega valovanja,
$\Delta \phi$ pa označuje razliko faz $\phi_1-\phi_2$. Le kadar je ta razlika neodvisna
od časa, lahko v eksperimentu izmerimo tudi tretji člen v enačbi~(\ref{eq:06_06}). Sicer se 
ta člen pri meritvi izpovpreči. V odvisnosti od (časovno neodvisnega) $\Delta \phi$,  
gostota svetlobnega toka zavzema vrednosti:
\beq
\left(\sqrt{j_1}- \sqrt{j_2}\right)^2 < j < \left(\sqrt{j_1} +\sqrt{j_2}\right)^2.
\label{eq:06_07}
\eeq
\begin{figure}[ht]
\centering
\def\svgwidth{120truemm} 
%\input{slike/06_kontrast.pdf_tex}
\caption{SLIKA}
\label{fig:06_kontrast}
\end{figure}

Vpeljemo vidljivost ali kontrast interferenčnega vzorca:
\beq
v = \frac{j_\mathrm{max}- j_\mathrm{min}}{j_\mathrm{max}+ j_\mathrm{min}},
\label{eq:06_08}
\eeq
ki lahko zavzema vrednosti med 0 in 1.








\section{Interferometri}
\section{Delitev valovno fronte}
\section{Delitev amplitude}
\section{Tanka plast}
\section{Fabry-Perotov interferometer}
\section{Večplastni nanosi}
\section{Uporaba interference}
